% !TeX program = lualatex
\documentclass[11pt,letterpaper]{article}

\usepackage{fontspec}
\usepackage{tabularx}
\usepackage{booktabs}
\usepackage{bucolors}
\usepackage[colorlinks=true,allcolors=black,urlcolor=bugold]{hyperref}
\usepackage{mathtools}
\usepackage{mdframed}
\usepackage{titling}
\usepackage{fancyhdr}
\usepackage{lastpage}
\usepackage[english]{babel}
\usepackage[sf,bf]{titlesec}
\usepackage[inline]{enumitem}
\usepackage[margin=1in,letterpaper]{geometry}
\usepackage{multicol}
\usepackage{soul}

\setmainfont{TeX Gyre Termes}[Ligatures=TeX]
\setsansfont{TeX Gyre Heros}[Ligatures=TeX]
\setmonofont{TeX Gyre Cursor}

\title{Analytical Chemistry 1}
\author{Dr.\ Daniel A.\ McCurry}
\date{Fall 2022}

\newcommand{\classnum}{CHEM321}

\pagestyle{fancy}
\lhead{}
\chead{}
\rhead{}
\lfoot{\footnotesize\sffamily McCurry --- \classnum\ --- FA22}
\cfoot{}
\rfoot{\footnotesize\sffamily\thepage~of~\pageref{LastPage}}
\renewcommand{\headrulewidth}{0pt}
\renewcommand{\footrulewidth}{0.4pt}

\pretitle{\noindent\color{bumaroon}
	\sffamily\bfseries\Large
	\classnum~Lab\newline
	\LARGE\expandafter\MakeUppercase\expandafter}
\posttitle{\par\medskip}
\preauthor{\noindent\sffamily}
\postauthor{ --- }
\predate{\sffamily}
\postdate{}

\setlength{\droptitle}{-2em}

\setcounter{secnumdepth}{0}
\setlist[description]{font=\sffamily\bfseries\small}
\urlstyle{same}

\begin{document}

\maketitle
\thispagestyle{fancy}

\noindent
\begin{tabularx}{\linewidth} {@{\qquad}>{\bfseries\sffamily}r
	>{\raggedright\arraybackslash}X@{\qquad}}
	\toprule
	Day and Time: & Wed., 3:00 pm -- 6:50 pm \\ 
		  & HSC 209 \\ \\
        Instructor: & Dr.\ Daniel A. McCurry\\
		    & 	Assistant Professor of Chemistry\\
		    & 	HSC 240\\
		    & 	(570) 389-5320\\
		    & 	\href{mailto:dmccurry@bloomu.edu}{\nolinkurl{dmccurry@bloomu.edu}}\\
		    & 	\href{https://bloomu.starfishsolutions.com/starfish-ops/dl/instructor/serviceCatalog.html?bookmark=connection/20001}{HuskySuccess
		     	Profile} \\ \\
	Office Hours: & \begin{minipage}[t]{\linewidth}
		\begin{tabular}[t] {@{}lr@{\,--\,}l}
			Mon./Tue.  & 3:00 & 5:00\,p.m. \\
			Thur. & 4:00 & 5:00\,p.m. \\
				\end{tabular}
			\end{minipage} \\
		      &   In-Person or via Zoom Meeting ID
                          ``\href{https://bloomu.zoom.us/my/dmccurry}{dmccurry}''\\
                      &    \href{https://bloomu.starfishsolutions.com/starfish-ops/dl/instructor/serviceCatalog.html?bookmark=connection/20001/schedule}{Schedule
		      an Appointment} (not required) \\ \\
		Text: & McCurry, D.A.; Hallen, C.P. \textit{CHEM321
	Analytical Chemistry Laboratory Manual}, Fall 2022 Ed. \\
	       	      & \footnotesize Available for purchase during the first
		      week of laboratory. \\ \\
	Materials: & Scientific or graphing calculator (simpler calculators
	are not appropriate). \\
		   & USB flash drive for transferring laboratory data \\
		   & Carbon- or carbonless-copy laboratory notebook \\
		   & Safety goggles \\
		   & Black or blue ink \emph{ballpoint} pen \\
		   & Permanent marker \\
	\bottomrule
\end{tabularx}
\begin{mdframed}
	Note that this lab syllabus \emph{expands} on the CHEM321 lecture
	syllabus. All policies and expectations from the lecture syllabus apply
	to the laboratory as well. This syllabus serves as an addendum to
	provide additional information pertinent only to students at the
	Bloomsburg campus.
\end{mdframed}

\section{Laboratory Overview}
The laboratory for analytical chemistry studies the theory and practice of
several common methods used to determine how much analyte is in a particular
sample of material. In several of the experiments that you will do this
semester, some component of a commercial product will be analyzed. In other
experiments, specially purchased or prepared unknowns will be used. For the
latter experiments, 20--50\,\% of your laboratory report grade will
depend simply on the accuracy and precision of your work. Why are these analyses
so rigorous (or why are analytical chemists so rigorous)? When you leave this
educational institution, it is possible that thousands of dollars or someone's
life will depend upon your analysis. 

In doing these experiments, you will touch upon some classical wet chemistry
methods of analysis as well as some modern instrumental techniques. Many
experiments also use statistical analysis to illustrate how good (or bad)
the results actually are. Some of these labs are designed to be long to see how
well you plan ahead for an analysis. Some are designed to be worked in groups of
two or three to see how well you can work with others. Still others are vague to
allow you the freedom to use your ingenuity and common sense to solve the
problem.

You are expected to be prepared to do an experiment when you arrive in the
laboratory. The instructor reserves the right to quiz at any time. These
quizzes will count as part of the lab grade.

\section{Evaluations and Grading}

Laboratory attendance is mandatory. If you do not attend a laboratory and do not
have a valid excuse, you will not be able to receive credit for the experiment
performed that day. The laboratory is graded as follows:

\subsection{Point Distribution}
\begin{tabular} {l r@{~}l r<{\,\%} | r<{\,\%}}
	&&& \multicolumn{1}{c}{\sffamily\textbf{Lab Grade}} &
	\multicolumn{1}{c}{\sffamily\textbf{Lecture Grade}} \\ \midrule
	Notebook            & & &  5 & 1.25 \\
	Cleanliness         & & &  5 & 1.25 \\
	Long Reports        & 2 @ & 10\,\% each & 20 & 2.50 \\
	Short Reports       & 5 @ & 7\,\% each & 35  & 5.00 \\
	Spreadsheet Reports & 5 @ & 5\,\% each & 25  & 8.75 \\
	ACS Exam            & & & 10               & 6.25 \\ \midrule
			    &&& 100 & 25
\end{tabular}

\subsection{Graded Items}

\begin{description}
	\item[Lab Notebook:] The carbon-/carbonless-copies from your laboratory
		notebook will be collected at the end of each laboratory. These
		will be graded on legibility, completeness (procedure, materials
		and equipment used), and quality of observations. 
	\item[Cleanliness:] You are responsible for keeping your bench and
		balance clean. Generally, if you exhibit good laboratory hygiene
		by cleaning up spills and putting materials back where they came
		from, you should anticipate receiving full credit for
		cleanliness.
	\item[Reports:] Reports will be graded for format (see laboratory
		manual), grammar, spelling, chemical logic, and accuracy of the
		analysis. Reports will be due at the \emph{beginning} of
		laboratory in a format as listed in the laboratory schedule.  It
		is expected that all data reduction and plotting be performed
		with modern spreadsheet software and presented in a logical
		table format.  Reports will be in three styles: long, short, and
		spreadsheet. See your laboratory manual for more details about
		the requirements for each report.
		
		Only electronic copies will be accepted through the appropriate
		BOLT assignment submission. Spreadsheets are to be submitted in
		their raw form, as a *.xlsx, *.xls, or *.ods. \emph{Be careful
			that you are, in fact, saving as one of these listed
		formats and not as *.csv!} You must submit your long and short
		reports in PDF format (*.pdf) as this will ensure all of your
		formatting is kept.  A late lab report will be graded with a
		25\,\% penalty for every 24 hours after the deadline, including
		weekends. Note that after the third late day, you cannot score
		above a 0\,\%, so the report will not be graded and no feedback
		will be provided.

	\item[ACS Final:] A standardized American Chemical Society Analytical
		Exam will be held during the last week of laboratory. This is to compare
		our class with the national standards as set by the ACS. These
		are often a bit difficult (the national average for these tends
		to hover around 50\,\%), so your score will be adjusted to
		accommodate the topics we covered in class. 
\end{description}

\section{Policies and Expectations}
All material from the lecture syllabus applies here, but do note the additional
safety information presented in your laboratory manual. You will be required to
sign a safety policy on the first day of laboratory acknowledging that you have
read and agreed to the laboratory safety policies and procedures.

In order to attend the laboratory, you must receive at least a 80\% on the
Laboratory Safety Module on BOLT by the second week of classes. This grade will
not factor in to your final grade for the laboratory.

\subsection{Attire}
The official ``uniform'' of lab is a t-shirt, jeans/khakis, and tennis shoes.
\begin{itemize}[nosep]
	\item A well-fitting sweatshirt is allowed.
	\item Exposed shoulders, mid-riff, and legs are \emph{never} acceptable.
	\item Safety glasses (with side shields, not screens) or goggles are
		required.
	\item If face masks are required, the mask must cover both your mouth
		and nose.
\end{itemize}
Your lab instructor has the right to deny admittance for clothing deemed
``unsafe'' for lab.

\subsection{Denial of Admission to or Removal from Laboratory}
The lab instructor has the authority to deny admission to lab to any student who
\begin{enumerate*}[label={(\arabic*)}]
	\item does not have the proper materials,
	\item is not properly clothed and equipped with safety glasses and a
		face mask (if required), 
	\item has not properly prepared for lab as evidenced by incompletion
		of assignments or unmaintained laboratory notebook,
	\item is in a condition that would pose a safety hazard to themself or others in the lab, or
	\item any other circumstance as deemed by the instructor (e.g.\ chronic lateness).
\end{enumerate*}
The instructor also has the authority to ask students to leave the lab who are
acting in a manner deemed unsafe by the instructor. Arrangements to do a lab make-up are at the
discretion of the instructor and may not be granted. 

\subsection{Use of Cell Phones and Smart Watches in Lab}
Use of your cell phone or a smart watch is prohibited in the laboratory. They
must be stowed in your bag until you leave the laboratory. Commonly used bases
in the laboratory will damage the touchscreen on these devices, rendering them
unusable. Further, accidental contamination of your device with a harmful
compound can cause severe bodily harm the next time you use it. Should you
disregard this rule, you will be asked to leave the laboratory and will lose
credit for all assessments for the day. There are no exceptions unless provided
for by Disability Services.

\subsection{Attendance and Participation}
You must attend the laboratory in order to receive credit for an experiment. If
you know you will be absent, please let your instructor know \emph{as soon as
possible}. Make-ups will be granted on a case-by-case basis. Laboratories
require additional setup time by the instructor, so early notification is
absolutely required if possible.

\subsection{Lab Notebooks and Technique}
To prepare you for scientific investigations in the future, you will be
required to keep a laboratory notebook which will be graded for completeness,
legibility, and timeliness. The lab manual has
complete instructions for the care and feeding of a laboratory notebook.  One
primary aim of the course is to help you improve your competence with
analytical manipulations.  The instructor will review your technique throughout
the semester. Your grade will be based upon the competence, dexterity, and
speed of your manipulations as well as your apparent preparedness upon arrival.
You are expected to come to lab prepared to do the days experiment with the
procedure outlined in your notebook.

\begin{mdframed}
	Again, you are expected to come to lab prepared \emph{with the procedure
	outlined in your notebook}. The time you have to perform the laboratory
	\emph{does not include} time to organize your notebook during lab. You
	are ultimately responsible for using the approximately 4 hours of
	laboratory time effectively. If you cannot complete your analysis within
	the time allotted, you have not planned adequately and you will not be
	granted any additional time to complete the analysis.
\end{mdframed}

\section{Academic Dishonesty}
\emph{Academic dishonesty is not tolerated.} If you are unclear about what is
dishonest, please see 
\href{https://www.bloomu.edu/prp-3512-academic-integrity-policy}{PRP 3512
Academic Integrity Policy} for clarification. Please note that your lab reports
must be your own work even if you are working with a partner. You must prepare
your own graphs, tables, and other figures.
Turnitin\texttrademark{} will be used to check submissions for similarity. Note
that there is not a specific threshold that you must remain under --- I will check
all similarities and will assess plagiarism based on my own discretion. Quoting
entire passages from literature sources is not appropriate in scientific
disciplines. You must paraphrase and use your own wording to demonstrate your
understanding of the material. If you are unsure
about my specific instructions, ask me.

\begin{mdframed}
	\centering\bfseries The minimum penalty for academic dishonesty is a
	course assignment of ``F'' for \emph{all} students involved.
\end{mdframed}

\section{Class Cancellation}
Please see BOLT for class updates if lab is canceled. In some instances, we may
skip the lab altogether, which means grade percentages will have to change. In
other cases, your instructor may provide data or an alternative assignment to
practice the skills that would otherwise be missed. If compression (see
\href{https://www.bloomu.edu/documents/prp5205}{PRP 5205 University Closing
Policy}) occurs, the lab will still be held as scheduled, with the number of
required trials shortened and grading adjusted appropriately to compensate.

\section{Fire Alarms}
In the event of a building evacuation, calmly and quickly leave the building via
the nearest exit. Ensure all
hot plates or other possible hazards are turned off and unplugged prior to
leaving. Your instructor will point exits out the first week of class.  Gather
with your class on the quad lawn in front of Bakeless. In the case of inclement
weather, we will meet in the Kehr Union Ballroom. Do not re-enter the
building for any reason. Do not gather around the exits. Do not enter a
building that has an alarm sounding. There is one, and only one, drill the first
week of the semester. There are \emph{no} random drills.

\section{Accommodations}
Please see the lecture syllabus regarding documented accommodations with 
\href{https://bloomu.prod.acquia-sites.com/offices-directory/disability-services}{University
Disability Services}. The laboratory requires specific practical skills, so your
instructor will make the best effort to accommodate individuals when possible.

\vfill

\begin{mdframed}
	\noindent
	The materials contained in this syllabus, in the lab manual, and the
	BOLT webpage for this course are intended only for those registered for
	the above course/semester. These materials cannot be used without the
	expressed written consent of Dr.\ McCurry.

	\bigskip

	\noindent
	The source code for some material has been licensed under the CC
	BY-NC-SA 4.0 license to provide you with an opportunity to view the
	original source materials and contribute to the content. If you would
	like to view the material and suggest improvements, please visit
	\url{https://github.com/danian95/CHEM321}.
\end{mdframed}

\end{document}
