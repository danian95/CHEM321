% !TeX program = lualatex
\documentclass[11pt,letterpaper]{article}

\usepackage{fontspec}
\usepackage{tabularx}
\usepackage{booktabs}
\usepackage{bucolors}
\usepackage[colorlinks=true,allcolors=black,urlcolor=bugold]{hyperref}
\usepackage{mathtools}
\usepackage{mdframed}
\usepackage{titling}
\usepackage{fancyhdr}
\usepackage{lastpage}
\usepackage[english]{babel}
\usepackage[sf,bf]{titlesec}
\usepackage[inline]{enumitem}
\usepackage[margin=1in,letterpaper]{geometry}
\usepackage{multicol}
\usepackage{soul}
\usepackage{nameref}

\setmainfont{TeX Gyre Termes}[Ligatures=TeX]
\setsansfont{TeX Gyre Heros}[Ligatures=TeX]
\setmonofont{TeX Gyre Cursor}

\title{Quantitative Analysis}
\author{Dr.\ Daniel A.\ McCurry}
\date{Fall 2023}

\newcommand{\classnum}{CHEM341}

\pagestyle{fancy}
\lhead{}
\chead{}
\rhead{}
\lfoot{\footnotesize\sffamily McCurry --- \classnum\ --- FA23}
\cfoot{}
\rfoot{\footnotesize\sffamily\thepage~of~\pageref{LastPage}}
\renewcommand{\headrulewidth}{0pt}
\renewcommand{\footrulewidth}{0.4pt}

\pretitle{\noindent\color{bumaroon}
	\sffamily\bfseries
	\LARGE\classnum:\enspace\expandafter\MakeUppercase\expandafter}
\posttitle{\par\medskip}
\preauthor{\noindent\sffamily}
\postauthor{ --- }
\predate{\sffamily}
\postdate{}

\setlength{\droptitle}{-4em}

\setcounter{secnumdepth}{0}
\setlist[description]{font=\sffamily\bfseries\small}
\urlstyle{same}

\begin{document}

\maketitle
\thispagestyle{fancy}

\noindent
\begin{tabularx}{\linewidth} {@{\qquad}>{\bfseries\sffamily}r
	*{2}{>{\raggedright\arraybackslash}X}}
	\toprule
	Day and Time: & Lecture (Tues./Thur.) & Lab (Wed.) \\
		      & 12:30 pm -- 1:45 pm & 2:00 pm -- 4:50 pm \\ 
		 & HSC G42 & HSC 209 \\ \\
	Instructor: & Dr.\ Daniel A. McCurry &
	\href{mailto:dmccurry@commonwealthu.edu}{\nolinkurl{dmccurry@commonwealthu.edu}}
	\\
		    & 	Associate Professor of Chemistry & (570) 389-5320 \\
		    & 	HSC 240 &
		    \href{https://bloomu.starfishsolutions.com/starfish-ops/dl/instructor/serviceCatalog.html?bookmark=connection/20001}{CU
			    Succeed
		     	Profile} \\ \\
	Office Hours: & 
		\begin{tabular}[t] {@{}lr@{\,--\,}l}
			Mon./Tue.  & 3:00 & 5:00\,p.m. \\
			Thur. & 9:00 & 10:00\,a.m. \\ 
		\end{tabular}
		      & Need another time? \newline
\href{https://outlook.office.com/bookwithme/user/38fcfec5771549768d6c1a6f66912778@commonwealthu.edu/meetingtype/scMRb4P_HUKzXVNHurMbRA2?anonymous}{Schedule
			an Appointment}
			 \\ 
		      & \multicolumn{2}{l}{Office hours may be held in either HSC 240 or HSC 209
		      (just walk in!)} \\ \\
%		      &   In-Person or via Zoom Meeting ID
%                          ``\href{https://bloomu.zoom.us/my/dmccurry}{dmccurry}''\\
%                      &    \href{https://bloomu.starfishsolutions.com/starfish-ops/dl/instructor/serviceCatalog.html?bookmark=connection/20001/schedule}{Schedule
%		      an Appointment} (not required) \\ \\
		Textbook: %& Harvey, D. \textit{Analytical Chemistry 2.1} (\href{https://chem.libretexts.org/@go/page/122341}{Open Access}) \\
			   & \multicolumn{2}{l}{Harris, D.C. \textit{Quantitative Chemical Analysis},
	10\textsuperscript{th} Ed.}  \\ & \multicolumn{2}{l}{(older editions OK, Inclusive
Access opt out by \textbf{8/28})}
	      \\ \\
	Materials: & \multicolumn{2}{l}{Scientific or graphing calculator
					(simpler calculators are not appropriate).}\\
		   & \multicolumn{2}{l}{Laptop for spreadsheet manipulation (certain topics
		only)} \\
		 & \multicolumn{2}{l}{Carbon- or carbonless-copy laboratory notebook} \\
		 & \multicolumn{2}{l}{Safety goggles} \\
		 & \multicolumn{2}{l}{Black or blue ink \emph{ballpoint} pen} \\
		 & \multicolumn{2}{l}{Permanent marker} \\
	\bottomrule
\end{tabularx}

\section{Course Description}
This course studies the methods of analysis useful in today's science
laboratories. In fact, you have used numerous analytical methods already in your
previous chemistry courses. This course stresses the theory and laboratory
practices of these analytical methods. In other words, is the difference between
two results \emph{significant}? What caused the results between two
\emph{identical} samples to differ? How can we report our results to a specific
level of \emph{confidence}? Calculations, basic analytical theory, error
analysis, and method advantages and limitations are emphasized throughout the
lecture and laboratory.

\subsection{Learning Objectives}
By the end of the semester, you should be able to\ldots\ (in no particular
order)
\begin{itemize}[noitemsep]
	\item Propose potential solution interferents and identify sample
		considerations in order to judge appropriate routes of chemical
		analysis.
	\item Apply statistical tests to data to determine significant
		differences or identify outliers that may skew results.
	\item Explain why proper sample handling and collection is imperative to
		delivering meaningful results.
	\item Summarize and organize equilibrium phenomena in solution to assess
		the validity of analytical techniques.
	\item Compare and contrast different analytical techniques
		(spectrophotometry, chromatography, potentiometry, titrations)
		and assess their suitability for particular analyses.
	\item Report experimental results to levels appropriate for the analysis
		used and consistent with the use objective.
	\item Verify experimental procedures with appropriate methods of
		standardization and quality control (matrix spike, external
		standard, internal standard, standard additions).
	\item Choose an appropriate analytical technique for a specific analysis
		and justify the choice using sound chemical knowledge.
\end{itemize}

\subsection{Course Community and Communication}
\emph{Your active involvement in the course is important for your success!} As
such, you are expected to read the relevant textbook sections prior to lecture.
Please use the Discussion Boards on Brightspace for questions so other students
can contribute to or read answers. All course announcements will be delivered on
the Brightspace course page. If you would like push
notifications on your phone when announcements are posted, please download the
\href{https://documentation.brightspace.com/EN/brightspace/requirements/all/pulse.htm}{Brightspace
Pulse} app.

%\subsection{Textbook}
%You are expected to read the textbook along with lecture. This course is using an open access
%textbook, \textit{Analytical Chemistry 2.1} by David Harvey. Open access means
%that it is freely available online. I opted to use this textbook because it
%\begin{enumerate*}[label={(\arabic*)}]
%	\item offers the same information as other textbooks,
%	\item has received positive feedback from other analytical chemistry
%		instructors,
%	\item is more affordable (free) than traditional textbooks, and
%	\item is in line with my teaching philosophy that education should be
%		attainable that all who want it.
%\end{enumerate*}
%You may find the textbook on Brightspace or online at
%\url{https://chem.libretexts.org/Bookshelves/Analytical_Chemistry/Analytical_Chemistry_2.1_(Harvey)}.
%
%Many other institutions use Daniel C. Harris's \textit{Quantitative Chemical
%Analysis}. Based on feedback from students, I have decided to \emph{optionally
%recommend} this textbook for those seeking more information. 
%This \emph{optional} textbook was selected as ``Inclusive Access,'' which
%significantly lowers the cost of accessing the textbook \emph{for the semester}
%(you do not own it and you will lose access after the semester). Unfortunately,
%Inclusive Access is explicitly ``Opt Out.'' As such, the cost of this
%\emph{optional} textbook will be automatically charged to your BU student
%account under “misc.\ charges”.  If you decide that you do not want to use this
%\emph{optional} textbook, you may Opt Out by \textbf{August 29}.  If you ``Opt
%Out'', you will not be charged for the textbook and will no longer have access.
%If you have any questions, please contact
%\href{mailto:inclusiveaccess@bloomu.edu}{inclusiveaccess@bloomu.edu}.
%
%If you plan on attending graduate school in chemistry, I suggest opting out of
%Inclusive Access and purchasing the
%hard copy of \textit{Quantitative Chemical Analysis}. Older editions are most
%likely cheaper and will contain most of the same information.

\section{Evaluations and Grading}
This course will be operating on a 100\,\% scale according to
the tables below. Grading evaluates how well you understand the course material
and your performance in the laboratory.  Accumulation of ``correct'' points on a
quiz or exam is based on the \emph{perception} of your understanding. This
perception is influenced by your logic, your terminology, your mathematical
accuracy and your legibility. The instructor must be convinced that you
understand correctly in order to award you full credit.

\begin{multicols}{2}
	\subsection{Point Distribution}
	\begin{tabular} {l r<{\,\%}}
		Exams               & 35 \\
		Homework and Quizzes      & 15 \\
		Final Exam               & 25 \\
		Laboratory Reports & 25 \\ \midrule
		& 100
	\end{tabular}

	\subsection{Grade Assignment}%
	\begin{tabular} {r@{\,--\,}l<{\,\%} l@{\hspace{0.5in}}r@{\,--\,}l<{\,\%} l}
		\multicolumn{3}{c}{} & 75 & 77 & C+ \\
		91 & 100 & A  & 71 & 74 & C  \\
		88 & 90 & A- & 68 & 70 & C- \\
		\multicolumn{6}{c}{} \\
		85 & 87 & B+ & 65 & 67 & D+ \\
		81 & 84 & B  & 60 & 64 & D  \\
		78 & 80 & B- & 0  & 59 & F
	\end{tabular}
\end{multicols}

\subsection{Graded Items}
\begin{description}
	\item[Exams:] There will be three (3) 70 minute exams held during our
		regularly scheduled lecture time. Some exams may include
		take-home components due to time constraints. Tardiness will not extend your
		exam time. If accommodations must be made, it is your
		responsibility to alert the instructor at least 72 hours
		\emph{before} the exam.
	\item[Quizzes:] Quizzes will be offered randomly throughout the semester
		in lecture or lab. Your instructor reserves the right to declare
		quizzes open- or closed-book and/or open- or closed-friend.
	\item[Homework:] There will be regular homework assignments posted to
		LON-CAPA (see Brightspace for details). You will have sufficient
		opportunities to resubmit answers without penalty before the
		deadline. Make sure to start early so you can come to office
		hours with questions. Essentially, the only reason why you
		should lose points on homework is because you did not turn it in
		on time!
	\item[Final Exam:] The final exam will consist of a \emph{comprehensive}
		115 minute final exam during the final exam week. This will be
		of a similar format to all of the lecture exams held throughout
		the semester.
	\item[Laboratory:] Laboratory attendance is mandatory. Reports will be
		graded for format, grammar, spelling, chemical logic, and
		accuracy of the analysis. Reports will be due at the
		\emph{beginning} of laboratory in a format as listed in the
		laboratory schedule. Only electronic copies (PDF) will be accepted
		through the appropriate Brightspace assignment submission.
		Spreadsheets are to be submitted in their raw form, as a *.xlsx,
		*.xls, or *.ods. \emph{Be careful that you are, in fact, saving
		as one of these listed formats and not as *.csv!} A late lab
		report will be graded with a 20\% penalty for every 24 hours
		after the deadline.
\end{description}

\subsection{Grading Errors}
Did I make a mistake? Let me know within 72 hours from the return of the graded
item so that I may re-grade it. \emph{The entire item will be re-graded}. This
timeline is necessary so I don't get a pile of re-grades during the last week of
class. You have until 4:30\,p.m.\ on the day of the final exam to bring clerical
errors (e.g., I typed a grade incorrectly into Brightspace) to my attention.

\section{University Policies}
As a student of Commonwealth University, you expected to know and abide by the
policies and rules set forth in the
\href{https://www.commonwealthu.edu/student-handbook/code-of-conduct}{Student
Code of Conduct} and the
\href{https://www.bloomu.edu/about/administration-and-governance/policies}{Bloomsburg
	University Policies, Procedures, and Guidelines}. Additional
	clarifications specific to this course are provided below.

\subsection{Attendance}
Physical presence is only part of going to class. Participation in class is
expected. You will be called upon at random and are advised to be prepared.
The best learning occurs in an active environment where everyone participates.
In order to be prepared, you must read the textbook prior to coming to
lecture.  You do not need to understand everything on the first perusal. Many
of your questions will probably be answered by the lecture or as you work the
problems.

If you are unable to complete an examination, quiz, or laboratory experiment at its scheduled time due to some
unavoidable circumstance, you are obligated to provide a documented valid excuse
to your instructor. \emph{Failure to provide a valid excuse on these terms will
	result in a score of zero for the graded item.}

	\begin{center}
		\renewcommand\arraystretch{1.25}
\begin{tabularx}{\linewidth} {X X}
	\toprule
	\bfseries Valid Excuse & \bfseries Time Frame to Provide Documentation
	\\ \midrule
	Personal illness or accident, illness or accident of a dependent child,
	or death or critical illness of an immediate family member &
	48\,hr after the missed graded event \\
	Participation in a university-sponsored activity &
	48\,hr in advance of the missed graded event \\
	Military duty & As soon as possible \\
	Religious observance & The second Friday of the semester \\
	Others, as merited on a case-by-case basis & 48\,hr after the
	missed graded event \\
	\bottomrule
\end{tabularx}
\end{center}

In instances where you are asked to leave the laboratory (e.g., you are
disruptive, acting in a dangerous manner contrary to the terms in the Safety
Regulations, etc.), you will not be afforded an opportunity to make up the
experiment and will receive a ``0'' on the report.

Immediate technical issues, such as Brightspace being inaccessible, can be submitted to
\url{https://helpdesk.commonwealthu.edu}. Please forward your correspondance
with IT to my email address to alert me that you are taking steps to resolve the issue.

\subsection{Academic Dishonesty}
\emph{Academic dishonesty is not tolerated.} 
All work must be your own. You are permitted to work with others on homework. Do
note that homework answers are randomized and you will most likely not have the
same answers as your classmates. Lab reports must be written individually. You
will work with partners and/or groups, but you must create your own
spreadsheets, graphs, and written reports.

Artificial intelligence (AI) is not to be used for any work in this course. Work
generated by AI and turned in for a grade will result in a failing grade for the
\emph{course}.

\begin{mdframed}
	\centering\bfseries The minimum penalty for academic dishonesty is a
	course assignment of ``F'' for \emph{all} students involved.
\end{mdframed}

\subsection{Technology in the Classroom}
Some assignments and units may necessitate a laptop computer. In all other
instances, if you are not using a tablet or laptop to take notes, you are
expected to pay attention and keep phones, tablets, and laptops away. Please be
respectful of others' learning by not using distracting software or websites
during class.

\begin{mdframed}
	\centering\bfseries Cell phones and personal electronics are not to be
	used in the laboratory. We will be using corrosive chemicals that will
	permanently damage your electronics. Any cell phones visible in the
	laboratory will be collected at the instructor's desk.
\end{mdframed}

\subsection{Laboratory}
You will receive a copy of the Safety Rules and Regulations on the first day of
laboratory. You are expected to read and sign these regulations. Failure to
employ safe laboratory practices as outlined in these Rules and Regulations will
result in your expulsion from the laboratory.

You must receive an 80\% or better on the laboratory safety module prior to the
\emph{second} week of lab. You will not be permitted entry into the laboratory
if the module is not complete.

\subsection{Fire Alarms}
Given the quantity of explosive compounds in Hartline Science Center, all fire
alarms will be treated as an emergency. Your instructor will point out exits on
the first day of lecture and laboratory. In any case, we will exit the room in
an orderly fashion and meet outside on the lawn in front of Bakeless, so that I
may get a headcount. Missing individuals will be reported to the police.

\vfill

\begin{mdframed}
	\noindent
	The materials contained in this syllabus, in the lab manual, and the
	Brightspace webpage for this course are intended only for those registered for
	the above course/semester. These materials cannot be used without the
	expressed written consent of Dr.\ McCurry.

	\bigskip

	\noindent
	The source code for some material has been licensed under the CC
	BY-NC-SA 4.0 license to provide you with an opportunity to view the
	original source materials and contribute to the content. If you would
	like to view the material and suggest improvements, please visit
	\url{https://github.com/danian95/CHEM321}.
\end{mdframed}

\end{document}
