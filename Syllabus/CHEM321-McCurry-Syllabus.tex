% !TeX program = lualatex
\documentclass[11pt,letterpaper]{article}

\usepackage{tabularx}
\usepackage{booktabs}
\usepackage{bucolors}
\usepackage[colorlinks=true,allcolors=black,urlcolor=bugold]{hyperref}
\usepackage{mathtools}
\usepackage{mdframed}
\usepackage{titling}
\usepackage{newtxtext}
\usepackage{fancyhdr}
\usepackage{lastpage}
\usepackage[english]{babel}
\usepackage[sf,bf]{titlesec}
\usepackage{enumitem}
\usepackage[margin=1in,letterpaper]{geometry}
\usepackage{multicol}

\title{Analytical Chemistry 1}
\author{Dr.\ Daniel A.\ McCurry}
\date{Fall 2021}

\newcommand{\classnum}{CHEM321}

\pagestyle{fancy}
\lhead{}
\chead{}
\rhead{}
\lfoot{\footnotesize\sffamily McCurry --- \classnum\ --- FA21}
\cfoot{}
\rfoot{\footnotesize\sffamily\thepage~of~\pageref{LastPage}}
\renewcommand{\headrulewidth}{0pt}
\renewcommand{\footrulewidth}{0.4pt}

\pretitle{\noindent\color{bumaroon}
	\sffamily\bfseries\Large
	\classnum\newline
	\LARGE\expandafter\MakeUppercase\expandafter}
\posttitle{\par\medskip}
\preauthor{\noindent\sffamily}
\postauthor{ --- }
\predate{\sffamily}
\postdate{}

\setlength{\droptitle}{-2em}

\setcounter{secnumdepth}{0}
\setlist[description]{font=\sffamily\bfseries}
\urlstyle{same}

\begin{document}

\maketitle
\thispagestyle{fancy}

\noindent
\begin{tabularx}{\linewidth} {@{\qquad}>{\bfseries\sffamily}r
	>{\raggedright\arraybackslash}X@{\qquad}}
	\toprule
	Lecture: & TR, 12:30 pm -- 1:45 pm \\
			    & HSC G42 \\ \\
	Lab:     & W, 2:00 pm -- 5:50 pm \\ 
		 & HSC 209 \\ \\
        Instructor: & Dr.\ Daniel A. McCurry\\
		    & 	Assistant Professor of Chemistry\\
		    & 	HSC 240\\
		    & 	(570) 389-5320\\
		    & 	\href{mailto:dmccurry@bloomu.edu}{\nolinkurl{dmccurry@bloomu.edu}}\\
		    & 	\href{https://bloomu.starfishsolutions.com/starfish-ops/dl/instructor/serviceCatalog.html?bookmark=connection/20001}{HuskySuccess
		     	Profile} \\ \\
	Office Hours: & \begin{minipage}[t]{\linewidth}
		\begin{tabular}[t] {@{}lr@{\,--\,}l@{~@~}l}
			Mon./Tues.  & 4:00 & 6:00\,p.m. & Andruss Library
			(1\textsuperscript{st} floor) \\
			Fri. & 9:00 & 10:00\,a.m. & HSC 240\\
				\end{tabular}
			\end{minipage} \\
		      &   In-Person or via Zoom Meeting ID
                          ``\href{https://bloomu.zoom.us/my/dmccurry}{dmccurry}''\\
                      &    \href{https://bloomu.starfishsolutions.com/starfish-ops/dl/instructor/serviceCatalog.html?bookmark=connection/20001/schedule}{Schedule
		      an Appointment} (not required) \\ \\
		Text: & Harvey, D. \textit{Analytical Chemistry 2.1} \\
	      & \footnotesize Open-access available at no cost
	      \href{https://chem.libretexts.org/@go/page/122341}{online}
	      \\ \\
	Lab Text: & McCurry, D.A.; Hallen, C.P. \textit{Back to the CHEM321
	Analytical Chemistry Laboratory}, Fall 2021 Ed. \\
	       	      & \footnotesize Available for purchase during the first
		      week of laboratory. \\ \\
	Materials: & Scientific or graphing calculator (simpler calculators
	are not appropriate). \\
		   & USB flash drive for transferring laboratory data \\
		   & Carbon- or carbonless-copy laboratory notebook \\
		   & Safety goggles \\
	\bottomrule
\end{tabularx}

\section{Course Description}
Analytical Chemistry studies the methods of analysis useful in today's science
laboratories. In fact, you have used numerous analytical methods already in your
previous chemistry courses. The course stresses the theory and laboratory
practices of these analytical methods. In other words, is the difference between
two results \emph{significant}? What caused the results between two
\emph{identical} samples to differ? How can we report our results to a specific
level of \emph{confidence}? Calculations, basic analytical theory, error
analysis, and method advantages and limitations are emphasized throughout the
lecture and laboratory.

\subsection{Learning Objectives}
With all of the techniques and methods for statistical analysis we'll cover this
semester, it can be easy to lose sight of the goals for the course. By the end
of the semester, you should be able to\ldots
\begin{itemize}[noitemsep]
	\item Identify results that are significantly different to a specific
		confidence interval.
	\item Identify appropriate measurement techniques to suit a particular
		analysis.
	\item Consider all aspects of chemical equilibrium that may affect your
		measurements.
	\item Exhibit healthy skepticism when confronted with data.
\end{itemize}

\subsection{Laboratory Overview}

The laboratory for analytical chemistry studies the theory and practice of
several common methods used to determine how much analyte is in a particular
sample of material. In several of the experiments that you will do this
semester, some component of a commercial product will be analyzed. In other
experiments, specially purchased or prepared unknowns will be used. For the
latter experiments, 20--50\,\% of your laboratory report grade will
depend simply on the accuracy and precision of your work. Why are these analyses
so rigorous (or why are analytical chemists so rigorous)? When you leave this
educational institution, it is possible that thousands of dollars or someone's
life will depend upon your analysis. Before you are certified capable of such
analyses by a grade and/or a letter of recommendation, you must prove yourself
worthy of that grade/recommendation.

In doing these experiments, you will touch upon some classical wet chemistry
methods of analysis as well as some modern instrumental techniques. Many
experiments also use statistical analysis to illustrate how good (or bad)
the results actually are. Some of these labs are designed to be long to see how
well you plan ahead for an analysis. Some are designed to be worked in groups of
two or three to see how well you can work with others. Still others are vague to
allow you the freedom to use your ingenuity and common sense to solve the
problem.

You are expected to be prepared to do an experiment when you arrive in the
laboratory. The instructor reserves the right to quiz at any time. These
quizzes will count as part of the lab grade.

\subsection{Course Community and Communication}
\emph{Your active involvement in the course is important for your success!}
Throughout the semester, I will encourage active participation both during and
outside of lecture. In order to facilitate some community-building, we will be
using the Discussion Boards on BOLT extensively. If you have a question with an
answer that could benefit the class as a whole, please post your question to the
Discussion Board, rather than sending an individual email to directly to me. I
would also be happy to help organize study groups if you are interested. Please
let me know if you would like to chat with other classmates and I will put you
in touch. Learning occurs best when it is driven by your own motivation and
curiosity, not by passively absorbing information I share in a brief 75-minute
period!

All course announcements will be delivered on the BOLT course page. It is your
responsibility to regularly check BOLT. I will \emph{not} send out email
announcements, so BOLT is the \emph{only} location you will need to check for
all important course information --- in my experience, a single location for all
information significantly decreases any confusion. If you would like push
notifications on your phone when announcements are posted, please download the
\href{https://documentation.brightspace.com/EN/brightspace/requirements/all/pulse.htm}{Brightspace
Pulse} app.

\section{Evaluations and Grading}
Unfortunately, we do need to include a metric for assessing your understanding
of the material. This course will be operating on a 100\,\% scale according to
the tables below. Grading evaluates how well you understand the course material
and your performance in the laboratory.  Accumulation of ``correct'' points on a
quiz or exam is based on the \emph{perception} of your understanding. This
perception is influenced by your logic, your terminology, your mathematical
accuracy and your legibility. The instructor must be convinced that you
understand correctly in order to award you full credit.

\begin{multicols}{2}
	\subsection{Point Distribution}
	\begin{tabular} {>{\sffamily\bfseries}l l r<{\,\%}}
		Lecture:    & Exams               & 35 \\
			    & Homework \& Quizzes     & 15 \\
			    & Final Exam               & 25 \\
		Laboratory: & Reports & 25 \\ \midrule
			    & & 100
	\end{tabular}

	\subsection{Grade Assignment}%
	\begin{tabular} {r@{\,--\,}l<{\,\%} l@{\hspace{0.5in}}r@{\,--\,}l<{\,\%} l}
		\multicolumn{3}{c}{} & 75 & 77 & C+ \\
		91 & 97 & A  & 71 & 74 & C  \\
		88 & 90 & A- & 68 & 70 & C- \\
		\multicolumn{6}{c}{} \\
		85 & 87 & B+ & 65 & 67 & D+ \\
		81 & 84 & B  & 60 & 64 & D  \\
		78 & 80 & B- & 0  & 59 & F
	\end{tabular}
\end{multicols}

\subsection{Graded Items}
\begin{description}
	\item[Exams:] There will be three (3) 70 minute exams held during our
		regularly scheduled lecture time. Tardiness will not extend your
		exam time. If accommodations must be made, it is your
		responsibility to alert the instructor at least 72 hours
		\emph{before} the exam.
	\item[Quizzes:] Quizzes will be held randomly throughout the semester
		in lecture and lab; the amount depends on class participation.
		In other words, if it seems like everyone is following along, I
		won't need many quizzes to gauge progress!
	\item[Homework:] There will be a few homework assignments announced in
		lecture to help guide your practice of the material. Generally,
		these will be reserved for in-depth problems that I anticipate
		being a bit challenging on an exam.
	\item[Final Exam:] The final exam will consist of two parts:
		\begin{enumerate}
			\item A standardized American Chemical Society
				Analytical Exam held during the last week of
				laboratory. This is to compare our class with
				the national standards as set by the ACS. These
				are often a bit difficult (the national average
				for these tends to hover around 50\,\%), so your
				score will be adjusted to accommodate the topics
				we covered in class. This section of the final
				is worth 20\,\% of your final exam grade (5\,\%
				of your final grade).
			\item A \emph{comprehensive} 115 minute final exam
				during the final exam week. This will be of a
				similar format to all of the lecture exams held
				throughout the semester and will constitute
				80\,\% of your final exam grade (20\,\% of your
				final grade).
		\end{enumerate}
	\item[Laboratory:] Laboratory attendance is mandatory. Reports will be
		graded for format (see laboratory manual), grammar, spelling,
		chemical logic, and accuracy of the analysis. Reports will be
		due at the \emph{beginning} of laboratory in a format as listed
		in the laboratory schedule.  It is expected that all data
		reduction and plotting be performed with modern spreadsheet
		software and presented in a logical table format.  Reports will
		be in three styles:
		\begin{center}
			\begin{tabularx}{\linewidth} {r >{\raggedright\arraybackslash}X}
			Formal or Long Form Reports: & all sections as listed in the lab manual \\ Informal or Short Form Reports: & Abstract,
			Results, Discussion, References, and
		Appendix (including answers to appropriate questions) \\
			Spreadsheet Reports: & results as a spreadsheet
		\end{tabularx}
		\end{center}
		Only electronic copies will be accepted through the appropriate
		BOLT assignment submission. Spreadsheets are to be submitted in
		their raw form, as a *.xlsx, *.xls, or *.ods. \emph{Be careful
			that you are, in fact, saving as one of these listed
		formats and not as *.csv!} It is \emph{strongly} recommended
		that you submit your long and short reports in PDF format
		(*.pdf) as this will ensure all of your formatting is kept.  A
		late lab report will be graded with a 20\,\% penalty for every
		24 hours after the deadline.
\end{description}

\subsection{Grading Errors}
Did I make a mistake? Let me know within 72 hours from the return of the graded
item so that I may re-grade it. \emph{The entire item will be re-graded}. This
timeline is necessary so I don't get a pile of re-grades during the last week of
class. You have until 4:30\,p.m.\ on the day of the final exam to bring clerical
errors (e.g., I typed a grade incorrectly into BOLT) to my attention.

\section{Important Dates}\label{importantdates}
\begin{center}
	\begin{tabular} {l l l}
		September 14 & Tuesday    & Exam 1 \\
		October 19   & Tuesday    & Exam 2 \\
		November 16  & Tuesday    & Exam 3 \\
		November 23  & Tuesday   & Thanksgiving Recess (No Class) \\
		November 24  & Wednesday & Thanksgiving Recess (No Class) \\
		November 25  & Thursday    & Thanksgiving Recess (No Class) \\
		December 1   & Wednesday & ACS Analytical Exam (in lab) \\
		December 2   & Thursday    & Last Day of Class \\
		December 8   & Tuesday    & Final Examination (10:15\,am --
		12:15\,pm)
	\end{tabular}
\end{center}

Please note, if you are unable to take an examination or quiz at
its scheduled time due to some unavoidable circumstance, you are obligated to
provide a documented valid excuse to your instructor. \emph{Failure to provide a
valid excuse on these terms will result in a score of zero for the quiz or
exam.}

	\begin{center}
		\renewcommand\arraystretch{1.25}
\begin{tabularx}{\linewidth} {X X}
	\toprule
	\bfseries Valid Excuse & \bfseries Time Frame to Provide Documentation
	\\ \midrule
	Personal illness or accident, illness or accident of a dependent child,
	or death or critical illness of an immediate family member &
	48\,hr after the missed graded event \\
	Participation in a university-sponsored activity &
	48\,hr in advance of the missed graded event \\
	Military duty & As soon as possible \\
	Religious observance & The second Friday of the semester \\
	Others, as merited on a case-by-case basis & 48\,hr after the
	missed graded event \\
	\bottomrule
\end{tabularx}
\end{center}

Immediate technical issues, such as BOLT being inaccessible, can be submitted to
\url{https://helpdesk.bloomu.edu}. At the bottom of the form, you must include
my email address (dmccurry@bloomu.edu) in the ``Additional Email Notifications''
to alert me that you are taking steps to resolve the issue.

\section{Policies and Expectations}
In order to maintain an atmosphere that is conducive to learning, all members
of the class are expected to demonstrate common courtesy during the lecture,
including but not exclusively:
\begin{enumerate}
	\item \textbf{Wear a mask that covers your nose and mouth.} Not all
		students are vaccinated and other, more infectious variants pose
		a health risk to all individuals. In order to ensure the
		wellbeing of your fellow classmates, please wear a well-fitting
		mask while inside.
	\item \textbf{Be on time for class.} If you are unavoidably late, enter
		the room quietly and quickly choose a seat.  Tardiness will not
		extend your exam or quiz time. We will not return to missed
		quizzes.
	\item \textbf{Once in class, stay for the duration.} If you must leave
		early, give me advance warning. If you become ill during an exam
		or quiz, the missed graded event policy will take effect with
		the consent of the professor. In either case, you will not be
		allowed to return that day. Take care of your personal needs
		\emph{before class}.
	\item \textbf{Silence your cell phone and/or other electronic devices.}
		Students with special circumstances need to speak with me right
		away.
	\item \textbf{Refrain from talking to your neighbors excessively.} Stay
		focused on the lecture materials. If you missed something, feel
		free to ask me to repeat it!
\end{enumerate}
\emph{Please review
\href{https://www.bloomu.edu/prp-3881-student-disruptive-behavior-policy}{PRP 3881 Student
Disruptive Behavior Policy} for further details regarding the consequences for
disruptive behavior.}

\subsection{Attendance and Participation}
Physical presence is only part of going to class. Participation in class is
expected. You will be called upon at random and are advised to be prepared.
The best learning occurs in an active environment where everyone participates.
In order to be prepared, you should read the textbook prior to coming to
lecture.  You do not need to understand everything on the first perusal. Many
of your questions will probably be answered by the lecture or as you work the
problems.

\subsection{Lab Notebooks and Technique}
To prepare you for scientific investigations in the future, you will be
required to keep a laboratory notebook which will be graded for completeness,
legibility, and timeliness. The lab manual has
complete instructions for the care and feeding of a laboratory notebook.  One
primary aim of the course is to help you improve your competence with
analytical manipulations.  The instructor will review your technique throughout
the semester. Your grade will be based upon the competence, dexterity, and
speed of your manipulations as well as your apparent preparedness upon arrival.
You are expected to come to lab prepared to do the days experiment with the
procedure outlined in your notebook.

\section{Academic Dishonesty}
\emph{Academic dishonesty is not tolerated.} If you are unclear about what is
dishonest, please see 
\href{https://www.bloomu.edu/prp-3512-academic-integrity-policy}{PRP 3512
Academic Integrity Policy} for clarification. Please note that your lab reports
must be your own work even if you are working with a partner. If you are unsure
about my specific instructions, ask me.

\begin{mdframed}
	\centering\bfseries The minimum penalty for academic dishonesty is a
	course assignment of ``F'' for \emph{all} students involved.
\end{mdframed}

\section{Class Cancellation}
If class is canceled on the day of an exam, it will be administered the next
time class meets.  If class is canceled the period before an exam, it will
still be administered on its scheduled day. If compression
(see \href{https://www.bloomu.edu/documents/prp5205}{PRP 5205 University Closing
Policy})
occurs on an exam day, the exam will be held as scheduled (with
grading adjusted to allow for the shortened time period).

\section{Fire Alarms}
In the event of a building evacuation, calmly and quickly leave the building via
the nearest exit. In the event of an evacuation during laboratory, ensure all
hot plates or other possible hazards are turned off and unplugged prior to
leaving. Your instructor will point exits out the first week of class.  Gather
with your class on the quad lawn in front of Bakeless. In the case of inclement
weather, we will meet in the Kehr Union Ballroom. Do not re-enter the
building for any reason. Do not gather around the exits. Do not enter a
building that has an alarm sounding. There is one, and only one, drill the first
week of the semester. There are \emph{no} random drills.


\section{Last Day to Withdraw}
The Registrar has set the last day of class as the withdrawal deadline.
Students are encouraged to review the latest withdrawal
(\href{https://www.bloomu.edu/prp-3462-undergraduate-course-withdrawal}{PRP 3462 Undergraduate
Course Withdrawal}) and course repeat
(\href{https://www.bloomu.edu/prp-3452-course-repeat}{PRP 3452 Course
Repeat}) policies of the
University prior to that date as there are strict limits on the number of
repeats one can have. 

\begin{mdframed}
	\centering
	All policies, rules, and procedures (PRP) are available at
	\url{https://www.bloomu.edu/about/administration-and-governance/policies}
\end{mdframed}

\section{Accommodations}
Bloomsburg University provides reasonable accommodations to students who have
documented disabilities. If you have a documented disability that requires
academic accommodations and are not registered with 
\href{https://bloomu.prod.acquia-sites.com/offices-directory/disability-services}{University
Disability Services}
please contact this office as soon as possible to establish your
eligibility. Please also provide documentation to Dr.\ McCurry as promptly as
possible.


\vfill

\begin{mdframed}
	\noindent
	The materials contained in this syllabus, in the lab manual, and the
	BOLT webpage for this course are intended only for those registered for
	the above course/semester. These materials cannot be used without the
	expressed written consent of Dr.\ McCurry.

	\noindent
	The source code for some material has been licensed under the Creative
	Commons Attribution-NonCommercial-ShareAlike 4.0 International (CC
	BY-NC-SA 4.0) license to provide you with an opportunity to view the
	original source materials and contribute to the content. If you would
	like to view the material and suggest improvements, please visit
	\url{https://git.damccurry.com/dan/CHEM321}. Ask Dr.\ McCurry for an
	account!
\end{mdframed}

\end{document}
