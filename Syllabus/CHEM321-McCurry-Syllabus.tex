% !TeX program = lualatex
\documentclass[11pt,letterpaper]{article}

\usepackage{fontspec}
\usepackage{tabularx}
\usepackage{booktabs}
\usepackage{bucolors}
\usepackage[colorlinks=true,allcolors=black,urlcolor=bugold]{hyperref}
\usepackage{mathtools}
\usepackage{mdframed}
\usepackage{titling}
\usepackage{fancyhdr}
\usepackage{lastpage}
\usepackage[english]{babel}
\usepackage[sf,bf]{titlesec}
\usepackage[inline]{enumitem}
\usepackage[margin=1in,letterpaper]{geometry}
\usepackage{multicol}
\usepackage{soul}
\usepackage{nameref}

\setmainfont{TeX Gyre Termes}[Ligatures=TeX]
\setsansfont{TeX Gyre Heros}[Ligatures=TeX]
\setmonofont{TeX Gyre Cursor}

\title{Analytical Chemistry 1}
\author{Dr.\ Daniel A.\ McCurry}
\date{Fall 2022}

\newcommand{\classnum}{CHEM321}

\pagestyle{fancy}
\lhead{}
\chead{}
\rhead{}
\lfoot{\footnotesize\sffamily McCurry --- \classnum\ --- FA22}
\cfoot{}
\rfoot{\footnotesize\sffamily\thepage~of~\pageref{LastPage}}
\renewcommand{\headrulewidth}{0pt}
\renewcommand{\footrulewidth}{0.4pt}

\pretitle{\noindent\color{bumaroon}
	\sffamily\bfseries\Large
	\classnum~Lecture\newline
	\LARGE\expandafter\MakeUppercase\expandafter}
\posttitle{\par\medskip}
\preauthor{\noindent\sffamily}
\postauthor{ --- }
\predate{\sffamily}
\postdate{}

\setlength{\droptitle}{-2em}

\setcounter{secnumdepth}{0}
\setlist[description]{font=\sffamily\bfseries\small}
\urlstyle{same}

\begin{document}

\maketitle
\thispagestyle{fancy}

\noindent
\begin{tabularx}{\linewidth} {@{\qquad}>{\bfseries\sffamily}r
	>{\raggedright\arraybackslash}X}
	\toprule
	Time: & Mon./Wed./Fri., 9:00 am -- 9:50 am \\ \\
	Location: & HSC G42 (BU) \\
		  & Grant 140 (MU) \\ 
		  & Zoom Meeting ID provided with instructor approval \\ \\
		  %& Zoom Meeting ID \href{https://bloomu.zoom.us/j/91337754874}{913 3775 4874} (MU
		% only) \\ \\
        Instructor: & Dr.\ Daniel A. McCurry\\
		    & 	Assistant Professor of Chemistry\\
		    & 	HSC 240\\
		    & 	(570) 389-5320\\
		    & 	\href{mailto:dmccurry@bloomu.edu}{\nolinkurl{dmccurry@bloomu.edu}}\\
		    & 	\href{https://bloomu.starfishsolutions.com/starfish-ops/dl/instructor/serviceCatalog.html?bookmark=connection/20001}{HuskySuccess
		     	Profile} \\ \\
	Office Hours: & \begin{minipage}[t]{\linewidth}
		\begin{tabular}[t] {@{}lr@{\,--\,}l}
			Mon./Tue.  & 3:00 & 5:00\,p.m. \\
			Thur. & 4:00 & 5:00\,p.m. \\
				\end{tabular}
			\end{minipage} \\
		      &   In-Person or via Zoom Meeting ID
                          ``\href{https://bloomu.zoom.us/my/dmccurry}{dmccurry}''\\
                      &    \href{https://bloomu.starfishsolutions.com/starfish-ops/dl/instructor/serviceCatalog.html?bookmark=connection/20001/schedule}{Schedule
		      an Appointment} (not required) \\ \\
		Text: & Harvey, D. \textit{Analytical Chemistry 2.1}
		(\href{https://chem.libretexts.org/@go/page/122341}{Open
		Access}) \\
		      & Harris, D.C. \textit{Quantitative Chemical Analysis},
		      10\textsuperscript{th} Ed. (Optional, opt out by
		      \textbf{8/29})
	      \\ \\
	Materials: & Scientific or graphing calculator (simpler calculators
	are not appropriate). \\
		   & Laptop for spreadsheet manipulation (certain topics only)
		   \\
	\bottomrule
\end{tabularx}

\section{Course Description}
Analytical Chemistry studies the methods of analysis useful in today's science
laboratories. In fact, you have used numerous analytical methods already in your
previous chemistry courses. This course stresses the theory and laboratory
practices of these analytical methods. In other words, is the difference between
two results \emph{significant}? What caused the results between two
\emph{identical} samples to differ? How can we report our results to a specific
level of \emph{confidence}? Calculations, basic analytical theory, error
analysis, and method advantages and limitations are emphasized throughout the
lecture and laboratory.

\subsection{Learning Objectives}
By the end
of the semester, you should be able to\ldots\ (in no particular order)
\begin{itemize}[noitemsep]
	\item Propose potential solution interferents and identify sample
		considerations in order to judge appropriate routes of chemical
		analysis.
	\item Apply statistical tests to data to determine significant
		differences or identify outliers that may skew results.
	\item Explain why proper sample handling and collection is imperative to
		delivering meaningful results.
	\item Summarize and organize equilibrium phenomena in solution to assess
		the validity of analytical techniques.
	\item Compare and contrast different analytical techniques
		(spectrophotometry, chromatography, potentiometry, titrations)
		and assess their suitability for particular analyses.
	\item Report experimental results to levels appropriate for the analysis
		used and consistent with the use objective.
	\item Verify experimental procedures with appropriate methods of
		standardization and quality control (matrix spike, external
		standard, internal standard, standard additions).
	\item Choose an appropriate analytical technique for a specific analysis
		and justify the choice using sound chemical knowledge.
\end{itemize}

\subsection{Course Community and Communication}
\emph{Your active involvement in the course is important for your success!}
Throughout the semester, I will encourage active participation both during and
outside of lecture. In order to facilitate some community-building, we will be
using the Discussion Boards on BOLT extensively. If you have a question with an
answer that could benefit the class as a whole, please post your question to the
Discussion Board, rather than sending an individual email to directly to me. I
would also be happy to help organize study groups if you are interested. Please
let me know if you would like to chat with other classmates and I will put you
in touch. Learning occurs best when it is driven by your own motivation and
curiosity, not by passively absorbing information I share in a brief 50-minute
period!

All course announcements will be delivered on the BOLT course page. It is your
responsibility to regularly check BOLT. I will \emph{not} send out email
announcements, so BOLT is the \emph{only} location you will need to check for
all important course information --- in my experience, a single location for all
information significantly decreases any confusion. If you would like push
notifications on your phone when announcements are posted, please download the
\href{https://documentation.brightspace.com/EN/brightspace/requirements/all/pulse.htm}{Brightspace
Pulse} app.

\subsection{Textbook}
You are expected to read the textbook along with lecture. This course is using an open access
textbook, \textit{Analytical Chemistry 2.1} by David Harvey. Open access means
that it is freely available online. I opted to use this textbook because it
\begin{enumerate*}[label={(\arabic*)}]
	\item offers the same information as other textbooks,
	\item has received positive feedback from other analytical chemistry
		instructors,
	\item is more affordable (free) than traditional textbooks, and
	\item is in line with my teaching philosophy that education should be
		attainable that all who want it.
\end{enumerate*}
You may find the textbook on BOLT or online at
\url{https://chem.libretexts.org/Bookshelves/Analytical_Chemistry/Analytical_Chemistry_2.1_(Harvey)}.

Many other institutions use Daniel C. Harris's \textit{Quantitative Chemical
Analysis}. Based on feedback from students, I have decided to \emph{optionally
recommend} this textbook for those seeking more information. 
This \emph{optional} textbook was selected as ``Inclusive Access,'' which
significantly lowers the cost of accessing the textbook \emph{for the semester}
(you do not own it and you will lose access after the semester). Unfortunately,
Inclusive Access is explicitly ``Opt Out.'' As such, the cost of this
\emph{optional} textbook will be automatically charged to your BU student
account under “misc.\ charges”.  If you decide that you do not want to use this
\emph{optional} textbook, you may Opt Out by \textbf{August 29}.  If you ``Opt
Out'', you will not be charged for the textbook and will no longer have access.
If you have any questions, please contact
\href{mailto:inclusiveaccess@bloomu.edu}{inclusiveaccess@bloomu.edu}.

If you plan on attending graduate school in chemistry, I suggest opting out of
Inclusive Access and purchasing the
hard copy of \textit{Quantitative Chemical Analysis}. Older editions are most
likely cheaper and will contain most of the same information.

\section{Evaluations and Grading}
Unfortunately, we do need to include a metric for assessing your understanding
of the material. This course will be operating on a 100\,\% scale according to
the tables below. Grading evaluates how well you understand the course material
and your performance in the laboratory.  Accumulation of ``correct'' points on a
quiz or exam is based on the \emph{perception} of your understanding. This
perception is influenced by your logic, your terminology, your mathematical
accuracy and your legibility. The instructor must be convinced that you
understand correctly in order to award you full credit.

\begin{multicols}{2}
	\subsection{Point Distribution}
	\begin{tabular} {l r<{\,\%}}
		Exams               & 30 \\
		Homework      & 10 \\
		Lecture Assignments & 10 \\
		Final Exam               & 25 \\
		Laboratory & 25 \\ \midrule
		& 100
	\end{tabular}

	\subsection{Grade Assignment}%
	\begin{tabular} {r@{\,--\,}l<{\,\%} l@{\hspace{0.5in}}r@{\,--\,}l<{\,\%} l}
		\multicolumn{3}{c}{} & 75 & 77 & C+ \\
		91 & 100 & A  & 71 & 74 & C  \\
		88 & 90 & A- & 68 & 70 & C- \\
		\multicolumn{6}{c}{} \\
		85 & 87 & B+ & 65 & 67 & D+ \\
		81 & 84 & B  & 60 & 64 & D  \\
		78 & 80 & B- & 0  & 59 & F
	\end{tabular}
\end{multicols}

\subsection{Graded Items}
\begin{description}
	\item[Exams:] There will be three (3) 50 minute exams held during our
		regularly scheduled lecture time. Tardiness will not extend your
		exam time. If accommodations must be made, it is your
		responsibility to alert the instructor at least 72 hours
		\emph{before} the exam.
		\begin{description}
			\item[MU:] Exams will be proctored at the Testing Center (South
		Hall, 2\textsuperscript{nd} floor, room 210) at the time of our
		regularly scheduled lecture.
\end{description}
	\item[Lecture Assignments:] It is impossible to learn by passively
		absorbing information. Although I will run through examples and
		practice with you, you will work in groups to solve some brief
		assignments regularly. Work will be collected and graded, with
		an emphasis placed on participation and attempts.
	\item[Homework:] There will be regular homework assignments posted to
		LON-CAPA (see the \nameref{homework} section below). You will have sufficient
		opportunities to resubmit answers without penalty before the
		deadline. Make sure to start early so you can come to office
		hours with questions. Essentially, the only reason why you
		should lose points on homework is because you did not turn it in
		on time!
	\item[Final Exam:] The final exam will consist of a \emph{comprehensive}
		115 minute final exam during the final exam week. This will be
		of a similar format to all of the lecture exams held throughout
		the semester.
	\item[Laboratory:] Please see your laboratory syllabus and instructor
		for specific laboratory requirements. Your lab instructor will
		assign a laboratory grade that will contribute towards 25\% of
		your final grade in this course.
\end{description}

\subsection{Homework}\label{homework}
This course is using LON-CAPA (\url{https://lon-capa.bloomu.edu}) for homework
assignments. You will be sent an email with your initial username and password.
Unfortunately, it is impossible for me to personally return timely feedback on
homework in addition to grading lab reports, lecture assignments, exams and
materials from other classes. LON-CAPA is able to automatically grade your
answers and will provide you multiple opportunities without penalty to answer
homework questions correctly. This will allow you ample time to seek extra help
if needed. Feel free to work with others (in fact, I encourage it!), but do note
that on most questions, everyone will have different answers.

LON-CAPA may be very different than other homework systems you have used in the
past. All content has been coded and tested by your instructor. If you are
having any issues with any specific problems or the website itself, please reach
out to me as soon as possible. I am using LON-CAPA for its incredible
flexibility in designing problems as well as its cost to students --- free.

\subsection{Grading Errors}
Did I make a mistake? Let me know within 72 hours from the return of the graded
item so that I may re-grade it. \emph{The entire item will be re-graded}. This
timeline is necessary so I don't get a pile of re-grades during the last week of
class. You have until 4:30\,p.m.\ on the day of the final exam to bring clerical
errors (e.g., I typed a grade incorrectly into BOLT) to my attention.

\section{Important Dates}\label{importantdates}
\begin{center}
	\begin{tabular} {l l l}
		September 5  & Monday     & Labor Day (No Class) \\
		September 12 & Monday     & Exam 1 \\
		October 17   & Monday     & Exam 2 \\
		November 21  & Monday     & Exam 3 \\
		November 23  & Wednesday  & Thanksgiving Recess (No Class) \\
		November 25  & Friday     & Thanksgiving Recess (No Class) \\
		December 2   & Friday     & Last Day of Class \\
		December 5   & Monday     & Final Examination (8:00\,am --
		10:00\,am)
	\end{tabular}
\end{center}

Please note, if you are unable to take an examination at
its scheduled time due to some unavoidable circumstance, you are obligated to
provide a documented valid excuse to your instructor. \emph{Failure to provide a
valid excuse on these terms will result in a score of zero for the 
exam.}

	\begin{center}
		\renewcommand\arraystretch{1.25}
\begin{tabularx}{\linewidth} {X X}
	\toprule
	\bfseries Valid Excuse & \bfseries Time Frame to Provide Documentation
	\\ \midrule
	Personal illness or accident, illness or accident of a dependent child,
	or death or critical illness of an immediate family member &
	48\,hr after the missed graded event \\
	Participation in a university-sponsored activity &
	48\,hr in advance of the missed graded event \\
	Military duty & As soon as possible \\
	Religious observance & The second Friday of the semester \\
	Others, as merited on a case-by-case basis & 48\,hr after the
	missed graded event \\
	\bottomrule
\end{tabularx}
\end{center}

Immediate technical issues, such as BOLT being inaccessible, can be submitted to
\url{https://helpdesk.bloomu.edu}. At the bottom of the form, you must include
my email address (dmccurry@bloomu.edu) in the ``Additional Email Notifications''
to alert me that you are taking steps to resolve the issue.

\section{Policies and Expectations}
In order to maintain an atmosphere that is conducive to learning, all members
of the class are expected to demonstrate common courtesy during the lecture,
including but not exclusively:
\begin{enumerate}
	\item \textbf{Following university protocol, when necessary, wear a mask
		that covers your nose and mouth.} Not all students are
		vaccinated and other, more infectious variants pose a health
		risk to all individuals. In order to ensure the well-being of
		your fellow classmates, please wear a well-fitting mask while
		inside.
	\item \textbf{Be on time for class.} If you are unavoidably late, enter
		the room quietly and quickly choose a seat.  Tardiness will not
		extend your exam time. We will not return to any missed lecture
		assignments.
	\item \textbf{Once in class, stay for the duration.} If you must leave
		early, give me advance warning. If you become ill during an
		exam, the missed graded event policy will take effect with the
		consent of the professor. In either case, you will not be
		allowed to return that day. Take care of your personal needs
		\emph{before class}.
	\item \textbf{Silence your cell phone and/or other electronic devices.}
		Students with special circumstances need to speak with me right
		away.
	\item \textbf{Refrain from talking to your neighbors excessively.} Stay
		focused on the lecture materials. If you missed something, feel
		free to ask me to repeat it!
\end{enumerate}
\emph{Please review
\href{https://www.bloomu.edu/prp-3881-student-disruptive-behavior-policy}{PRP 3881 Student
Disruptive Behavior Policy} for further details regarding the consequences for
disruptive behavior.}

\subsection{Attendance and Participation}
Physical presence is only part of going to class. Participation in class is
expected. You will be called upon at random and are advised to be prepared.
The best learning occurs in an active environment where everyone participates.
In order to be prepared, you should read the textbook prior to coming to
lecture.  You do not need to understand everything on the first perusal. Many
of your questions will probably be answered by the lecture or as you work the
problems.

\section{Academic Dishonesty}
\emph{Academic dishonesty is not tolerated.} If you are unclear about what is
dishonest, please see 
\href{https://www.bloomu.edu/prp-3512-academic-integrity-policy}{PRP 3512
Academic Integrity Policy} for clarification. All exam work must be your own.
You are permitted to work with others on lecture assignments and homework. Do
note that homework answers are randomized and you will most likely not have the
same answers as your classmates.

\begin{mdframed}
	\centering\bfseries The minimum penalty for academic dishonesty is a
	course assignment of ``F'' for \emph{all} students involved.
\end{mdframed}

\section{Class Cancellation}
If class is canceled on the day of an exam, it will be administered the next
time class meets.  If class is canceled the period before an exam, anticipate
that it will still be administered on its scheduled day. If compression
(see \href{https://www.bloomu.edu/documents/prp5205}{PRP 5205 University Closing
Policy})
occurs on an exam day, the exam will be held as scheduled (with
grading adjusted to allow for the shortened time period).

\section{Fire Alarms}
In the event of a building evacuation, calmly and quickly leave the building via
the nearest exit. 
\begin{description}
	\item[BU:] Your instructor will point exits out the first week of class.
		Gather with your class on the quad lawn in front of Bakeless. In
		the case of inclement weather, we will meet in the Kehr Union
		Ballroom. Do not re-enter the building for any reason. Do not
		gather around the exits. Do not enter a building that has an
		alarm sounding. There is one, and only one, drill the first week
		of the semester. There are \emph{no} random drills.
	\item[MU:] Please follow instructions given per Mansfield campus fire
		policies.
\end{description}


\section{Last Day to Withdraw}
The Registrar has set the last day of class as the withdrawal deadline.
Students are encouraged to review the latest withdrawal
(\href{https://www.bloomu.edu/prp-3462-undergraduate-course-withdrawal}{PRP 3462 Undergraduate
Course Withdrawal}) and course repeat
(\href{https://www.bloomu.edu/prp-3452-course-repeat}{PRP 3452 Course
Repeat}) policies of the
University prior to that date as there are strict limits on the number of
repeats one can have. 

Note that withdrawal should be taken as a \emph{last resort}. Your instructor
maintains 5 office hours per week and is willing to meet for additional
appointments to help you succeed in this class. Do not wait until the last minute to seek help.

\begin{mdframed}
	\centering
	All policies, rules, and procedures (PRP) are available at
	\url{https://www.bloomu.edu/about/administration-and-governance/policies}
\end{mdframed}

\section{Accommodations}
Reasonable accommodations may be made to students who have documented
disabilities. If you have a documented disability that requires academic
accommodations and are not registered with
\href{https://bloomu.prod.acquia-sites.com/offices-directory/disability-services}{University
Disability Services}, it is recommended that you contact the Disability Services office during
the \textit{first two weeks of classes} or immediately upon diagnosis to ensure
accommodations are met in an efficient, appropriate, and timely manner.
Note that you must contact the Disability Services office to renew accommodation
letters at the start of each semester. Please also provide documentation to Dr.\
McCurry as promptly as possible.

\vfill

\begin{mdframed}
	\noindent
	The materials contained in this syllabus, in the lab manual, and the
	BOLT webpage for this course are intended only for those registered for
	the above course/semester. These materials cannot be used without the
	expressed written consent of Dr.\ McCurry.

	\bigskip

	\noindent
	The source code for some material has been licensed under the CC
	BY-NC-SA 4.0 license to provide you with an opportunity to view the
	original source materials and contribute to the content. If you would
	like to view the material and suggest improvements, please visit
	\url{https://github.com/danian95/CHEM321}.
\end{mdframed}

\end{document}
