% !TEX program = xelatex
\documentclass{article}

\title{Chapter 2 Practice Problem Solutions}
\date{}
\author{D.A. McCurry}

\usepackage{analchem}
\usepackage{enumitem}
\usepackage[letterpaper,margin=1in]{geometry}
\sisetup{math-micro=\text{µ},text-micro=µ}
\usepackage{graphicx}
%\DeclareSIUnit{\ounce}{oz}
%\DeclareSIUnit{\calorie}{cal}
%\DeclareSIUnit{\inch}{in.}

\begin{document}

\maketitle

\begin{enumerate}[start=1,leftmargin=0pt]
	\item \begin{enumerate}[label=(\alph*)]
		\item See answer in back of book.
		\item See Lab Manual pg. iv.
	\end{enumerate}
	\item We primarily use nitrile gloves in laboratory. See compatibility
		at
		\url{https://eta-safety.lbl.gov/sites/all/files/VWR%20Chemical%20Resistance%20Gloves%20Chart.pdf}.
	\item See answer in back of book.
\end{enumerate}

\begin{enumerate}[start=6,leftmargin=0pt]
	\item See pg. 27 in text.
	\item According to the buoyancy equation (pg. 29),
		\begin{align*}
			m &= \frac{m^{\prime} \bigg(1 -
					\frac{d_a}{d_w}\bigg)}{\bigg(1 -
			\frac{d_a}{d}\bigg)} \\
			&= \frac{m^{\prime} \bigg(1 -
					\frac{d_a}{\SI{8.0}{\gram\per\milli\liter}}\bigg)}{\bigg(1
			- \frac{d_a}{\SI{8.0}{\gram\per\milli\liter}}\bigg)} \\
			m &= m^\prime
		\end{align*}
		So the ratio of the corrected mass to the measured mass is 1.
	\item We can use the buoyancy equation (pg. 29) to figure this one out:
		\begin{align*}
			m &= \frac{m^{\prime} \bigg(1 -
					\frac{d_a}{d_w}\bigg)}{\bigg(1 -
			\frac{d_a}{d}\bigg)} \\
			&= \frac{(\SI{14.82}{\gram})\bigg(1 -
					\frac{\SI{0.0012}{\gram\per\milli\liter}}{\SI{8.0}{\gram\per\milli\liter}}\bigg)}{\bigg(1
					-
			\frac{\SI{0.0012}{\gram\per\milli\liter}}{\SI{0.626}{\gram\per\milli\liter}}\bigg)}
			\\
			&= \SI{14.84623624}{\gram} \approx \boxed{\SI{14.85}{\gram}}
		\end{align*}
	\item This is actually referring to Figure 2-7 (a typo). From the
		Figure 2-7, we see that substances with lower densities require
		a much larger buoyancy correction than those with higher
		densities. As such, \ch{Li} ($d =
		\SI{0.53}{\gram\per\milli\liter}$) should require the \emph{greatest}
		buoyancy correction. Values closest to the density of the
		reference material (here \SI{8.0}{\gram\per\milli\liter}) will
		have the smallest buoyancy correction. As such, \ch{PbO2} ($d =
		\SI{9.4}{\gram\per\milli\liter}$) will require the \emph{smallest}
		buoyancy correction.
	\item Buoyancy equation (pg. 29):
		\begin{align*}
			m &= \frac{m^{\prime} \bigg(1 -
				\frac{d_a}{d_w}\bigg)}{\bigg(1 -
				\frac{d_a}{d}\bigg)} \\
			&= \frac{(\SI{4.2366}{\gram})\bigg(1 -
				\frac{\SI{0.0012}{\gram\per\milli\liter}}{\SI{8.0}{\gram\per\milli\liter}}\bigg)}{\bigg(1
				-
				\frac{\SI{0.0012}{\gram\per\milli\liter}}{\SI{1.636}{\gram\per\milli\liter}}\bigg)}
			\\
			&= \SI{4.239073855}{\gram} \approx \boxed{\SI{4.239}{\gram}}
		\end{align*}
		If we did not correct for buoyancy, the calculated molarity would be too low by \SI{0.06}{\percent}
		\begin{align*}
		\frac{|4.239073855-4.2366|}{4.239073855} \times 100 &= \SI{0.058358387}{\percent} \\
		&\approx \boxed{\SI{0.06}{\percent}}
		\end{align*}
	\item See Table 2-6 in text (pg. 37).
\end{enumerate}

\begin{enumerate}[start=24,leftmargin=0pt]
	\item We need to first determine the mass of \emph{just} water in the flask and then correct for the volume of water (Table 2-7, pg. 38).
	\begin{align*}
	\underbrace{(\SI{20.2144}{\gram} - \SI{10.2634}{\gram})}_{\mathclap{\text{mass of water \emph{only}}}} \times \frac{\SI{1.0029}{\milli\liter}}{\SI{1}{\gram}} &= \SI{9.9798579}{\milli\liter} \\
	&\approx \boxed{\SI{9.9799}{\milli\liter}}
	\end{align*}
	\item For the first part, we're essentially looking at a \% error calculation:
	\begin{align*}
	\frac{|1.0020 - 1.0040|}{1.0020} \times 100 &= \SI{0.1996007}{\percent} \\
	&\approx \boxed{\SI{0.2}{\percent}}
	\intertext{Rather than using the percentage, I just multiplied by the ratios of corrections:}
	\frac{\SI{0.5000}{\mole}}{\SI{1000}{\milli\liter}} \times
	\frac{\SI{1.0020}{\milli\liter}}{\SI{1}{\gram}} \times
	\frac{\SI{1}{\gram}}{\SI{1.0040}{\milli\liter}}
	\times \frac{\SI{1000}{\milli\liter}}{\SI{1}{\liter}} &= \boxed{\SI{0.4990}{\Molar}}
	\end{align*}
	I find this conceptually easier for me, but you could also take the \SI{0.2}{\percent} times the original concentration and then subtract that answer from \SI{0.5000}{\Molar}.
\end{enumerate}

\begin{enumerate}[start=31,leftmargin=0pt]
	\item Here's a screenshot of what my spreadsheet looks like:
	
		\includegraphics[width=\linewidth]{2-31.png}
		
\end{enumerate}

\end{document}
