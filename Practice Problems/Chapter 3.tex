% !TEX program = xelatex
\documentclass{article}

\title{Chapter 3 Practice Problem Solutions}
\date{}
\author{D.A. McCurry}

\usepackage{analchem}
\usepackage[inline]{enumitem}
\usepackage[letterpaper,margin=1in]{geometry}
\sisetup{math-micro=\text{µ},text-micro=µ}
\usepackage{amssymb}
%\usepackage{graphicx}
%\DeclareSIUnit{\ounce}{oz}
%\DeclareSIUnit{\calorie}{cal}
%\DeclareSIUnit{\inch}{in.}

\setlist[enumerate,1]{leftmargin=0pt}
\setlist[enumerate,2]{label = (\alph*)}

\begin{document}

\maketitle

\begin{enumerate}[label={3-\Alph*.}]
	\item This wasn't assigned but I accidentally answered it, so here it is anyway.
	\begin{enumerate}
		\item Sig fig rules for subtraction (3 SFs):
		\begin{center}
		\begin{math}
			\begin{array} {c S[table-format=2.4]<{\,\si{\gram}}}
				& 12.5296 \\
				- & 12.4372 \\ \midrule
				& 0.0924
			\end{array}
		\end{math}
		\end{center}
	\item Propagation of error for subtraction:
	\begin{align*}
	e &= \sqrt{e_1^2 + e_2^2} \\
	&= \sqrt{0.0003^2 + 0.0003^2} \\
	&= \SI{0.000424264}{\gram}
	\intertext{And then to determine relative percent error:}
	\frac{0.000424264}{0.0924} \times 100 &= \SI{0.459160248}{\percent}
	\end{align*}
	Thus, our answer to a reasonable number of digits is \boxed{\SI{0.0924(4)}{\gram}} or
	\boxed{\num{0.0924}\,(\pm \SI{0.5}{\percent})\,\si{\gram}}
		\end{enumerate}
\end{enumerate}

\begin{enumerate}[start=1]
	\item \begin{enumerate*}
		\item 5
		\item 4
		\item 3
		\end{enumerate*}
	\item \begin{enumerate*}
		\item \num{1.237}
		\item \num{1.238}
		\item \num{0.14}
		\item \num{2.1}
		\item \num{2.01} (However, the book follows the ``round to the even digit rule'')
		\end{enumerate*}
	\item Following the rule we learned in CHEM115,
	\begin{enumerate*}
		\item 0.217
		\item 0.217
		\item 0.217
		\end{enumerate*}.
	Understand that I have learned it both ways and now you have, too. Let the mathematicians and statisticians argue about which method is more correct. Whether you round up 1 digit or not with an insignificant 5 is often not going to matter as this last digit has the most uncertainty in it.
\end{enumerate}

\begin{enumerate}[start=5]
	\item \begin{enumerate*}
		\item \num{3.71}
		\item \num{10.7}
		\item \num{4.0e1}
		\item \num{2.85e-6}
		\item \num{12.6251}
		\item \num{6.0e-4}
		\item \num{242}
	\end{enumerate*}
	\item \begin{enumerate}
		\item \ch{BaF2}:
			\begin{center}
				\begin{tabular} {
					r @{ $\times$ }
					c @{ @ }
					S[table-format=3.7,table-space-text-post={\,\si{\gram\per\mole}}]<{\,\si{\gram\per\mole}} @{ $=$ }
					S[table-format=3.7]<{\,\si{\gram\per\mole}}
				}
					1 & Ba & 137.327    & 137.327 \\
					2 & F  & 18.9984032 & 37.9968064 \\ \midrule
					\multicolumn{3}{l}{} & 175.324 \\
				  \end{tabular}
				\end{center}
		\item \ch{C6H4O4}:
			\begin{center}
				\begin{tabular} {
					r @{ $\times$ }
					c @{ @ }
					S[table-format=2.5,table-space-text-post={\,\si{\gram\per\mole}}]<{\,\si{\gram\per\mole}} @{ $=$ }
					S[table-format=3.5]<{\,\si{\gram\per\mole}}
				}
					6 & C & 12.0106 & 72.0636 \\
					4 & H & 1.00798 & 4.03192 \\
					4 & O & 15.9994 & 63.9976 \\ \midrule
					\multicolumn{3}{l}{} & 140.0931 \\
				  \end{tabular}
				\end{center}
		\end{enumerate}
	\item \begin{enumerate*}
		\item \num{12.3}
		\item \num{75.5}
		\item \num{5.520e3}
		\item \num{3.04}
		\item \num{3.04e-10}
		\item \num{11.9}
		\item \num{4.600}
		\item \num{4.9e-7}
	\end{enumerate*}
\item The data in the table \emph{is} the data for Figure 3-3. Just check to
	see if it looks correct. While the gridlines make it easier, reports
	require graphs to have \emph{no} gridlines. Note that directions for
	more modern versions of Excel may require some adjustments.
\end{enumerate}

\begin{enumerate}[start=10]
	\item Systematic error is a constant, correctable error. Random error
		is always present. More trials should bring random error down.
	\item The apparent mass will be lower because the crucible apparently
		weighs more than it should. This is an example of systematic
		error.
	\item \begin{enumerate*}
		\item systematic
		\item systematic
		\item random
		\item random
	\end{enumerate*}
\end{enumerate}

\begin{enumerate}[start=14]
	\item The relative uncertainty is \SI{0.16789}{\percent}, so the absolute uncertainty is:
		\begin{align*}
			\frac{\num{0.16789}}{100} \times \num{3.12356} &= \num{0.005244145} \\
			&\approx \num{0.005}
		\end{align*}

		Expressed to appropriate digits, our answer is \boxed{\num{3.124(5)}} or
		\boxed{\num{3.124} \pm \SI{0.2}{\percent}}.
	\item \begin{enumerate}
		\item\label{3-15a} \begin{align*}
				6.2 - 4.1 &= 2.1 \\
				e = \sqrt{0.2^2 + 0.1^2} &= \num{0.223606798} \\
				\%e = \frac{\num{0.223606798}}{2.1} \times 100 &= \SI{10.64794275}{\percent} \\
				\therefore &\quad \boxed{\num{2.1(2)}} \\
				&\quad \boxed{\num{2.1} \pm \SI{11}{\percent}}
			\end{align*}
		\item \begin{align*}
				9.42 \times 0.016 &= \num{0.15088} \\
				\%e = \sqrt{\bigg(\frac{0.05}{9.43}\times 100 \bigg)^2
					+ \bigg(\frac{0.001}{0.016}\times 100 \bigg)^2
				} &= \SI{6.272450566}{\percent} \\
				e = \frac{\num{6.272450566}}{100} \times \num{0.15088} &=
				\num{0.009463873} \\
				\therefore &\quad \boxed{\num{0.151(9)}} \\
				&\quad \boxed{\num{0.151} \pm \SI{6}{\percent}}
			\end{align*}
		\item We need to consider each operation sequentially (order of
			operations). The first operation is the same as
			\ref{3-15a}, so this question can be rewritten as:
			\begin{align*}
				\num{2.1(2)} \div \num{9.43(5)} &= \num{0.222693531} \\
				\%e = \sqrt{\bigg(\frac{0.05}{9.43}\times 100 \bigg)^2
					+ \bigg(\frac{2.1}{0.02}\times 100 \bigg)^2
				} &= \SI{10.661136}{\percent} \\
				e = \frac{\num{10.661136}}{100} \times \num{0.222693531} &=
				\num{0.02374166} \\
				\therefore &\quad \boxed{\num{0.22(2)}} \\
				&\quad \boxed{\num{0.22} \pm \SI{11}{\percent}}
			\end{align*}
		\item Again, order of operations applies:
			\begin{align*}
				\num{6.2(2)e-3} + \num{4.1(1)e-3} &= \num{0.0103} \\
				e = \sqrt{(\num{0.2e-3})^2 + (\num{0.1e-3})^2} &= \num{0.000223607} \\
				\num{9.43} \times \num{0.0103} &= \num{0.097129} \\
				\%e = \sqrt{\bigg(\frac{0.05}{9.43}\times 100 \bigg)^2
					+ \bigg(\frac{0.000223607}{0.0103}\times 100 \bigg)^2
				} &= \SI{2.23475181}{\percent} \\
				e = \frac{\num{2.23475181}}{100} \times \num{0.097129} &=
				\num{0.002170592} \\
				\therefore &\quad \boxed{\num{0.097(2)}} \\
				&\quad \boxed{\num{0.097} \pm \SI{2}{\percent}}
			\end{align*}
		\end{enumerate}
	\item \begin{enumerate}
		\item If each operation follows the same propagation rules, we
			can just combine them into one step as follows:
			\begin{align*}
				9.23 + 4.21 - 3.26 &= 10.18 \\
				e = \sqrt(0.03^2 + 0.02^2 + 0.06^2) &= 0.07 \\
				\%e = \frac{0.07}{10.18} \times 100 &= \SI{0.68762279}{\percent} \\
				\therefore &\quad \boxed{\num{10.18(7)}} \\
				&\quad \boxed{10.18 \pm \SI{0.7}{\percent}}
			\end{align*}
		\item \begin{align*}
				91.3 \times 40.3 \div 21.2 &= \num{174.378673} \\
				\%e = \sqrt{\bigg(\frac{1.0}{91.3}\times 100 \bigg)^2
					+ \bigg(\frac{0.2}{40.3}\times 100 \bigg)^2
					+ \bigg(\frac{0.2}{21.1}\times 100 \bigg)^2
				} &= \SI{1.531144971}{\percent} \\
				e = \frac{\num{1.531144971}}{100} \times \num{174.378673} &=
				\num{2.669990281} \\
				\therefore &\quad \boxed{\num{174(3)}} \\
				&\quad \boxed{174 \pm \SI{2}{\percent}}
			\end{align*}
		\item Order of operations:
			\begin{align*}
				4.97 - 1.86 &= 3.11 \\
				e = \sqrt{0.05^2 + 0.01^2} &= \num{0.050990195} \\
				3.11 \div 21.1 &= \num{0.147393365} \\
				\%e = \sqrt{\bigg(\frac{\num{0.050990195}}{3.11}\times 100 \bigg)^2
					+ \bigg(\frac{0.2}{21.1}\times 100 \bigg)^2
				} &= \SI{1.893831214}{\percent} \\
				e = \frac{\num{1.893831214}}{100} \times \num{0.147393365} &=
				\num{0.002791382} \\
				\therefore &\quad \boxed{\num{0.147(3)}} \\
				&\quad \boxed{0.147 \pm \SI{2}{\percent}}
			\end{align*}
		\item \begin{align*}
				\num{2.0164} + \num{1.233} + \num{4.61} &= \num{7.8594} \\
				e = \sqrt{\num{0.0008}^2 + 0.002^2 + 0.01^2} &= \num{0.010229369} \\
				\%e = \frac{\num{0.010229369}}{\num{7.8594}} \times 100 &= \num{0.130154585} \\
				\therefore &\quad \boxed{\num{7.86(1)}} \\
				&\quad \boxed{7.86 \pm \SI{0.1}{\percent}}
			\end{align*}
		\item \begin{align*}
				\num{2.0164e3} + \num{1.233e2} + \num{4.61e1} &= \num{2.1858e3} \\
				e = \sqrt{(\num{0.0008e3})^2 + (\num{0.002e2})^2 + (\num{0.01e1})^2} &= \num{0.830662386} \\
				\%e = \frac{\num{0.830662386}}{\num{2.1858e3}} \times 100 &= \num{0.038002671} \\
				\therefore &\quad \boxed{\num{2185.8(8)}} \\
				&\quad \boxed{2185.8 \pm \SI{0.04}{\percent}}
			\end{align*}
		\item \begin{align*}
				3.14^{\frac{1}{3}} &= \num{1.464344351} \\
				\%e = \frac{1}{3} \times \frac{0.05}{3.14} \times 100 &= \num{0.530785563} \\
				e = \frac{\num{0.530785563}}{100} \times \num{1.464344351} &=
				\num{0.007772528} \\
				\therefore &\quad \boxed{\num{1.464(8)}} \\
				&\quad \boxed{1.464 \pm \SI{0.5}{\percent}}
			\end{align*}
		\item \begin{align*}
				\log 3.14 &= \num{0.496929648} \\
				e = \frac{1}{\ln 10} \frac{0.05}{3.14} &= \num{0.006915517} \\
				\%e = \frac{\num{0.006915517}}{\num{0.496929648}} \times 100 &=
				\num{1.391649151} \\
				\therefore &\quad \boxed{\num{0.497(7)}} \\
				&\quad \boxed{0.497 \pm \SI{1}{\percent}}
			\end{align*}
		\end{enumerate}
\end{enumerate}

\begin{enumerate}[start=18]
	\item \begin{enumerate}
		\item Formula mass of \ch{NaCl}:
			\begin{center}
				\begin{tabular} {
					r @{ $\times$ }
					c @{ @ }
					S[table-format=2.9(1),table-space-text-post={\,\si{\gram\per\mole}}]<{\,\si{\gram\per\mole}} @{ $=$ }
					S[table-format=2.9(1)]<{\,\si{\gram\per\mole}}
				}
				1 & Na & 22.98976928(2) & 22.98976928(2) \\
				1 & Cl & 35.452(6)      & 35.452(6) \\ \midrule
				\multicolumn{3}{l}{}    & 58.442(6)
			\end{tabular}
			\end{center}
		\item Need to follow multiplication and division propagation rules:
			\begin{align*}
				\frac{\SI{2.634(2)}{\gram}}{\SI{0.10000(8)}{\liter}}
				\times
				\frac{\SI{1}{\mole}}{\SI{58.442(6)}{\gram}} &=
				\SI{0.450705041}{\Molar} \\
				\%e = \sqrt{\bigg(\frac{0.002}{2.634} \times 100 \bigg)^2
					+ \bigg(\frac{0.00008}{0.10000} \times 100 \bigg)^2
				+ \bigg(\frac{0.006}{58.442} \times 100 \bigg)^2}
				&= \num{0.110773603} \\
				e = \frac{\num{0.110773603}}{100} \times \num{0.450705041}
				&= \num{0.000499262} \\
				\therefore &\quad \boxed{\SI{0.4507(5)}{\Molar}} \\
				&\quad \boxed{\num{0.4507} (\pm \SI{0.1}{\percent})\,\si{\Molar}}
			\end{align*}
	\end{enumerate}
	\item This is a big one\ldots\ Let's solve this stepwise. The buoyancy equation is
		\begin{align*}
			m &= \frac{m^\prime \bigg(1 - \frac{d_a}{d_w}\bigg)}{\bigg(1 - \frac{d_a}{d}\bigg)}
			\intertext{Let's solve for each term separately. Notice
			that the ``1'' does not have any error associated with
		it:}
		\frac{d_a}{d_w} &= \frac{\SI{0.0012(1)}{\gram\per\milli\liter}}{\SI{8.0(5)}{\gram\per\milli\liter}} = \underbrace{\num{0.00015(2)}}_{\mathclap{
		e_{\frac{d_a}{d_w}} = \sqrt{\big(\frac{0.0001}{0.0012}\big)^2 + \big(\frac{0.5}{8.0}\big)^2} \times \num{0.00015} = \num{0.00002}}} \\
			1 - \frac{d_a}{d_w} &= \num{0.99985(2)} \\ \\
		\frac{d_a}{d} &= \frac{\SI{0.0012(1)}{\gram\per\milli\liter}}{\SI{0.9972995(0)}{\gram\per\milli\liter}} = \underbrace{\num{0.0012(1)}}_{\mathclap{
		e_{\frac{d_a}{d}} = \sqrt{\big(\frac{0.0001}{0.0012}\big)^2 + \big(\frac{0}{0.9972995}\big)^2} \times \num{0.0012} = \num{0.0001}}} \\
			1 - \frac{d_a}{d} &= \num{0.9988(1)} \\ \\
			m &= \frac{(\SI{1.0346(2)}{\gram})(\num{0.99985(2)})}{\num{0.9988(1)}} = \underbrace{\boxed{\SI{1.0357(2)}{\gram}}}_{\mathclap{e_m = \sqrt{\big(\frac{\num{0.00002}}{\num{0.00015}}\big)^2 + \big(\frac{\num{0.0001}}{\num{0.0012}}\big)^2 + \big(\frac{\num{0.0002}}{\num{1.0346}}\big)^2} \times \SI{1.0357}{\gram} = \SI{0.0002}{\gram}}}
		\end{align*}
	\item \begin{enumerate}
		\item This should be routine by now:
			\begin{center}
				\begin{tabular} {
					r @{ $\times$ }
					c @{ @ }
					S[table-format=2.8(2),table-space-text-post={\,\si{\gram\per\mole}}]<{\,\si{\gram\per\mole}} @{ $=$ }
					S[table-format=3.8(2)]<{\,\si{\gram\per\mole}}
				}
					2 & Na & 22.98976928(2) & 45.97953856(4) \\
					1 & C  & 12.0106(10)    & 12.0106(10)    \\
					3 & O  & 15.9994(4)     & 47.9982(12)     \\ \midrule
					\multicolumn{3}{l}{}    & 105.988(2)     \\
				  \end{tabular}
				\end{center}
				The textbook seems to have this wrong according
				to its own explanation. Harris seems to have
				propagated this error assuming it was random
				instead of systematic!
			\item \label{3-20b} \begin{align*}
					\SI{0.967(9)}{\gram} \times
					\frac{\SI{1}{\mole}}{\SI{105.988(2)}{\gram}}
					\times
					\frac{\SI{2}{\mole}~\ch{HCl}}{\SI{1}{\mole}~\ch{Na2CO3}}
					\times
					\frac{1}{\SI{27.35(4)}{\milli\liter}}
					\times
					\frac{\SI{1000}{\milli\liter}}{\SI{1}{\liter}}
					&= \boxed{\SI{0.667(1)}{\Molar}}
				\end{align*}
				The error was calculated as:
				\begin{align*}
					e &= \sqrt{\bigg(\frac{0.0009}{0.9674}\bigg)^2
						+ \bigg(\frac{0.04}{27.35}\bigg)^2
					+ \bigg(\frac{0.002}{105.988}\bigg)^2}
					\times \SI{0.667}{\Molar} = \SI{0.001}{\Molar}
				\end{align*}
			\item I think this is a poorly worded problem. It's not
				clear how Harris determines that he can
				translate this into a mole ratio from a weight
				ratio. I would add an additional term to
				\ref{3-20b} as follows:
				\begin{align*}
					\SI{0.967(9)}{\gram}~\text{stock} &\times
					\frac{\SI{100.00(5)}{\gram}~\ch{Na2CO3}}{\SI{100}{\gram}~\text{stock}}
					\times
					\frac{\SI{1}{\mole}}{\SI{105.988(2)}{\gram}}
					\\
					&\times
					\frac{\SI{2}{\mole}~\ch{HCl}}{\SI{1}{\mole}~\ch{Na2CO3}}
					\times
					\frac{1}{\SI{27.35(4)}{\milli\liter}}
					\times
					\frac{\SI{1000}{\milli\liter}}{\SI{1}{\liter}}
					= \boxed{\SI{0.667(1)}{\Molar}}
				\end{align*}
				The error was calculated as:
				\begin{align*}
					e &= \sqrt{\bigg(\frac{0.0009}{0.9674}\bigg)^2
						+ \bigg(\frac{0.04}{27.35}\bigg)^2
					+ \bigg(\frac{0.002}{105.988}\bigg)^2
					+ \bigg(\frac{0.05}{100.00}\bigg)^2}
					\times \SI{0.667}{\Molar} = \SI{0.001}{\Molar}
				\end{align*}
				The same answer? Yes! The relative error in the
				weight percent is so small (\SI{0.05}{\percent})
				that this doesn't affect our result
				\emph{significantly}.
		\end{enumerate}
\end{enumerate}

\end{document}
