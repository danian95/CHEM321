\documentclass{beamer}

\usepackage{lecture}
\usepackage{bigdelim}
\usepackage{tikz}
\usepackage{pdfpages}

\title{Statistics}
\subtitle{Chapter 3}
\institute{CHEM321 - Analytical Chemistry I}

\begin{document}

\begin{frame}[allowframebreaks]{Case 3: Example 2}
	Five ore samples were analyzed by atomic absorption and UV spectroscopy
	Are these two methods equivalent?

	\begin{center}
		\sisetup{table-format=2.2}
		\begin{tabu} to \textwidth {c S S}
			{sample} & {titration (\%)} & {Na ISE (\%)} \\
			\tabucline[1.5pt]{-}
			1 & 36.25 & 36.15 \\
			2 & 38.92 & 38.86 \\
			3 & 27.63 & 27.58 \\
			4 & 31.45 & 31.41 \\
			5 & 40.69 & 40.58 \\
		\end{tabu}
	\end{center}

	\framebreak

	\begin{center}
		\sisetup{table-format=2.2}
		\begin{tabu} to \textwidth {c S S S[table-format=+1.2]}
			{sample} & {titration (\%)} & {Na ISE (\%)} &
			{difference} \\ \tabucline[1.5pt]{-}
			1 & 36.25 & 36.15 & +0.10 \\
			2 & 38.92 & 38.86 & +0.06 \\
			3 & 27.63 & 27.58 & +0.05 \\
			4 & 31.45 & 31.41 & +0.04 \\
			5 & 40.69 & 40.58 & +0.11 \\
		\end{tabu}
	\end{center}

	\begin{align*}
		\text{mean difference} &= 0.72 \\
		\text{std dev difference} &= 0.031 \\
		t_{\text{calc}} &= \dfrac{|\bar{d}|}{s_d} \sqrt{n} = 5.19
	\end{align*}

	\fbox{\parbox{\textwidth}{\centering Since $t$ (4 DF) is 4.604 (99\%)
	and 5.598 (99.5\%), $t_\text{calc} < t_\text{table}$ up to 99\% and
	these means and therefore methods are statistically different from
	50--99\% confidence. \emph{BUT} they would be found statistically
	the same at 99.5 and 99.9\% confidence.}}
\end{frame}

\begin{frame}{Recommendations for the Treatment of Outliers}
	\begin{itemize}
		\item Re-examine all data to see if a gross error occurred
			(proper lab notebook a must!)
		\item Estimate the precision that is usually expected from the
			technique to be sure outlier is actually questionable.
		\item Repeat the analysis if time and sample are available.
		\item If more data can not be secured, use a Grubbs test.
		\item If statistics indicate retention, consider reporting the
			median value instead of the mean.
	\end{itemize}
\end{frame}

\begin{frame}{The Grubbs Test}
	Recommended by the International Standards Organization and the American
	Society for Testing and Materials for the testing of potential
	outliers.\footnote{Historically, a $Q$ test has been used, but this has
	fallen out of favor.}

	\begin{align*}
		G_{\text{calc}} = \dfrac{|\text{questionable value} -
		\bar{x}|}{s}
	\end{align*}

	\begin{center}
	\begin{tabu} to \textwidth {c S[table-format=1.3]}
		\# Observations & {$G$ (95\% confidence)} \\
		\tabucline[1.5pt]{-}
		4 & 1.463 \\
		5 & 1.672 \\
		6 & 1.822 \\
		7 & 1.938 \\
		8 & 2.032 \\
		9 & 2.110 \\
		10 & 2.176
	\end{tabu}
	\end{center}
\end{frame}

\begin{frame}{Least Squares Analysis}
	At one point, I needed to know this\ldots

	\vspace{1em}

	You do not have to -- use Excel (or alternative)!

	\vspace{1em}

	You \alert{do} have to know how to add error bars to graphs. Reproduce
	Figure 4-16 on Harris p. 88.
\end{frame}
 
\end{document}
