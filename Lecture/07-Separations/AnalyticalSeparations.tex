% !TEX program = xelatex
%\documentclass[notes=show]{beamer}
%\documentclass[notes=hide]{beamer}
%\documentclass[notes=only]{beamer}
\documentclass[11pt,letterpaper]{article}
\usepackage{beamerarticle}

\usepackage{analchem}
\usepackage{lecture}
\usepackage{pgfplots}
\usepackage{multicol}

\title{Introduction to Analytical Separations}
\subtitle{Chapter 23}
\institute{Bloomsburg University}
\author{CHEM321 --- Analytical Chemistry I}
\date{Fall 2020}

\begin{document}

\maketitle
\mode<article>{\thispagestyle{fancy}}

\begin{frame}{Fall 2018 UV/Vis Beverage Lab}
	Sodium benzoate and caffeine concentrations were determined using
	UV-visible spectroscopy. For the sodium benzoate:

	\begin{tabularx}{\linewidth} {p{0.5\linewidth} *{3}{S[table-format=3.0]}}
		\toprule & \multicolumn{2}{c}{\bfseries mg/12 fl oz} \\
		\cline{2-3}
		{\bfseries Soda} & {\bfseries Class} &
		{\bfseries Reported} & {\bfseries \% Diff} \\ \midrule
		Cherry Vanilla Pepsi &		230 &   38 & 600 \\
		Mountain Dew Kickstart	&	430 &	54 & 800 \\
		Mountain Dew	&		210 &	54 & 390 \\
		Mountain Dew Ice &		18  &   54 & 33  \\
		Pepsi		&		110 &	38 & 280 \\
		\bottomrule
	\end{tabularx}

	\begin{itemize}[<+->]
		\item What went wrong?
		\item We need some way to \alert{separate} the components.
	\end{itemize}
\end{frame}

\begin{frame}{How do we separate things?}
	\begin{itemize}[<+->]
		\item Filtration
			\begin{itemize}
				\item Good for separate states of matter.
			\end{itemize}
		\item Evaporation/Distillation
			\begin{itemize}
				\item Need good range of boiling points.
			\end{itemize}
		\item Solvent Extraction
			\begin{itemize}
				\item Requires differences in solubility between
					different compounds.
				\item With an infinite range of solvents, we can
					separate everything!
					{\footnotesize (Let's pretend this is
					possible for now\ldots)}
			\end{itemize}
	\end{itemize}
\end{frame}

\frame{\section{Solvent Extractions}
	\begin{learningobjectives}
	\item Determine the most efficient means of extracting an analyte using
		different solvents.
	\item Calculate the extraction efficiency of a solvent extraction.
	\item Explain what affects the efficiency of an extraction.
	\item Identify the limitations of simple solvent extractions.
	\end{learningobjectives}
}

\vspace{\stretch{-1}}

\begin{frame}{Solvent Extraction}
	\begin{columns}
		\column{0.5\textwidth}
	\begin{itemize}
		\item A solute (or analyte) is transferred from one phase to
			another.
		\item Two \alert{immiscible} liquids are required.
	\end{itemize}

		\column{0.5\textwidth}
		\begin{center}
			\includegraphics[scale=0.55]{immiscible.png}
		\end{center}
	\end{columns}

	\begin{block}{Partition Coefficient}
		The equilibrium constant for the reaction in which a solute is
		partitioned between two phases.

		\begin{equation*}
			K =
			\frac{\mathcal{A}_{\ch{S}_2}}{\mathcal{A}_{\ch{S}_1}}
			\approx \frac{[\ch{S}]_2}{[\ch{S}]_1}
		\end{equation*}

		For $K > 1$, S prefers Phase 2. For $K < 1$, S prefers Phase 1.
	\end{block}
\end{frame}

\vspace{\stretch{-1}}

\begin{frame}{Extraction Efficiency}
	\only<+>{%
		If solute S in $V_1$ \si{\mL} of solvent 1 is extracted
			with $V_2$ \si{\mL} of solvent 2, we can calculate the
			\alert{fraction} of S remaining in solvent 1:
			\begin{align*}
				K = \frac{[\ch{S}]_2}{[\ch{S}]_1} &= \frac{(1 -
				q)m/V_2}{qm/V_1}
				\intertext{where $m$ is the number of moles of S
				and $q$ is the fraction remaining ---
				rearranging:}
				q &= \frac{V_1}{V_1 + KV_2}
			\end{align*}
		}

		\only<+>{%
		If we replace solvent 2 and run another extraction, we are
			taking a ``fraction of a fraction''. In other words, we
			multiply:
			\begin{align*}
				q \cdot q &= \left( \frac{V_1}{V_1 + KV_2}
				\right)^2 \\
				\intertext{And if we keep going\ldots}
					q^n &= \left( \frac{V_1}{V_1 + KV_2}
				\right)^n
			\end{align*}
		So the amount of solute remaining in a phase is
			\alert{exponentially} related to the number of
			extractions we perform.
		}
\end{frame}

\vspace{\stretch{-1}}

\begin{frame}[t]{Extraction Efficiency Example}
	Solute A has a partition coefficient of 3 between toluene and water,
	with three times as much in the toluene phase. Suppose that
	\SI{100}{\milli\liter} of a \SI{0.010}{\Molar} aqueous solution of A are
	extracted with toluene. What fraction of A remains in the aqueous phase
	if \emph{one} extraction with \emph{\SI{500}{\mL}} is performed or if
	\emph{five} extractions with \emph{\SI{100}{\mL}} are performed?
\end{frame}

\note{
	For a single extraction, we can use
	\begin{align*}
		q &= \frac{V_1}{V_1 + KV_2} \\
		&= \frac{100}{100 + (3)(500)} = 0.062 \approx \SI{6}{\percent}
		\intertext{and for 5 extractions, we take $q^n$, or $q^5$,}
		q^5 &= \bigg( \frac{V_1}{V_1 + KV_2} \bigg)^5 \\
		&= \bigg( \frac{100}{100 + (3)(100)} \bigg)^5 = \num{0.00098}
		\approx \SI{0.1}{\percent}
	\end{align*}

	It is more efficient to do several small extractions than one big one.
	}

\begin{frame}{Effects of Charge on Partition Coefficients}
	\begin{itemize}[<+->]
		\item Charged species prefer polar solvents.
		\item pH can alter charge:
			\begin{reaction*}
				HA + H2O <=> A- + \Oxo 
			\end{reaction*}
				The conjugate base is more soluble in more
					polar solvents.
				\item The different species can be
					\alert{distributed} between two phases:
					\begin{align*}
						\text{Distribution Coefficient}
						= D &= \frac{\text{total
						concentration in phase
						2}}{\text{total concentration in
						phase 1}} \\
						&=
						\frac{[\ch{HA}]_2}{[\ch{HA}]_1 +
						[\ch{A-}]_1}
					\end{align*}
%					{\footnotesize (See the
%					\alert{Challenge} on page 606 -- good
%					math manipulation practice for the
%					final!)}
		\item Similarly, chelation of metals can alter charge:
			\begin{reactions*}
				$n$L- + M^{$n+$} &<=> ML_{$n$} \\
				Ca^2+ + EDTA^4- &<=> CaEDTA^2-
			\end{reactions*}
			This is a great method of \alert{preconcentration} from
			environmental samples.
	\end{itemize}
\end{frame}

\vspace{\stretch{-1}}
\clearpage

\begin{frame}{Issues with Solvent Extractions}
	\begin{itemize}
		\item It can take quite a bit of time to perform $n$ extractions
			with a large number of solutes.
			\begin{itemize}
				\item Cherry Vanilla Pepsi has high fructose
					corn syrup, caramel color, ``natural
					flavor'', phosphoric acid, potassium
					sorbate, potassium citrate, citric acid,
					caffeine, potassium benzoate, and
					calcium disodium EDTA. 
			\end{itemize}
		\item We don't actually have infinite solvents for every solute.
%
%			\bigskip
%			\pause
%
%		\item Recall the microfluidic devices from the previous chapter:
%			\begin{itemize}[<+->]
%				\item What helped speed up analysis?
%				\item ``Let the solvent do the work.''
%			\end{itemize}
	\end{itemize}
\end{frame}

\vspace{\stretch{-1}}

\begin{frame}{Chromatography}
	\only<+>{%
	\begin{columns}
		\column{0.6\linewidth}
		Pigments from plants could be separated as
			they each interacted to a different extent with another
			phase leaving colored bands behind.
			\begin{center}
				\begin{tabular} {>{\bfseries}l l}
					chromatos & ``color'' \\
					graphein & ``to write''
				\end{tabular}
			\end{center}
		You may have observed this before in thin layer
			chromatography (TLC)
				\begin{itemize}
					\item The key features are the
						\alert{stationary} powder on the
						glass slide and the
						\alert{mobile} solvent.
				\end{itemize}
		\column{0.3\linewidth}
		\mode<presentation>{\includegraphics[scale=0.5]{tlc.jpg}}
	\end{columns}
}

\only<+>{%
		Rather than a small plate, we can fill a column with the
			\alert{stationary phase} and continuously run the
			\alert{mobile phase} through it.
			\begin{center}
		\includegraphics[scale=0.6]{fig23-5.png}
			\end{center}
		A detector placed at the end of the column can be used to
			quantify analyte.
}
\end{frame}

\begin{frame}{Stationary Phases for Different Applications}
	\begin{center}
		\only<+>{\includegraphics[scale=0.8]{fig23-6-1.png}}

		\mode<article>{\clearpage}

		\only<+>{\includegraphics[scale=0.8]{fig23-6-2.png}}

		\only<+>{\includegraphics[scale=0.8]{fig23-6-3.png}}
	\end{center}
\end{frame}

\vspace{\stretch{-1}}

\frame{\section{Physical Basis of Chromatography}
	\begin{learningobjectives}
	\item Explain the difference between the mobile phase volume and column
		volume.
	\item Compare different peaks on a chromatogram in terms of retention
		time.
	\item Relate chromatography back to partition coefficients.
	\end{learningobjectives}
}

\vspace{\stretch{-1}}

\begin{frame}{Chromatography Parameters}
	\only<+>{%
	Consider a column with an inner diameter of \SI{0.60}{\cm} in which
	\SI{20}{\percent} of the volume is occupied by
	the mobile phase, thus the volume of the column is
	\begin{align*}
		V &= \pi r^2 \times \text{length} \\
		\frac{V}{\text{length}} &= \pi (\SI{0.30}{\cm})^2 =
		\SI{0.283}{\mL\per\cm}
	\end{align*}
	and the volume occupied by the mobile phase is $
		\SI{20}{\percent} \times \SI{0.283}{\mL\per\cm} =
		\SI{0.0565}{\mL}$.
	}

	\only<+>{%
	\begin{itemize}
		\item The \alert{volume flow rate} is the volume of
			solvent that travels per minute through the
			column. For example, \SI{0.30}{\mL\per\minute}.
		\item The \alert{linear velocity} is the distance, in
			centimeters, traveled by the solvent in 1 min.
			At a flow rate of \SI{0.30}{\mL\per\minute},
			\begin{align*}
				\frac{\SI{0.30}{\mL}}{\SI{1}{\minute}}
				\times
				\frac{\SI{1}{\cm}}{\SI{0.0565}{\mL}} =
				\SI{5.3}{\cm\per\minute}
			\end{align*}
	\end{itemize}
}
	
\only<+>{%	
	\begin{itemize}
		\item The column geometry/material and mobile phase flow rate
			directly affect the \alert{time} that the solute is
			exposed to the stationary phase.
		\item \alert{More} interactions with the stationary phase mean a
			\alert{longer} period of time before we see the solute
			at the detector.
		\item In separations, we measure in terms of \alert{retention}.
	\end{itemize}
	
	\begin{center}
		\includegraphics[width=0.8\linewidth]{retention.png}
	\end{center}
}
\end{frame}

\vspace{\stretch{-1}}

\begin{frame}{Retention Time}
	\only<1-3>{%
	\begin{center}
		\includegraphics[scale=0.55]{retention.png}
	\end{center}

	\begin{description}[<+->]
		\item[Retention time ($\bm{t_r}$):] The time
			elapsed between injection and arrival at the
			detector.
%		\item \textbf{Retention volume ($\bm{V_t}$):} The volume
%			of mobile phase required to elute the solute
%			from the column.
		\item[Minimum possible time ($\bm{t_m}$):] The
			time elapsed for an \alert{unretained} solute to
			travel through the column.
		\item[Adjusted retention time ($\bm{t'_t}$):] A
			correction to account for delays in pressing the
			``Start'' button. Yes, really.
			\begin{equation*}
				t'_r = t_r - t_m
			\end{equation*}
	\end{description}
}

\only<4->{%
	\begin{description}
		\item[Relative retention ($\bm{\alpha}$):] The ratio of
			\alert{adjusted} retention times for two components.
			\begin{equation*}
				\alpha = \frac{t'_{r_2}}{t'_{r_1}} =
				\frac{k_2}{k_1}
			\end{equation*}
			where $t'_{r_2} > t'_{r_1}$ so that $\alpha > 1$.
		\item[Retention factor ($\bm{k}$):] The relative
			retention time expressed in multiples of $t_m$.
			\begin{equation*}
				k = \frac{t_r - t_m}{t_m} = \frac{t'_r}{t_m}
			\end{equation*}
	\end{description}
}
\end{frame}

\vspace{\stretch{-1}}

\begin{frame}{How is $t_r$ Related to $K$?}
	\only<+>{%
		If the solute is not retained at all, $t_r = t_m$,
		thus
	\begin{align*}
		k &= \frac{t_r - t_m}{t_m} = \frac{t_m - t_m}{t_m} =
		\frac{0}{t_m} = 0
		\intertext{If the solute spends 50\% of its time in the
		stationary phase and 50\% in the mobile phase, it will take
		twice as long to get through the column, so $t_r = 2t_m$
		and}
		k &= \frac{2t_m - t_m}{t_m} = \frac{t_m}{t_m} = 1
		\intertext{So, effectively}
		k &= \frac{\text{time solute spends in stationary
		phase}}{\text{time solute spends in mobile phase}}
	\end{align*}
}

\only<+>{%
At any given time, there are a certain number of
		moles in each phase. If a solute spends more time in a specific
		phase, there should be more moles of solute in that phase.
		Thus,
	\begin{align*}
		\frac{\text{time in stationary phase}}{\text{time
		in mobile phase}} &= \frac{\text{moles 
		in stationary phase}}{\text{moles in mobile phase}} \\
		k &= \frac{c_\text{s}V_\text{s}}{c_\text{m}V_\text{m}}
	\end{align*}
Recall that the partition coefficient is the ratio of
		concentrations in each phase, so
	\begin{align*}
		k &= K\frac{V_\text{s}}{V_\text{m}} = \frac{t_r - t_m}{t_m} =
		\frac{t'_r}{t_m}
	\end{align*}
}

\only<+>{%
	Because $t'_r \propto k \propto K$, relative retention can be expressed
	as
	\begin{align*}
		\alpha = \frac{t'_{r_2}}{t'_{r_1}} = \frac{k_2}{k_1} =
		\frac{K_2}{K_1}
	\end{align*}

	\begin{block}{Physical basis of chromatography}
		The greater the ratio of partition coefficients between mobile
		and stationary phases, the greater the separation between two
		components of a mixture.
	\end{block}
}
\end{frame}

%\begin{frame}[t]{Retention Time Example}
%	Unretained methane gave a sharp spike in \SI{42}{\second}, whereas
%	benzene required \SI{251}{\second}. The open tubular chromatography
%	column has an inner diameter of \SI{250}{\micro\meter} and is coated on
%	the inside with a layer of stationary phase \SI{1.0}{\um} thick.
%	Estimate the partition coefficient, $K$, for benzene between stationary
%	and mobile phases and state what fraction of the time benzene spends in
%	the mobile phase.
%\end{frame}

\frame{\section{Achieving High Quality Separations}
	\begin{learningobjectives}
	\item Calculate the efficiency of chromatographic separations in terms
		of resolution.
	\item Explain how resolution can decrease.
	\item Calculate the height equivalent and number of theoretical plates
		for a chromatogram.
	\end{learningobjectives}
}

\begin{frame}{Efficiency of Separations}
	\only<1>{
	There are two factors that contribute to separation efficiency:
	\begin{enumerate}
		\item Difference in retention time between compounds.
		\item Broadness of the peaks
	\end{enumerate}
	}

	\begin{center}
		\includegraphics[width=0.9\linewidth]{gaussianpeak.png}
	\end{center}

	\only<1>{Ideal peaks demonstrate a Gaussian shape.}

	\only<2->{
		Broadness is measured in terms of width, either
		\begin{enumerate}
			\item at the \alert{baseline} between tangents drawn to
				the steepest parts of the peak, $w = 4\sigma$.
			\item at \alert{half height}, $w_{1/2} = 2.35\sigma$.
				\visible<3>{\textbf{Why?}}
		\end{enumerate}
		}
\end{frame}

\begin{frame}{Resolution}
	\begin{align*}
		\parbox{0.25\linewidth}{\centering \footnotesize Resolution
		between 2 peaks} = \frac{\text{separation}}{\text{broadness}} =
		\frac{\Delta t_r}{w_\text{av}} = \frac{\Delta V_r}{w_\text{av}}
		= \frac{0.589\Delta t_r}{w_{1/2\text{av}}}
	\end{align*}
%	where
%	\begin{itemize}[]
%		\item $\Delta t_r$ or $\Delta V_r$ is the separation between
%			peaks (in time or volume)
%		\item $w_\text{av}$ is the average peak width of the two peaks.
%	\end{itemize}

	\begin{center}
		\includegraphics[width=0.9\linewidth]{resolution.png}
	\end{center}
\end{frame}

\vspace{\stretch{-1}}

\begin{frame}{Diffusion}
	If compounds elute according to their affinity to the
	stationary phase, why do they broaden?
	\begin{itemize}
		\item \alert{Diffusion} is the net transport of a solute from a
			region of high concentration to a region of low
			concentration.
	\end{itemize}

	\begin{center}
		\includegraphics[scale=0.6]{diffusion.png}
	\end{center}
\end{frame}

\vspace{\stretch{-1}}

\begin{frame}{Fick's First Law of Diffusion}
		The rate at which moles travel through a defined region of
			solution, the \alert{flux}, is proportional to the
			concentration gradient:
			\begin{equation*}
				J = -D \frac{dc}{dx}
			\end{equation*}
		The proportionality constant, D, is the \alert{diffusion
			coefficient}, defined through the Stokes-Einstein
			equation as:
			\begin{equation*}
				D = \frac{kT}{f} = \frac{kT}{6\pi\eta r} =
				\frac{\text{thermal energy}}{\text{friction
				coefficient}}
			\end{equation*}
		In short, the standard deviation of the peak width can be
			related as
			\begin{equation*}
				\sigma = \sqrt{2Dt}
			\end{equation*}
			This suggests that longer elution times lead to broader
			peaks!
\end{frame}

\vspace{\stretch{-1}}

\begin{frame}{Height Equivalent to a Theoretical Plate (HETP)}
	\only<+>{%
	\begin{columns}
		\column{0.6\textwidth}
		\begin{itemize}
			\item The efficiency of the column used is often
				compared in terms of \alert{HETP} or \alert{number of
				plates}.
			\item This is from historical usage in distillation.
			\item At each plate, we should have smaller amounts of
				impurity, so the number of plates is effectively
				the number of solvent extractions.
			\item For very tall plates, more time is allowed for
				vapor to diffuse, causing broader peaks.
		\end{itemize}
		\column{0.4\textwidth}
		\begin{center}
		\includegraphics[scale=0.45]{industrial-fractional-distillation.png}

		{\tiny https://en.wikipedia.org/wiki/Distillation}
		\end{center}
	\end{columns}
}

\only<+>{%
	\begin{itemize}
		\item In terms of chromatography, if we assume that solute
			travels a distance $x$ at linear velocity $u_x$, then
			the time it has been on the column is $t = x/u_x$ and
			therefore
			\begin{align*}
				\sigma &= \sqrt{2Dt} \\
				\sigma^2 &= 2Dt = 2D\frac{x}{u_x} =
				\underbrace{\bigg( \frac{2D}{u_x}
				\bigg)}_{\mathclap{\text{Plate height} \equiv
				H}} x = Hx \\
				H &= \frac{\sigma^2}{x} = \frac{\text{variance
				of the band}}{\text{distance traveled}}
			\end{align*}
		\item The \alert{smaller} the plate height, the \alert{narrower}
			the peak.
		\item Common values are \SIrange{\sim0.1}{1}{\mm} in GC,
			\SI{\sim10}{\um} in HPLC, and \SI{<1}{\um} in CE.
	\end{itemize}
}

\only<+>{%
	\begin{itemize}
		\item Provided that plate height is in terms of distance
			traveled, we can calculate the number of plates, $N$, in
			a column of length $L$:
			\begin{equation*}
				N = \frac{\text{length of column}}{\text{plate
				height}} = \frac{L}{H} = \frac{Lx}{\sigma^2} =
				\frac{L^2}{\sigma^2} = \frac{16L^2}{w^2}
			\end{equation*}
			because $x = L$ for the entire column and $\sigma =
			w/4$.
		\item Because retention time is dependent on column length (how
			long it takes for a solute to elute from the column), we
			can write
			\begin{equation*}
				N = \frac{16t_r^2}{w^2} = \frac{t_r^2}{\sigma^2}
				= \frac{5.55t_r^2}{w^2_{1/2}}
			\end{equation*}
			We can calculate the number of plates based on the
			analytes we are trying to separate.
	\end{itemize}
}

\only<+>{%
	\begin{itemize}
		\item Because plate height is a result of peak broadness and
			separation time, it can be related directly to
			resolution with the Purnell equation:
			\begin{equation*}
				\text{Resolution} = \frac{\sqrt{N}}{4}
				\frac{\alpha - 1}{\alpha} \bigg( \frac{k_2}{1 +
				k_2} \bigg)
			\end{equation*}
			where
			\begin{itemize}
				\item $\alpha$ is the relative retention of the
					peaks ($\alpha =
					\frac{t'_{r_2}}{t'_{r_1}} =
					\frac{k_2}{k_1}$)
				\item $k_2$ is the retention factor of the
					\alert{more retained} component ($k =
					\frac{t'_r}{t_m}$)
			\end{itemize}
	\end{itemize}
}

\only<+>{%
	\begin{equation*}
		\text{Resolution} = \frac{\sqrt{N}}{4}
		\frac{\alpha - 1}{\alpha} \bigg( \frac{k_2}{1 +
		k_2} \bigg)
	\end{equation*}

	The Purnell equation tells us a few important things:
	\begin{itemize}
		\item Increasing the column length increases resolution because
			$\text{Resolution} \propto \sqrt{N}$
		\item If $k_2$ is made larger (most likely through column
			material choice), resolution improves.
		\item If $\alpha$ is increased (possibly through changing
			temperature or mobile phase), the peaks move apart and
			resolution improves.
	\end{itemize}
}
\end{frame}

\begin{frame}{Diffusion is Not the Only Reason for Broadening}
	\begin{itemize}
		\item The error present in every step of the analysis is
			additive.
			\begin{equation*}
				\sigma^2_\text{obs} = \sigma^2_1 + \sigma^2_2 +
				\sigma^2_3 + \cdots = \sum\sigma_i^2
			\end{equation*}
		\item The injection volume introduces some error as we cannot
			instantly inject all of the material at once. Similarly,
			we cannot detect the entire volume of solute at once.
			\begin{equation*}
				\sigma^2_\text{injection} =
				\sigma^2_\text{detector} = \frac{(\Delta
				V)^2}{12}
			\end{equation*}
		\item The tubing connecting the injection port, column, and
			detector all demonstrate \alert{laminar flow}.
			\begin{equation*}
				\sigma^2_\text{tubing} = \frac{\pi
				d_t^4l_tu_v}{384D}
			\end{equation*}
	\end{itemize}
\end{frame}

\begin{frame}{The van Deemter Equation}
	\begin{columns}[onlytextwidth]
		\column{0.55\textwidth}
		The contributions to band broadening are summarized in the
		\alert{van Deemter Equation}:
		\begin{align*}
			H &\approx A + B\tfrac{1}{u_x} + Cu_x
			\shortintertext{where}
			A &= \text{Multiple paths} \\
			B\tfrac{1}{u_x} &= \text{Longitudinal diffusion} \\
			Cu_x &= \text{Equilibration time}
		\end{align*}

		At \alert{optimum linear velocity}, plate hight is lowest and
		the number of plates is highest.
		\column{0.45\textwidth}
		\begin{center}
		\includegraphics[scale=0.7]{vandeemter.png}
		\end{center}
	\end{columns}
\end{frame}

\begin{frame}{Multiple Flow Paths ($A$)}
	\begin{center}
		\includegraphics[width=0.9\linewidth]{multipleflow.png}
	\end{center}

	\begin{itemize}
		\item When a column is packed with many stationary phase
			particles, some solute can randomly flow through a
			shorter path length than others.
		\item This term is \alert{independent} of flow rate.
	\end{itemize}
\end{frame}

\begin{frame}{Longitudinal Diffusion ($\frac{B}{u_x}$)}
	\begin{columns}[onlytextwidth]
		\column{0.6\textwidth}
		\begin{itemize}
			\item Fick's First Law: areas of high
				concentration over time diffused to areas of low
				concentration.
			\item Band broadening was related to
				\begin{align*}
					\sigma^2 &= 2D_mt = \frac{2D_mL}{u_x}
					\shortintertext{thus,}
					H_\text{long. diff.} &=
					\frac{\sigma^2}{L} = \frac{2D_m}{u_x}
					\equiv \frac{B}{u_x}
				\end{align*}
		\end{itemize}
		\column{0.4\textwidth}
		\begin{center}
			\includegraphics[scale=0.7]{longitudinal.png}
		\end{center}
	\end{columns}
	\begin{itemize}
		\item This term is \alert{inversely proportional} to
			flow rate --- the faster the flow, the less time
			is available for solute to diffuse.
	\end{itemize}
\end{frame}

\begin{frame}{Equilibration Time ($Cu_x$)}
	\begin{columns}[onlytextwidth]
		\column{0.6\textwidth}
		\begin{itemize}
			\item Time is required for equilibrium at each
				infinitesimally small point along the column.
			\item Time must be spent for the solute to diffuse to
				the surface of the stationary phase.
				\begin{align*}
					H_\text{mass transfer} = Cu_x = (C_m +
					C_s)u_x
				\end{align*}
		\end{itemize}
		\column{0.4\textwidth}
		\begin{center}
			\includegraphics[scale=0.6]{masstransfer.png}
		\end{center}
	\end{columns}
	\begin{itemize}
		\item If the flow rate is too high, there is insufficient time
			for this mass transfer to occur, thus this term is
			\alert{directly proportional} to flow rate.
	\end{itemize}
\end{frame}

%\begin{frame}{Let's take a look at the AOAC lab\ldots}
\begin{frame}{Separation of Ethanol and Isopropanol}
	\only<1|article:0>{\documentclass{standalone}

\usepackage{tikz}
\usepackage{pgfplots}
\usepackage{mathtools}
\usepackage{mhchem}
\usepackage{siunitx}
\DeclareSIUnit{\Molar}{\textsc{m}}

\setlength{\textwidth}{3in}
\setlength{\textheight}{4in}

\begin{document}

\begin{tikzpicture}
	\begin{axis} [
			width=\linewidth,
			height=18em,
			xmax=5.75,
			xmin=0,
			ymax=4750,
			ymin=0,
			xlabel={$t$ (\si{\minute})},
			ylabel={\si{\milli\volt}},
			label style={font=\footnotesize},
			tick label style={font=\footnotesize},
			xtick={0,0.5,...,6},
			ytick={0,500,...,4500},
			no markers,
    			every axis plot/.append style={ultra thick},
			]
		\addplot table [
			x=t,
			y=mV,
			col sep=comma
			]
		{beer-gc.csv};
	\end{axis}
\end{tikzpicture}

\end{document}
}
	\only<2>{\documentclass{standalone}

\usepackage{tikz}
\usepackage{pgfplots}
\usepackage{mathtools}
\usepackage{mhchem}
\usepackage{siunitx}
\DeclareSIUnit{\Molar}{\textsc{m}}

\setlength{\textwidth}{3in}
\setlength{\textheight}{4in}

\begin{document}

\begin{tikzpicture}
	\begin{axis} [
			width=\linewidth,
			height=18em,
			xmax=5.25,
			xmin=4,
			ymax=4750,
			ymin=0,
			xlabel={$t$ (\si{\minute})},
			ylabel={\si{\milli\volt}},
			label style={font=\footnotesize},
			tick label style={font=\footnotesize},
			xtick={0,0.1,...,6},
			ytick={0,500,...,4500},
			no markers,
    			every axis plot/.append style={ultra thick},
			]
		\addplot table [
			x=t,
			y=mV,
			col sep=comma
			]
		{beer-gc.csv};
	\end{axis}
\end{tikzpicture}

\end{document}
}

	What is the number of theoretical plates?
\end{frame}

\note{
	\begin{columns}
		\column{0.35\linewidth}
		\begin{align*}
			N &= \frac{16t_r^2}{w^2} \\
			N &= \frac{5.55t_r^2}{w^2_{1/2}} \\
			N &= \frac{L}{H} \qquad
			H = \frac{L}{N}
		\end{align*}
		\column{0.55\linewidth}

		\textbf{Column Specs:}

		Equity-1701

		Fused Silica Capillary Column

		$\SI{30}{\meter} \times \SI{0.53}{\milli\meter} \times
		\SI{1.0}{\um~film~thickness}$
	\end{columns}
	\bigskip
		\textbf{My values:}

		\begin{tabularx}{\linewidth} {X S[table-format=1.4]
			S[table-format=1.6] S[table-format=4] S[table-format=2]}
			\toprule
			& {$t_r$ (min)} & {$w_{1/2}$ (min)} & {$N$} & {$H$ (mm)}
			\\ \midrule
			Ethanol & 4.225 & 0.046875 & 2113 & 14 \\
			Propanol & 4.8375 & 0.05 & 2598 & 12 \\
			\bottomrule
		\end{tabularx}

		\bigskip

	How did we do?
	}

%	\begin{frame}{Chromatography is Only for Separations}
%	\begin{itemize}
%		\item How do we \alert{detect} what has come out of the column?
%		\item What makes a chemical signal turn into a \alert{digital}
%			signal?
%		\item We must \alert{interface} a chromatographic technique (GC,
%			HPLC, CE) with some type of detector.
%
%			\bigskip
%
%			\pause
%
%		\item Let's revisit the construction of a spectrophotometer.
%	\end{itemize}
%\end{frame}
%
%\begin{frame}{Anatomy of a UV/Vis Spectrometer}
%	\begin{center}
%		\includegraphics[width=0.8\linewidth]{Harr9e_fig_18_05.jpg}
%	\end{center}
%
%	\begin{itemize}
%		\only<1>{
%		\item A \alert{single-beam} spectrometer is the simplest design,
%			in which light from the \alert{source} is split into
%			component wavelengths through a \alert{monochromator}
%			before passing through the \alert{sample} and striking
%			the \alert{detector}.
%		\item We can easily replace a sample cuvette with a \alert{flow
%			cell} for continuous flow applications.}
%		\only<2>{
%		\item Disadvantages:
%			\begin{itemize}
%				\item A reference sample must continuously be
%					checked to account for deviations in
%					source intensity or instrument
%					\alert{drift}.
%				\item If multiple wavelengths are to be
%					monitored, the reference will have to be
%					checked at each wavelength.
%			\end{itemize}}
%	\end{itemize}
%\end{frame}
%
%\begin{frame}[allowframebreaks]{Anatomy of a Double-Beam Spectrometer}
%	\begin{center}
%		\includegraphics[width=\linewidth]{Harr9e_fig_20_01.jpg}
%	\end{center}
%
%	\begin{itemize}
%		\item Using \alert{choppers}, we can alternate the path that the
%			light follows.
%		\item A reference cuvet placed in the second path permits
%			continuous monitoring of changes in the light source or
%			detector response.
%	\end{itemize}
%
%	\framebreak
%
%	\begin{center}
%		\includegraphics[width=0.8\linewidth]{Harr9e_fig_20_02.jpg}
%	\end{center}
%\end{frame}
%
%\begin{frame}[allowframebreaks]{Components of Optical Instruments}{Skoog,
%	Holler, Crouch. Principles of Instrumental Analysis,
%	7\textsuperscript{th} Ed.}
%	\includegraphics[width=\linewidth]{02_ch07_Fig02a.jpg}
%
%	\includegraphics[width=\linewidth]{02_ch07_Fig02b.jpg}
%
%	\includegraphics[width=\linewidth]{03_ch07_Fig03a.jpg}
%
%	\includegraphics[width=\linewidth]{03_ch07_Fig03b.jpg}
%\end{frame}
%
%\begin{frame}{Common Light Sources}
%	\begin{columns}
%		\column{0.45\linewidth}
%		\only<1-3|handout:1>{
%			\textbf{UV-Visible-Near IR Region}
%		\begin{itemize}
%			\item<1-> \ch{H2}/\ch{D2}
%				\begin{itemize}
%					\item 160--375 nm
%					\item must use Quartz windows and
%						cuvettes
%				\end{itemize}
%			\item<2-> Xe arc lamps
%		        	\begin{itemize}
%					\item 250--600 nm, max I at 500 nm
%				\end{itemize}
%			\item<3-> W filament
%				\begin{itemize}
%					\item 320--2500 nm, needs close V
%						control
%				\end{itemize}
%		\end{itemize}}
%
%		\only<4|handout:0>{
%			\begin{center}
%				\includegraphics[width=\linewidth]{nernst-glower.png}
%			\end{center}}
%
%		\only<5|handout:0>{
%			\begin{center}
%				\includegraphics[width=\linewidth]{globar.png}
%			\end{center}}
%
%		\only<6|handout:0>{
%			\begin{center}
%				\includegraphics[width=\linewidth]{incandescent-wire.png}
%			\end{center}}
%
%
%
%		\column{0.45\linewidth}
%		
%		\only<4-6|handout:1>{
%			\textbf{IR Region}
%		\begin{itemize}
%			\item<4-> Nernst glower
%				\begin{itemize}
%					\item rare earth oxides
%				\end{itemize}
%			\item<5-> globar
%				\begin{itemize}
%					\item silicon carbide rod
%				\end{itemize}
%			\item<6-> incandescent wire
%				\begin{itemize}
%					\item nichrome wire
%				\end{itemize}
%		\end{itemize}}
%		
%		\only<1|handout:0>{
%			\begin{center}
%				\includegraphics[width=0.7\linewidth]{deuterium.png}
%			\end{center}}
%
%		\only<2|handout:0>{
%			\begin{center}
%				\includegraphics[width=\linewidth]{xe-arc.png}
%			\end{center}}
%
%		\only<3|handout:0>{
%			\begin{center}
%				\includegraphics[width=\linewidth]{w-filament.png}
%			\end{center}}
%
%	\end{columns}
%\end{frame}
%
%\begin{frame}{Deuterium and Tungsten-Filament Emission Spectra}
%	\begin{center}
%		\includegraphics[width=0.8\linewidth]{Harr9e_fig_20_04.jpg}
%	\end{center}
%\end{frame}
%
%\begin{frame}{Arc Lamp Emission Spectra}
%	\begin{center}
%		\includegraphics[width=0.8\linewidth]{arc-emission.png}
%	\end{center}
%\end{frame}
%
%\begin{frame}{Lasers!}
%	\begin{center}
%	\bfseries
%	\alert{L}ight \alert{A}mplification by \alert{S}timulated
%	\alert{E}mission of \alert{R}adiation
%
%		\bigskip
%
%		\includegraphics[width=0.8\linewidth]{04_ch07_Fig04.jpg}\footnotetext{Skoog
%		et al. Principles of Instrumental Analysis,
%		7\textsuperscript{th} Ed.}
%	\end{center}
%
%	\begin{multicols}{2}
%		\begin{itemize}
%			\item Monochromatic
%			\item High power at one wavelength
%			\item Collimated - all waves parallel
%			\item Polarized - electric field oscillates in one plane
%			\item Coherent - all waves in phase
%		\end{itemize}
%	\end{multicols}
%\end{frame}
%
%\begin{frame}
%	\includegraphics[width=\linewidth]{ruby-laser.jpg}
%\end{frame}
%
%\begin{frame}{Wavelength Selection}
%	\begin{columns}
%		\column{0.4\linewidth}
%		A \alert{monochromator} disperses light into the
%		component wavelengths and selects a narrow band
%		of wavelengths (ideally one -- mono, but this is
%		not physically possible) to pass through an exit
%		slit.
%		\column{0.5\linewidth}
%		\includegraphics[width=\linewidth]{20_ch07_Fig18.jpg}
%	\end{columns}
%\end{frame}
%
%\begin{frame}[allowframebreaks]{Czerny-Turner Grating Monochromator}
%	\begin{center}
%		\includegraphics[width=0.8\linewidth]{Harr9e_fig_20_06.jpg}
%	\end{center}
%
%	\framebreak
%
%	\begin{center}
%		\includegraphics[width=0.8\linewidth]{Harr9e_fig_20_07.jpg}
%	\end{center}
%
%	\begin{equation*}
%		\text{Grating equation:} \qquad n\lambda = d(\sin\theta +
%		\sin\phi)
%	\end{equation*}
%
%	where $n$ is the order of diffraction and $d$ is the distance between
%	grooves. For each incident angle, $\theta$, there is a series of
%	reflection angles, $\phi$, at which a given $\lambda$ will produce
%	maximum constructive interference.
%
%	\framebreak
%
%	\begin{center}
%		\includegraphics[width=0.7\linewidth,trim={0 1.5in 0 0},clip]{order-of-light.png}
%	\end{center}
%
%	\begin{equation*}
%		n\lambda = a - b
%	\end{equation*}
%
%	where $n$ is the order of diffraction and $a - b$ is the difference in
%	pathlength between grooves.
%\end{frame}
%
%\begin{frame}{Angular Dispersion of Gratings}
%	Dispersion is a ``measure'' of the angle the grating must be rotated to
%	change wavelength of the exiting light by a unit wavelength.
%
%	i.e. angular separation $\Delta \phi$ obtained for two wavelengths
%	separated by $\Delta \lambda$.
%
%	\begin{equation*}
%		\frac{\Delta\phi}{\Delta\lambda} = \frac{n}{d\cos\phi}
%	\end{equation*}
%
%	Smaller $d$ values (\#\si{\nano\meter}/blaze) of the grating generate
%	larger dispersions leading to higher \alert{resolutions}.
%\end{frame}
%
%\begin{frame}{Resolution of a Grating}
%	The ability of a grating to disperse wavelengths -- ideally the number
%	of grating groves illuminated (lines/\si{\milli\meter}).
%
%	\begin{equation*}
%		\frac{\lambda}{\Delta \lambda} = n N
%	\end{equation*}
%
%	where $N$ is the number of grooves of the grating that are illuminated.
%	The more grooves in a grating, the better the resolution between closely
%	spaced wavelengths.
%	
%	\alert{Resolution is directly related to the size of
%	the illuminated region.}
%\end{frame}
%
%\begin{frame}{Resolution is Related to Bandpass}
%%	\begin{itemize}
%%		\item Spectral slit width
%%			\begin{equation*}
%%				d\lambda_\text{eff} = WD^{-1}
%%			\end{equation*}
%%			where $W$ is physical slit width and $D^{-1}$ is the
%%			reciprocal linear dispersion of the grating.
%%		\item If $D^{-1}$ is \SI{1.2}{\nano\meter\per\milli\meter}, what
%%			slit width is needed to resolve
%%			\SIlist{589.0;589.6}{\nano\meter}?
%%	\end{itemize}
%	\begin{columns}
%		\column{0.4\linewidth}
%		\begin{itemize}
%			\item We want to keep the slit width large to allow
%				increased light to reach the detector and
%				improve \alert{signal to noise}.
%			\item The slit width needs to be minimized to improve
%				resolution.
%			\item Choosing a monochromator bandwidth that is
%				$\frac{1}{5}$ as wide as the absorption peak is
%				generally OK.
%		\end{itemize}
%		\column{0.5\linewidth}
%	\begin{center}
%		\includegraphics[width=\linewidth]{Harr9e_fig_20_10.jpg}
%	\end{center}
%	\end{columns}
%\end{frame}
%
%\begin{frame}{Filters}
%	\begin{itemize}
%		\item Different orders of diffraction ($\frac{\lambda}{2}$,
%			$\frac{\lambda}{3}$, \ldots) can still pass through
%			monochromator.
%		\item A \alert{filter} can be added to remove this interference.
%		\item The simplest filter is colored glass that can absorb at
%			the undesired $\lambda$.
%	\end{itemize}
%
%	\begin{center}
%		\includegraphics[width=\linewidth,trim={0 0.25in 0 0},clip]{bandpass-filter.jpg}
%	\end{center}
%\end{frame}
%
%\begin{frame}[allowframebreaks]{So how do we detect the light?}
%	\begin{columns}
%		\column{0.45\linewidth}
%	\begin{itemize}
%		\item A \alert{phototube} is a vacuum tube with a
%			\alert{photoemissive cathode}. When the cathode is
%			struck by light, electrons are ejected and flow to the
%			positively charged electrode (the anode).
%		\item The response is sensitive to the wavelength of the
%			incident photon.
%	\end{itemize}
%		\column{0.45\linewidth}
%		\includegraphics[width=\linewidth]{phototube.png}
%	\end{columns}
%
%	\framebreak
%
%	\includegraphics[width=\linewidth]{Harr9e_fig_20_13.jpg}
%
%	A single beam spectrometer needs to be recalibrated for each $\lambda$
%	whereas a double beam spectrometer can automatically readjust for
%	100\%T.
%\end{frame}
%
%\begin{frame}{Photomultiplier Tubes}
%	For \emph{very} low intensity applications, a \alert{photomultiplier
%	tube} has the ability to produce $>10^6$ electrons from a single photon.
%
%	\begin{center}
%	\includegraphics[scale=0.75]{Harr9e_fig_20_14a.jpg} \quad
%	\includegraphics[scale=0.75]{Harr9e_fig_20_14b.jpg}
%	\end{center}
%\end{frame}
%
%\begin{frame}[allowframebreaks]{Photodiode Arrays}
%	\begin{itemize}
%		\item Phototubes are slow (and large).
%		\item A photodiode array (PDA) can record an entire spectrum in
%			under a second.
%		\item This is \emph{very} useful for chromatography when we need
%			to scan samples eluting from a column.
%	\end{itemize}
%
%	\begin{center}
%		\includegraphics[width=\linewidth]{Harr9e_fig_20_15ab.jpg} \quad 
%	\end{center}
%
%	\framebreak
%
%	\begin{center}
%	\includegraphics[width=0.8\linewidth]{Harr9e_fig_20_16.jpg}
%	\end{center}
%
%	A \alert{polychromator} can be used to direct component wavelengths to
%	\alert{different regions} of the PDA.
%\end{frame}
%
%\begin{frame}[allowframebreaks]{Charge Coupled Devices}
%	\begin{itemize}
%		\item Even more sensitivity can be gained by using a
%			\alert{charge coupled device} (CCD).
%		\item CCDs \alert{store} electrons in each \alert{pixel} of what
%			is most often a \alert{two-dimensional} array.
%		\item CCDs can use the columns to average multiple scans at once
%			if operated in a one-dimensional mode or they can be
%			used for two-dimensional imaging.
%	\end{itemize}
%
%	\begin{center}
%		\includegraphics[width=0.9\linewidth]{Harr9e_fig_20_17.jpg}
%	\end{center}
%
%	\framebreak
%
%	\begin{center}
%		\includegraphics[width=0.6\linewidth]{Harr9e_fig_20_22.jpg}
%	\end{center}
%
%	The Ocean Optics USB 4000 spectrometer used in lab uses a polychromator
%	and CCD detector (along with a few filters and lenses here and there).
%\end{frame}
%
%\begin{frame}{Recall the Z-Cell}
%	\begin{columns}
%		\column{0.45\linewidth}
%		\begin{itemize}
%			\item UV/Vis spectrophotometers are easily interfaced to
%				a HPLC or LC via fiber optics.
%			\item The eulate needs to travel through a known path
%				length and the absorbance can be read through
%				a PDA.
%			\item Most often, ultraviolet light is used as this can
%				detect most compounds of interest.
%		\end{itemize}
%		\column{0.45\linewidth}
%		\includegraphics[width=\linewidth]{Harr9e_fig_25_21.jpg}
%	\end{columns}
%\end{frame}
%
%\begin{frame}{But wait! How did we detect ethanol/propanol in GC?}
%	\textbf{Some Common GC Detectors:}
%	\begin{itemize}
%		\item Thermal Conductivity Detector
%			\begin{itemize}
%				\item Measures changes in thermal conductivity
%					of gas
%				\item Universal -- detects all analytes
%				\item Nondestructive -- can be used \alert{in
%					tandem} with other techniques
%				\item Low sensitivity
%			\end{itemize}
%		\item \alert{Flame Ionization Detector} (what we used in lab)
%			\begin{itemize}
%				\item Eluate is burned in \ch{H2} and air,
%					producing \ch{CHO+} (for organic
%					compounds)
%				\item Insensitive to nonhydrocarbons
%				\item More sensitive than thermal conductivity
%			\end{itemize}
%		\item Electron Capture Detector
%			\begin{itemize}
%				\item Conductivity of a gas plasma is measured
%				\item Useful for halogen-containing molecules
%					and organometallics
%				\item Insensitive to hydrocarbons
%				\item Extremely sensitive
%			\end{itemize}
%	\end{itemize}
%\end{frame}
%
%\begin{frame}{Thermal Conductivity Detector}
%	\begin{center}
%		\includegraphics[width=0.8\linewidth]{Harr9e_fig_24_19.jpg}
%	\end{center}
%\end{frame}
%
%\begin{frame}{Flame Ionization Detector}
%	\begin{center}
%		\includegraphics[width=0.5\linewidth]{Harr9e_fig_24_20.jpg}
%	\end{center}
%\end{frame}
%
%\begin{frame}{Electron Capture Detector}
%	\begin{center}
%		\includegraphics[width=0.5\linewidth]{Harr9e_fig_24_22.jpg}
%	\end{center}
%\end{frame}
%
%\begin{frame}{Mass Spectrometry}{A Standalone Technique or \alert{Tandem}
%	Detector}
%	\begin{itemize}
%		\item You may have seen in the past HPLC- or GC-MS
%		\item or perhaps a GC-MS-MS
%		\item or perhaps a GC-MS-MS-MS\ldots
%		\item \alert{Mass spectrometry} is a very sensitive detector
%			that provides both \alert{qualitative} and
%			\alert{quantitative} analysis.
%		\item MS analyzes the mass to charge ratio ($m/z$) of analyte.
%			If $z =$
%			\begin{itemize}
%				\item[1] then $m/z$ is equal to the mass of the
%					ion.
%				\item[2] then $m/z$ is half of the mass.
%			\end{itemize}
%		\item Peak area is proportional to the abundance of the specific
%			\alert{mass fragment}.
%	\end{itemize}
%\end{frame}
%
%\begin{frame}{Mass Spectrometry}{A Detector for GC}
%	\begin{center}
%		\includegraphics[width=0.8\linewidth]{Harr9e_fig_24_24.jpg}
%	\end{center}
%
%	Selected ion monitoring permits viewing a \alert{single} $m/z$ value,
%	greatly simplifying the chromatogram.
%\end{frame}
%
%\begin{frame}[allowframebreaks]{How does MS work?}
%	\begin{enumerate}
%		\item Ions are created by an \alert{ion source}:
%			\begin{itemize}
%				\item \textbf{Electron ionization} produces a
%					molecular ion, \ch{M^{+.}} and many
%					\alert{fragments}. (Hard ionization)
%				\item \textbf{Chemical ionization} produces
%					\ch{MH+} and few \alert{fragments}.
%					(Soft ionization)
%			\end{itemize}
%		\item Ions are accelerated by an electric field and separated
%			based on their mass-to-charge ratio in the \alert{mass
%			selector}.
%		\item At the \alert{electron multiplier detector}, each arriving
%			ion starts a cascade of electrons that reach the anode,
%			where current is measured.
%		\item The \alert{mass spectrum} shows detector response as a
%			function of $m/z$.
%	\end{enumerate}
%
%	\framebreak
%
%	\begin{center}
%		\includegraphics[width=0.8\linewidth]{Harr9e_fig_22_02.jpg}
%	\end{center}
%\end{frame}
%
%\begin{frame}{Electron Ionization}
%	\begin{itemize}
%		\item Electrons emitted from a hot filament are accelerated
%			through \SI{70}{\volt} before interacting with incoming
%			molecules, \ch{M}.
%			\begin{reaction*}
%				M + !(\SI{70}{eV})(\el{})
%				-> !(\text{molecular~ion})( M^{+.} )
%				+ !(\SI{\sim55}{eV})(\el{})
%				+ !(\SI{0.1}{eV})(\el{})
%			\end{reaction*}
%		\item The resulting molecular ion, \ch{M^{+.}}, can have so much
%			extra energy that it breaks into fragments.
%			\begin{reaction*}
%				M^{+.} -> A+ + B+ + C+
%			\end{reaction*}
%		\item \ch{M^{+.}} and its fragment ions are then accelerated to
%			the mass analyzer.
%		\item There might be so little \ch{M^{+.}} that its peak is
%			small or absent in the mass spectrum.
%	\end{itemize}
%\end{frame}
%
%\begin{frame}{Chemical Ionization}
%	\begin{itemize}
%		\item Produces less fragmentation than electron ionization.
%		\item The ionization source is filled with a reagent gas such as
%			methane, isobutane, or ammonia, at a pressure of about
%			\SI{1}{\milli\bar}.
%		\item Energetic electrons (100--200~eV) convert \ch{CH4} into
%			a variety of reactive products:
%			\begin{reactions*}
%				CH4 + \el{} &-> CH4^{+.} + 2 \el{} \\
%				CH4^{+.} + CH4 &-> \alert{CH5^+} + ^.CH3 \\
%				CH4^{+.} &-> CH3^+ + H^. \\
%				CH3^+ + CH4 &-> \alert{C2H5^+} + H2
%			\end{reactions*}
%		\item \alert{\ch{CH5^+}} reacts with analyte \ch{M} to give
%			\ch{MH+}, the most abundant ion in the chemical
%			ionization mass spectrum.
%			\begin{reactions*}
%				M + \alert{CH5^+} &-> MH^+ + CH4
%			\end{reactions*}
%	\end{itemize}
%\end{frame}
%
%\begin{frame}[allowframebreaks]{The Mass Spectrum}
%	\begin{center}
%		\includegraphics[width=\linewidth]{Harr9e_fig_22_04.jpg}
%	\end{center}
%
%	\begin{itemize}
%		\item The molecular ion, \ch{M^{+.}}, tells us the molecular
%			mass of the unknown.
%		\item When electron ionization causes some compounds to not
%			exhibit a molecular ion, \ch{M^{+.}}, fragments can
%			provide clues to the structure of an unknown.
%		\item The \alert{nitrogen rule}, the \alert{ratio of two
%			isotopes}, and the \alert{isotopic patterns} are useful
%			clues for deciphering the mass spectrum.
%	\end{itemize}
%\end{frame}
%
%\begin{frame}{The Nitrogen Rule}
%	\begin{itemize}
%		\item If a compound has an \alert{odd} number of nitrogen (in
%			addition to C, H, halogens, O, S, Si, and P) atoms, then
%			\ch{M^{+.}} has an odd nominal mass.
%		\item If a compound has an \alert{even} number of nitrogen
%			atoms, \ch{M^{+.}} has an even nominal mass.
%		\item A molecular ion at $m/z$ 128 can have 0 or 2 \ch{N} atoms,
%			but it cannot have 1 \ch{N} atom.
%	\end{itemize}
%\end{frame}
%
%\begin{frame}{Ratio of Two Isotopes}
%	\begin{columns}
%		\column{0.45\linewidth}
%		\begin{itemize}
%			\item Aromatic compounds usually have significant
%				intensity for \ch{M^{+.}}.
%			\item \ch{M^{+.}} is the \alert{base peak} (most
%				intense) in the spectra of benzene.
%		\end{itemize}
%		\column{0.45\linewidth}
%		\includegraphics[width=\linewidth]{Harr9e_fig_22_06.jpg}
%	\end{columns}
%	\begin{itemize}
%		\item We can predict the intensity of the \alert{M+1} peak from
%			isotopic ratios.
%			\begin{equation*}
%				\text{Intensity} = \underbrace{n \times
%				\SI{1.08}{\percent}}_{\mathclap{\text{From
%				\ch{^{13}C}}}}
%				+ \underbrace{m \times
%				\SI{0.012}{\percent}}_{\mathclap{\text{From
%				\ch{^{2}H}}}}
%			\end{equation*}
%	\end{itemize}
%\end{frame}
%
%\begin{frame}{Rings + Double Bonds}
%	If we know the composition of a molecular ion, and we want
%	to propose its structure, we can use the following
%	equation:
%	\begin{equation*}
%		\text{R} + \text{DB} = c - h/2 + n/2 + 1
%	\end{equation*}
%	where $c$ is the number of Group 14 atoms (that make 4
%	bonds), $h$ is the
%	number of H and halogen atoms (that make 1 bond), and
%	$n$ is the number of Group 15 atoms (that make 3 bonds).
%
%	\bigskip
%
%	\includegraphics[width=0.45\linewidth]{Harr9e_22UNEQ01.jpg}
%\end{frame}
%
%\note{
%	\begin{align*}
%		\text{R + DB} &= c - h/2 + n/2 + 1 \\
%		&= (14 + 1) - \frac{22 + 1 + 1}{2} + \frac{1 + 1}{2} + 1 = 5
%	\end{align*}
%
%	\(\underbrace{\ch{C_{14}Si}}_c
%	\underbrace{\ch{H_{22}ClBr}}_h
%	\underbrace{\ch{NAs}}_n\ch{O3S} \)}
%
%\begin{frame}[allowframebreaks]{Some Notes on Interpreting Mass Spectra}
%	\begin{itemize}
%		\item The molecular ion is found from the highest $m/z$ value of
%			any ``significant'' peak that cannot be attributed to
%			isotopes or background signals.
%		\item Intensities of the isotopic peaks must be consistent with
%			the proposed formula.
%		\item The peak for the heaviest \alert{fragment ion} should not
%			correspond to an improbable mass loss from \ch{M^{+.}}.
%			Common mass losses:
%		
%			\begin{center}
%			\begin{tabular} {c c}
%				\ch{CH3} & 15 \\
%				\ch{OH} or \ch{NH3} & 17 \\
%				\ch{H2O} & 18 \\
%				\ch{C2H5} & 29 \\
%				\ch{OCH3} & 31
%			\end{tabular}
%			\end{center}
%
%		\item If a fragment ion contains $x$ atoms of element \ch{X},
%			then there \emph{must} be at least $x$ atoms of \ch{X}
%			in the molecular ion.
%	\end{itemize}
%
%	\framebreak
%
%	\begin{center}
%		\includegraphics[width=\linewidth]{Harr9e_fig_22_12.jpg}
%	\end{center}
%\end{frame}
%
%\begin{frame}[allowframebreaks]{Let's try a few!}{Identify Major Peaks}
%	{\centering
%		\includegraphics[width=0.9\linewidth]{ethylbenzene.pdf}
%		\includegraphics[width=0.9\linewidth]{1-pentanol.pdf}
%		\includegraphics[width=0.9\linewidth]{methylene-chloride.pdf}
%		\includegraphics[width=0.9\linewidth]{1-decanol.pdf}
%		\par}
%\end{frame}
%
%\note{
%	\begin{tabularx}{\linewidth} {l c X}
%		\bfseries Compound & \bfseries MW & \bfseries Peaks \\ \midrule
%		\chemfig{[,0.4]*6(=-=(-CH_2CH_3)-=-)} & 106 & Base at 91 (lost methyl),
%		molecular ion at 106 \\
%		\ch{CH3(CH2)8CH2OH} & 158 & Base at 41 (\ch{C3H5+}), \ch{C8H16+}
%		at 112, \ch{M+} at 158, difference of 14 between many \\
%		\ch{CH2Cl2} & 84 & Base peak at 49 (\ch{CH2Cl+}), \ch{M+} at 84
%		\\
%		\ch{CH3(CH2)4OH} & 88 & Base at 42 (M - \ch{H2O} and
%		\ch{CH2=CH2}, 31 (\ch{CH2=OH+}), 55 (M - \ch{H2O} and
%		\ch{CH3}), 70 (M - \ch{H2O})
%	\end{tabularx}
%	}
%
%\begin{frame}{Types of Mass Spectrometers}
%	\begin{itemize}
%		\item Magnetic sector (what we saw when MS was first introduced)
%		\item Transmission quadrupole
%		\item Time-of-flight (TOF)
%		\item Quadrupole ion trap (QIT)
%		\item Linear ion trap
%		\item Orbitrap
%	\end{itemize}
%\end{frame}
%
%\begin{frame}{Double-Focusing Mass Spectrometer}{Magnetic Sector}
%	Higher resolution can be be attained by using an electric field with the
%	magnetic sector so that ions with a narrow range of kinetic energy are
%	selected.
%
%	\begin{center}
%		\includegraphics[width=0.5\linewidth]{Harr9e_fig_22_13.jpg}
%	\end{center}
%\end{frame}
%
%\begin{frame}{Transmission Quadrupole}{Cheap and good for GC!}
%	\begin{center}
%		\includegraphics[width=\linewidth]{Harr9e_fig_22_14.jpg}
%	\end{center}
%\end{frame}
%
%\begin{frame}{Time-of-Flight}{Almost like a prism\ldots}
%	\begin{center}
%		\includegraphics[width=\linewidth]{Harr9e_fig_22_16.jpg}
%	\end{center}
%\end{frame}
%
%\begin{frame}{Quadrupole Ion Trap}{Holds ions until ready to analyze}
%	\begin{center}
%		\includegraphics[width=0.5\linewidth]{Harr9e_fig_22_18.jpg}
%	\end{center}
%\end{frame}
%
%\begin{frame}{Linear Quadrupole Ion Trap}{Holds more than the QIT}
%	\begin{center}
%		\includegraphics[width=0.45\linewidth]{Harr9e_fig_22_19a.jpg}
%		\quad
%		\includegraphics[width=0.45\linewidth]{Harr9e_fig_22_19b.jpg}
%	\end{center}
%\end{frame}
%
%\begin{frame}{Orbitrap}{No magnetic field or oscillating electric field
%	necessary!}
%	\begin{center}
%		\includegraphics[width=0.9\linewidth]{Harr9e_fig_22_20.jpg}
%	\end{center}
%\end{frame}
%
%\begin{frame}{A final note on sample introduction\ldots}
%	\begin{itemize}
%		\item One interesting application for mass spectrometry is the
%			sample-prep free introduction.
%		\item \alert{Desorption electrospray ionization (DESI)} bombards
%			a sample with charged droplets to force ions from a
%			surface into the spectrometer.
%	\end{itemize}
%
%	\begin{center}
%		\includegraphics[width=0.8\linewidth]{Harr9e_fig_22_39.jpg}
%	\end{center}
%\end{frame}
%
%\begin{frame}{Interested in Mass Spec?}
%	\begin{itemize}
%		\item Planning on going to graduate school?
%		\item Take a look at \alert{Graham Cooks} at \alert{Purdue}.
%	\end{itemize}
%
%	\begin{center}
%		\includegraphics[width=0.9\linewidth]{miniature1.jpg}
%	\end{center}
%
%	\footnotetext{https://aston.chem.purdue.edu}
%\end{frame}
%
%	
%\
\end{document}
