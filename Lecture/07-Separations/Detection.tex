\documentclass[handout]{beamer}

\usepackage{lecture}
\usepackage{multicol}

\title{Methods of Detection}
\subtitle{Primarily Chapter 20, but also some 22, 24, and 25}
\institute{Bloomsburg University}
\author{CHEM321 - Analytical Chemistry I}
\date{December 4, 2018}

\begin{document}

\begin{frame}
	\titlepage
\end{frame}

\begin{frame}{Chromatography is Only for Separations}
	\begin{itemize}
		\item How do we \alert{detect} what has come out of the column?
		\item What makes a chemical signal turn into a \alert{digital}
			signal?
		\item We must \alert{interface} a chromatographic technique (GC,
			HPLC, CE) with some type of detector.

			\bigskip

			\pause

		\item Let's revisit the construction of a spectrophotometer.
	\end{itemize}
\end{frame}

\begin{frame}{Anatomy of a UV/Vis Spectrometer}
	\begin{center}
		\includegraphics[width=0.8\linewidth]{Harr9e_fig_18_05.jpg}
	\end{center}

	\begin{itemize}
		\only<1>{
		\item A \alert{single-beam} spectrometer is the simplest design,
			in which light from the \alert{source} is split into
			component wavelengths through a \alert{monochromator}
			before passing through the \alert{sample} and striking
			the \alert{detector}.
		\item We can easily replace a sample cuvette with a \alert{flow
			cell} for continuous flow applications.}
		\only<2>{
		\item Disadvantages:
			\begin{itemize}
				\item A reference sample must continuously be
					checked to account for deviations in
					source intensity or instrument
					\alert{drift}.
				\item If multiple wavelengths are to be
					monitored, the reference will have to be
					checked at each wavelength.
			\end{itemize}}
	\end{itemize}
\end{frame}

\begin{frame}[allowframebreaks]{Anatomy of a Double-Beam Spectrometer}
	\begin{center}
		\includegraphics[width=\linewidth]{Harr9e_fig_20_01.jpg}
	\end{center}

	\begin{itemize}
		\item Using \alert{choppers}, we can alternate the path that the
			light follows.
		\item A reference cuvet placed in the second path permits
			continuous monitoring of changes in the light source or
			detector response.
	\end{itemize}

	\framebreak

	\begin{center}
		\includegraphics[width=0.8\linewidth]{Harr9e_fig_20_02.jpg}
	\end{center}
\end{frame}

\begin{frame}[allowframebreaks]{Components of Optical Instruments}{Skoog,
	Holler, Crouch. Principles of Instrumental Analysis,
	7\textsuperscript{th} Ed.}
	\includegraphics[width=\linewidth]{02_ch07_Fig02a.jpg}

	\includegraphics[width=\linewidth]{02_ch07_Fig02b.jpg}

	\includegraphics[width=\linewidth]{03_ch07_Fig03a.jpg}

	\includegraphics[width=\linewidth]{03_ch07_Fig03b.jpg}
\end{frame}

\begin{frame}{Common Light Sources}
	\begin{columns}
		\column{0.45\linewidth}
		\only<1-3|handout:1>{
			\textbf{UV-Visible-Near IR Region}
		\begin{itemize}
			\item<1-> \ch{H2}/\ch{D2}
				\begin{itemize}
					\item 160--375 nm
					\item must use Quartz windows and
						cuvettes
				\end{itemize}
			\item<2-> Xe arc lamps
		        	\begin{itemize}
					\item 250--600 nm, max I at 500 nm
				\end{itemize}
			\item<3-> W filament
				\begin{itemize}
					\item 320--2500 nm, needs close V
						control
				\end{itemize}
		\end{itemize}}

		\only<4|handout:0>{
			\begin{center}
				\includegraphics[width=\linewidth]{nernst-glower.png}
			\end{center}}

		\only<5|handout:0>{
			\begin{center}
				\includegraphics[width=\linewidth]{globar.png}
			\end{center}}

		\only<6|handout:0>{
			\begin{center}
				\includegraphics[width=\linewidth]{incandescent-wire.png}
			\end{center}}



		\column{0.45\linewidth}
		
		\only<4-6|handout:1>{
			\textbf{IR Region}
		\begin{itemize}
			\item<4-> Nernst glower
				\begin{itemize}
					\item rare earth oxides
				\end{itemize}
			\item<5-> globar
				\begin{itemize}
					\item silicon carbide rod
				\end{itemize}
			\item<6-> incandescent wire
				\begin{itemize}
					\item nichrome wire
				\end{itemize}
		\end{itemize}}
		
		\only<1|handout:0>{
			\begin{center}
				\includegraphics[width=0.7\linewidth]{deuterium.png}
			\end{center}}

		\only<2|handout:0>{
			\begin{center}
				\includegraphics[width=\linewidth]{xe-arc.png}
			\end{center}}

		\only<3|handout:0>{
			\begin{center}
				\includegraphics[width=\linewidth]{w-filament.png}
			\end{center}}

	\end{columns}
\end{frame}

\begin{frame}{Deuterium and Tungsten-Filament Emission Spectra}
	\begin{center}
		\includegraphics[width=0.8\linewidth]{Harr9e_fig_20_04.jpg}
	\end{center}
\end{frame}

\begin{frame}{Arc Lamp Emission Spectra}
	\begin{center}
		\includegraphics[width=0.8\linewidth]{arc-emission.png}
	\end{center}
\end{frame}

\begin{frame}{Lasers!}
	\begin{center}
	\bfseries
	\alert{L}ight \alert{A}mplification by \alert{S}timulated
	\alert{E}mission of \alert{R}adiation

		\bigskip

		\includegraphics[width=0.8\linewidth]{04_ch07_Fig04.jpg}\footnotetext{Skoog
		et al. Principles of Instrumental Analysis,
		7\textsuperscript{th} Ed.}
	\end{center}

	\begin{multicols}{2}
		\begin{itemize}
			\item Monochromatic
			\item High power at one wavelength
			\item Collimated - all waves parallel
			\item Polarized - electric field oscillates in one plane
			\item Coherent - all waves in phase
		\end{itemize}
	\end{multicols}
\end{frame}

\begin{frame}
	\includegraphics[width=\linewidth]{ruby-laser.jpg}
\end{frame}

\begin{frame}{Wavelength Selection}
	\begin{columns}
		\column{0.4\linewidth}
		A \alert{monochromator} disperses light into the
		component wavelengths and selects a narrow band
		of wavelengths (ideally one -- mono, but this is
		not physically possible) to pass through an exit
		slit.
		\column{0.5\linewidth}
		\includegraphics[width=\linewidth]{20_ch07_Fig18.jpg}
	\end{columns}
\end{frame}

\begin{frame}[allowframebreaks]{Czerny-Turner Grating Monochromator}
	\begin{center}
		\includegraphics[width=0.8\linewidth]{Harr9e_fig_20_06.jpg}
	\end{center}

	\framebreak

	\begin{center}
		\includegraphics[width=0.8\linewidth]{Harr9e_fig_20_07.jpg}
	\end{center}

	\begin{equation*}
		\text{Grating equation:} \qquad n\lambda = d(\sin\theta +
		\sin\phi)
	\end{equation*}

	where $n$ is the order of diffraction and $d$ is the distance between
	grooves. For each incident angle, $\theta$, there is a series of
	reflection angles, $\phi$, at which a given $\lambda$ will produce
	maximum constructive interference.

	\framebreak

	\begin{center}
		\includegraphics[width=0.7\linewidth,trim={0 1.5in 0 0},clip]{order-of-light.png}
	\end{center}

	\begin{equation*}
		n\lambda = a - b
	\end{equation*}

	where $n$ is the order of diffraction and $a - b$ is the difference in
	pathlength between grooves.
\end{frame}

\begin{frame}{Angular Dispersion of Gratings}
	Dispersion is a ``measure'' of the angle the grating must be rotated to
	change wavelength of the exiting light by a unit wavelength.

	i.e. angular separation $\Delta \phi$ obtained for two wavelengths
	separated by $\Delta \lambda$.

	\begin{equation*}
		\frac{\Delta\phi}{\Delta\lambda} = \frac{n}{d\cos\phi}
	\end{equation*}

	Smaller $d$ values (\#\si{\nano\meter}/blaze) of the grating generate
	larger dispersions leading to higher \alert{resolutions}.
\end{frame}

\begin{frame}{Resolution of a Grating}
	The ability of a grating to disperse wavelengths -- ideally the number
	of grating groves illuminated (lines/\si{\milli\meter}).

	\begin{equation*}
		\frac{\lambda}{\Delta \lambda} = n N
	\end{equation*}

	where $N$ is the number of grooves of the grating that are illuminated.
	The more grooves in a grating, the better the resolution between closely
	spaced wavelengths.
	
	\alert{Resolution is directly related to the size of
	the illuminated region.}
\end{frame}

\begin{frame}{Resolution is Related to Bandpass}
%	\begin{itemize}
%		\item Spectral slit width
%			\begin{equation*}
%				d\lambda_\text{eff} = WD^{-1}
%			\end{equation*}
%			where $W$ is physical slit width and $D^{-1}$ is the
%			reciprocal linear dispersion of the grating.
%		\item If $D^{-1}$ is \SI{1.2}{\nano\meter\per\milli\meter}, what
%			slit width is needed to resolve
%			\SIlist{589.0;589.6}{\nano\meter}?
%	\end{itemize}
	\begin{columns}
		\column{0.4\linewidth}
		\begin{itemize}
			\item We want to keep the slit width large to allow
				increased light to reach the detector and
				improve \alert{signal to noise}.
			\item The slit width needs to be minimized to improve
				resolution.
			\item Choosing a monochromator bandwidth that is
				$\frac{1}{5}$ as wide as the absorption peak is
				generally OK.
		\end{itemize}
		\column{0.5\linewidth}
	\begin{center}
		\includegraphics[width=\linewidth]{Harr9e_fig_20_10.jpg}
	\end{center}
	\end{columns}
\end{frame}

\begin{frame}{Filters}
	\begin{itemize}
		\item Different orders of diffraction ($\frac{\lambda}{2}$,
			$\frac{\lambda}{3}$, \ldots) can still pass through
			monochromator.
		\item A \alert{filter} can be added to remove this interference.
		\item The simplest filter is colored glass that can absorb at
			the undesired $\lambda$.
	\end{itemize}

	\begin{center}
		\includegraphics[width=\linewidth,trim={0 0.25in 0 0},clip]{bandpass-filter.jpg}
	\end{center}
\end{frame}

\begin{frame}[allowframebreaks]{So how do we detect the light?}
	\begin{columns}
		\column{0.45\linewidth}
	\begin{itemize}
		\item A \alert{phototube} is a vacuum tube with a
			\alert{photoemissive cathode}. When the cathode is
			struck by light, electrons are ejected and flow to the
			positively charged electrode (the anode).
		\item The response is sensitive to the wavelength of the
			incident photon.
	\end{itemize}
		\column{0.45\linewidth}
		\includegraphics[width=\linewidth]{phototube.png}
	\end{columns}

	\framebreak

	\includegraphics[width=\linewidth]{Harr9e_fig_20_13.jpg}

	A single beam spectrometer needs to be recalibrated for each $\lambda$
	whereas a double beam spectrometer can automatically readjust for
	100\%T.
\end{frame}

\begin{frame}{Photomultiplier Tubes}
	For \emph{very} low intensity applications, a \alert{photomultiplier
	tube} has the ability to produce $>10^6$ electrons from a single photon.

	\begin{center}
	\includegraphics[scale=0.75]{Harr9e_fig_20_14a.jpg} \quad
	\includegraphics[scale=0.75]{Harr9e_fig_20_14b.jpg}
	\end{center}
\end{frame}

\begin{frame}[allowframebreaks]{Photodiode Arrays}
	\begin{itemize}
		\item Phototubes are slow (and large).
		\item A photodiode array (PDA) can record an entire spectrum in
			under a second.
		\item This is \emph{very} useful for chromatography when we need
			to scan samples eluting from a column.
	\end{itemize}

	\begin{center}
		\includegraphics[width=\linewidth]{Harr9e_fig_20_15ab.jpg} \quad 
	\end{center}

	\framebreak

	\begin{center}
	\includegraphics[width=0.8\linewidth]{Harr9e_fig_20_16.jpg}
	\end{center}

	A \alert{polychromator} can be used to direct component wavelengths to
	\alert{different regions} of the PDA.
\end{frame}

\begin{frame}[allowframebreaks]{Charge Coupled Devices}
	\begin{itemize}
		\item Even more sensitivity can be gained by using a
			\alert{charge coupled device} (CCD).
		\item CCDs \alert{store} electrons in each \alert{pixel} of what
			is most often a \alert{two-dimensional} array.
		\item CCDs can use the columns to average multiple scans at once
			if operated in a one-dimensional mode or they can be
			used for two-dimensional imaging.
	\end{itemize}

	\begin{center}
		\includegraphics[width=0.9\linewidth]{Harr9e_fig_20_17.jpg}
	\end{center}

	\framebreak

	\begin{center}
		\includegraphics[width=0.6\linewidth]{Harr9e_fig_20_22.jpg}
	\end{center}

	The Ocean Optics USB 4000 spectrometer used in lab uses a polychromator
	and CCD detector (along with a few filters and lenses here and there).
\end{frame}

\begin{frame}{Recall the Z-Cell}
	\begin{columns}
		\column{0.45\linewidth}
		\begin{itemize}
			\item UV/Vis spectrophotometers are easily interfaced to
				a HPLC or LC via fiber optics.
			\item The eulate needs to travel through a known path
				length and the absorbance can be read through
				a PDA.
			\item Most often, ultraviolet light is used as this can
				detect most compounds of interest.
		\end{itemize}
		\column{0.45\linewidth}
		\includegraphics[width=\linewidth]{Harr9e_fig_25_21.jpg}
	\end{columns}
\end{frame}

\begin{frame}{But wait! How did we detect ethanol/propanol in GC?}
	\textbf{Some Common GC Detectors:}
	\begin{itemize}
		\item Thermal Conductivity Detector
			\begin{itemize}
				\item Measures changes in thermal conductivity
					of gas
				\item Universal -- detects all analytes
				\item Nondestructive -- can be used \alert{in
					tandem} with other techniques
				\item Low sensitivity
			\end{itemize}
		\item \alert{Flame Ionization Detector} (what we used in lab)
			\begin{itemize}
				\item Eluate is burned in \ch{H2} and air,
					producing \ch{CHO+} (for organic
					compounds)
				\item Insensitive to nonhydrocarbons
				\item More sensitive than thermal conductivity
			\end{itemize}
		\item Electron Capture Detector
			\begin{itemize}
				\item Conductivity of a gas plasma is measured
				\item Useful for halogen-containing molecules
					and organometallics
				\item Insensitive to hydrocarbons
				\item Extremely sensitive
			\end{itemize}
	\end{itemize}
\end{frame}

\begin{frame}{Thermal Conductivity Detector}
	\begin{center}
		\includegraphics[width=0.8\linewidth]{Harr9e_fig_24_19.jpg}
	\end{center}
\end{frame}

\begin{frame}{Flame Ionization Detector}
	\begin{center}
		\includegraphics[width=0.5\linewidth]{Harr9e_fig_24_20.jpg}
	\end{center}
\end{frame}

\begin{frame}{Electron Capture Detector}
	\begin{center}
		\includegraphics[width=0.5\linewidth]{Harr9e_fig_24_22.jpg}
	\end{center}
\end{frame}

\begin{frame}{Mass Spectrometry}{A Standalone Technique or \alert{Tandem}
	Detector}
	\begin{itemize}
		\item You may have seen in the past HPLC- or GC-MS
		\item or perhaps a GC-MS-MS
		\item or perhaps a GC-MS-MS-MS\ldots
		\item \alert{Mass spectrometry} is a very sensitive detector
			that provides both \alert{qualitative} and
			\alert{quantitative} analysis.
		\item MS analyzes the mass to charge ratio ($m/z$) of analyte.
			If $z =$
			\begin{itemize}
				\item[1] then $m/z$ is equal to the mass of the
					ion.
				\item[2] then $m/z$ is half of the mass.
			\end{itemize}
		\item Peak area is proportional to the abundance of the specific
			\alert{mass fragment}.
	\end{itemize}
\end{frame}

\begin{frame}{Mass Spectrometry}{A Detector for GC}
	\begin{center}
		\includegraphics[width=0.8\linewidth]{Harr9e_fig_24_24.jpg}
	\end{center}

	Selected ion monitoring permits viewing a \alert{single} $m/z$ value,
	greatly simplifying the chromatogram.
\end{frame}

\begin{frame}[allowframebreaks]{How does MS work?}
	\begin{enumerate}
		\item Ions are created by an \alert{ion source}:
			\begin{itemize}
				\item \textbf{Electron ionization} produces a
					molecular ion, \ch{M^{+.}} and many
					\alert{fragments}. (Hard ionization)
				\item \textbf{Chemical ionization} produces
					\ch{MH+} and few \alert{fragments}.
					(Soft ionization)
			\end{itemize}
		\item Ions are accelerated by an electric field and separated
			based on their mass-to-charge ratio in the \alert{mass
			selector}.
		\item At the \alert{electron multiplier detector}, each arriving
			ion starts a cascade of electrons that reach the anode,
			where current is measured.
		\item The \alert{mass spectrum} shows detector response as a
			function of $m/z$.
	\end{enumerate}

	\framebreak

	\begin{center}
		\includegraphics[width=0.8\linewidth]{Harr9e_fig_22_02.jpg}
	\end{center}
\end{frame}

\begin{frame}{Electron Ionization}
	\begin{itemize}
		\item Electrons emitted from a hot filament are accelerated
			through \SI{70}{\volt} before interacting with incoming
			molecules, \ch{M}.
			\begin{reaction*}
				M + !(\SI{70}{eV})(\el{})
				-> !(\text{molecular~ion})( M^{+.} )
				+ !(\SI{\sim55}{eV})(\el{})
				+ !(\SI{0.1}{eV})(\el{})
			\end{reaction*}
		\item The resulting molecular ion, \ch{M^{+.}}, can have so much
			extra energy that it breaks into fragments.
			\begin{reaction*}
				M^{+.} -> A+ + B+ + C+
			\end{reaction*}
		\item \ch{M^{+.}} and its fragment ions are then accelerated to
			the mass analyzer.
		\item There might be so little \ch{M^{+.}} that its peak is
			small or absent in the mass spectrum.
	\end{itemize}
\end{frame}

\begin{frame}{Chemical Ionization}
	\begin{itemize}
		\item Produces less fragmentation than electron ionization.
		\item The ionization source is filled with a reagent gas such as
			methane, isobutane, or ammonia, at a pressure of about
			\SI{1}{\milli\bar}.
		\item Energetic electrons (100--200~eV) convert \ch{CH4} into
			a variety of reactive products:
			\begin{reactions*}
				CH4 + \el{} &-> CH4^{+.} + 2 \el{} \\
				CH4^{+.} + CH4 &-> \alert{CH5^+} + ^.CH3 \\
				CH4^{+.} &-> CH3^+ + H^. \\
				CH3^+ + CH4 &-> \alert{C2H5^+} + H2
			\end{reactions*}
		\item \alert{\ch{CH5^+}} reacts with analyte \ch{M} to give
			\ch{MH+}, the most abundant ion in the chemical
			ionization mass spectrum.
			\begin{reactions*}
				M + \alert{CH5^+} &-> MH^+ + CH4
			\end{reactions*}
	\end{itemize}
\end{frame}

\begin{frame}[allowframebreaks]{The Mass Spectrum}
	\begin{center}
		\includegraphics[width=\linewidth]{Harr9e_fig_22_04.jpg}
	\end{center}

	\begin{itemize}
		\item The molecular ion, \ch{M^{+.}}, tells us the molecular
			mass of the unknown.
		\item When electron ionization causes some compounds to not
			exhibit a molecular ion, \ch{M^{+.}}, fragments can
			provide clues to the structure of an unknown.
		\item The \alert{nitrogen rule}, the \alert{ratio of two
			isotopes}, and the \alert{isotopic patterns} are useful
			clues for deciphering the mass spectrum.
	\end{itemize}
\end{frame}

\begin{frame}{The Nitrogen Rule}
	\begin{itemize}
		\item If a compound has an \alert{odd} number of nitrogen (in
			addition to C, H, halogens, O, S, Si, and P) atoms, then
			\ch{M^{+.}} has an odd nominal mass.
		\item If a compound has an \alert{even} number of nitrogen
			atoms, \ch{M^{+.}} has an even nominal mass.
		\item A molecular ion at $m/z$ 128 can have 0 or 2 \ch{N} atoms,
			but it cannot have 1 \ch{N} atom.
	\end{itemize}
\end{frame}

\begin{frame}{Ratio of Two Isotopes}
	\begin{columns}
		\column{0.45\linewidth}
		\begin{itemize}
			\item Aromatic compounds usually have significant
				intensity for \ch{M^{+.}}.
			\item \ch{M^{+.}} is the \alert{base peak} (most
				intense) in the spectra of benzene.
		\end{itemize}
		\column{0.45\linewidth}
		\includegraphics[width=\linewidth]{Harr9e_fig_22_06.jpg}
	\end{columns}
	\begin{itemize}
		\item We can predict the intensity of the \alert{M+1} peak from
			isotopic ratios.
			\begin{equation*}
				\text{Intensity} = \underbrace{n \times
				\SI{1.08}{\percent}}_{\mathclap{\text{From
				\ch{^{13}C}}}}
				+ \underbrace{m \times
				\SI{0.012}{\percent}}_{\mathclap{\text{From
				\ch{^{2}H}}}}
			\end{equation*}
	\end{itemize}
\end{frame}

\begin{frame}{Rings + Double Bonds}
	If we know the composition of a molecular ion, and we want
	to propose its structure, we can use the following
	equation:
	\begin{equation*}
		\text{R} + \text{DB} = c - h/2 + n/2 + 1
	\end{equation*}
	where $c$ is the number of Group 14 atoms (that make 4
	bonds), $h$ is the
	number of H and halogen atoms (that make 1 bond), and
	$n$ is the number of Group 15 atoms (that make 3 bonds).

	\bigskip

	\includegraphics[width=0.45\linewidth]{Harr9e_22UNEQ01.jpg}
\end{frame}

\note{
	\begin{align*}
		\text{R + DB} &= c - h/2 + n/2 + 1 \\
		&= (14 + 1) - \frac{22 + 1 + 1}{2} + \frac{1 + 1}{2} + 1 = 5
	\end{align*}

	\(\underbrace{\ch{C_{14}Si}}_c
	\underbrace{\ch{H_{22}ClBr}}_h
	\underbrace{\ch{NAs}}_n\ch{O3S} \)}

\begin{frame}[allowframebreaks]{Some Notes on Interpreting Mass Spectra}
	\begin{itemize}
		\item The molecular ion is found from the highest $m/z$ value of
			any ``significant'' peak that cannot be attributed to
			isotopes or background signals.
		\item Intensities of the isotopic peaks must be consistent with
			the proposed formula.
		\item The peak for the heaviest \alert{fragment ion} should not
			correspond to an improbable mass loss from \ch{M^{+.}}.
			Common mass losses:
		
			\begin{center}
			\begin{tabu} to \linewidth {c c}
				\ch{CH3} & 15 \\
				\ch{OH} or \ch{NH3} & 17 \\
				\ch{H2O} & 18 \\
				\ch{C2H5} & 29 \\
				\ch{OCH3} & 31
			\end{tabu}
			\end{center}

		\item If a fragment ion contains $x$ atoms of element \ch{X},
			then there \emph{must} be at least $x$ atoms of \ch{X}
			in the molecular ion.
	\end{itemize}

	\framebreak

	\begin{center}
		\includegraphics[width=\linewidth]{Harr9e_fig_22_12.jpg}
	\end{center}
\end{frame}

\begin{frame}[allowframebreaks]{Let's try a few!}{Identify Major Peaks}
	{\centering
		\includegraphics[width=0.9\linewidth]{ethylbenzene.pdf}
		\includegraphics[width=0.9\linewidth]{1-pentanol.pdf}
		\includegraphics[width=0.9\linewidth]{methylene-chloride.pdf}
		\includegraphics[width=0.9\linewidth]{1-decanol.pdf}
		\par}
\end{frame}

\note{
	\extrarowsep=^1.5em_0.25em
	\begin{tabu} to \linewidth {l c X}
		\rowfont{\bfseries} Compound & MW & Peaks \\ \tabucline-
		\chemfig{[,0.4]*6(=-=(-CH_2CH_3)-=-)} & 106 & Base at 91 (lost methyl),
		molecular ion at 106 \\
		\ch{CH3(CH2)8CH2OH} & 158 & Base at 41 (\ch{C3H5+}), \ch{C8H16+}
		at 112, \ch{M+} at 158, difference of 14 between many \\
		\ch{CH2Cl2} & 84 & Base peak at 49 (\ch{CH2Cl+}), \ch{M+} at 84
		\\
		\ch{CH3(CH2)4OH} & 88 & Base at 42 (M - \ch{H2O} and
		\ch{CH2=CH2}, 31 (\ch{CH2=OH+}), 55 (M - \ch{H2O} and
		\ch{CH3}), 70 (M - \ch{H2O})
	\end{tabu}
	}

\begin{frame}{Types of Mass Spectrometers}
	\begin{itemize}
		\item Magnetic sector (what we saw when MS was first introduced)
		\item Transmission quadrupole
		\item Time-of-flight (TOF)
		\item Quadrupole ion trap (QIT)
		\item Linear ion trap
		\item Orbitrap
	\end{itemize}
\end{frame}

\begin{frame}{Double-Focusing Mass Spectrometer}{Magnetic Sector}
	Higher resolution can be be attained by using an electric field with the
	magnetic sector so that ions with a narrow range of kinetic energy are
	selected.

	\begin{center}
		\includegraphics[width=0.5\linewidth]{Harr9e_fig_22_13.jpg}
	\end{center}
\end{frame}

\begin{frame}{Transmission Quadrupole}{Cheap and good for GC!}
	\begin{center}
		\includegraphics[width=\linewidth]{Harr9e_fig_22_14.jpg}
	\end{center}
\end{frame}

\begin{frame}{Time-of-Flight}{Almost like a prism\ldots}
	\begin{center}
		\includegraphics[width=\linewidth]{Harr9e_fig_22_16.jpg}
	\end{center}
\end{frame}

\begin{frame}{Quadrupole Ion Trap}{Holds ions until ready to analyze}
	\begin{center}
		\includegraphics[width=0.5\linewidth]{Harr9e_fig_22_18.jpg}
	\end{center}
\end{frame}

\begin{frame}{Linear Quadrupole Ion Trap}{Holds more than the QIT}
	\begin{center}
		\includegraphics[width=0.45\linewidth]{Harr9e_fig_22_19a.jpg}
		\quad
		\includegraphics[width=0.45\linewidth]{Harr9e_fig_22_19b.jpg}
	\end{center}
\end{frame}

\begin{frame}{Orbitrap}{No magnetic field or oscillating electric field
	necessary!}
	\begin{center}
		\includegraphics[width=0.9\linewidth]{Harr9e_fig_22_20.jpg}
	\end{center}
\end{frame}

\begin{frame}{A final note on sample introduction\ldots}
	\begin{itemize}
		\item One interesting application for mass spectrometry is the
			sample-prep free introduction.
		\item \alert{Desorption electrospray ionization (DESI)} bombards
			a sample with charged droplets to force ions from a
			surface into the spectrometer.
	\end{itemize}

	\begin{center}
		\includegraphics[width=0.8\linewidth]{Harr9e_fig_22_39.jpg}
	\end{center}
\end{frame}

\begin{frame}{Interested in Mass Spec?}
	\begin{itemize}
		\item Planning on going to graduate school?
		\item Take a look at \alert{Graham Cooks} at \alert{Purdue}.
	\end{itemize}

	\begin{center}
		\includegraphics[width=0.9\linewidth]{miniature1.jpg}
	\end{center}

	\footnotetext{https://aston.chem.purdue.edu}
\end{frame}

	
\end{document}
