% !TEX program = xelatex
\documentclass[notes=hide]{beamer}
%\documentclass[notes=only]{beamer}
%\documentclass[notes=onlyslideswithnotes]{beamer}
%\documentclass[letterpaper,11pt]{article}
%\usepackage{beamerarticle}

\usepackage{analchem}
\usepackage{lecture}

\title{Chemical Measurements}
\subtitle{Chapter 1}
\author{D.A. McCurry}
\institute{Department of Chemistry and Biochemistry \\ Bloomsburg University}
\date{August 29, 2018}

\begin{document}

\maketitle
\mode<article>{\thispagestyle{fancy}}

\mode<presentation>{\begin{frame}{There is a big shift towards miniaturatization!}
	\centering
	
	\includegraphics[width=0.9\linewidth]{miniaturization.jpeg}

	\footnotetext{Yang, Y. et al. \textit{Anal. Chem.} \textbf{2017,} 89
	(1), 71-91}
\end{frame}}


\section{Units}

\begin{frame}{Measurements}
	\begin{itemize}
		\item Units are \emph{very} important in Analytical Chemistry!

		\item Common units and terms:

	\begin{center}
	\begin{tabular} {l l c c c}
		length: & meter & \si{\meter} \\
		mass: & kilogram & \si{\kilo\gram} \\
		time: & second & \si{\second} \\
		electric current: & ampere & \si{\ampere} \\
		temperature: & kelvin & \si{\kelvin} \\
		pressure: & pascal & \si{\pascal} & \emph{or} &
			\si{\newton\per\meter\squared} \\
	\end{tabular}
	\end{center}

	\begin{block}{Syst\`{e}me International d'Unit\'{e}s (SI units)}
		see Tables 1-1, 1-2; page 11 in Harris
	\end{block}
	\end{itemize}
\end{frame}

\begin{frame}[allowframebreaks]{Improper units can be catastrophic!}

	\mode<presentation>{
		\includegraphics[width=\linewidth]{Gimli_glider.JPG}
	
	\footnotesize By Source, Fair use,
	https://en.wikipedia.org/w/index.php?curid=18952968
}

	\framebreak

	\includegraphics[width=\linewidth]{MarsOrbiter.pdf}
\end{frame}

\begin{frame}{Concentrations}
	The amount of \emph{solute} dissolved in a specific amount of
	\emph{solution} or \emph{solvent}.
	
	\begin{align*}
		\intertext{Some common examples you have probably seen before:}
		\text{molarity (\si{\Molar})} &=
		\dfrac{\text{\si{\mole}~solute}} {\text{\si{\liter}~solution}}
		\\
		\text{molality (\si{\molal})} &=
		\dfrac{\text{\si{\mole}~solute}}
		{\text{\si{\kilo\gram}~solvent}}
		\visible<2>{
		\intertext{And some you might not have:}
		\text{formality (\si{\formal})} &=
		\dfrac{\text{\si{\mole}~\alert{ionic~compounds}}}
		{\text{\si{\liter}~solution}} \\
		\text{normality (\si{\normal})} &= \dfrac{\text{reactive
		\alert{equivalents}}}{\text{\si{\liter}~solution}}
		}
	\end{align*}
\end{frame}

\begin{frame}{Why bother with formality?}
	\mode<presentation>{
	\only<1>{\begin{center}
		\includegraphics[width=\linewidth]{formality-scale.png}
	\end{center}
	}}

	\only<2->{
		\begin{itemize}[<+(1)->]
		\item Long answer: it is more reflective of the actual
			concentration of the species present in solution
			\begin{itemize}
				\item Strong electrolytes dissociate completely
					in solution:
					\begin{reaction*}
						!(\SI{1}{\formal})(NaCl) ->[ H2O
						] !(\SI{1}{\Molar})( Na+ ) +
						!(\SI{1}{\Molar})( Cl- )
					\end{reaction*}
				\item Weak electrolytes do not dissociate
					completely in solution:
					\begin{reaction*}
						!(\SI{1}{\formal})( HF ) <=>[ H2O ]
						!(\SI{<1}{\Molar})( H+ ) +
						!(\SI{<1}{\Molar})( F- )
					\end{reaction*}
			\end{itemize}
		\item Short answer: most often we don't -- just go by molarity
	\end{itemize}
}
\end{frame}

\begin{frame}{Why bother with normality?}
	\begin{itemize}
		\item For some reactions, it may be more convenient to consider
			the \emph{reactive equivalents} offered by a reactant
		\item In particular, \emph{acid-base titrations} can be a bit
			simpler to think of in terms of normality

			\begin{center}
				\ch{H2SO4 + 2 NaOH -> Na2SO4 + 2 H2O}
			\end{center}

			\emph{Both} \ch{H+} in \ch{H2SO4} will react with
			\ch{OH-} to produce 2 \emph{equivalents} of \ch{H2O}

			\begin{align*}
				\therefore \SI{1}{\Molar}~\ch{H2SO4} &=
				\SI{2}{\normal}~\ch{H2SO4}
			\end{align*}
	\end{itemize}
\end{frame}

\begin{frame}[t]{Conversions}
	A concentrated HF solution is \SI{48.1}{\percent} by weight \ch{HF} and
	has a density of \SI{1.15}{\gram\per\milli\liter}. What are the formal
	and molal concentrations of \ch{HF}?

	\mode<article>{\vspace{15em}}

\note<+>{
	\begin{enumerate}
		\item Assume \SI{100.0}{\gram}~solution $\therefore$
			\SI{48.1}{\gram}~HF and \SI{51.9}{\gram}~\ch{H2O}
		\item Calculate moles HF: \SI{2.393}{\mole}~\ch{HF}
		\item This is enough to provide molality:
			\begin{align*}
				\dfrac{\SI{2.393}{\mole}~\ch{HF}}
				{\SI{0.0519}{\kilo\gram}~\text{solvent}}
				= \SI{46.1}{\molal}~\ch{HF}
			\end{align*}
		\item Use density to find volume of solution:
			\begin{align*}
				\SI{100.0}{\gram}~\text{solution} \times
				\dfrac{\SI{0.001}{\liter}}{\SI{1.15}{\gram}} =
				\SI{0.08696}{\liter}
			\end{align*}
		\item Calculate formality:
			\begin{align*}
				\dfrac{\SI{2.393}{\mole}}{\SI{0.08696}{\liter}}
				= \SI{27.5}{\formal}~\ch{HF}
			\end{align*}
	\end{enumerate}
}
\end{frame}

\begin{frame}{Exponent Removal}
	The concentration of barium ion in a produced water sample was found to
	be \SI{2.51e-7}{\Molar}.

	\begin{itemize}[<+->]
		\item Prefixes (kilo, milli, centi, \ldots)
			\begin{align*}
				\SI{2.51e-7}{\Molar} =
				\dfrac{\SI{2.51e-7}{\mole}}{\si{\liter}}
				\times
				\dfrac{\SI{e6}{\micro\mole}}{\SI{1}{\mole}} =
				\SI{0.251}{\micro\Molar}
			\end{align*}
		\item p-function: $\text{p}X = -\log [X]$
			\begin{align*}
				\text{pBa} = -\log(\num{2.51e-7}) = 6.600
			\end{align*}
		\item parts-per-concentrations (ppm, ppb, ppt, \ldots)
			\begin{align*}
				\SI{2.51e-7}{\Molar} \times
				\SI{137327}{\milli\gram\per\mole}~\ch{Ba} &=
				\SI{0.0344}{\milli\gram\per\liter}~\ch{Ba^{2+}}
				\\
				&= \SI{0.0344}{ppm}\footnotemark[1]~\ch{Ba^{2+}} \\
				&= \SI{34.4}{ppb}\footnotemark[1]~\ch{Ba^{2+}}
			\end{align*}

			\only<3>{
			\footnotetext[1]{Only in water ($d =
			\SI{1.00}{\gram\per\milli\liter}$), ppm =
	mg/L;  ppb = \si{\micro\gram}/L;  ppt = ng/L.}}
	\end{itemize}
\end{frame}

\begin{frame}{Why is \SI{1}{\milli\gram\per\liter} = \SI{1}{ppm}?}

	\mode<article>{\vspace{15em}}

\note{
	\begin{align*}
		\text{ppm} &= \dfrac{\SI{1}{part}}{\SI[inter-unit-product={ }]{1}{million~parts}} =
			\dfrac{\SI{1}{part}}{\SI{e6}{parts}} \\
			\shortintertext{}
		d_{\ch{H2O}} &= \dfrac{\SI{1}{\gram}}{\SI{1}{\milli\liter}} =
		\dfrac{\SI{1}{\gram}}{\SI{e-3}{\liter}} \times
		\dfrac{\SI{1}{\milli\gram}}{\SI{0.001}{\gram}} =
		\dfrac{\SI{1}{\milli\gram}}{\SI{e-6}{\liter}} =
		\SI{e6}{\milli\gram\per\liter}\\
	\end{align*}

	$\therefore$ in \SI{1}{\liter}~\ch{H2O}, we have
\SI{e6}{\milli\gram}~\ch{H2O}, or ``\SI[inter-unit-product={ }]{1}{million~parts}''.}
\end{frame}

\begin{frame}[t]{Problem}
	A pesticide (\SI{408.8}{amu}) is extracted from a \SI{1.000}{\liter}
	ground water sample into \SI{25.00}{\milli\liter} of hexane. GC
	analysis indicates the latter solution contains \SI{52.1}{ppb}
	pesticide. What is the molar concentration in the original ground
	water?

	\pause

	\begin{center}
		(The density of hexane is \SI{0.6548}{\gram\per\milli\liter})
	\end{center}

	\mode<article>{\vspace{15em}}

\note<+>{
	\begin{align*}
		\si{ppb} &= \dfrac{\text{mass of substance}}{\text{mass of sample}} \times \num{e9} \\
	\text{mass of sample}  &= \SI{25.00}{\milli\liter}~\text{hexane} \times \dfrac{\SI{0.6548}{\gram}}{\SI{1}{\milli\liter}} = \SI{16.37}{\gram}~\text{hexane} \\
\SI{52.1}{ppb} &= \dfrac{x}{\SI{16.37}{\gram}} \times \num{e9} \therefore x = \SI{8.529e-7}{\gram}~\text{pesticide} \\
		& \SI{8.529e-7}{\gram}~\text{pesticide} \times \dfrac{\SI{1}{\mole}}{\SI{408.8}{\gram}} = \SI{2.086e-9}{\mole} \\
		& \therefore \fbox{\SI{2.09e-9}{\Molar} or \SI{2.09}{\nano\Molar}}
	\end{align*}
	}
\end{frame}

\section{Dilutions}

\begin{frame}[t]{Dilutions}
	\begin{align*}
		M_\text{conc} V_\text{conc} = M_\text{dil} V_\text{dil}
	\end{align*}

	How would you prepare a liter of \SI{0.200}{\formal}~\ch{HCl} from
	\SI{11.8}{\formal} stock?

	\pause

	\mode<presentation>{\vskip0pt plus 1filll}
	\mode<article>{\vspace{15em}}

	\begin{block}{Lab Technique:}
		Transfer $V_\text{conc}$ concentrated \ch{HCl} to a
		\SI{1.00}{\liter} volumetric flask and dilute to volume. Shake
		well at intermediate and final volume.
	\end{block}

	\note<.>{
	\begin{align*}
		(\SI{11.8}{\formal})(V_\text{conc}) &=
		(\SI{0.200}{\formal})(\SI{1.00}{\liter}) \\
		V_\text{conc} &=
		\dfrac{(\SI{0.200}{\formal})(\SI{1.00}{\liter})}{\SI{11.8}{\formal}} \\
		V_\text{conc} &= \SI{17.0}{\milli\liter}
	\end{align*}
	}
\end{frame}

\end{document}
		
