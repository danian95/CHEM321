% !TEX program = xelatex
%\documentclass[notes=show]{beamer}
\documentclass[notes=onlyslideswithnotes,notes=hide]{beamer}
%\documentclass[letterpaper,11pt]{article}
%\usepackage{beamerarticle}

\usepackage{analchem}
\usepackage{lecture}
\usepackage{multicol}
\usepackage{tabu}

\title{Activity and the Systematic Treatment of Equilibrium}
\subtitle{Chapter 8}
\institute{CHEM321 - Analytical Chemistry I \\ Bloomsburg University}
\author{D.A. McCurry}
\date{Fall 2019}

\begin{document}

\maketitle
\mode<article>{\thispagestyle{fancy}}

\begin{frame}[allowframebreaks=0.7]{Why is the equilibrium constant not enough?}
	\begin{itemize}
		\item Consider the solubility of mercury(I) bromide, $K_{sp} =
			\num{5.6e-23}$.
		\item The equilibrium is:
			\begin{align*}
				\ch{Hg2Br2\sld{} <=> Hg2^{2+}\aq{} + 2
				Br^{-}\aq{}}
				\qquad K_{sp} &= \num{5.6e-23} \\
				&= [\ch{Hg2^{2+}}][\ch{Br-}]^2
			\end{align*}
		\item Using ICE, $x = [\ch{Hg2^{2+}}] = \SI{2.4e-8}{\Molar}$
		\item Yet when a very soluble salt such as \ch{KNO3} is added,
			more solid dissolves!
		\item This is not a Le Ch\^atelier response since neither
			\ch{K+} nor \ch{NO3-} is a common ion nor do either
			react with mercury(I) or bromide ions.
		\item In fact, if \SI{0.100}{\Molar}~\ch{KNO3} is added, the
			solubility increases such that $[\ch{Hg2^{2+}}] =
			\SI{4.1e-8}{\Molar}$, a 46\% increase in solubility!
		\item We find that whenever an \emph{inert salt} is added to a
			solution of a sparingly soluble salt, the solubility of
			the latter is increased.
	\end{itemize}

	\begin{block}{Inert salt}
		Any electrolyte that is neither a common ion nor will react with
		the ions of the sparingly soluble species.
	\end{block}
\end{frame}

\begin{frame}[allowframebreaks]{Ionic vs. Hydrated Radii}
	\begin{columns}
		\column{0.45\linewidth}
		\begin{itemize}
			\item Every dissolved species is surrounded by water
				molecules.
			\item For ionic species, the number of \ch{H2O}
				molecules depends on the ions charge density.
		\end{itemize}
		\column{0.45\linewidth}
		\includegraphics[scale=0.75]{estimated-hydration.jpg}
	\end{columns}

	\framebreak

	\begin{columns}
		\column{0.45\linewidth}
		\begin{itemize}
			\item A smaller ion of like charge has a larger charge
				density thus \ch{Li+} attracts more water than
				\ch{K+} and a 3+ ion attracts more than a 2+.
			\item In fact, the largest hydrated ion is \ch{H3O+} at
				\SI{900}{\pico\meter}!
				\begin{itemize}
					\item By comparison, hydrated \ch{Li+}
						is only about
						\SI{600}{\pico\meter}.
				\end{itemize}
		\end{itemize}
		\column{0.45\linewidth}
		\includegraphics[scale=0.25]{ionic-v-hydrated-radii.jpg}
	\end{columns}
\end{frame}

\begin{frame}{Ionic Atmosphere}
	\begin{columns}
		\column{0.6\linewidth}
	\begin{itemize}[<+->]
		\item Ions attract (or repel) other ions.
			\begin{itemize}
				\item Otherwise we would not be able to have
					insoluble salts!
			\end{itemize}
		\item What happens when an inert electrolyte is added to the
			solution?
			\begin{itemize}
				\item The ionic atmosphere causes \emph{less}
					attraction between the ions of the
					insoluble salt!
				\item Less attraction means less crashing and a
					\emph{higher} solubility!
			\end{itemize}
		\item The more ions we have in solution, the less attraction we
			have, thus the greater solubility.
	\end{itemize}
		\column{0.4\linewidth}
		\includegraphics[scale=0.85]{ionic-atmosphere.jpg}
	\end{columns}
\end{frame}

\begin{frame}{The Equilibrium ``Constant'' is Not Constant!}
	\begin{columns}
		\column{0.6\linewidth}
		\begin{itemize}
			\item We will see the same effect for all equilibria.
			\item For example, from CHEM 116:
				\begin{align*}
					\ch{Fe^{3+} + SCN- <=> FeSCN^{2+}}
				\end{align*}
			\item Note that as the ionic atmosphere increases, we
				get more of the reactant ions present shifting
				the equilibrium position.
		\end{itemize}
		\column{0.4\linewidth}
		\includegraphics[scale=0.35]{FeSCN.jpg}
	\end{columns}
\end{frame}

\begin{frame}{Ionic Strength}
	\begin{itemize}
		\item We will measure the size or strength of the ionic
			atmosphere by calculating the \emph{ionic strength}
			($\mu$) of the solution.
		\item $\mu$ is the total ion concentration in the solution:
			\begin{align*}
				\mu = \frac{1}{2}(c_1z_1^2 + c_2z_2^2 + \cdots)
				= \frac{1}{2}\sum_{i}c_iz_i^2
			\end{align*}
			where $c_i$ is the ion concentration of the
			$i$\textsuperscript{th} species and $z_i$ is its ionic
			charge.
	\end{itemize}
\end{frame}

\begin{frame}{Ionic Strength Example}
	\begin{itemize}
		\item Find the ionic strength of the following solutions:
			\begin{itemize}
				\item \SI{0.10}{\Molar} \ch{AgNO3}
				\item \SI{0.10}{\Molar} \ch{Na2SO4}
				\item \SI{0.20}{\Molar} \ch{KCl} and
					\SI{0.10}{\Molar} \ch{K2SO4}
			\end{itemize}
		\item Recognize that these values are approximations. All salts
			over 1\%~(w/w) will have some soluble ion pairs. Using
			\SI{0.025}{\formal} solutions as an example,
			\begin{itemize}
				\item \ch{NaCl} is 99.6\% dissociated
				\item \ch{Na2SO4} is 96\%
				\item \ch{MgSO4} is 65\%
				\item \ch{La2(SO4)3} is 4\%
			\end{itemize}
			See Box 8-1, page 164
	\end{itemize}

	\mode<article>{\vspace{15em}}

\note{
	\begin{multicols}{2}
	\textbf{\ch{AgNO3}}
	\begin{align*}
		\mu &= \frac{1}{2} \sum_{i} c_i z_i^2 \\
		&= \frac{1}{2} (0.10 \times 1^2 + 0.10 \times (-1)^2) \\
		&= \SI{0.10}{\Molar}
	\end{align*}

	\textbf{\ch{Na2SO4}}
	\begin{align*}
		\mu &= \frac{1}{2} \sum_{i} c_i z_i^2 \\
		&= \frac{1}{2} (0.20 \times 1^2 + 0.10 \times (-2)^2) \\
		&= \SI{0.30}{\Molar}
	\end{align*}

	\textbf{\ch{KCl} and \ch{K2SO4}}
	\begin{align*}
		\mu &= \frac{1}{2} \sum_{i} c_i z_i^2 \\
		&= \frac{1}{2} (0.20 \times 1^2 + 0.2 \times (-1)^2 \\
		&\qquad {} + 0.20
		\times 1^2 + 0.10 \times (-2)^2) \\
		&= \SI{0.5}{\Molar}
	\end{align*}
	\end{multicols}
	}
\end{frame}

\begin{frame}{Working Ionic Strengths into K's}
	\begin{itemize}
		\item To account for the effect ionic strength has on the
			chemistry of a salt, we replace molar concentrations
			with activities:
			\begin{align*}
				\mathcal{A}_{\ch{C}} = [\ch{C}]\gamma_{\ch{C}}
			\end{align*}
			where the molar concentration of species \ch{C}
			([\ch{C}]) is multiplied by its activity coefficient
			($\gamma_{\ch{C}}$).
		\item The \emph{extended Debye-Huckel equation} gives the
			activity coefficient which is dependent on the size of
			the ionic strength.
			\begin{align*}
				\log \gamma = \dfrac{-0.51z^2\sqrt{\mu}}{1 +
				(\alpha \sqrt{\mu}/305)} \qquad\text{(at
				\SI{25}{\celsius})}
			\end{align*}
			where $\alpha$ is the size of the ion in
			\si{\pico\meter}. Note that this relation only works
			well up to \SI{0.10}{\Molar} ionic strength.
	\end{itemize}
\end{frame}

\begin{frame}{The \emph{Real} Equilibrium Constant}
	\begin{itemize}
		\item Consider the solubity of mercury (I) bromide, with
			equilibrium,
			\begin{align*}
				\ch{Hg2Br2(s) <=> Hg2^{2+}(aq) + 2 Br^{-}(aq)}
			\end{align*}
		\item The new equilibrium constant including activity
			coefficients is
			\begin{align*}
				K_{sp} = \mathcal{A}_{\ch{Hg2^{2+}}}
				\mathcal{A}_{\ch{Br-}}^2 =
				[\ch{Hg2^{2+}}]\gamma_{\ch{Hg2^{2+}}}
				[\ch{Br-}]^2\gamma_{\ch{Br-}}^2
			\end{align*}
	\end{itemize}

	\pause

	\begin{block}{General Form of the Equilibrium Constant}
		\begin{align*}
			K = \dfrac{\mathcal{A}^c_{\ch{C}}
			\mathcal{A}^d_{\ch{D}}}{\mathcal{A}^a_{\ch{A}}
			\mathcal{A}^b_{\ch{B}}} = \dfrac{[\ch{C}]^c
			\gamma_{\ch{C}}^c [\ch{D}]^d \gamma_{\ch{D}}^d}
			{[\ch{A}]^a \gamma_{\ch{A}}^a [\ch{B}]^b
			\gamma_{\ch{B}}^b}
		\end{align*}
	\end{block}
\end{frame}


\begin{frame}{Effect of $\mu$, $z_i$, and $\alpha$ on $\gamma$}
	\begin{columns}
		\column{0.4\linewidth}
		\includegraphics[scale=0.85]{activity-v-ionic-strength.png}
		\column{0.6\linewidth}
		\begin{enumerate}
			\item<1-> As ionic strength increase, the activity
				coefficient decreases.

				\only<1|handout:0>{\begin{itemize}
					\item The activity coefficient
						($\gamma$) approaches unity as
						the ionic strength ($\mu$)
						approaches 0.
					\item At $\mu \leq \SI{0.0001}{\Molar}$,
						$\gamma \approx 1$.
				\end{itemize}}

			\item<2-> As the charge on an ion increases, the
				departure of its activity coefficient from unity
				increases.

				\only<2|handout:0>{\begin{itemize}
					\item Activity coefficients are much
						more important for highly
						charged species ($\pm3$ or
						$\pm4$).
					\item Let $\gamma_i = 1$ for all
						uncharged species at any $\mu$.
				\end{itemize}}

			\item<3-> The smaller the hydrated radius of the ion,
				the more important the activity coefficient
				becomes.

				\only<3|handout:0>{\begin{itemize}
					\item At fixed ionic strength, however,
						species of similar charge will
						have similar activity
						coefficients.
				\end{itemize}}

			\item<4-> Activity coefficient values are independent of
				counter ions.

				\only<4|handout:0>{\begin{itemize}
					\item The key is the species itself --
						its hydrated size, its charge,
						its being affected by ionic
						strength!
				\end{itemize}}
		\end{enumerate}
	\end{columns}
\end{frame}

\begin{frame}[t]{Activity Example}
	What is the solubility of calcium fluoride in \SI{0.050}{\formal} sodium
	fluoride? Calculate the solubility both with and without activity
	correction.

	\smallskip

	$K_{sp} = \num{3.9e-11}$ for calcium fluoride

	\mode<article>{\vfill}

\note{
	\footnotesize
	\begin{tabu} to \linewidth {c r c r c r}
		& \ch{CaF2(s)} & \ch{<=>} & \ch{Ca^{2+}(aq)} & \ch{+} &
		\ch{2 F^{-}(aq)} \\
		I & & & 0 & & 0.050 \\
		C & $-x$ & & $+x$ & & $+2x$ \\ \tabucline{2-}
		E & $-x$ & & $x$ & & $0.050 + 2x$
	\end{tabu}

	\begin{columns}
		\column{0.5\linewidth}
		\begin{align*}
			K_{sp} &= [\ch{Ca^{2+}}][\ch{F-}]^2 \\
			\num{3.9e-11} &= (x)(0.050+2x)^2 \\
			x &= \SI{1.6e-8}{\Molar}
		\end{align*}
		\column{0.5\linewidth}
		\begin{align*}
			K_{sp} &= [\ch{Ca^{2+}}]\gamma_{\ch{Ca^{2+}}}
			[\ch{F-}]^2\gamma_{\ch{F-}}^2 \\
			\num{3.9e-11} &= (x)(0.050+2x)^2
			\gamma_{\ch{Ca^{2+}}}\gamma_{\ch{F-}}^2 \\
			x &= \dfrac{\num{3.9e-11}}{0.050^2 \times 0.485 \times
			0.81^2} \\
			&= \SI{4.9e-8}{\Molar}
		\end{align*}
	\end{columns}

	\begin{align*}
		\mu &= 0.5 \left( [\ch{Na+}]\times1^2 +
		[\ch{F-}]\times-1^2 + [\ch{Ca^{2+}}]\times2^2 \right) \\
		&= 0.5 \left(0.050 + (0.050+2x) + 4x\right) \\
		&= \SI{0.050}{\Molar} \text{~and via Table 8-1,~}
		\gamma_{\ch{Ca^{2+}}} = 0.485 \text{~and~}
		\gamma_{\ch{F-}} = 0.81
	\end{align*}
	}
\end{frame}

\begin{frame}{pH Revisited}
	\begin{itemize}
		\item pH is also affected by ionic strength because it is an
			equilibrium involving charged species.
		\item What is the real pH of pure distilled water?
			\begin{align*}
				\ch{2 H2O <=> H3O+ + OH-} \qquad K_w =
				[\ch{H3O+}]\gamma_{\ch{H3O+}}
				[\ch{OH-}]\gamma_{\ch{OH-}}
			\end{align*}
		\item At very low ionic strength, $\gamma \approx 1$
		\item At \SI{1.0e-7}{\Molar},
			\begin{align*}
				\text{pH} &= -\log \mathcal{A}_{\ch{H3O+}} \\
				&= -\log [\ch{H3O+}]\gamma_{\ch{H3O+}}
				\\
				&= -\log (\num{1.0e-7})(1.00) \\
				&= 7.00
			\end{align*}
	\end{itemize}
\end{frame}

\begin{frame}[t]{pH Example}
	What is the pH of real distilled water with \SI{0.50}{\Molar}~\ch{KCl}
	added?

	\mode<article>{\vfill}

	\note{
	\begin{align*}
		\ch{2 H2O <=> H3O+ + OH-} \qquad K_w &=
				[\ch{H3O+}]\gamma_{\ch{H3O+}}
				[\ch{OH-}]\gamma_{\ch{OH-}} \\
		\intertext{$\mu$ = \SI{0.50}{\Molar}, $\gamma_{\ch{H3O+}} =
		0.28$, and $\gamma_{\ch{OH-}} = 0.38$}
		\num{1.0e-14} &= (0.28x)(0.38x) \\
		\num{9.4e-14} &= x^2 \\
		x &= [\ch{H3O+}] = \SI{3.1e-7}{\Molar} \\
		pH &= -\log \mathcal{A}_{\ch{H3O+}} = -\log
		[\ch{H3O+}]\gamma_{\ch{H3O+}} \\
		&= -\log (\num{3.1e-7})(0.28) \\
		&= 7.06
	\end{align*}
	}
\end{frame}

\mode<article>{\clearpage}

\begin{frame}{Systematic Treatment of Equilibrium}
	\begin{itemize}
		\item Many chemical systems are complex due to the number of
			simultaneous reactions occuring.
		\item Recall:
			 \begin{center}
				 \scriptsize
                \begin{tabular} {r @{\ch{<=>}} l               
                r@{ = }S[table-format=1.2e-2]}
                     \ch{CaCO3(s)} & \ch{Ca^{2+}(aq)} + \ch{CO3^{2-}(aq)} & $K_{sp}$ & 4.5e-9 \\
                     \ch{CO3^{2-}(aq)} + \ch{H2O(l)} & \ch{HCO3^{-}(aq)} + \ch{OH^{-}(aq)} & $K_b$ & 2.1e-4 \\
                     \ch{CO2(aq)} + \ch{H2O(l)} & \ch{H2CO3(aq)} & $K_h$ &         1.7e-3 \\
                     \ch{H2CO3(aq)} + \ch{H2O(l)} & \ch{HCO3^{-}(aq)} + \ch{H3O^{+}(aq)} & $K_a$ & 4.46e-7 \\
                     \ch{H3O^{+}(aq)} + \ch{OH^{-}(aq)} & \ch{2 H2O(l)} & $1/K_w$ & 1.0e14  \\ \midrule
                     \ch{CaCO3(s) + CO2(aq) + H2O(l)} & \ch{Ca^{2+}(aq) +    
                2 HCO3^{-}(aq)} & $K_{eq}$ & 7.2e-8 \\                          
        \end{tabular}                                                              
        \end{center}
		\item How do we solve a problem with so many unknowns?
	\end{itemize}
\end{frame}

\begin{frame}{$n$ equations\ldots}
	\begin{itemize}[<+->]
		\item In order to solve a problem containing 1 unknown, you
			simply need 1 equation containing that value:
			\begin{align*}
				\si{\gram}~\ch{Ca^{2+}} =
				y~\si{\mole}~\ch{Y^{4-}} \times
				\dfrac{\SI{1}{\mole}~\ch{Ca^{2+}}}
				{\SI{1}{\mole}~\ch{Y^{4-}}} \times
				\dfrac{\SI{40.08}{\gram}~\ch{Ca^{2+}}}
				{\SI{1}{\mole}}
			\end{align*}
		\item To solve for 2 unknowns, you need 2 equations:
			\begin{align*}
				A^1_{\text{T}} &=
				\epsilon^1_{\ch{Mn}}bC_{\ch{Mn}} +
				\epsilon^1_{\ch{Cr}}bC_{\ch{Cr}} \\
				A^2_{\text{T}} &=
				\epsilon^2_{\ch{Mn}}bC_{\ch{Mn}} +
				\epsilon^2_{\ch{Cr}}bC_{\ch{Cr}}
			\end{align*}
		\item To solve $n$ unknowns, you need $n$ equations.
			\begin{itemize}
				\item Include \emph{all} the \alert{equilibrium
					constants}
				\item Include a statement of electroneutrality
					-- the \alert{charge balance}
				\item Include a statement of the conservation of
					matter -- the \alert{mass balance}
			\end{itemize}
	\end{itemize}
\end{frame}

\begin{frame}{Charge Balance}
	\begin{itemize}
		\item An algebraic statement of electroneutrality.
		\item The sum of the positive charges in solution \emph{equals}
			the sum of negative charges in solution.

			\begin{align*}
				\sum(\text{positive charges}) &=
				\sum(\text{negative charges}) \\
				n_1[\ch{C1}] + n_2[\ch{C2}] + \cdots &=
				m_1[\ch{A1}] + m_2[\ch{A2}] + \cdots
			\end{align*}
		
			where
		
			\begin{tabu} to \linewidth {X}
				$n$ is the charge of each \emph{cation} of
				concentration [\ch{C}] \\
				$m$ is the charge of each \emph{anion} of
				concentration [\ch{A}]
			\end{tabu}
	
		\item Activity coefficients \emph{do not} appear in the charge
			balance!
	\end{itemize}
\end{frame}

\begin{frame}[t]{Charge Balance Example 1}
	Write the charge balance for an aqueous solution containing
	\SI{0.10}{\Molar}~\ch{Mg(NO3)2}, \SI{0.20}{\Molar}~\ch{HCN},
	\SI{0.15}{\Molar}~\ch{KCN}, \SI{0.10}{\Molar}~\ch{Ag4Fe(CN)6},
	\ch{H3O+}, \ch{OH-}, ignoring any further equilibria for this problem.
	Show that it balances.

	\mode<article>{\vfill}

	\note{\scriptsize
	What species do we have present?
	\begin{align*}
		\ch{Mg(NO3)2 &<=> Mg^{2+} + 2 NO3-} \\
		\ch{HCN &<=> H+ + CN-} \\
		\ch{KCN &<=> K+ + CN-} \\
		\ch{Ag4Fe(CN)6 &<=> 4 Ag+ + Fe(CN)6^{4-}} \\
		\ch{2 H2O &<=> H3O+ + OH-}
	\end{align*}

	\begin{align*}
		2[\ch{Mg^{2+}}] + [\ch{H3O+}] + [\ch{K+}] + [\ch{Ag+}] &=
		[\ch{NO3-}] + [\ch{CN-}] + 4[\ch{Fe(CN)6^{4-}}] + [\ch{OH-}] \\
		2(0.10) + (0.20 + \num{1e-7}) + 0.15 + 0.40 &= 0.20 + (0.20 +
		0.15) + 4(0.10) + \num{1e-7}
		\\
		\SI{0.95}{\Molar} &= \SI{0.95}{\Molar}
	\end{align*}
	}
\end{frame}

\begin{frame}[t]{Charge Balance Example 2}
	Write the charge balance for a solution of sodium dihydrogen phosphate
	(including equilibria).

	\mode<article>{\vfill}

	\note{\scriptsize
	What species do we have present?
	\begin{align*}
		\ch{NaH2PO4 &<=> Na+ + H2PO4-} \\
		\ch{H2PO4- &<=> H3O+ + HPO4^{2-}} \\
		\ch{H2PO4- + H3O+ &<=> H3PO4} \\
		\ch{HPO4^{2-} &<=> H3O+ + PO4^{3-}} \\
		\ch{2 H2O &<=> H3O+ + OH-}
	\end{align*}


	\begin{align*}
		[\ch{Na+}] + [\ch{H3O+}] &= [\ch{H2PO4-}] + 2[\ch{HPO4^{2-}}] +
		3[\ch{PO4^{3-}}] + [\ch{OH-}]
	\end{align*}
	}
\end{frame}

\begin{frame}{Mass Balance}
	\begin{itemize}
		\item A statement of the conservation of matter.
		\item The quantity of all species in a solution containing a
			particular atom (or group of atoms) \emph{must equal}
			the amount of that atom (or group) delivered to the
			solution.
		\item We generally have two mass balances, one for the anion and
			another for the cation. These can be combined into a
			single expression.
	\end{itemize}
\end{frame}

\mode<article>{\clearpage}

\begin{frame}[t]{Mass Balance Example 1}
	Write the mass balances for an aqueous solution containing
	\SI{0.10}{\formal}~\ch{Mg(NO3)2}.

	\mode<article>{\vspace{\fill}}

\note{
	\begin{center}
		\ch{Mg(NO3)2 <=> Mg^{2+} + 2 NO3-}
	\end{center}

	\begin{align*}
		\SI{0.10}{\formal} &= [\ch{Mg^{2+}}] \\
		2[\ch{Mg^{2+}}] &= [\ch{NO3-}] \\
		\SI{0.20}{\formal} &= [\ch{NO3-}]
	\end{align*}
	}
\end{frame}

\begin{frame}[t]{Mass Balance Example 2}
	Write the mass balances for a \SI{0.20}{\formal} solution of sodium
	sulfate.

	\mode<article>{\vspace{\fill}}

	\note{
	\begin{reaction*}
		Na2SO4 <=> 2 Na+ + SO4^{2-} + HSO4-
	\end{reaction*}

	\begin{align*}
		\SI{0.20}{\formal} &= [\ch{SO4^{2-}}] + [\ch{HSO4-}] \\
		2[\text{total sulfates}] &= [\ch{Na+}] \\
		\SI{0.40}{\formal} &= [\ch{Na+}]
	\end{align*}
	}
\end{frame}
	
\begin{frame}[t]{Mass Balance Example 3}
	Write the mass balance for a saturated solution of copper(II) iodate.

	\mode<article>{\vspace{\fill}}

	\note{
		\begin{reaction*}
			Cu(IO3)2 <=> Cu^{2+} + 2 IO3-
		\end{reaction*}
	
		\begin{align*}
			2[\ch{Cu^{2+}}] &= [\ch{IO3-}] \\
		\end{align*}
	}
\end{frame}

\begin{frame}[t]{Mass Balance Example 4}
	Write the mass balance for a saturated silver phosphate solution.

	\mode<article>{\vspace{\fill}}

\note{
	\begin{reactions*}
		Ag3PO4 &<=> 3 Ag+ + PO4^{3-} \\
		PO4^{3-} + H2O &<=> HPO4^{2-} + OH- \\
		HPO4^{2-} + H2O &<=> H2PO4- + OH- \\
		H2PO4- + H2O &<=> H3PO4 + OH-
	\end{reactions*}

	\begin{align*}
		3[\text{total phosphates}] &= [\ch{Ag+}] \\
		3\big( [\ch{PO4^{3-}}] + [\ch{HPO4^{2-}}] + [\ch{H2PO4-}] +
		[\ch{H3PO4}]
		\big) &= [\ch{Ag+}]
	\end{align*}
	}
\end{frame}

\mode<article>{\clearpage}

\begin{frame}{The Systematic Treatment of Equilibrium}
	\begin{enumerate}
		\item Write the \emph{pertinent reactions}.
		\item Write the \emph{charge balance} equation.
		\item Write the \emph{mass balance} equation(s).
		\item Write the \emph{equilibrium constant} for each chemical
			reaction.
			\begin{itemize}
				\item This is the only step where activity
					coefficients may appear.
			\end{itemize}
		\item \emph{Count the equations and unknowns}.
			\begin{itemize}
				\item The numbers should equal or you need to
					find more equilibria or fix some
					concentrations.
			\end{itemize}
		\item \emph{Solve} for all the unknowns.
	\end{enumerate}
\end{frame}

\begin{frame}[t]{Systematic Treatment Example 1}
	Calculate the solubility of calcium fluoride in a pH 4.00 buffer.

	\mode<article>{\vfill}
	
	\note{
	\begin{multicols}{2}
		\footnotesize
		\begin{enumerate}
			\item Pertinent reactions
				\begin{reactions*}
					CaF2 &<=> Ca^{2+} + 2 F- \\
					F- + H2O &<=> HF + OH- \\
					2 H2O &<=> H3O+ + OH-
				\end{reactions*}
			\item Charge balance?
			\item Mass balance
				\begin{align*}
					2[\ch{Ca^{2+}}] &= [\ch{HF}] + [\ch{F-}]
					\\
					[\ch{H3O+}] &= \SI{1.0e-4}{\Molar}
				\end{align*}
			\item Equilibrium constants
				\begin{align*}
					K_{sp} &= [\ch{Ca^{2+}}][\ch{F-}]^2 =
					\num{3.9e-11} \\
					K_b &=
					\dfrac{[\ch{HF}][\ch{OH-}]}{[\ch{F-}]} =
					\num{1.5e-11} \\
					K_w &= [\ch{H3O+}][\ch{OH-}] =
					\num{1.0e-14}
				\end{align*}
			\item Count equations and unknowns
				\begin{itemize}
					\item 5 unknowns 
					\item 5 equations
				\end{itemize}
			\item Solve
		\end{enumerate}
	\end{multicols}
	}

\end{frame}
	\note{
		\footnotesize
		$[\ch{H3O+}] = \SI{1.0e-4}{\Molar}$ is both an equation and a
		solution.
		\begin{align*}
			[\ch{OH-}] &= \dfrac{\num{1.0e-14}}{\num{1.0e-4}} =
			\SI{1.0e-10}{\Molar}
		\end{align*}
		$K_b$ contains [\ch{OH-}], so:
		\begin{align*}
			\dfrac{\num{1.5e-11}}{\num{1.0e-10}} = 0.15 =
			\dfrac{[\ch{HF}]}{[\ch{F-}]} \qquad \therefore
			0.15[\ch{F-}] = [\ch{HF}]
		\end{align*}
		From mass balance,
		\begin{align*}
			2[\ch{Ca^{2+}}] &= [\ch{HF}] + [\ch{F-}] \\
			2[\ch{Ca^{2+}}] &= 0.15[\ch{F-}] +
			[\ch{F-}] = 1.15[\ch{F-}] \\
			1.74[\ch{Ca^{2+}}] &= [\ch{F-}]
		\end{align*}
		}

	\note{
		\footnotesize
		Using $K_{sp}$,
		\begin{align*}
			\num{3.9e-11} &= [\ch{Ca^{2+}}](1.74[\ch{Ca^{2+}}])^2 \\
			&= 3.02[\ch{Ca^{2+}}]^3 \\
			[\ch{Ca^{2+}}] &= \SI{2.3e-4}{\Molar}
		\end{align*}
		Therefore,
		\begin{align*}
			[\ch{F-}] &= 1.74[\ch{Ca^{2+}}] = \SI{4.1e-4}{\Molar} \\
			[\ch{HF}] &= 0.15[\ch{F-}] = \SI{6.1e-5}{\Molar} \\
		\end{align*}
		
		\begin{framed}
			\begin{columns}
				\column{0.5\textwidth}
				\begin{align*}
					[\ch{H3O+}] &= \SI{1.0e-4}{\Molar} \\
					[\ch{OH-}] &= \SI{1.0e-10}{\Molar} \\
					[\ch{F-}] &= \SI{4.1e-4}{\Molar}
				\end{align*}
				\column{0.5\textwidth}
				\begin{align*}
					[\ch{HF}] &= \SI{6.1e-5}{\Molar} \\
					[\ch{Ca^{2+}}] &= \SI{2.3e-4}{\Molar}
				\end{align*}
			\end{columns}
		\end{framed}
		}

%\begin{frame}<handout:0>{Quiz}
%	\begin{enumerate}
%		\item Write the $K$ expressions for the following
%			reactions:
%
%			\begin{align*}
%				\ch{Pb(NO3)2(aq) &<=> Pb^{2+}(aq) + 2 NO3-(aq)}
%				&\qquad &K_{d1}	\\
%				\ch{Na2SO4(aq) &<=> 2Na+(aq) + SO4^{2-}(aq)}
%				&\qquad &K_{d2} \\
%				\ch{NaNO3(aq) &<=> Na+(aq) + NO3-(aq)} &\qquad
%				&K_{d3} \\
%				\ch{PbSO4(s) &<=> Pb^{2+}(aq) + SO4^{2-}(aq)}
%				&\qquad &K_{sp}
%			\end{align*}
%		
%		\item Write the $K_{eq}$ for the following reaction in terms of
%			the $K$ found above:
%
%			\begin{align*}
%				\ch{Pb(NO3)2(aq) + Na2SO4(aq) &<=> PbSO4(s) +
%				2NaNO3(aq)}
%			\end{align*}
%	\end{enumerate}
%\end{frame}

		\mode<article>{\clearpage}

\begin{frame}[t]{Systematic Treatment Example 2}
	Calculate the solubility of silver phosphate in a pH~5.00 buffer.

	\mode<article>{\vfill}

	\note{
	\scriptsize
	\begin{multicols}{2}
		\begin{enumerate}
			\item Pertinent reactions
				\begin{reactions*}
					Ag3PO4 &<=> 3 Ag+ + PO4^{3-} \\
					PO4^{3-} + H2O &<=> HPO4^{2-} + OH- \\
					HPO4^{2-} + H2O &<=> H2PO4- + OH- \\
					H2PO4- + H2O &<=> H3PO4 + OH- \\
					2 H2O &<=> H3O+ + OH-
				\end{reactions*}
			\item Charge balance
			\item Mass balance
				\begin{align*}
					[\ch{Ag+}] &= 3(\ch{[PO4^{3-}] +
					[HPO4^{2-}]} \\
					&\qquad + \ch{[H2PO4^-] + [H3PO4]}) \\
					[\ch{H3O+}] &= \SI{1.0e-5}{\Molar}
				\end{align*}
			\item Equilibrium constants
				\begin{align*}
					K_{sp} &= [\ch{Ag+}]^3[\ch{PO4^{3-}}] =
					\num{2.8e-18} \\
					K_{b1} &=
					\dfrac{[\ch{HPO4^{2-}}][\ch{OH-}]}{[\ch{PO4^{3-}}]} =
					\num{0.0237} \\
					K_{b2} &=
					\dfrac{[\ch{H2PO4-}][\ch{OH-}]}{[\ch{HPO4^{2-}}]} =
					\num{1.58e-7} \\
					K_{b3} &=
					\dfrac{[\ch{H3PO4}][\ch{OH-}]}{[\ch{H2PO4-}]}
					= \num{1.41e-12} \\
					K_w &= [\ch{H3O+}][\ch{OH-}] =
					\num{1.0e-14}
				\end{align*}
			\item Count equations and unknowns
				\begin{itemize}\tiny
					\item 7 unknowns 
					\item 7 equations
				\end{itemize}
			\item Solve
		\end{enumerate}
	\end{multicols}
	}
\end{frame}

\note{\scriptsize
	$[\ch{H3O+}] = \SI{1.0e-5}{\Molar}$ is both an equation and a
	solution.
	\begin{align*}
		[\ch{OH-}] &= \dfrac{\num{1.0e-14}}{\num{1.0e-5}} =
		\SI{1.0e-9}{\Molar}
	\end{align*}
	All $K_b$ contain [\ch{OH-}], so:
	\begin{align*}
		\dfrac{\num{1.41e-12}}{\num{1.0e-9}} = \num{1.41e-3} &=
		\dfrac{[\ch{H3PO4}]}{[\ch{H2PO4-}]} \\
		(\num{1.41e-3})[\ch{H2PO4-}] &= [\ch{H3PO4}] \\
		(\num{1.58e2})[\ch{HPO4^{2-}}] &= [\ch{H2PO4-}] \\
		(\num{2.37e7})[\ch{PO4^{3-}}] &= [\ch{HPO4^{2-}}]
	\end{align*}
}

\note{\scriptsize
	From mass balance,
	\begin{align*}
		[\ch{Ag+}] &= 3(\ch{[PO4^{3-}] + [HPO4^{2-}] + [H2PO4^-] +
		[H3PO4]}) \\
		&= 3(\ch{[PO4^{3-}]} + (\num{2.37e7})\ch{[PO4^{3-}]} +
		(\num{1.58e2})\ch{[PO4^{2-}]} + (\num{1.41e-3})\ch{[H2PO4-]}) \\
		&= 3(\ch{[PO4^{3-}]} + (\num{2.37e7})\ch{[PO4^{3-}]}
		(\num{1.58e2})(\num{2.37e7})\ch{[PO4^{3-}]} \\
		& \qquad\qquad {} +
		(\num{1.41e-3})(\num{1.58e2})\ch{[PO4^{2-}]}) \\
		&= 3(\ch{[PO4^{3-}]} + (\num{2.37e7})\ch{[PO4^{3-}]} +
		(\num{1.58e2})(\num{2.37e7})\ch{[PO4^{3-}]}  \\
		& \qquad\qquad {} +
		(\num{1.41e-3})(\num{1.58e2})(\num{2.37e7})\ch{[PO4^{3-}]}) \\
		&= \num{1.13e10}\ch{[PO4^{3-}]}
	\end{align*}
	}

\note{\scriptsize
	Using $K_{sp}$,
	\begin{align*}
		\num{2.8e-18} &=
		((\num{1.13e10})[\ch{PO4^{3-}}])^3[\ch{PO4^{3-}}] \\
		&= (\num{1.44e30})[\ch{PO4^{3-}}]^4 \\
		[\ch{PO4^{3-}}] &= \SI{1.18e-12}{\Molar}
	\end{align*}
	Therefore,
	\begin{align*}
		[\ch{HPO4^{2-}}] &= (\num{2.37e7})[\ch{PO4^{3-}}] =
		\SI{2.80e-5}{\Molar} \\
		[\ch{H2PO4-}] &= (\num{1.58e2})[\ch{HPO4^{2-}}] =
		\SI{4.41e-3}{\Molar} \\
		[\ch{H3PO4}] &= (\num{1.41e-3})[\ch{H2PO4-}] =
		\SI{6.20e-6}{\Molar} \\
		[\ch{Ag+}] &= 3[\text{total phosphates}] = \SI{1.33e-2}{\Molar}
	\end{align*}
	
%	\begin{framed}
%		\begin{columns}
%			\column{0.5\textwidth}
%			\begin{align*}
%				[\ch{H3O+}] &= \SI{1.0e-4}{\Molar} \\
%				[\ch{OH-}] &= \SI{1.0e-10}{\Molar} \\
%				[\ch{F-}] &= \SI{4.1e-4}{\Molar}
%			\end{align*}
%			\column{0.5\textwidth}
%			\begin{align*}
%				[\ch{HF}] &= \SI{6.1e-5}{\Molar} \\
%				[\ch{Ca^{2+}}] &= \SI{2.3e-4}{\Molar}
%			\end{align*}
%		\end{columns}
%	\end{framed}
	}


\begin{frame}{Some for you to try:}
	\begin{enumerate}
		\item Calculate the solubility of aluminum phosphate ($K_{sp} =
			\num{9.84e-22}$) at pH~4.00.
		\item Using a spreadsheet, calculate the solubility of \ch{CaF2}
			between pH's 1--10. Finally, plot [\ch{Ca^{2+}}] vs pH.
		\item Calculate the solubility of barium oxalate at pH~2.00.
	\end{enumerate}
\end{frame}

\end{document}
