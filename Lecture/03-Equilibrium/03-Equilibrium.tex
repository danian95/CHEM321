% !TEX program = xelatex
%\documentclass[notes=onlyslideswithnotes,notes=hide]{beamer}
%\documentclass[notes=hide]{beamer}
%\documentclass[notes=show]{beamer}
\documentclass[notes=only]{beamer}
%\documentclass[handout]{beamer}
%\documentclass[letterpaper,11pt]{article}
%\usepackage{beamerarticle}

%\usepackage{cancel}
\newcommand<>{\latecancel}[1]{\alt#2{\cancel{#1}}{\vphantom{\cancel{#1}}{#1}}}

\usepackage{newtxtext}
\usepackage{bucolors}
\usepackage{multicol}
\usepackage{analchem}
\usepackage{lecture}
\sisetup{
	separate-uncertainty=true,
	multi-part-units=single,
	add-decimal-zero=false,
	round-mode=off}
\usepackage{bigdelim}
\usepgfplotslibrary{fillbetween}
\NewChemEqConstant{\Kh}{K-gas}{\ensuremath{\mathrm{h}}}
%\NewChemEqConstant{\Keq}{K-eq}{\ensuremath{\mathrm{eq}}}

\graphicspath{{./Figures/}}

\title{Equilibrium}
\author{D.A. McCurry}
\institute{Department of Chemistry and Biochemistry \\ Bloomsburg University}
\date{Fall 2021}

\begin{document}

\maketitle
\mode<article>{\thispagestyle{fancy}}

\mode<presentation>{{
		\usebackgroundtemplate{\includegraphics[height=\paperheight,trim={2in 0 0 0},clip]{hotwaterheater.jpg}}
	\begin{frame}[b,plain]
		\footnotesize{\color{white}Source: A random YouTube image from a search.}
		\bigskip
		\null
	\end{frame}
}}

%\begin{frame}
%	\begin{columns}
%		\column{0.5\textwidth}
%		\includegraphics[width=\textwidth]{equilibrium-poster.jpg}
%		\column{0.5\textwidth}
%		{\Large
%		``The only thing more powerful than the system, is the man that
%		will overthrow it.''}
%		
%		\pause
%
%		\vspace{3em}
%
%		{\small
%		(But given enough time, the system will return to equilibrium)}
%	\end{columns}
%\end{frame}

\frame{\section{Equilibrium Constant Review}
	\begin{learningobjectives}
	\item Manipulate equilibrium constants to describe a specific reaction.
	\item Explain the relationship between the equilibrium constant and thermodynamics.
	\item Determine reaction direction due to changes in temperature or species concentration.
	\item Use a ladder diagram to determine the predominant species in a
		solution.
	\end{learningobjectives}
}

\vspace{\stretch{-1}}

\begin{frame}[allowframebreaks]{The Equilibrium Constant}
	\begin{block}{Law of Mass Action}
		\centering
		\parbox[c]{0.4\linewidth}{
			\begin{center}
			\ch{aA + bB <=> cC + dD}
			\end{center}
			}
		\qquad
		\parbox[c]{0.4\linewidth}{
			\begin{align*}
				K = \dfrac{[\ch{C}]^c[\ch{D}]^d}
				{[\ch{A}]^a[\ch{B}]^b}
			\end{align*}
			}
	\end{block}

	\bigskip

	\begin{columns}
		\column{0.45\textwidth}
			What does it mean if\ldots
			\begin{itemize}
				\item[\dots] $K < 1$?
				\item[\dots] $K > 1$?
			\end{itemize}
		\column{0.45\textwidth}
		$K$ is \alert{dimensionless}:
			\begin{equation*}
				\ch{[A]} \equiv
				\frac{\ch{[A]}}{\ch{[A]^{\standardstate}}}
			\end{equation*}
	\end{columns}

	\framebreak

	To maintain the dimensionless character,

	\begin{enumerate}
		\item The concentrations of all solutes should be express in
			\si{\Molar} or \si{\formal}.
		\item The concentrations of all gases should be expressed in
			bar. (atm are close, but be consistent)
		\item The concentrations of pure solids, pure liquids, and
			solvents are omitted because they are \alert{invariant}.
	\end{enumerate}
\end{frame}

\begin{frame}[t]{Manipulating Equilibrium Constants}
	Consider the following reaction:
	\begin{align*}
		\ch{HA} &\ch{<=>} \ch{H+ + A-} &&& K_1 &=
		\dfrac{[\ch{H+}][\ch{A-}]}{[\ch{HA}]} \\
		\visible<+(1)->{%
		\intertext{What happens if we reverse it?}
	\ch{H+ + A-} &\ch{<=>} \ch{HA} \\}
		\visible<+(1)->{%
		\intertext{What happens if we add a new reaction to it?}
	\ch{{} + H+ + C} &\ch{<=>} \ch{CH+} &&& K_2
	}
	\end{align*}
	\note<+(1)>{%
		Reversing the reaction \alert{inverses} the value of
		$K$:
		\begin{align*}
		\ch{H+ + A-} &\ch{<=>} \ch{HA} &&& K_1' &=
		\dfrac{[\ch{HA}]}{[\ch{H+}][\ch{A-}]} = \frac{1}{K_1} \\
		\intertext{Adding reaction, $K_2$, to $K_1$, the new $K$ is the
		product of the two individual values:}
		\intertext{\small\centering
			\begin{tabularx}{0.3\linewidth} {l c l @{\quad\ch{<=>}\quad} l
			c l c}
			\ch{HA} & & & \cancel{\ch{H+}} & + & \ch{A-} & $K_1$ \\
			\cancel{\ch{H+}} & + & \ch{C} & \ch{CH+} & & & $K_2$ \\ 
		\end{tabularx}}
		\ch{HA + C} &\ch{<=>} \ch{A- + CH+} &&& K_3 &= 
		\dfrac{[\ch{A-}][\ch{CH+}]}{[\ch{HA}][\ch{C}]} = K_1K_2
	\end{align*}
}
\end{frame}

\begin{frame}[t]{Manipulation Example}
	Find \Keq{} for the reaction,
	\begin{reaction*}
		CaCO3\sld{} + CO2\aq{} + H2O\lqd{} <=> Ca^{2+}\aq{} + 2 HCO3^{-}\aq{}
	\end{reaction*}

	\onslide<+(1)->
	Consider that we have\ldots
	\begin{enumerate}[<+->]
		\item Dissolution of \ch{CaCO3} \textrightarrow{} Solubility (\Ksp)
		\item Weak base (\ch{CO3^{2-}}) in \water{} \textrightarrow{} Base
			dissociation (\Kb)
		\item Dissolution of \ch{CO2} \textrightarrow{} Gas solubility (\Kh)
		\item Weak acid (\ch{H2CO3}) in \water{} \textrightarrow{} Acid
			dissociation (\Ka)
     	\end{enumerate}

	\note<1>{\scriptsize
	\begin{center}
		\begin{tabularx}{\textwidth} {c<{.} r @{\ch{<=>}} l
		X@{ = }S[table-format=1.2e-2]}
		1 &	\ch{CaCO3(s)} & \ch{Ca^{2+}(aq)} + \ch{CO3^{2-}(aq)} & $\Ksp{}$ & 4.5e-9 \\
		2 &	\ch{CO3^{2-}(aq)} + \ch{H2O(l)} & \ch{HCO3^{-}(aq)} + \ch{OH^{-}(aq)} & $\Kb{}$ & 2.1e-4 \\
		3 &	\ch{CO2(aq)} + \latecancel<7->{\ch{H2O(l)}} & \latecancel<7->{\ch{H2CO3(aq)}} & $K_h$ & 	1.7e-3 \\
		4 &	\ch{H2CO3(aq)} + \ch{H2O(l)} & \ch{HCO3^{-}(aq)} + \ch{H3O^{+}(aq)} & $\Ka{}$ & 4.46e-7 \\
		5 &	\ch{H3O^{+}(aq)} + \ch{OH^{-}(aq)} & \ch{2 H2O(l)} & $1/\Kw{}$ & 1.0e14  \\ \midrule
		6 &	\ch{CaCO3(s) + CO2(aq) + H2O(l)} & \ch{Ca^{2+}(aq) +
		2 HCO3^{-}(aq)} & \Keq & 7.2e-8 \\
	\end{tabularx}
	\end{center}
}
\end{frame}

\begin{frame}[allowframebreaks]{Equilibrium and Thermodynamics}
	\begin{itemize}
		\item \textbf{Enthalpy} ($\bm{\Delta H}$) is the heat absorbed
		or released when a reaction takes place under \alert{constant} applied \alert{pressure}.
			\begin{center}
				\parbox{0.8\linewidth}{
					\begin{itemize}
						\item [$+\Delta H$] Heat is
							absorbed (endothermic)
						\item [$-\Delta H$] Heat is
							released (exothermic)
					\end{itemize}
					}
			\end{center}
		\item \textbf{Entropy} ($\bm{\Delta S}$) is the heat absorbed
			divided by the temperature when a \alert{reversible}
			reaction takes place at a \alert{constant temperature}.
			\begin{align*}
				\Delta S = \dfrac{q_{\text{rev}}}{T}
			\end{align*}
	\end{itemize}

	\framebreak
	
	\begin{itemize}
		\item The \textbf{Gibbs free energy} ($\Delta G$) declares
			whether a reaction is favorable:
			\begin{align*}
				\Delta G = \Delta H - T \Delta S
			\end{align*}
			\begin{center}
				\parbox{0.8\linewidth}{
					\begin{itemize}
						\item [$+\Delta G$] Unfavorable,
							non-spontaneous
						\item [$-\Delta G$] Favorable,
							spontaneous
					\end{itemize}
					}
			\end{center}
			The free energy is related to the equilibrium constant,
			\Keq,
			\begin{align*}
				\Delta G^\circ = -RT \ln \Keq
			\end{align*}
			where $R$ is the universal gas constant
			(\SI{8.314}{\joule\per\mole\per\kelvin})
				
			Note that this relation is for the
			\alert{standard}
			Gibbs free energy:
			\begin{align*}
				\Delta G^\circ = \Delta H^\circ
				- T \Delta S^\circ
			\end{align*}
	\end{itemize}
\end{frame}

\mode<presentation>{{
	\usebackgroundtemplate{\includegraphics[width=\paperwidth]{elephantfoot.jpg}}
	\begin{frame}[b,plain]
		\begin{block}{Nuclear Meltdown}
		The Chernobyl disaster was a \textbf{highly favorable} (in terms of
	$\Delta G$) chemical reaction.
\end{block}
\end{frame}}}

\begin{frame}{Le Ch\^atelier's Principle}
	\begin{itemize}
		\item If stress is applied to a system at equilibrium, the equilibrium will shift in order to reduce the
			added stress.
		\item The \alert{reaction quotient} predicts the direction of
			the reaction when equilibrium is perturbed:
			\begin{align*}
				\ch{aA + bB <=> cC + dD} \qquad Q =
				\dfrac{[\ch{C}]^c[\ch{D}]^d}
				{[\ch{A}]^a[\ch{B}]^b}
			\end{align*}
			What happens when\ldots
			\begin{itemize}
				\item[\ldots] $Q > K$
				\item[\ldots] $Q < K$
				\item[\ldots] $Q = K$
			\end{itemize}
	\end{itemize}
\end{frame}

\begin{frame}[t]{Reaction Quotient Example}
	\begin{align*}
		\ch{KNO3(s) <=> K^{+}(aq) + NO3^{-}(aq)}
	\end{align*}
	If both \ch{K+} and \ch{NO3-} concentrations are \SI{3.30}{\Molar}, are
	we at equilibrium at \SI{28.4}{\celsius} where $\Keq = 10.9$?

	\mode<article>{\vspace{\stretch{1}}}

	\note<1>{
		\begin{align*}
			Q &= [\ch{K+}][\ch{NO3-}] \\
			&= (3.30)(3.30) = 10.9
		\end{align*}

		\begin{center}
			\fbox{$Q = K \therefore$ the reaction is at
			equilibrium.}
		\end{center}}

	\begin{itemize}[<+(1)->]
		\item What happens if we make [\ch{K+}] = \SI{5.50}{\Molar}?

			\mode<article>{\vspace{\stretch{1}}}

	\note<2>{
		\begin{align*}
			Q &= [\ch{K+}][\ch{NO3-}] \\
			&= (5.50)(3.30) = 18.2
		\end{align*}

		\begin{center}
			\fbox{\parbox{0.8\linewidth}{
				\centering
				$Q \neq K \therefore$ the reaction is not at
				equilibrium.
		
				$Q > K \therefore$ the reaction shifts towards
				the reactants.}
				}
		\end{center}}

		\mode<article>{\clearpage}

		\item When does it stop?

			\mode<article>{\vspace{\stretch{1}}}
	\end{itemize}

	\note<3>{
		\begin{align*}
			10.9 &= (5.50 - x)(3.30 - x) \\
			10.9 &= x^2 - 8.80x + 18.2 \\
			0 &= x^2 - 8.80x + 7.3 \\
			x &= 0.93 \text{ or } 7.8
		\end{align*}

		\begin{center}
		\fbox{\parbox{0.8\linewidth}{\centering
			The reaction precipitates \SI{0.93}{\mole}~\ch{KNO3},
			yielding [\ch{K+}]~=~\SI{4.57}{\Molar} and
			[\ch{NO3-}]~=~\SI{2.37}{\Molar}.}}
		\end{center}}
\end{frame}

\begin{frame}{Le Ch\^atelier's Principle: $\Delta T$}
	\Keq{} is dependent on temperature as per
	\begin{align*}
		\Delta G^\circ &= -RT \ln \Keq \\
		\intertext{rearranging \ldots}
		\Keq{} &= e^{-\Delta G^\circ/RT} = e^{-(\Delta H^\circ -
		T\Delta S^\circ)/RT} &= e^{(-\Delta H^\circ/RT + \Delta
		S^\circ/R)} \\
		& &= e^{-\Delta H^\circ/RT} \cdot e^{\Delta S^\circ/R}
	\end{align*}
	Therefore,
	\begin{itemize}
		\item \Keq{} of an endothermic reaction ($\Delta H^\circ > 0$)
			increases if $T\uparrow$
		\item \Keq{} of an exothermic ($\Delta H^\circ < 0$)
			decreases if $T\uparrow$
	\end{itemize}
\end{frame}

\vspace{\stretch{-1}}

\frame{\section{Applications of Equilibrium Constants}
	\begin{learningobjectives}
	\item Create a $K$ expression for any reaction at equilibrium.
	\item Understand how to solve specific $K$ equilibria:
		\begin{itemize}
			\item solubility
			\item formation
			\item acid-base
		\end{itemize}
	\end{learningobjectives}
}

\begin{frame}[t]{Applications of Equilibrium Constants}
	\begin{description}
		\item<1->[Solubility:]
		\ch{Ag2CrO4(s) <=> 2 Ag^{+}(aq) +
			CrO4^{2-}(aq)} \quad $\Ksp{} =
			\num{1.2e-12}$
		\item<2->[Formation:]
		\ch{Ca^{2+}(aq) + EDTA^{4-}(aq)
			<=> CaEDTA^{2-}(aq)} \quad $\Kf{} =
			\num{4.9e10}$
		\item<3->[Acid:]
		\ch{CH3COOH(aq) + H2O(l) <=>
			CH3COO^{-}(aq) + H3O^{+}(aq)} \quad $\Ka{} =
			\num{1.8e-5}$
		\item<4->[Base:]
			\ch{NH3(aq) + H2O(l) <=> NH4^{+}(aq)
			+ OH^{-}(aq)} \quad $\Kb{} = \num{1.8e-5}$
		\item<5->[Hydrolysis:]
		\ch{2 H2O(l) <=> H3O^{+}(aq) +
			OH^{-}(aq)} \quad $\Kw{} = \num{1.00e-14}$
		\item<6->[Others:]
			$	K_{\mathrm{x}} =
				\frac{\prod[\text{products}]}
				{\prod[\text{reactants}]}$
	\end{description}
\end{frame}

\begin{frame}[t]{Solubility Product Example}
	What is the solubility of silver chromate in pure water and in a
	solution of \SI{0.0500}{\Molar} aluminum chromate?
	\begin{align*}
		\ch{Ag2CrO4(s) <=> 2 Ag^{+}(aq) + CrO4^{2-}(aq)} \qquad \Ksp{} =
		\num{1.2e-12}
	\end{align*}

	\note<1>{\footnotesize
	\textbf{Pure water:}

	\begin{center}
		\begin{tabular} {c*{5}{r}}
			& \ch{Ag2CrO4(s)} & \ch{<=>} & \ch{2 Ag^{+}(aq)} &+&
			\ch{CrO4^{2-}(aq)} \\
			I & solid && 0 && 0 \\
			C & $-x$ && $+2x$ && $+x$ \\ \midrule
			E & $\text{solid}-x$ && $2x$ && $x$
		\end{tabular}
	\end{center}

	\begin{align*}
		\Ksp{} &= [\ch{Ag^{+}}]^2[\ch{CrO4^{2-}}] \\
		\num{1.2e-12} &= (2x)^2(x) \\
		&= 4x^3 \\
		x &= \sqrt[3]{\num{1.2e-12}/4} \\
		&= \num{6.69e-5} \equiv \num{7e-5}
	\end{align*}}

	\visible<2-|article:0>{
	\begin{center}
		\textbf{\usebeamercolor[fg]{alerted text}The common ion effect!}
	\end{center}}

	\mode<article>{\clearpage}


\note<2>{\footnotesize
	\textbf{Aluminum chromate solution:}

	\begin{center}
		\ch{Al2(CrO4)3(s) -> 2 Al^{3+}(aq) + 3 CrO4^{2-}(aq)}

		$ \SI{0.0500}{\Molar}~\ch{Al2(CrO4)3} \times
		\dfrac{\SI{3}{\mol}~\ch{CrO4^{2-}}}{\SI{1}{\mole}~\ch{Al2(CrO4)3}}$

		\begin{tabular} {c*{5}{r}}
			& \ch{Ag2CrO4(s)} & \ch{<=>} & \ch{2 Ag^{+}(aq)} &+&
			\ch{CrO4^{2-}(aq)} \\
			I & solid && 0 && 0.1500 \\
			C & $-x$ && $+2x$ && $+x$ \\ \midrule
			E & $\text{solid}-x$ && $2x$ && $0.1500 + x$
		\end{tabular}
	\end{center}

	\begin{columns}
		\column{0.5\linewidth}
		\begin{align*}
			\Ksp{} &= [\ch{Ag+}]^2[\ch{CrO4^{2-}}] \\
			\num{1.2e-12} &= (2x)^2(0.1500 + x) \\
			&= 4x^2 \times 0.1500 \\
		\end{align*}
		\column{0.5\linewidth}
		\begin{align*}
			x &= \sqrt{\num{8e-12}/4} \\
			&= \num{1.4e-6}
		\end{align*}

		$x$ is 0.0009\% of 0.1500 -- no worries!

	\end{columns}
	}


\end{frame}

\begin{frame}{Formation: Lewis Acid/Base Reactions}
	If anion \ch{X-} precipitates metal \ch{M+}, often a high concentration
	of \ch{X-} causes solid \ch{MX} to redissolve.
	\begin{itemize}
		\item Increased solubility is due to formation of \alert{complex
			ions}.
		\item Lewis acids and bases react to form
			\alert{coordinate covalent bonds}.
	\end{itemize}

	For example,
	\begin{align*}
		\ch{Pb^{2+}\aq{} + I^{-}\aq{} &<=> PbI^{+}\aq{}} \\
		\ch{Pb^{2+}\aq{} + 2 I^-\aq{} &<=> PbI2\aq{} <=> PbI2\sld{}} \\
		\ch{Pb^{2+}\aq{} + 3 I^-\aq{} &<=> PbI3^-\aq{}}
	\end{align*}

	\visible<+(1)->{\textbf{What product is present?}}
\end{frame}

\begin{frame}{Formation: Stepwise vs Cumulative}
	There can only be one concentration of \ch{Pb^{2+}} in solution, so
	solving one equilibrium must satisfy all equilibria.
	\begin{itemize}[<+(1)->]
		\item $K_{\mathrm{i}}$ will be used for \alert{stepwise} formation
			constants:
			\begin{center}
			\begin{tabular} {l@{ + \ch{I^-\aq{} <=>} }l c}
				\ch{Pb^{2+}\aq{}} & \ch{PbI^+\aq{}} & $K_1$ \\
				\ch{PbI^+\aq{}} & \ch{PbI2\aq{}} & $K_2$ \\
				\ch{PbI2\aq{}} & \ch{PbI3^-\aq{}} & $K_3$ \\
				\ch{PbI3^-\aq{}} & \ch{PbI4^{2-}\aq{}} & $K_4$
			\end{tabular}
			\end{center}
		\item $\beta_{\mathrm{i}}$ will be used for \alert{cumulative} formation
			constants:
			\begin{center}
				\begin{tabular} {l@{ + }r@{ \ch{<=>}
					}l l}
					\ch{Pb^{2+}\aq{}} & \ch{I^-\aq{}} &
					\ch{PbI^+\aq{}} &
					$K_1$ \\
					\ch{Pb^{2+}\aq{}} & \ch{2 I^-\aq{}} & \ch{PbI2\aq{}} & $\beta_2 =
				K_1K_2$ \\
					\ch{Pb^{2+}\aq{}} & \ch{3 I^-\aq{}} &
					\ch{PbI3^-\aq{}} & $\beta_3 =
				K_1K_2K_3$ \\
					\ch{Pb^{2+}\aq{}} & \ch{4 I^-\aq{}} & \ch{PbI4^{2-}\aq{}} &
				$\beta_4 = K_1K_2K_3K_4$
			\end{tabular}
			\end{center}
	\end{itemize}
\end{frame}

\begin{frame}{Formation: Competing Equilibria}
	\begin{columns}
		\column{0.5\textwidth}
		\begin{itemize}
			\item As [\ch{I-}] increases, [\ch{Pb_{total}}]
				decreases due to common ion effect.
			\item At very high [\ch{I-}], the \ch{PbI2(s)}
				redissolves due to complexation.
		\end{itemize}
		\column{0.5\textwidth}
		\includegraphics[scale=0.4]{PbI2-complex.jpg}
	\end{columns}
\end{frame}

\begin{frame}[t]{Formation Example}
	Find the concentrations of \ch{PbI+}, \ch{PbI2\aq{}}, \ch{PbI3-}, and
	\ch{PbI4^{2-}} in a solution saturated with \ch{PbI2\sld} and containing
	dissolved \ch{I-} with a concentration of \SI{0.0010}{\Molar}.

	\medskip

	\begin{tabular}{L @{ $=$ } S[table-format=1.1e-1]}
		K_\text{sp}  & 7.9e-9 \\
		K_1     & 1.0e2  \\
		\beta_2 & 1.4e3  \\
		\beta_3 & 8.3e3  \\
		\beta_4 & 3.0e4
	\end{tabular}

	\mode<article>{\vspace{\stretch{1}}}

	\note{ \small
		\begin{align*}
			K_1     &= \frac{[\ch{PbI+     }]}{[\ch{Pb^{2+}}][\ch{I-}]  } &
			\beta_2 &= \frac{[\ch{PbI2\aq  }]}{[\ch{Pb^{2+}}][\ch{I-}]^2} &
			\beta_3 &= \frac{[\ch{PbI3-    }]}{[\ch{Pb^{2+}}][\ch{I-}]^3} &
			\beta_4 &= \frac{[\ch{PbI4^{2-}}]}{[\ch{Pb^{2+}}][\ch{I-}]^4} &
		\end{align*}
		\begin{align*}
			\intertext{We need [\ch{Pb^{2+}}]!}
			K_\text{sp} &= [\ch{Pb^{2+}}][\ch{I-}]^2 &
			\therefore [\ch{Pb^{2+}}] &= \frac{K_\text{sp}}{[\ch{I-}]^2}
			= \frac{\num{7.9e-9}}{\num{0.0010}^2} = \SI{7.9e-3}{\Molar} \\
			&& [\ch{PbI+}] &= K_1[\ch{Pb^{2+}}][\ch{I-}] = \SI{7.9e-4}{\Molar} \\
			&& [\ch{PbI2\aq}] &= \beta_2[\ch{Pb^{2+}}][\ch{I-}]^2 = \SI{1.1e-5}{\Molar} \\
			&& [\ch{PbI3-  }] &= \beta_3[\ch{Pb^{2+}}][\ch{I-}]^3 = \SI{6.6e-8}{\Molar} \\
			&& [\ch{PbI4^{2+}}] &= \beta_4[\ch{Pb^{2+}}][\ch{I-}]^4 = \SI{2.4e-10}{\Molar} \\
		\end{align*}
	}
\end{frame}

\clearpage

\begin{frame}[t]{Acid Dissociation Example}
	What is the pH of a 5.0\% (w/w) vinegar solution? ($\Ka{} = \num{1.8e-5}$)

	\mode<article>{\vspace{15em}}

	\note{\footnotesize
		\vspace{-2em}\begin{align*}
		\dfrac{\SI{5}{\gram}~\ch{HAc}}{\SI{100.0}{\gram~sol}} \times
		\overbrace{\dfrac{\SI{1000}{\gram~sol}}{\SI{1}{\liter}}}^{\mathclap{\text{assume}~d = d_{\ch{H2O}} = \SI{1}{\gram\per\milli\liter}}} \times
		\dfrac{\SI{1}{\mole}~\ch{HAc}}{\SI{60.06}{\gram}} =
		\SI{0.83}{\Molar}
	\end{align*}

	\begin{center}
		\begin{tabular} {@{}c*{5}{r}}
			& \ch{CH3COOH(aq)} & \ch{+ H2O(l) <=>} &
			\ch{CH3COO^{-}(aq)} &+& \ch{H3O^{+}(aq)} \\
			I & 0.83 && 0 && 0 \\
			C & $-x$ && $+x$ && $+x$ \\ \midrule
			E & $0.83-x$ && $x$ && $x$
		\end{tabular}
	\end{center}

	\begin{columns}
		\column{0.5\linewidth}
		\begin{align*}
			\num{1.8e-5} &= \dfrac{[\ch{CH3COO-}][\ch{H3O+}]}{\ch{CH3COOH}}
			\\
			&= \dfrac{(x)(x)}{0.83-x} = \dfrac{x^2}{0.83} \\
			x &= \SI{3.9e-3}{\Molar}~\ch{H3O+}
		\end{align*}
		\column{0.5\linewidth}
		$x$ is 0.5\% of 0.83
		\begin{align*}
			\text{pH} &= -\log [\ch{H3O+}] \\
			&= -log(0.0039) \\
			&= 2.41
		\end{align*}
	\end{columns}
	}
\end{frame}

\begin{frame}[t]{Base Dissociation Example}
	What is the pH of a solution that is \SI{0.175}{\Molar}~\ch{NH3}? ($\Kb{}
	= \num{1.8e-5}$)

	\mode<article>{\vspace{18em}}

\note{
	\begin{center}
		\begin{tabular} {@{}c*{5}{r}}
			& \ch{NH3(aq)} & \ch{ + H2O(l) <=>} &
			\ch{NH4^{+}(aq)} &+& \ch{OH^{-}(aq)} \\
			I & 0.175 && 0 && 0 \\
			C & $-x$ && $+x$ && $+x$ \\ \midrule
			E & $0.175-x$ && $x$ && $x$
		\end{tabular}
	\end{center}

	\begin{columns}
		\column{0.5\linewidth}
		\begin{align*}
			\num{1.8e-5} &= \dfrac{[\ch{NH4+}][\ch{OH-}]}{\ch{NH3}}
			\\
			&= \dfrac{(x)(x)}{0.175-x} = \dfrac{x^2}{0.175} \\
			x &= \SI{1.8e-3}{\Molar}~\ch{OH-}
		\end{align*}
		\column{0.5\linewidth}
		\begin{align*}
			\text{pOH} &= -\log [\ch{OH-}] \\
			&= -\log(0.0018) = 2.75 \\
			\text{pH} &= 14.00 - \text{pOH} = 11.25
		\end{align*}
	\end{columns}
	}
\end{frame}

\begin{frame}{Recall: Acid/Base Strength}
	\begin{itemize}
		\item Acid/base dissociation equilibria are considered for
			\alert{weak} acids and bases.
		\item Strong acids and bases dissociate \alert{completely} ---
			i.e. $\Ka{}$ or $\Kb{}$ are so large that it is not worth
			calculating
	\end{itemize}

	{\small
		\begin{tabularx}{\textwidth} {*{2}{l>{\raggedright\arraybackslash}X}}
		\toprule
		\multicolumn{2}{l}{\textit{Acids}} &
		\multicolumn{2}{l}{\textit{Bases}} \\
		\bfseries Name & \bfseries Formula & \bfseries Name & \bfseries Formula \\
		\midrule
		\ch{HCl} & Hydrochloric acid & \ch{LiOH} & Lithium hydroxide \\
		\ch{HBr} & Hydrogen bromide & \ch{NaOH} & Sodium hydroxide \\
		\ch{HI} & Hydrogen iodide & \ch{KOH} & Potassium hydroxide \\
		\ch{H2SO4} & Sulfuric acid & \ch{RbOH} & Rubidium hydroxide \\
		\ch{HNO3} & Nitric acid & \ch{CsOH} & Cesium hydroxide \\
		\ch{HClO4} & Perchloric acid & \ch{R4NOH} & Quaternary ammonium
		hydroxide \\
		\bottomrule
	\end{tabularx}
}
\end{frame}

\begin{frame}[allowframebreaks]{Polyprotic species}
	\begin{itemize}
		\item Some compounds can donate or accept more than one proton.
			For example,
			\begin{align*}
				\ch{H3PO4 + H2O &<=> H3O+ + H2PO4-} \\
				\ch{H2PO4- + H2O &<=> H3O+ + HPO4^{2-}} \\
				\ch{HPO4^{2-} + H2O &<=> H3O+ + PO4^{3-}}
			\end{align*}
		\item It is harder to remove (or add) each successive proton,
			thus
			\begin{equation*}
				K_\text{a1} >
				K_\text{a2} >
				K_\text{a3} > \cdots \quad\text{and}\quad
				K_\text{b1} >
				K_\text{b2} >
				K_\text{b3} > \cdots
			\end{equation*}
		\item Which \Ka\ and \Kb\ do we use to calculate \Kw?
	\end{itemize}

	\framebreak

	For \ch{H3PO4},
	\begin{reaction*}
		H3PO4     <=>[ $K$_{a1} ][ $K$_{b3} ]
		H2PO4-    <=>[ $K$_{a2} ][ $K$_{b2} ]
		HPO4^{2-} <=>[ $K$_{a3} ][ $K$_{b1} ]
		PO4^{3-}
	\end{reaction*}
	\begin{equation*}
		K_\text{a1}K_\text{b3} =
		K_\text{a2}K_\text{b2} =
		K_\text{a3}K_\text{b1} =
		K_\text{w}
	\end{equation*}

	\vspace{2em}

	For EDTA (\ch{H6Y^{2+}}) --- a hexavalent acid,

	\medskip

	\resizebox{\linewidth}{!}{%
			\ch{%
			H6Y^{2+} <=>[ $K$_{a1} ][ $K$_{b6} ]
		H5Y+     <=>[ $K$_{a2} ][ $K$_{b5} ]
		H4Y      <=>[ $K$_{a3} ][ $K$_{b4} ]
		H3Y-     <=>[ $K$_{a4} ][ $K$_{b3} ]
		H2Y^{2-} <=>[ $K$_{a5} ][ $K$_{b2} ]
		HY^{3-}  <=>[ $K$_{a6} ][ $K$_{b1} ] Y^{4-}
}}

	\medskip

	\resizebox{\linewidth}{!}{%
		$K_\text{a1}K_\text{b6} = 
		K_\text{a2}K_\text{b5} = 
		K_\text{a3}K_\text{b4} = 
		K_\text{a4}K_\text{b3} =
		K_\text{a5}K_\text{b2} = 
		K_\text{a6}K_\text{b1} =
		K_\text{w}$
}
\end{frame}

\begin{frame}[t]{Ladder Diagrams}
	An easy way of finding the \alert{predominant} specie in a mixture.

	\begin{example}
		\begin{columns}
			\column{0.4\linewidth}
			\begin{center}
		\resizebox{!}{20pt}{%
		\chemfig{^+H_3N-CH(-[::-90]CH_2-[:-90]CH_2-[:-90]CH_2-[:-90]NH
		-[:-90]C(=[:-90]NH_2^+)-NH_3^+)-C(=[:90]O)-OH}
	}
\end{center}
\column{0.5\linewidth}
\begin{align*}
	K_\text{a1} &= \num{1.50e-2} \\
	K_\text{a2} &= \num{1.02e-9} \\
	K_\text{a3} &= \num{3.3e-13}
\end{align*}
\end{columns}
	\end{example}
\end{frame}

%%% END PART 1
\frame{\section{Systematic Treatment of Equilibrium}
	\begin{learningobjectives}
	\item Balance all charged particles in a solution.
	\item Balance all mass across products and reactants.
	\item Determine the concentration of all species in a solution at equilibrium.
	\end{learningobjectives}
}

\begin{frame}[t]{How do we solve complex equilibria?}
	Recall:

	\resizebox{\linewidth}{!}{%
                \begin{tabular} {r @{\ch{<=>}} l               
                r@{ = }S[table-format=1.2e-2]}
                     \ch{CaCO3(s)} & \ch{Ca^{2+}(aq)} + \ch{CO3^{2-}(aq)} & $\Ksp{}$ & 4.5e-9 \\
                     \ch{CO3^{2-}(aq)} + \ch{H2O(l)} & \ch{HCO3^{-}(aq)} + \ch{OH^{-}(aq)} & $\Kb{}$ & 2.1e-4 \\
                     \ch{CO2(aq)} + \ch{H2O(l)} & \ch{H2CO3(aq)} & $K_h$ &         1.7e-3 \\
                     \ch{H2CO3(aq)} + \ch{H2O(l)} & \ch{HCO3^{-}(aq)} + \ch{H3O^{+}(aq)} & $\Ka{}$ & 4.46e-7 \\
                     \ch{H3O^{+}(aq)} + \ch{OH^{-}(aq)} & \ch{2 H2O(l)} & $1/\Kw{}$ & 1.0e14  \\ \midrule
                     \ch{CaCO3(s) + CO2(aq) + H2O(l)} & \ch{Ca^{2+}(aq) +    
                2 HCO3^{-}(aq)} & \Keq & 7.2e-8 \\                          
        \end{tabular}
	}

	\bigskip

	If you need to solve for the concentration of each species in the above
	reactions, what do?
\end{frame}

\begin{frame}{$n$ equations\ldots}
	\begin{itemize}[<+->]
		\item In order to solve a problem containing 1 unknown, you
			simply need 1 equation containing that value:
			\begin{align*}
				\si{\gram}~\ch{Ca^{2+}} =
				y~\si{\mole}~\ch{Y^{4-}} \times
				\dfrac{\SI{1}{\mole}~\ch{Ca^{2+}}}
				{\SI{1}{\mole}~\ch{Y^{4-}}} \times
				\dfrac{\SI{40.08}{\gram}~\ch{Ca^{2+}}}
				{\SI{1}{\mole}}
			\end{align*}
		\item To solve for 2 unknowns, you need 2 equations:
			\begin{align*}
				A^1_{\text{T}} &=
				\epsilon^1_{\ch{Mn}}bC_{\ch{Mn}} +
				\epsilon^1_{\ch{Cr}}bC_{\ch{Cr}} \\
				A^2_{\text{T}} &=
				\epsilon^2_{\ch{Mn}}bC_{\ch{Mn}} +
				\epsilon^2_{\ch{Cr}}bC_{\ch{Cr}}
			\end{align*}
		\item To solve $n$ unknowns, you need $n$ equations.
			\begin{itemize}
				\item Include \emph{all} the \alert{equilibrium
					constants}
				\item Include a statement of electroneutrality
					--- the \alert{charge balance}
				\item Include a statement of the conservation of
					matter --- the \alert{mass balance}
			\end{itemize}
	\end{itemize}
\end{frame}

\begin{frame}{Charge Balance}
	\begin{itemize}
		\item An algebraic statement of electroneutrality.
		\item The sum of the positive charges in solution \alert{equals}
			the sum of negative charges in solution.

			\begin{align*}
				\sum(\text{positive charges}) &=
				\sum(\text{negative charges}) \\
				n_1[\ch{C1}] + n_2[\ch{C2}] + \cdots &=
				m_1[\ch{A1}] + m_2[\ch{A2}] + \cdots
			\end{align*}
		
			where
		
			\begin{tabularx}{\linewidth} {c X}
				$n$ & is the charge of each \alert{cation} of
				concentration \ch{[C]} \\
				$m$ & is the charge of each \alert{anion} of
				concentration \ch{[A]}
			\end{tabularx}
	\end{itemize}
\end{frame}

\vspace{\stretch{-1}}

\begin{frame}[t]{Charge Balance Example 1}
	Write the charge balance for an aqueous solution containing
	\SI{0.10}{\Molar}~\ch{Mg(NO3)2}, \SI{0.20}{\Molar}~\ch{HCN},
	\SI{0.15}{\Molar}~\ch{KCN}, \SI{0.10}{\Molar}~\ch{Ag4Fe(CN)6},
	\ch{H3O+}, \ch{OH-}, ignoring any further equilibria for this problem.
	Show that it balances.

	\mode<article>{\vspace{\stretch{1}}}

	\note{\scriptsize
	What species do we have present?
	\begin{align*}
		\ch{Mg(NO3)2 &<=> Mg^{2+} + 2 NO3-} \\
		\ch{HCN &<=> H+ + CN-} \\
		\ch{KCN &<=> K+ + CN-} \\
		\ch{Ag4Fe(CN)6 &<=> 4 Ag+ + Fe(CN)6^{4-}} \\
		\ch{2 H2O &<=> H3O+ + OH-}
	\end{align*}

	\begin{align*}
		2[\ch{Mg^{2+}}] + [\ch{H3O+}] + [\ch{K+}] + [\ch{Ag+}] &=
		[\ch{NO3-}] + [\ch{CN-}] + 4[\ch{Fe(CN)6^{4-}}] + [\ch{OH-}] \\
		2(0.10) + (0.20 + \num{1e-7}) + 0.15 + 0.40 &= 0.20 + (0.20 +
		0.15) + 4(0.10) + \num{1e-7}
		\\
		\SI{0.95}{\Molar} &= \SI{0.95}{\Molar}
	\end{align*}
	}
\end{frame}

\begin{frame}[t]{Charge Balance Example 2}
	Write the charge balance for a solution of sodium dihydrogen phosphate
	(including equilibria).

	\mode<article>{\vspace{\stretch{1}}}

	\note{\scriptsize
	What species do we have present?
	\begin{align*}
		\ch{NaH2PO4 &<=> Na+ + H2PO4-} \\
		\ch{H2PO4- &<=> H3O+ + HPO4^{2-}} \\
		\ch{H2PO4- + H3O+ &<=> H3PO4} \\
		\ch{HPO4^{2-} &<=> H3O+ + PO4^{3-}} \\
		\ch{2 H2O &<=> H3O+ + OH-}
	\end{align*}


	\begin{align*}
		[\ch{Na+}] + [\ch{H3O+}] &= [\ch{H2PO4-}] + 2[\ch{HPO4^{2-}}] +
		3[\ch{PO4^{3-}}] + [\ch{OH-}]
	\end{align*}
	}
\end{frame}

\clearpage

\begin{frame}{Mass Balance}
	\begin{itemize}
		\item A statement of the conservation of matter.
		\item The quantity of all species in a solution containing a
			particular atom (or group of atoms) \alert{must equal}
			the amount of that atom (or group) delivered to the
			solution.
		\item We generally have two mass balances, one for the anion and
			another for the cation. These can be combined into a
			single expression.
	\end{itemize}
\end{frame}

\vspace{\stretch{-1}}

\begin{frame}[t]{Mass Balance Example 1}
	Write the mass balances for an aqueous solution containing
	\SI{0.10}{\formal}~\ch{Mg(NO3)2}.

	\mode<article>{\vspace{\stretch{1}}}

\note{
	\begin{center}
		\ch{Mg(NO3)2 <=> Mg^{2+} + 2 NO3-}
	\end{center}

	\begin{align*}
		\SI{0.10}{\formal} &= [\ch{Mg^{2+}}] \\
		2[\ch{Mg^{2+}}] &= [\ch{NO3-}] \\
		\SI{0.20}{\formal} &= [\ch{NO3-}]
	\end{align*}
	}
\end{frame}

\begin{frame}[t]{Mass Balance Example 2}
	Write the mass balances for a \SI{0.20}{\formal} solution of sodium
	sulfate.

	\mode<article>{\vspace{\stretch{1}}}

	\note{
	\begin{reaction*}
		Na2SO4 <=> 2 Na+ + SO4^{2-} + HSO4-
	\end{reaction*}

	\begin{align*}
		\SI{0.20}{\formal} &= [\ch{SO4^{2-}}] + [\ch{HSO4-}] \\
		2[\text{total sulfates}] &= [\ch{Na+}] \\
		\SI{0.40}{\formal} &= [\ch{Na+}]
	\end{align*}
	}
\end{frame}
	
\begin{frame}[t]{Mass Balance Example 3}
	Write the mass balance for a saturated solution of copper(II) iodate.

	\mode<article>{\vspace{\stretch{1}}}

	\note{
		\begin{reaction*}
			Cu(IO3)2 <=> Cu^{2+} + 2 IO3-
		\end{reaction*}
	
		\begin{align*}
			2[\ch{Cu^{2+}}] &= [\ch{IO3-}] \\
		\end{align*}
	}
\end{frame}

\clearpage

\begin{frame}[t]{Mass Balance Example 4}
	Write the mass balance for a saturated silver phosphate solution.

	\mode<article>{\vspace{\stretch{1}}}

\note{
	\begin{reactions*}
		Ag3PO4 &<=> 3 Ag+ + PO4^{3-} \\
		PO4^{3-} + H2O &<=> HPO4^{2-} + OH- \\
		HPO4^{2-} + H2O &<=> H2PO4- + OH- \\
		H2PO4- + H2O &<=> H3PO4 + OH-
	\end{reactions*}

	\begin{align*}
		3[\text{total phosphates}] &= [\ch{Ag+}] \\
		3\big( [\ch{PO4^{3-}}] + [\ch{HPO4^{2-}}] + [\ch{H2PO4-}] +
		[\ch{H3PO4}]
		\big) &= [\ch{Ag+}]
	\end{align*}
	}
\end{frame}

\begin{frame}{The Systematic Treatment of Equilibrium}
	\begin{enumerate}
		\item Write the \textbf{\usebeamercolor[fg]{alerted text}pertinent reactions}.
		\item Write the \textbf{\usebeamercolor[fg]{alerted text}charge balance} equation.
		\item Write the \textbf{\usebeamercolor[fg]{alerted text}mass balance} equation(s).
		\item Write the \textbf{\usebeamercolor[fg]{alerted text}equilibrium constant} for each chemical
			reaction.
		\item \textbf{\usebeamercolor[fg]{alerted text}Count} the equations and unknowns.
			\begin{itemize}
				\item The numbers should equal or you need to
					find more equilibria or fix/approximate some
					concentrations.
			\end{itemize}
		\item \textbf{\usebeamercolor[fg]{alerted text}Solve} for all the unknowns.
	\end{enumerate}
\end{frame}

\begin{frame}[t]{Systematic Treatment Example 1}
	What is the \pH{} of a \SI{0.50}{\Molar} solution of \ch{HF}?

	\note{%
	\begin{multicols}{2}
		\footnotesize
		\begin{enumerate}
			\item Pertinent reactions
				\begin{reactions*}
					HF + H2O &<=> H3O+ + F- \\
					2 H2O &<=> H3O+ + OH-
				\end{reactions*}
			\item Charge balance
				\begin{align*}
					[\Oxo] &= [\Hyd] + [\ch{F-}]
				\end{align*}
			\item Mass balance
				\begin{align*}
					\SI{0.50}{\Molar} &= [\ch{HF}] +
					[\ch{F-}]
				\end{align*}
			\item Equilibrium constants
				\begin{align*}
					\Ka{} &=
					\dfrac{[\ch{F-}][\Oxo]}{[\ch{HF}]} =
					\num{6.8e-4} \\
					\Kw{} &= [\ch{H3O+}][\ch{OH-}] =
					\num{1.0e-14}
				\end{align*}
			\item Count equations and unknowns
				\begin{itemize}
					\item 4 unknowns 
					\item 4 equations
				\end{itemize}
			\item Solve
		\end{enumerate}
	\end{multicols}
	}

\end{frame}

\begin{frame}[t]{Systematic Treatment Example 2}
	Calculate the solubility of calcium fluoride in a pH 4.00 buffer.

	\mode<article>{\vspace{\stretch{3}}}
	
	\note{
	\begin{multicols}{2}
		\footnotesize
		\begin{enumerate}
			\item Pertinent reactions
				\begin{reactions*}
					CaF2 &<=> Ca^{2+} + 2 F- \\
					F- + H2O &<=> HF + OH- \\
					2 H2O &<=> H3O+ + OH-
				\end{reactions*}
			\item Charge balance?
			\item Mass balance
				\begin{align*}
					2[\ch{Ca^{2+}}] &= [\ch{HF}] + [\ch{F-}]
					\\
					[\ch{H3O+}] &= \SI{1.0e-4}{\Molar}
				\end{align*}
			\item Equilibrium constants
				\begin{align*}
					\Ksp{} &= [\ch{Ca^{2+}}][\ch{F-}]^2 =
					\num{3.9e-11} \\
					\Kb{} &=
					\dfrac{[\ch{HF}][\ch{OH-}]}{[\ch{F-}]} =
					\num{1.5e-11} \\
					\Kw{} &= [\ch{H3O+}][\ch{OH-}] =
					\num{1.0e-14}
				\end{align*}
			\item Count equations and unknowns
				\begin{itemize}
					\item 5 unknowns 
					\item 5 equations
				\end{itemize}
			\item Solve
		\end{enumerate}
	\end{multicols}
	}

\end{frame}
	\note{
		\footnotesize
		$[\ch{H3O+}] = \SI{1.0e-4}{\Molar}$ is both an equation and a
		solution.
		\begin{align*}
			[\ch{OH-}] &= \dfrac{\num{1.0e-14}}{\num{1.0e-4}} =
			\SI{1.0e-10}{\Molar}
		\end{align*}
		$\Kb{}$ contains [\ch{OH-}], so:
		\begin{align*}
			\dfrac{\num{1.5e-11}}{\num{1.0e-10}} = 0.15 =
			\dfrac{[\ch{HF}]}{[\ch{F-}]} \qquad \therefore
			0.15[\ch{F-}] = [\ch{HF}]
		\end{align*}
		From mass balance,
		\begin{align*}
			2[\ch{Ca^{2+}}] &= [\ch{HF}] + [\ch{F-}] \\
			2[\ch{Ca^{2+}}] &= 0.15[\ch{F-}] +
			[\ch{F-}] = 1.15[\ch{F-}] \\
			1.74[\ch{Ca^{2+}}] &= [\ch{F-}]
		\end{align*}
		}

	\note{
		\footnotesize
		Using $\Ksp{}$,
		\begin{align*}
			\num{3.9e-11} &= [\ch{Ca^{2+}}](1.74[\ch{Ca^{2+}}])^2 \\
			&= 3.02[\ch{Ca^{2+}}]^3 \\
			[\ch{Ca^{2+}}] &= \SI{2.3e-4}{\Molar}
		\end{align*}
		Therefore,
		\begin{align*}
			[\ch{F-}] &= 1.74[\ch{Ca^{2+}}] = \SI{4.1e-4}{\Molar} \\
			[\ch{HF}] &= 0.15[\ch{F-}] = \SI{6.1e-5}{\Molar} \\
		\end{align*}
		
		\begin{framed}
			\begin{columns}
				\column{0.5\textwidth}
				\begin{align*}
					[\ch{H3O+}] &= \SI{1.0e-4}{\Molar} \\
					[\ch{OH-}] &= \SI{1.0e-10}{\Molar} \\
					[\ch{F-}] &= \SI{4.1e-4}{\Molar}
				\end{align*}
				\column{0.5\textwidth}
				\begin{align*}
					[\ch{HF}] &= \SI{6.1e-5}{\Molar} \\
					[\ch{Ca^{2+}}] &= \SI{2.3e-4}{\Molar}
				\end{align*}
			\end{columns}
		\end{framed}
		}

%\begin{frame}<handout:0>{Quiz}
%	\begin{enumerate}
%		\item Write the $K$ expressions for the following
%			reactions:
%
%			\begin{align*}
%				\ch{Pb(NO3)2(aq) &<=> Pb^{2+}(aq) + 2 NO3-(aq)}
%				&\qquad &K_{d1}	\\
%				\ch{Na2SO4(aq) &<=> 2Na+(aq) + SO4^{2-}(aq)}
%				&\qquad &K_{d2} \\
%				\ch{NaNO3(aq) &<=> Na+(aq) + NO3-(aq)} &\qquad
%				&K_{d3} \\
%				\ch{PbSO4(s) &<=> Pb^{2+}(aq) + SO4^{2-}(aq)}
%				&\qquad &\Ksp{}
%			\end{align*}
%		
%		\item Write the \Keq for the following reaction in terms of
%			the $K$ found above:
%
%			\begin{align*}
%				\ch{Pb(NO3)2(aq) + Na2SO4(aq) &<=> PbSO4(s) +
%				2NaNO3(aq)}
%			\end{align*}
%	\end{enumerate}
%\end{frame}

		\mode<article>{\clearpage}

\begin{frame}[t]{Systematic Treatment Example 3}
	Calculate the solubility of silver phosphate in a pH~5.00 buffer.

	\mode<article>{\vfill}

	\note{
	\scriptsize
	\begin{multicols}{2}
		\begin{enumerate}
			\item Pertinent reactions
				\begin{reactions*}
					Ag3PO4 &<=> 3 Ag+ + PO4^{3-} \\
					PO4^{3-} + H2O &<=> HPO4^{2-} + OH- \\
					HPO4^{2-} + H2O &<=> H2PO4- + OH- \\
					H2PO4- + H2O &<=> H3PO4 + OH- \\
					2 H2O &<=> H3O+ + OH-
				\end{reactions*}
			\item Charge balance
			\item Mass balance
				\begin{align*}
					[\ch{Ag+}] &= 3(\ch{[PO4^{3-}] +
					[HPO4^{2-}]} \\
					&\qquad + \ch{[H2PO4^-] + [H3PO4]}) \\
					[\ch{H3O+}] &= \SI{1.0e-5}{\Molar}
				\end{align*}
			\item Equilibrium constants
				\begin{align*}
					\Ksp{} &= [\ch{Ag+}]^3[\ch{PO4^{3-}}] =
					\num{2.8e-18} \\
					K_{b1} &=
					\dfrac{[\ch{HPO4^{2-}}][\ch{OH-}]}{[\ch{PO4^{3-}}]} =
					\num{0.0237} \\
					K_{b2} &=
					\dfrac{[\ch{H2PO4-}][\ch{OH-}]}{[\ch{HPO4^{2-}}]} =
					\num{1.58e-7} \\
					K_{b3} &=
					\dfrac{[\ch{H3PO4}][\ch{OH-}]}{[\ch{H2PO4-}]}
					= \num{1.41e-12} \\
					\Kw{} &= [\ch{H3O+}][\ch{OH-}] =
					\num{1.0e-14}
				\end{align*}
			\item Count equations and unknowns
				\begin{itemize}\tiny
					\item 7 unknowns 
					\item 7 equations
				\end{itemize}
			\item Solve
		\end{enumerate}
	\end{multicols}
	}
\end{frame}

\note{\scriptsize
	$[\ch{H3O+}] = \SI{1.0e-5}{\Molar}$ is both an equation and a
	solution.
	\begin{align*}
		[\ch{OH-}] &= \dfrac{\num{1.0e-14}}{\num{1.0e-5}} =
		\SI{1.0e-9}{\Molar}
	\end{align*}
	All $\Kb{}$ contain [\ch{OH-}], so:
	\begin{align*}
		\dfrac{\num{1.41e-12}}{\num{1.0e-9}} = \num{1.41e-3} &=
		\dfrac{[\ch{H3PO4}]}{[\ch{H2PO4-}]} \\
		(\num{1.41e-3})[\ch{H2PO4-}] &= [\ch{H3PO4}] \\
		(\num{1.58e2})[\ch{HPO4^{2-}}] &= [\ch{H2PO4-}] \\
		(\num{2.37e7})[\ch{PO4^{3-}}] &= [\ch{HPO4^{2-}}]
	\end{align*}
}

\note{\scriptsize
	From mass balance,
	\begin{align*}
		[\ch{Ag+}] &= 3(\ch{[PO4^{3-}] + [HPO4^{2-}] + [H2PO4^-] +
		[H3PO4]}) \\
		&= 3(\ch{[PO4^{3-}]} + (\num{2.37e7})\ch{[PO4^{3-}]} +
		(\num{1.58e2})\ch{[PO4^{2-}]} + (\num{1.41e-3})\ch{[H2PO4-]}) \\
		&= 3(\ch{[PO4^{3-}]} + (\num{2.37e7})\ch{[PO4^{3-}]}
		(\num{1.58e2})(\num{2.37e7})\ch{[PO4^{3-}]} \\
		& \qquad\qquad {} +
		(\num{1.41e-3})(\num{1.58e2})\ch{[PO4^{2-}]}) \\
		&= 3(\ch{[PO4^{3-}]} + (\num{2.37e7})\ch{[PO4^{3-}]} +
		(\num{1.58e2})(\num{2.37e7})\ch{[PO4^{3-}]}  \\
		& \qquad\qquad {} +
		(\num{1.41e-3})(\num{1.58e2})(\num{2.37e7})\ch{[PO4^{3-}]}) \\
		&= \num{1.13e10}\ch{[PO4^{3-}]}
	\end{align*}
	}

\note{\scriptsize
	Using \Ksp{},
	\begin{align*}
		\num{2.8e-18} &=
		((\num{1.13e10})[\ch{PO4^{3-}}])^3[\ch{PO4^{3-}}] \\
		&= (\num{1.44e30})[\ch{PO4^{3-}}]^4 \\
		[\ch{PO4^{3-}}] &= \SI{1.18e-12}{\Molar}
	\end{align*}
	Therefore,
	\begin{align*}
		[\ch{HPO4^{2-}}] &= (\num{2.37e7})[\ch{PO4^{3-}}] =
		\SI{2.80e-5}{\Molar} \\
		[\ch{H2PO4-}] &= (\num{1.58e2})[\ch{HPO4^{2-}}] =
		\SI{4.41e-3}{\Molar} \\
		[\ch{H3PO4}] &= (\num{1.41e-3})[\ch{H2PO4-}] =
		\SI{6.20e-6}{\Molar} \\
		[\ch{Ag+}] &= 3[\text{total phosphates}] = \SI{1.33e-2}{\Molar}
	\end{align*}
	
%	\begin{framed}
%		\begin{columns}
%			\column{0.5\textwidth}
%			\begin{align*}
%				[\ch{H3O+}] &= \SI{1.0e-4}{\Molar} \\
%				[\ch{OH-}] &= \SI{1.0e-10}{\Molar} \\
%				[\ch{F-}] &= \SI{4.1e-4}{\Molar}
%			\end{align*}
%			\column{0.5\textwidth}
%			\begin{align*}
%				[\ch{HF}] &= \SI{6.1e-5}{\Molar} \\
%				[\ch{Ca^{2+}}] &= \SI{2.3e-4}{\Molar}
%			\end{align*}
%		\end{columns}
%	\end{framed}
	}


%\begin{frame}{Some for you to try:}
%	\begin{enumerate}
%		\item Calculate the solubility of aluminum phosphate ($\Ksp{} =
%			\num{9.84e-22}$) at pH~4.00.
%		\item Using a spreadsheet, calculate the solubility of \ch{CaF2}
%			between pH's 1--10. Finally, plot [\ch{Ca^{2+}}] vs pH.
%		\item Calculate the solubility of barium oxalate at pH~2.00.
%	\end{enumerate}
%\end{frame}

%%% END PART 2
%
%\frame{\section{Activity}
%	\begin{learningobjectives}
%	\item Explain why equilibrium constants differ from tabulated values at higher concentrations.
%	\item Calculate ionic strength of solutions.
%	\item Use ionic strength to predict the activity of species in solution.
%	\end{learningobjectives}
%}
%
%\begin{frame}[t]{Why is the equilibrium constant not enough?}
%	Consider the solubility of mercury(I) bromide, $\Ksp{} =
%	\num{5.6e-23}$.
%	\note{%
%	\begin{itemize}
%		\item The equilibrium is:
%			\begin{align*}
%				\ch{Hg2Br2\sld{} <=> Hg2^{2+}\aq{} + 2
%				Br^{-}\aq{}}
%				\qquad \Ksp{} &= \num{5.6e-23} \\
%				&= [\ch{Hg2^{2+}}][\ch{Br-}]^2
%			\end{align*}
%		\item Using ICE, $x = [\ch{Hg2^{2+}}] = \SI{2.4e-8}{\Molar}$
%		\item When \ch{KNO3} is added, the Solubility can increase such that $[\ch{Hg2^{2+}}] =
%			\SI{4.1e-8}{\Molar}$, a 46\% increase in solubility!
%		\item We find that whenever an \alert{inert salt} is added to a
%			solution of a sparingly soluble salt, the solubility of
%			the latter is increased.
%	\end{itemize}
%}
%
%	\vfill
%
%	\visible<+(1)->{What \emph{should} happen when we add \ch{KNO3}?}
%
%	\vfill
%
%	\begin{block}<+(1)->{Inert salt}
%		Any electrolyte that is neither a common ion nor will react with
%		the ions of the sparingly soluble species.
%	\end{block}
%\end{frame}
%
%\begin{frame}{Ionic vs. Hydrated Radii}
%	\only<+>{%
%		\begin{columns}
%			\column{0.5\textwidth}
%			\begin{itemize}
%				\item Every dissolved species is surrounded by water
%					molecules.
%				\item For ionic species, the number of \ch{H2O}
%					molecules depends on the ions charge density.
%			\end{itemize}
%			\column{0.5\textwidth}
%			\includegraphics[scale=0.75]{estimated-hydration.jpg}
%		\end{columns}
%	}
%
%	\only<+>{%
%		\begin{columns}
%			\column{0.5\textwidth}
%			\begin{itemize}
%				\item A smaller ion of like charge has a larger charge
%					density thus \ch{Li+} attracts more water than
%					\ch{K+} and a 3+ ion attracts more than a 2+.
%				\item In fact, the largest hydrated ion is \ch{H3O+} at
%					\SI{900}{\pico\meter}!
%					\begin{itemize}
%						\item By comparison, hydrated \ch{Li+}
%							is only about
%							\SI{600}{\pico\meter}.
%					\end{itemize}
%			\end{itemize}
%			\column{0.5\textwidth}
%			\includegraphics[scale=0.25]{ionic-v-hydrated-radii.jpg}
%		\end{columns}
%	}
%\end{frame}
%
%\begin{frame}{Ionic Atmosphere}
%	\begin{columns}
%		\column{0.6\textwidth}
%	\begin{itemize}[<+->]
%		\item Ions attract (or repel) other ions.
%%			\begin{itemize}
%%				\item Otherwise we would not be able to have
%%					insoluble salts!
%%			\end{itemize}
%		\item What happens when an inert electrolyte is added to the
%			solution?
%			\begin{itemize}
%				\item The ionic atmosphere causes \alert{less}
%					attraction between the ions of the
%					insoluble salt!
%				\item Less attraction means less crashing and a
%					\alert{higher} solubility!
%			\end{itemize}
%		\item The more ions we have in solution, the less attraction we
%			have, thus the greater solubility.
%	\end{itemize}
%		\column{0.4\textwidth}
%		\includegraphics[scale=0.85]{ionic-atmosphere.jpg}
%	\end{columns}
%\end{frame}
%
%\begin{frame}{The Equilibrium ``Constant'' is Not Constant!}
%	\begin{columns}
%		\column{0.6\linewidth}
%		\begin{itemize}
%			\item We will see the same effect for all equilibria.
%			\item For example, from CHEM 116:
%				\begin{align*}
%					\ch{Fe^{3+} + SCN- <=> FeSCN^{2+}}
%				\end{align*}
%			\item Note that as the ionic atmosphere increases, we
%				get more of the reactant ions present shifting
%				the equilibrium position.
%		\end{itemize}
%		\column{0.4\linewidth}
%		\includegraphics[scale=0.35]{FeSCN.jpg}
%	\end{columns}
%\end{frame}
%
%\begin{frame}{Ionic Strength}
%	\begin{itemize}
%		\item We will measure the size or strength of the ionic
%			atmosphere by calculating the \alert{ionic strength}
%			($\mu$) of the solution.
%		\item $\mu$ is the total ion concentration in the solution:
%			\begin{align*}
%				\mu = \frac{1}{2}(c_1z_1^2 + c_2z_2^2 + \cdots)
%				= \frac{1}{2}\sum_{i}c_iz_i^2
%			\end{align*}
%			where $c_i$ is the ion concentration of the
%			$i$\textsuperscript{th} species and $z_i$ is its ionic
%			charge.
%	\end{itemize}
%\end{frame}
%
%\begin{frame}{Ionic Strength Example}
%	\begin{itemize}
%		\item Find the ionic strength of the following solutions:
%			\begin{itemize}
%				\item \SI{0.10}{\Molar} \ch{AgNO3}
%				\item \SI{0.10}{\Molar} \ch{Na2SO4}
%				\item \SI{0.20}{\Molar} \ch{KCl} and
%					\SI{0.10}{\Molar} \ch{K2SO4}
%			\end{itemize}
%		\item Recognize that these values are approximations. All salts
%			over 1\%~(w/w) will have some soluble ion pairs. Using
%			\SI{0.025}{\formal} solutions as an example,
%			\begin{itemize}
%				\item \ch{NaCl} is 99.6\% dissociated
%				\item \ch{Na2SO4} is 96\%
%				\item \ch{MgSO4} is 65\%
%				\item \ch{La2(SO4)3} is 4\%
%			\end{itemize}
%			See Box 8-1, page 164
%	\end{itemize}
%
%	\mode<article>{\vspace{15em}}
%
%\note{
%	\begin{multicols}{2}
%	\textbf{\ch{AgNO3}}
%	\begin{align*}
%		\mu &= \frac{1}{2} \sum_{i} c_i z_i^2 \\
%		&= \frac{1}{2} (0.10 \times 1^2 + 0.10 \times (-1)^2) \\
%		&= \SI{0.10}{\Molar}
%	\end{align*}
%
%	\textbf{\ch{Na2SO4}}
%	\begin{align*}
%		\mu &= \frac{1}{2} \sum_{i} c_i z_i^2 \\
%		&= \frac{1}{2} (0.20 \times 1^2 + 0.10 \times (-2)^2) \\
%		&= \SI{0.30}{\Molar}
%	\end{align*}
%
%	\textbf{\ch{KCl} and \ch{K2SO4}}
%	\begin{align*}
%		\mu &= \frac{1}{2} \sum_{i} c_i z_i^2 \\
%		&= \frac{1}{2} (0.20 \times 1^2 + 0.2 \times (-1)^2 \\
%		&\qquad {} + 0.20
%		\times 1^2 + 0.10 \times (-2)^2) \\
%		&= \SI{0.5}{\Molar}
%	\end{align*}
%	\end{multicols}
%	}
%\end{frame}
%
%\begin{frame}{Working Ionic Strengths into K's}
%	\begin{itemize}
%		\item To account for the effect ionic strength has on the
%			chemistry of a salt, we replace molar concentrations
%			with activities:
%			\begin{align*}
%				\mathcal{A}_{\ch{C}} = [\ch{C}]\gamma_{\ch{C}}
%			\end{align*}
%			where the molar concentration of species \ch{C}
%			([\ch{C}]) is multiplied by its activity coefficient
%			($\gamma_{\ch{C}}$).
%		\item The extended \alert{Debye-Huckel equation} gives the
%			activity coefficient which is dependent on the size of
%			the ionic strength.
%			\begin{align*}
%				\log \gamma = \dfrac{-0.51z^2\sqrt{\mu}}{1 +
%				(\alpha \sqrt{\mu}/305)} \qquad\text{(at
%				\SI{25}{\celsius})}
%			\end{align*}
%			where $\alpha$ is the size of the ion in
%			\si{\pico\meter}. Note that this relation only works
%			well up to \SI{0.10}{\Molar} ionic strength.
%	\end{itemize}
%\end{frame}
%
%\vspace{\stretch{-1}}
%
%\begin{frame}{The Real Equilibrium Constant}
%	\begin{itemize}
%		\item Consider the solubity of mercury (I) bromide, with
%			equilibrium,
%			\begin{align*}
%				\ch{Hg2Br2(s) <=> Hg2^{2+}(aq) + 2 Br^{-}(aq)}
%			\end{align*}
%		\item The new equilibrium constant including activity
%			coefficients is
%			\begin{align*}
%				\Ksp{} = \mathcal{A}_{\ch{Hg2^{2+}}}
%				\mathcal{A}_{\ch{Br-}}^2 =
%				[\ch{Hg2^{2+}}]\gamma_{\ch{Hg2^{2+}}}
%				[\ch{Br-}]^2\gamma_{\ch{Br-}}^2
%			\end{align*}
%	\end{itemize}
%
%	\pause
%
%	\begin{block}{General Form of the Equilibrium Constant}
%		\begin{align*}
%			K = \dfrac{\mathcal{A}^c_{\ch{C}}
%			\mathcal{A}^d_{\ch{D}}}{\mathcal{A}^a_{\ch{A}}
%			\mathcal{A}^b_{\ch{B}}} = \dfrac{[\ch{C}]^c
%			\gamma_{\ch{C}}^c [\ch{D}]^d \gamma_{\ch{D}}^d}
%			{[\ch{A}]^a \gamma_{\ch{A}}^a [\ch{B}]^b
%			\gamma_{\ch{B}}^b}
%		\end{align*}
%	\end{block}
%\end{frame}
%
%\vspace{\stretch{-1}}
%
%\begin{frame}{Effect of $\mu$, $z_i$, and $\alpha$ on $\gamma$}
%	\begin{columns}
%		\column{0.4\textwidth}
%		\begin{center}
%			\includegraphics[scale=0.85]{activity-v-ionic-strength.png}
%		\end{center}
%		\column{0.6\textwidth}
%		\begin{enumerate}
%			\item<1-> As ionic strength increase, the activity
%				coefficient decreases.
%
%				\only<1|handout:0>{\begin{itemize}
%					\item The activity coefficient
%						($\gamma$) approaches unity as
%						the ionic strength ($\mu$)
%						approaches 0.
%					\item At $\mu \leq \SI{0.0001}{\Molar}$,
%						$\gamma \approx 1$.
%				\end{itemize}}
%
%			\item<2-> As the charge on an ion increases, the
%				departure of its activity coefficient from unity
%				increases.
%
%				\only<2|handout:0>{\begin{itemize}
%					\item Activity coefficients are much
%						more important for highly
%						charged species ($\pm3$ or
%						$\pm4$).
%					\item Let $\gamma_i = 1$ for all
%						uncharged species at any $\mu$.
%				\end{itemize}}
%
%			\item<3-> The smaller the hydrated radius of the ion,
%				the more important the activity coefficient
%				becomes.
%
%				\only<3|handout:0>{\begin{itemize}
%					\item At fixed ionic strength, however,
%						species of similar charge will
%						have similar activity
%						coefficients.
%				\end{itemize}}
%
%			\item<4-> Activity coefficient values are independent of
%				counter ions.
%
%				\only<4|handout:0>{\begin{itemize}
%					\item The key is the species itself --
%						its hydrated size, its charge,
%						its being affected by ionic
%						strength!
%				\end{itemize}}
%		\end{enumerate}
%	\end{columns}
%\end{frame}
%
%\vspace{\stretch{-1}}
%
%\begin{frame}[t]{Activity Example}
%	What is the solubility of calcium fluoride in \SI{0.050}{\formal} sodium
%	fluoride? Calculate the solubility both with and without activity
%	correction.
%
%	\smallskip
%
%	$\Ksp{} = \num{3.9e-11}$ for calcium fluoride
%
%	\mode<article>{\vfill}
%
%\note{
%	\footnotesize
%	\begin{tabular} {c r c r c r}
%		& \ch{CaF2(s)} & \ch{<=>} & \ch{Ca^{2+}(aq)} & \ch{+} &
%		\ch{2 F^{-}(aq)} \\
%		I & & & 0 & & 0.050 \\
%		C & $-x$ & & $+x$ & & $+2x$ \\ \cline{2-5}
%		E & $-x$ & & $x$ & & $0.050 + 2x$
%	\end{tabular}
%
%	\begin{columns}
%		\column{0.5\linewidth}
%		\begin{align*}
%			\Ksp{} &= [\ch{Ca^{2+}}][\ch{F-}]^2 \\
%			\num{3.9e-11} &= (x)(0.050+2x)^2 \\
%			x &= \SI{1.6e-8}{\Molar}
%		\end{align*}
%		\column{0.5\linewidth}
%		\begin{align*}
%			\Ksp{} &= [\ch{Ca^{2+}}]\gamma_{\ch{Ca^{2+}}}
%			[\ch{F-}]^2\gamma_{\ch{F-}}^2 \\
%			\num{3.9e-11} &= (x)(0.050+2x)^2
%			\gamma_{\ch{Ca^{2+}}}\gamma_{\ch{F-}}^2 \\
%			x &= \dfrac{\num{3.9e-11}}{0.050^2 \times 0.485 \times
%			0.81^2} \\
%			&= \SI{4.9e-8}{\Molar}
%		\end{align*}
%	\end{columns}
%
%	\begin{align*}
%		\mu &= 0.5 \left( [\ch{Na+}]\times1^2 +
%		[\ch{F-}]\times-1^2 + [\ch{Ca^{2+}}]\times2^2 \right) \\
%		&= 0.5 \left(0.050 + (0.050+2x) + 4x\right) \\
%		&= \SI{0.050}{\Molar} \text{~and via Table 8-1,~}
%		\gamma_{\ch{Ca^{2+}}} = 0.485 \text{~and~}
%		\gamma_{\ch{F-}} = 0.81
%	\end{align*}
%	}
%\end{frame}
%
%\begin{frame}{pH Revisited}
%	\begin{itemize}
%		\item pH is also affected by ionic strength because it is an
%			equilibrium involving charged species.
%		\item What is the real pH of pure distilled water?
%			\begin{align*}
%				\ch{2 H2O <=> H3O+ + OH-} \qquad \Kw{} =
%				[\ch{H3O+}]\gamma_{\ch{H3O+}}
%				[\ch{OH-}]\gamma_{\ch{OH-}}
%			\end{align*}
%		\item At very low ionic strength, $\gamma \approx 1$
%		\item At \SI{1.0e-7}{\Molar},
%			\begin{align*}
%				\text{pH} &= -\log \mathcal{A}_{\ch{H3O+}} \\
%				&= -\log [\ch{H3O+}]\gamma_{\ch{H3O+}}
%				\\
%				&= -\log (\num{1.0e-7})(1.00) \\
%				&= 7.00
%			\end{align*}
%	\end{itemize}
%\end{frame}
%
%\vspace{\stretch{-1}}
%
%\begin{frame}[t]{pH Example}
%	What is the pH of real distilled water with \SI{0.50}{\Molar}~\ch{KCl}
%	added?
%
%	\mode<article>{\vfill}
%
%	\note{
%	\begin{align*}
%		\ch{2 H2O <=> H3O+ + OH-} \qquad \Kw{} &=
%				[\ch{H3O+}]\gamma_{\ch{H3O+}}
%				[\ch{OH-}]\gamma_{\ch{OH-}} \\
%		\intertext{$\mu$ = \SI{0.50}{\Molar}, $\gamma_{\ch{H3O+}} =
%		0.28$, and $\gamma_{\ch{OH-}} = 0.38$}
%		\num{1.0e-14} &= (0.28x)(0.38x) \\
%		\num{9.4e-14} &= x^2 \\
%		x &= [\ch{H3O+}] = \SI{3.1e-7}{\Molar} \\
%		pH &= -\log \mathcal{A}_{\ch{H3O+}} = -\log
%		[\ch{H3O+}]\gamma_{\ch{H3O+}} \\
%		&= -\log (\num{3.1e-7})(0.28) \\
%		&= 7.06
%	\end{align*}
%	}
%\end{frame}
%
\end{document}
