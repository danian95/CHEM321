% !TEX program = xelatex
%\documentclass[notes=show]{beamer}
%\documentclass[notes=onlyslideswithnotes,notes=hide]{beamer}
\documentclass[11pt,letterpaper]{article}
\usepackage{beamerarticle}

\usepackage{analchem}
\usepackage{lecture}
\usepackage{tabu}
\usepackage{multicol}

\title{Monoprotic Acid-Base Equilibria}
\subtitle{Chapter 9}
\institute{CHEM321 - Analytical Chemistry I \\ Bloomsburg University}
\author{D.A. McCurry}
\date{Fall 2019}

\begin{document}

\maketitle
\mode<article>{\thispagestyle{fancy}}

\begin{frame}{\ch{HNO3} -- One of our Strong Acids}
	\begin{columns}
		\column{0.5\linewidth}
		\begin{center}
			\includegraphics[scale=0.3]{HNO3Raman.jpeg}
			*undissociated \ch{HNO3}
		\end{center}
		\column{0.5\linewidth}
		\begin{center}
			\includegraphics[scale=0.5]{HNO3Dissociation.jpeg}
			\includegraphics[scale=0.5]{HNO3TempKa.jpeg}
		\end{center}
	\end{columns}
\end{frame}

\begin{frame}{The Hazard of $\text{pH} = -\log[\ch{H3O+}]$}
	As we've seen, calculating pH of strong acids and bases is very easy
	because they dissociate 100\%:

	\begin{align*}
		\ch{!( "\SI{0.0100}{\Molar}" )( HNO3 ) &-> !( "\SI{0.0100}{\Molar}" )( H3O+ ) + !( "\SI{0.0100}{\Molar}" )( NO3- )} \\
		\shortintertext{So,}
		\text{pH} &= -\log(0.0100) = 2.000\\
		\intertext{}
		\visible<2->{
		\ch{!( "\SI{1.00e-8}{\Molar}" )( HNO3 ) &-> !( "\SI{1.00e-8}{\Molar}" )( H3O+ ) + !( "\SI{1.00e-8}{\Molar}" )( NO3- )} \\
		\shortintertext{So,}
		\text{pH} &= -\log(\num{1.00e-8}) = 8.000}
	\end{align*}

	\pause

	What happened? Why did this stop working?
\end{frame}

\begin{frame}{A Systematic Approach to pH}
	\mode<article>{\vfill}
	\note{
	\begin{enumerate}
		\item Pertinent Reactions
			\begin{align*}
				\ch{!( "\SI{1.00e-8}{\Molar}" )( HNO3 ) &-> !( "\SI{1.00e-8}{\Molar}" )( H3O+ ) + !( "\SI{1.00e-8}{\Molar}" )( NO3- )} \\
				\ch{2 H2O &<=> H3O+ + OH-}
			\end{align*}
	\end{enumerate}
	\begin{multicols}{2}
		\begin{enumerate}
			\setcounter{enumi}{1}
			\item Charge Balance
				\begin{align*}
					[\ch{H3O+}] = [\ch{OH-}] + [\ch{NO3-}]
				\end{align*}
			\item Mass Balance
				\begin{align*}
					[\ch{NO3-}] = \SI{1.00e-8}{\Molar}
				\end{align*}
			\item Write K's
				\begin{align*}
					K_w = [\ch{H3O+}][\ch{OH-}]
				\end{align*}
			\item Count: 3 unknowns and 3 equations!
			\item Solve
		\end{enumerate}
	\end{multicols}
	}

\end{frame}

\note{
	\begin{align*}
		\shortintertext{Substitute the MB into the CB:}
		[\ch{H3O+}] &= [\ch{OH-}] + \SI{1.00e-8}{\Molar} \\[1em]
		\shortintertext{Rearrange for easier solving:}
		[\ch{OH-}] &= [\ch{H3O+}] - \SI{1.00e-8}{\Molar} \\[1em]
		\shortintertext{Plug into $K_w$ expression:}
		\num{1.0e-14} &= [\ch{H3O+}]([\ch{H3O+}] - \num{1.00e-8}) \\
		0 &= [\ch{H3O+}]^2 - \num{1.00e-8}[\ch{H3O+}] - \num{1.0e-14} \\
		[\ch{H3O+}] &= \SI{1.05e-7}{\Molar} \text{ and }
		\fbox{\text{pH}=6.98}
	\end{align*}

	There are 2 sources of \ch{H3O+} in any aqueous acid solution -- the acid
	and the water.
}

\begin{frame}{Spreadsheet software aids in pH determination}
	\begin{center}
		\mode<article>{\small}
		\begin{tabular} {S[table-format=1.1e-2] *{2}{S[table-format=2.2]}}
			{[Strong Acid]} & {$\approx$ pH} & {$=$ pH} \\ \midrule
			1.0e-1 & 1.00 & 1.00 \\
			1.0e-2 & 2.00 & 2.00 \\
			1.0e-3 & 3.00 & 3.00 \\
			1.0e-4 & 4.00 & 4.00 \\
			1.0e-5 & 5.00 & 5.00 \\
			1.0e-6 & 6.00 & 6.00 \\
			1.0e-7 & 7.00 & 6.79 \\
			1.0e-8 & 8.00 & 6.98 \\
			1.0e-9 & 9.00 & 7.00 \\
			1.0e-10 & 10.00 & 7.00 \\
			1.0e-11 & 11.00 & 7.00 \\
			1.0e-12 & 12.00 & 7.00
		\end{tabular}
		\qquad
		\begin{tabular} {S[table-format=1.1e-2] *{2}{S[table-format=2.2]}}
			{[Strong Base]} & {$\approx$ pH} & {$=$ pH} \\ \midrule
			1.0e-1 & 13.00 & 13.00 \\
			1.0e-2 & 12.00 & 12.00 \\
			1.0e-3 & 11.00 & 11.00 \\
			1.0e-4 & 10.00 & 10.00 \\
			1.0e-5 & 9.00 & 9.00 \\
			1.0e-6 & 8.00 & 8.00 \\
			1.0e-7 & 7.00 & 7.21 \\
			1.0e-8 & 6.00 & 7.02 \\
			1.0e-9 & 5.00 & 7.00 \\
			1.0e-10 & 4.00 & 7.00 \\
			1.0e-11 & 3.00 & 7.00 \\
			1.0e-12 & 2.00 & 7.00
		\end{tabular}
	\end{center}
\end{frame}

\mode<article>{\clearpage}

\begin{frame}{Even better\ldots we can plot it!}
	\centering
	\includegraphics[width=\textwidth]{strongacid-base.eps}
\end{frame}

\begin{frame}[allowframebreaks]{How much \ch{H3O+} comes from water?}
	\begin{center}
		\small
		\begin{tabular} {S[table-format=1.1e-2]
			S[table-format=1.1e-2] S[table-format=3.3]
			S[table-format=1.1e-2] S[table-format=3.10]}
			{[Strong Acid]} & \multicolumn{2}{c}{From acid} & \multicolumn{2}{c}{From
			water} \\
			& [\ch{H3O+}] & \% & [\ch{H3O+}] & \% \\ \midrule
			1.0e-1 & 1.0e-1 & 100 & 1.0e-13 & 0.0000000001 \\
			1.0e-2 & 1.0e-2 & 100 & 1.0e-12 & 0.00000001 \\
			1.0e-3 & 1.0e-3 & 100 & 1.0e-11 & 0.000001 \\
			1.0e-4 & 1.0e-4 & 100 & 1.0e-10 & 0.0001 \\
			1.0e-5 & 1.0e-5 & 100 & 1.0e-9 & 0.01 \\
			1.0e-6 & 1.0e-6 & 99 & 9.9e-9 & 1 \\
			1.0e-7 & 1.0e-7 & 62 & 6.2e-8 & 38 \\
			1.0e-8 & 1.0e-8 & 10 & 9.5e-8 & 90 \\
			1.0e-9 & 1.0e-9 & 1 & 1.0e-7 & 99 \\
			1.0e-10 & 1.0e-10 & 0.1 & 1.0e-7 & 100 \\
			1.0e-11 & 1.0e-11 & 0.01 & 1.0e-7 & 100 \\
			1.0e-12 & 1.0e-12 & 0.001 & 1.0e-7 & 100
		\end{tabular}
	\end{center}

	\framebreak

	\noindent
	If we have \SI{1.0e-2}{\Molar} \ch{HNO3}, the \ch{OH-} concentration
	follows,

	\begin{align*}
		[\ch{OH-}] &= \dfrac{\num{1.0e-14}}{[\ch{H3O+}]} \\
		&= \dfrac{\num{1.0e-14}}{\num{1.0e-2}} = \SI{1.0e-12}{\Molar}
	\end{align*}

	The \emph{only} source of \ch{OH-} is the dissociation of water, so
	water must \emph{also contribute} \SI{1.0e-12}{\Molar} \ch{H3O+}!
\end{frame}

\begin{frame}{Weak Acids and Bases}
	\begin{block}{Weak Acid Equilibrum}
		\begin{align*}
			\ch{HA + H2O <=>[ $K_a$ ] H3O+ + A-} \qquad K_a =
			\dfrac{[\ch{H+}][\ch{A-}]}{[\ch{HA}]}
		\end{align*}
	\end{block}

	\begin{block}{Weak Base Equilibrum}
		\begin{align*}
			\ch{B + H2O <=>[ $K_b$ ] BH+ + OH-} \qquad K_b =
			\dfrac{[\ch{BH+}][\ch{OH-}]}{[\ch{B}]}
		\end{align*}
	\end{block}

	Conjugate Acid-Base Pairs: $K_a \times K_b = K_w$
\end{frame}

\begin{frame}[t]{A Typical Weak Acid Problem}
	Find the pH of a solution of the weak acid, HA, given the formal
	concentration of HA and the value of \Ka.

	\mode<article>{\vfill}

	\note{
		\begin{multicols}{2}
		\begin{enumerate}
			\item Pertinent equations:
				{\footnotesize
				\begin{align*}
					\ch{HA + H2O &<=>[ $K_a$ ] H3O+ + A-} \\
					\ch{2 H2O &<=>[ $K_w$ ] H3O+ + OH-}
				\end{align*}}
			\item Charge Balance:
				{\footnotesize
				\begin{align*}
					[\ch{H3O+}] = [\ch{OH-}] + [\ch{A-}]
				\end{align*}}
			\item Mass Balance:
				{\footnotesize
				\begin{align*}
					\si{\formal} = [\ch{A-}] + [\ch{HA}]
				\end{align*}}
			\item Equilibria:
				{\footnotesize
				\begin{align*}
					\Ka &= \frac{[\Oxo][\ch{A-}]}{[\ch{HA}]} \\
					\Kw &= [\Oxo][\Hyd]
				\end{align*}}
			\item Count: 4 unknowns and 4 equations, then
			\item Solve!
		\end{enumerate}
		\end{multicols}
		\begin{align*}
			0 = [\Oxo]^3 + [\Oxo]^2\Ka -
			[\Oxo](\si{\formal}\Ka + \Kw) - \Kw\Ka
		\end{align*}
	}
\end{frame}

\begin{frame}[t]{``Wait!''}
	The Good Chemist comes galloping down from the mountain on her white
	stallion to rescue us. ``There is no reason to solve a cubic equation.
	We can make an excellent, simplifying approximation.''

	\begin{itemize}
		\item The amount of \ch{H3O+} from the acid is much larger than
			that from water \emph{at normal lab concentrations}.
			\begin{align*}
				[\ch{A-}] >> [\ch{OH-}] \qquad\therefore [\ch{H3O+}]
				\approx [\ch{A-}]
			\end{align*}
		\item Let $x = [\Oxo] = [\ch{A-}]$
%			If $x = [\ch{H3O+}]$, then
%			\begin{align*}
%				[\ch{H3O+}] &= [\ch{A-}] = x \\
%				\textsc{F} &= [\ch{A-}] + [\ch{HA}] \qquad\therefore
%				[\ch{HA}] = \textsc{F} - x \\
%				K_a &= \dfrac{[\ch{H+}][\ch{A-}]}{[\ch{HA}]} =
%				\dfrac{x^2}{\textsc{F}-x}
%			\end{align*}
	\end{itemize}

	\mode<article>{\vspace{15em}}

	\note{\begin{multicols}{2}
		\begin{enumerate}
			\item Pertinent equations:
				{\footnotesize
				\begin{align*}
					\ch{HA + H2O &<=>[ $K_a$ ] H3O+ + A-} \\
					\ch{2 H2O &<=>[ $K_w$ ] H3O+ + OH-}
				\end{align*}}
			\item Charge Balance:
				{\footnotesize
				\begin{align*}
					x &= [\Hyd] + x \\
					\therefore [\Hyd] &= 0
				\end{align*}}
			\item Mass Balance:
				{\footnotesize
				\begin{align*}
					\si{\formal} &= x + [\ch{HA}] \\
					\therefore [\ch{HA}] &= \si{\formal} - x
				\end{align*}}
			\item Equilibria:
				{\footnotesize
				\begin{align*}
					\Ka &= \frac{x^2}{[\ch{HA}]} = \boxed{\frac{x^2}{\si{\formal} - x}} \\
					\Kw &= x [\ch{OH-}] \approx 0
				\end{align*}}
			\item Count: 4 unknowns and 4 equations, then
			\item Solve!
		\end{enumerate}
		\end{multicols}
	}
\end{frame}

\mode<article>{\clearpage}

\begin{frame}[t]{Weak Acid Example}
	What is the \pH{} of a \SI{0.0500}{\Molar} \para-hydroxybenzoic acid ($\pKa =
	4.54$) solution?

	\mode<article>{\vspace{20em}}

	\note{\begin{align*}
		\Ka &= \num{2.9e-5} = \dfrac{x^2}{\si{\formal} - x} =
		\dfrac{x^2}{\num{0.0500} - x} \\
		0 &= x^2 + \num{2.9e-5}x - \num{1.45e-6} \\
		\intertext{After some quadratic magic:}
		x &= [\ch{H3O+}] = [\ch{A-}] = \SI{1.2e-3}{\Molar} \\
		\pH &= \boxed{2.92} \\
		\therefore \pOH &= 11.08 \\
		[\ch{OH-}] &= \SI{8.4e-12}{\Molar}
	\end{align*}
	\SI{8.4e-12}{\Molar} \ch{H3O+} comes from \ch{H2O}, so it is a valid
assumption that 100\% of \ch{H3O+} comes from HA and none from \ch{H2O}!}
\end{frame}

\begin{frame}{Fraction of Dissociation}
	\begin{columns}
		\column{0.55\linewidth}
		Amount dissociated relative to the mass balance or total amount.
		\begin{align*}
			\alpha = \frac{[\ch{A-}]}{[\ch{HA}] + [\ch{A-}]} =
			\frac{x}{\si{\formal}}
		\end{align*}
		For the last example,
		\begin{align*}
			\alpha =
			\frac{\SI{1.2e-3}{\Molar}}{\SI{0.0500}{\Molar}} =
			\fbox{0.024 = 2.4\%}
		\end{align*}
		\begin{center}
			\includegraphics[scale=0.5]{hydroxybenzoic-acid.jpeg}
		\end{center}
		\column{0.38\linewidth}
		\begin{center}
			\includegraphics[scale=0.33]{frac-dissociation.jpeg}
		\end{center}
	\end{columns}
\end{frame}

\begin{frame}{Weak Base Equilibria}
	We can make the same assumption as with the weak acids, i.e., that
	nearly all the \ch{OH-} comes from the weak base and very little from
	the water.

	\begin{align*}
		\ch{B + H2O &<=> HB+ + OH-} &\qquad \Kb &=
		\frac{[\ch{HB+}][\ch{OH-}]}{[\ch{B}]} \\
		[\ch{B}] &= \si{\formal} - [\ch{HB+}] = \si{\formal} - x &\qquad
		\therefore \Kb &= \frac{x^2}{\si{\formal}-x}
	\end{align*}
\end{frame}

\begin{frame}[t]{Weak Base Example}
	What is the pH of a \SI{0.00372}{\Molar} cocaine solution?

	\begin{center}
		\includegraphics[scale=0.25]{base-association.jpeg}
	\end{center}

	\mode<article>{\vfill}

	\note{
		\begin{reaction*}
%			B + \water <=> HB+ + \Hyd
			!($0.00372-x$)( B ) + \water <=> !($x$)( HB+ ) + !($x$)(\Hyd)
		\end{reaction*}
		
		\begin{align*}
			\Kb &= \frac{x^2}{\si{\formal} - x} \\
			\num{2.6e-6} &= \frac{x^2}{0.00372-x} \\
			\intertext{via quadratic,}
			x &= \SI{9.71e-5}{\Molar}~\Hyd \\
			\therefore [\Oxo] &= \SI{1.03e-10}{\Molar} \text{~and~} \pH{} = 9.99
		\end{align*}
	}
\end{frame}

\begin{frame}{Fraction of Association}
	\begin{align*}
		\alpha = \frac{[\ch{HB+}]}{[\ch{B}] + [\ch{HB+}]} =
		\dfrac{x}{\si{\formal}}
	\end{align*}

	For the last example,

	\begin{align*}
		\alpha =
		\frac{\SI{9.71e-5}{\Molar}}{\SI{3.73e-3}{\Molar}} =
		\fbox{0.0260 = 2.60\%}
	\end{align*}

	\mode<article>{\clearpage}

	\begin{block}{Fraction of Association/Dissociation}
		\begin{align*}
			\frac{x}{\si{\formal}} = \frac{\text{amount
			associated/dissociated}}{\text{total amount of species}}
		\end{align*}
	\end{block}
\end{frame}

\begin{frame}{Buffers}
	\begin{itemize}
		\item What is a buffer?
		\item What is it used for?
	\end{itemize}

	\begin{align*}
		\Ka &= \frac{[\Oxo][\ch{A-}]}{[\ch{HA}]} \\
		\log\Ka &= \log[\Oxo] + \log\frac{[\ch{A-}]}{[\ch{HA}]} \\
		-\log[\Oxo] &= -\log\Ka + \log\frac{[\ch{A-}]}{[\ch{HA}]} \\
		\pH &= \pKa + \log\frac{[\ch{A-}]}{[\ch{HA}]} \\
		\text{or, for a base:} \qquad
		\pH &= \pKa + \log\frac{[\ch{B}]}{[\ch{HB+}]} \\
	\end{align*}
\end{frame}

\begin{frame}{The Henderson-Hasselbalch Equation}
	{\Large
	\begin{align*}
		\pH &= \pKa + \log\frac{[\ch{A-}]}{[\ch{HA}]}
	\end{align*}}

	\begin{tikzpicture}
		\node[align=left,text width=0.5\linewidth](text) {
			\begin{enumerate}
				\item It is the easiest route to pH when your solution contains
					a weak conjugate pair.
				\item It tells us a lot about buffer capacity in the solution.
				\item We will use it extensively to look at the \emph{principle
					species} in a solution.
				\item Read Box 9-3 (9\textsuperscript{th} Ed.)
			\end{enumerate}};
		\node[right = of text] {
			\begin{tabular} {r@{/}l l}
				\multicolumn{2}{c}{[\ch{A-}]/[\ch{HA}]} &
				\multicolumn{1}{c}{\pH} \\ \midrule
				100 & 1   & $\pKa + 2$ \\
				10  & 1   & $\pKa + 1$ \\
				1   & 1   & $\pKa$ \\
				1   & 10  & $\pKa - 1$ \\
				1   & 100 & $\pKa - 2$
			\end{tabular}};
	\end{tikzpicture}
\end{frame}

\mode<presentation>{
\begin{frame}{Alkaline Water}
	\centering
	\includegraphics[width=0.7\textwidth]{alkaline-water.pdf}
	\footnote{https://www.mayoclinic.org}
\end{frame}}

\begin{frame}[t]{Henderson-Hasselbalch Example}
	We have \SI{200.}{\milli\liter} of a solution containing
	\SI{0.250}{\Molar} citric acid and \SI{0.200}{\Molar} sodium citrate. To
	this, \SI{10.0}{\milli\liter} of \SI{0.750}{\Molar}~\ch{NaOH} is added.
	What is the change in \pH?

	\mode<article>{\vfill}

	\note{
		\begin{enumerate}
			\item Initial \pH:
				\begin{align*}
					\pH &= \pKa + \log\frac{[\ch{A-}]}{[\ch{HA}]} \\
					&= 3.13 + \log\frac{0.200}{0.250} = 3.03
				\end{align*}
			\item Adding base, so reacting? \textbf{Citric Acid}
			\item ICE:
		\begin{tabular} {*{7}{r} l}
			\ch{HA} & + & \ch{OH-} & \ch{<=>} & \ch{A-} & + &
			\ch{H2O} \\
			50.0 & & 7.50 & & 40.0 & & & \si{\milli\mole} each \\
			-7.50 & & -7.50 & & +7.50 & & & \si{\milli\mole} each 
		\end{tabular}
	\end{enumerate}

		\begin{align*}
			\text{pH} = 3.13 + \log\dfrac{47.5}{42.5} = 3.18 \qquad
			\fbox{\textrm{d}\text{pH} = +0.15}
		\end{align*}
		}
\end{frame}

\begin{frame}{Preparing a buffer}
	\begin{itemize}[<+->]
		\item You can calculate the appropriate mass of each component
			and dissolve in appropriate volume to make the
			appropriate buffer.
			\begin{itemize}
				\item Activity coefficients
				\item \pKa is often listed at
					\SI{25}{\celsius} which \emph{may not
					be true} in lab
				\item Approximations (``Good chemist'') might
					not be accurate enough
				\item Other ion contributions
			\end{itemize}
		\item Instead, weigh a quantity of acid (or base) for desired
			concentration and add water to \emph{near desired}
			volume. Add a \emph{strong} base (or acid) until the
			desired \pH is reached, then dilute to desired volume.
			\begin{itemize}
				\item Strong species will alter
					[\ch{A-}]/[\ch{HA}] ratio by
					\emph{consuming} (neutralizing)
					species.
				\item Why do we have to add water first?
			\end{itemize}
	\end{itemize}
\end{frame}

\mode<article>{\clearpage}

\begin{frame}{The Real Henderson-Hasselbalch Equation}
	{\Large
	\begin{align*}
		\pH &= \pKa +
		\log\frac{[\ch{A-}]\gamma_{\ch{A-}}}{[\ch{HA}]}
	\end{align*}}

	\begin{itemize}
		\item Recall activity coefficients!
		\item Ionic strength affects \pH, therefore diluting buffers may
			significantly change \pH.
	\end{itemize}
\end{frame}

\begin{frame}{Buffer Capacity}
	\begin{columns}
		\column{0.5\linewidth}
		A measure of how well a solution resists changes in \pH{} when a
		strong acid or base is added.
		\begin{align*}
			\beta = \frac{\textrm{d}C_b}{\textrm{d}\pH} =
			-\frac{\textrm{d}C_a}{\textrm{d}\pH} 
		\end{align*}
		where $C_b$ and $C_a$ are the moles of the strong acid and base
		needed to change 1 liter of buffer by 1 pH unit.
		\column{0.4\linewidth}
		\begin{center}
			\includegraphics[scale=0.33]{buffercap.jpeg}
			
			\footnotesize
			\SI{0.100}{\formal} \ch{HA} with $\pKa = 5$
		\end{center}
	\end{columns}
\end{frame}


\end{document}
