% !TEX program = xelatex
\documentclass[notes]{beamer}
%\documentclass[notes=onlyslideswithnotes,notes=hide]{beamer}
%\documentclass[notes=only]{beamer}
%\documentclass[11pt,letterpaper]{article}
%\usepackage{beamerarticle}

\usepackage{analchem}
\usepackage{lecture}
\usepackage{cancel}

\newcommand<>{\latecancel}[1]{\alt#2{\cancel{#1}}{\vphantom{\cancel{#1}}{#1}}}

\title{Chemical Equilibrium}
\subtitle{Chapter 6}
\institute{CHEM321 - Analytical Chemistry I \\ Bloomsburg University}
\author{D.A. McCurry}
\date{Fall 2019}

\begin{document}

\maketitle
\mode<article>{\thispagestyle{fancy}}

\mode<presentation>{
	\begin{frame}
		\includegraphics[width=\linewidth]{hotwaterheater.jpg}
		
		\footnotesize{Source: A random YouTube image from a search.}
	\end{frame}
}

%\begin{frame}
%	\begin{columns}
%		\column{0.5\textwidth}
%		\includegraphics[width=\textwidth]{equilibrium-poster.jpg}
%		\column{0.5\textwidth}
%		{\Large
%		``The only thing more powerful than the system, is the man that
%		will overthrow it.''}
%		
%		\pause
%
%		\vspace{3em}
%
%		{\small
%		(But given enough time, the system will return to equilibrium)}
%	\end{columns}
%\end{frame}

\section{Equilibrium Constant}

\begin{frame}[allowframebreaks]{The Equilibrium Constant}
	\begin{block}{Law of Mass Action}
		\centering
		\parbox[c]{0.4\linewidth}{
			\begin{center}
			\ch{aA + bB <=> cC + dD}
			\end{center}
			}
		\qquad
		\parbox[c]{0.4\linewidth}{
			\begin{align*}
				K = \dfrac{[\ch{C}]^c[\ch{D}]^d}
				{[\ch{A}]^a[\ch{B}]^b}
			\end{align*}
			}
	\end{block}

	\begin{itemize}
		\item Since the ratio is product species over reactant species,
			if
			\begin{itemize}
				\item $K < 1$, the reaction is not favored
					(non-spontaneous)
				\item $K > 1$, the reaction is favored
					(spontaneous)
			\end{itemize}
		\item Due to their original thermodynamic derivation and the
			correct units, equilibrium constants are
			\emph{dimensionless}.
			\begin{itemize}
				\item{} [A](aq) really
					means [A]/(\SI{1}{\Molar})
				\item{} [A](g) really
					means [A]/(\SI{1}{bar})
			\end{itemize}
	\end{itemize}

	\framebreak

	To maintain the dimensionless character,

	\begin{enumerate}
		\item The concentrations of all solutes should be express in
			\si{\Molar} or \si{\formal}.
		\item The concentrations of all gases should be expressed in
			bar. (atm are close, but be consistent)
		\item The concentrations of pure solids, pure liquids, and
			solvents are omitted because they are \emph{invariant}.
	\end{enumerate}
\end{frame}

\begin{frame}{Manipulating Equilibrium Constants}
	\begin{align*}
		\intertext{For the following reaction,}
		\ch{HA} &\ch{<=>} \ch{H+ + A-} &&& K_1 &=
		\dfrac{[\ch{H+}][\ch{A-}]}{[\ch{HA}]} \\
		\intertext{Reversing the reaction \emph{inverses} the value of
		$K$:}
		\ch{H+ + A-} &\ch{<=>} \ch{HA} &&& K_1' &=
		\dfrac{[\ch{HA}]}{[\ch{H+}][\ch{A-}]} = \frac{1}{K_1} \\
		\intertext{Adding reaction, $K_2$, to $K_1$, the new $K$ is the
		product of the two individual values:}
		\intertext{\small\centering
			\begin{tabularx}{0.3\linewidth} {l c l @{\quad\ch{<=>}\quad} l
			c l c}
			\ch{HA} & & & \cancel{\ch{H+}} & + & \ch{A-} & $K_1$ \\
			\cancel{\ch{H+}} & + & \ch{C} & \ch{CH+} & & & $K_2$ \\ 
		\end{tabularx}}
		\ch{HA + C} &\ch{<=>} \ch{A- + CH+} &&& K_3 &= 
		\dfrac{[\ch{A-}][\ch{CH+}]}{[\ch{HA}][\ch{C}]} = K_1K_2
	\end{align*}
\end{frame}

\begin{frame}[t]{Manipulation Example}
	Find $K_{eq}$ for the reaction,
	\begin{reaction*}
		CaCO3\sld{} + CO2\aq{} + H2O\lqd{} <=> Ca^{2+}\aq{} + 2 HCO3^{-}\aq{}
	\end{reaction*}

	\mode<article>{\clearpage}

	\note{\scriptsize
	\begin{center}
		\begin{tabularx}{\textwidth} {c<{.} r @{\ch{<=>}} l
		X@{ = }S[table-format=1.2e-2]}
		1 &	\ch{CaCO3(s)} & \ch{Ca^{2+}(aq)} + \ch{CO3^{2-}(aq)} & $K_{sp}$ & 4.5e-9 \\
		2 &	\ch{CO3^{2-}(aq)} + \ch{H2O(l)} & \ch{HCO3^{-}(aq)} + \ch{OH^{-}(aq)} & $K_b$ & 2.1e-4 \\
		3 &	\ch{CO2(aq)} + \latecancel<7->{\ch{H2O(l)}} & \latecancel<7->{\ch{H2CO3(aq)}} & $K_h$ & 	1.7e-3 \\
		4 &	\ch{H2CO3(aq)} + \ch{H2O(l)} & \ch{HCO3^{-}(aq)} + \ch{H3O^{+}(aq)} & $K_a$ & 4.46e-7 \\
		5 &	\ch{H3O^{+}(aq)} + \ch{OH^{-}(aq)} & \ch{2 H2O(l)} & $1/K_w$ & 1.0e14  \\ \midrule
%	\end{tabu}
%
%	\begin{tabu} to \textwidth {r l X[r]@{ = }S[table-format=1.2e-2]}
%		\tabucline{-}
%		\phantom{\widthof{\ch{H3O^{+}(aq) + OH-(aq) <=>}}} &
%		\phantom{\widthof{\ch{HCO3^{-}(aq) + H3O^{+}(aq)}}} \\
		6 &	\ch{CaCO3(s) + CO2(aq) + H2O(l)} & \ch{Ca^{2+}(aq) +
		2 HCO3^{-}(aq)} & $K_{eq}$ & 7.2e-8 \\
	\end{tabularx}
	\end{center}
%
%%	\begin{align*}
%%		\MoveEqLeft \ch{CaCO3(s) + CO2(aq) + H2O(l)} \\
%%		&
%%	\end{align*}
	\begin{enumerate}
			\small
		\item Dissolve \ch{CaCO3} -- Solubility $K_{sp}$
		\item Weak base (\ch{CO3^{2-}}) reacts with \ch{H2O} -- Base
		     dissociation
		\item Add \ch{CO2}
		\item Weak acid (\ch{H2CO3}) reacts with \ch{H2O} -- Acid
		     dissociation
		\item Remove excess \ch{H3O+} and \ch{OH-} -- Hydrolysis
		\item Sum
	\end{enumerate}
}
\end{frame}

\section{Equilibrium and Thermodynamics}

\begin{frame}{Equilibrium and Thermodynamics}
	\begin{block}{Harris, 9\textsuperscript{th} Ed., pg. 121}
		``Equilibrium is controlled by the thermodynamics of a chemical
		reaction.  The heat absorbed or released (\emph{enthalpy}) and
		the dispersal of energy into molecular motion (\emph{entropy})
		independently contribute to the degree to which the reaction
		is favored or disfavored.''
	\end{block}

	\begin{itemize}
		\item \textbf{Enthalpy ($\bm{\Delta H}$)} is the heat absorbed
			or released when a reaction takes place \emph{under
			constant applied pressure}.
			\begin{center}
				\parbox{0.8\linewidth}{
					\begin{itemize}
						\item [$+\Delta H$] Heat is
							absorbed (endothermic)
						\item [$-\Delta H$] Heat is
							released (exothermic)
					\end{itemize}
					}
			\end{center}
		\item \textbf{Entropy ($\bm{\Delta S}$)} is the heat absorbed
			divided by the temperature when a \emph{reversible}
			reaction takes place \emph{at a constant temperature}.
			\begin{align*}
				\Delta S = \dfrac{q_{\text{rev}}}{T}
			\end{align*}
	\end{itemize}
\end{frame}

\begin{frame}{Free Energy}
	\begin{itemize}
		\item The \emph{Gibbs free energy} ($\Delta G$) declares
			whether a reaction is favorable:
			\begin{align*}
				\Delta G = \Delta H - T \Delta S
			\end{align*}
			\begin{center}
				\parbox{0.8\linewidth}{
					\begin{itemize}
						\item [$+\Delta G$] Unfavorable,
							non-spontaneous
						\item [$-\Delta G$] Favorable,
							spontaneous
					\end{itemize}
					}
			\end{center}
		\item The free energy is related to the equilibrium constant,
			$K_{eq}$,
			\begin{align*}
				\Delta G^\circ = -RT \ln K_{eq}
			\end{align*}
			where $R$ is the universal gas constant
			(\SI{8.314}{\joule\per\mole\per\kelvin})
				
			Note that this relation is for the
			\emph{standard}\footnote{The reference state of the
			materials} Gibbs free energy:
			\begin{align*}
				\Delta G^\circ = \Delta H^\circ
				- T \Delta S^\circ
			\end{align*}
	\end{itemize}
\end{frame}

\mode<presentation>{
\begin{frame}{Nuclear Meltdown}
	\begin{center}
		\includegraphics[width=0.8\textwidth]{elephantfoot.jpg}
	\end{center}

	The Chernobyl disaster was a \textbf{highly favorable} (in terms of
	$\Delta G$) chemical reaction.
\end{frame}}

\mode<article>{\clearpage}

\begin{frame}{Le Ch\^atelier's Principle}
	\begin{itemize}
		\item If stress\footnote{$\Delta$ concentrations, heat, volume,
			inert reagents, change in pressure, \ldots} is applied
			to a system at equilibrium, the equilibrium will shift
			in order to reduce the added stress.
		\item The \emph{reaction quotient} predicts the direction of
			the reaction when equilibrium is perturbed:
			\begin{align*}
				\ch{aA + bB <=> cC + dD} \qquad Q =
				\dfrac{[\ch{C}]^c[\ch{D}]^d}
				{[\ch{A}]^a[\ch{B}]^b}
			\end{align*}
			when A, B, C, and/or D are \emph{not} at equilibrium
			concentrations.
			\begin{center}
				\parbox{0.8\linewidth}{
					\begin{itemize}
						\item [$Q > K$] Reaction
							proceeds to the left
							(produces more
							reactants)
						\item [$Q < K$] Reaction
							proceeds to the right
							(produces more products)
						\item [$Q = K$] Reaction is at
							equilibrium
					\end{itemize}
					}
			\end{center}
	\end{itemize}
\end{frame}

\begin{frame}[t]{Reaction Quotient Example}
	\begin{align*}
		\ch{KNO3(s) <=> K^{+}(aq) + NO3^{-}(aq)}
	\end{align*}
	If both \ch{K+} and \ch{NO3-} concentrations are \SI{3.30}{\Molar}, are
	we at equilibrium at \SI{28.4}{\celsius} where $K_{eq} = 10.9$?

	\mode<article>{\vspace{10em}}

	\note<1>{
		\begin{align*}
			Q &= [\ch{K+}][\ch{NO3-}] \\
			&= (3.30)(3.30) = 10.9
		\end{align*}

		\begin{center}
			\fbox{$Q = K \therefore$ the reaction is at
			equilibrium.}
		\end{center}}

	\begin{itemize}[<+(1)->]
		\item What happens if we make [\ch{K+}] = \SI{5.50}{\Molar}?

			\mode<article>{\vspace{10em}}

	\note<2>{
		\begin{align*}
			Q &= [\ch{K+}][\ch{NO3-}] \\
			&= (5.50)(3.30) = 18.2
		\end{align*}

		\begin{center}
			\fbox{\parbox{0.8\linewidth}{
				\centering
				$Q \neq K \therefore$ the reaction is not at
				equilibrium.
		
				$Q > K \therefore$ the reaction shifts towards
				the reactants.}
				}
		\end{center}}

		\mode<article>{\clearpage}

		\item When does it stop?

			\mode<article>{\vspace{18em}}
	\end{itemize}

	\note<3>{
		\begin{align*}
			10.9 &= (5.50 - x)(3.30 - x) \\
			10.9 &= x^2 - 8.80x + 18.2 \\
			0 &= x^2 - 8.80x + 7.3 \\
			x &= 0.93 \text{ or } 7.8
		\end{align*}

		\begin{center}
		\fbox{\parbox{0.8\linewidth}{\centering
			The reaction precipitates \SI{0.93}{\mole}~\ch{KNO3},
			yielding [\ch{K+}]~=~\SI{4.57}{\Molar} and
			[\ch{NO3-}]~=~\SI{2.37}{\Molar}.}}
		\end{center}}
\end{frame}

\begin{frame}{Le Ch\^atelier's Principle: $\Delta T$}
	$K_{eq}$ is dependent on temperature as per
	\begin{align*}
		\Delta G^\circ &= -RT \ln K_{eq} \\
		\intertext{rearranging \ldots}
		K_{eq} &= e^{-\Delta G^\circ/RT} = e^{-(\Delta H^\circ -
		T\Delta S^\circ)/RT} &= e^{(-\Delta H^\circ/RT + \Delta
		S^\circ/R)} \\
		& &= e^{-\Delta H^\circ/RT} \cdot e^{\Delta S^\circ/R}
	\end{align*}
	Therefore,
	\begin{itemize}
		\item $K_{eq}$ of an endothermic reaction ($\Delta H^\circ > 0$)
			increases if $T$ is raised
		\item $K_{eq}$ of an exothermic reaction ($\Delta H^\circ < 0$)
			decreases if $T$ is raised
	\end{itemize}
\end{frame}

\section{Applications of Equilibrium Constants}

\begin{frame}[t]{Applications of Equilibrium Constants}
	\begin{itemize}
		\item<1-> Solubility, $K_{sp}$
			\only<1|handout:0>{
				\begin{itemize}
					\item Appendix F
					\item \ch{Ag2CrO4(s) <=> 2 Ag^{+}(aq) +
						CrO4^{2-}(aq)} \quad $K_{sp} =
						\num{1.2e-12}$
				\end{itemize}}
		\item<2-> Formation, $K_f$
			\only<2|handout:0>{
				\begin{itemize}
					\item \ch{Ca^{2+}(aq) + EDTA^{4-}(aq)
						<=> CaEDTA^{2-}(aq)} \quad $K_f =
						\num{4.9e10}$
				\end{itemize}}
		\item<3-> Acid, $K_a$
			\only<3|handout:0>{
				\begin{itemize}
					\item Appendix G
					\item \ch{CH3COOH(aq) + H2O(l) <=>
						CH3COO^{-}(aq) + H3O^{+}(aq)} \quad $K_a =
						\num{1.8e-5}$
				\end{itemize}}
		\item<4-> Base, $K_b$
			\only<4|handout:0>{
				\begin{itemize}
					\item \ch{NH3(aq) + H2O(l) <=> NH4^{+}(aq)
						+ OH^{-}(aq)} \quad $K_b = \num{1.8e-5}$
					\item $K_aK_b = K_w$
				\end{itemize}}
		\item<5-> Hydrolysis, $K_w$
			\only<5|handout:0>{
				\begin{itemize}
					\item \ch{2 H2O(l) <=> H3O^{+}(aq) +
						OH^{-}(aq)} \quad $K_w = \num{1.00e-14}$
				\end{itemize}}
		\item<6-> Others
			\only<6|hanout:0>{
				\begin{itemize}
					\item All equilibrium constants are of
						the same format:
						\begin{align*}
							K_x =
							\dfrac{\prod[\text{products}]}
							{\prod[\text{reactants}]}
						\end{align*}
					\item It doesn't really matter what it's
						called!
				\end{itemize}}
	\end{itemize}
\end{frame}

\begin{frame}[t]{Solubility Product Example}
	What is the solubility of silver chromate in pure water and in a
	solution of \SI{0.0500}{\Molar} aluminum \alert<2>{chromate}?

	\begin{align*}
		\ch{Ag2CrO4(s) <=> 2 Ag^{+}(aq) + CrO4^{2-}(aq)} \qquad K_{sp} =
		\num{1.2e-12}
	\end{align*}

	\note<1>{
	\textbf{Pure water:}

	\begin{center}
		\begin{tabular} {c*{5}{r}}
			& \ch{Ag2CrO4(s)} & \ch{<=>} & \ch{2 Ag^{+}(aq)} &+&
			\ch{CrO4^{2-}(aq)} \\
			I & solid && 0 && 0 \\
			C & $-x$ && $+2x$ && $+x$ \\ \midrule
			E & $\text{solid}-x$ && $2x$ && $x$
		\end{tabular}
	\end{center}

	\begin{align*}
		K_{sp} &= [\ch{Ag^{+}}]^2[\ch{CrO4^{2-}}] \\
		\num{1.2e-12} &= (2x)^2(x) \\
		&= 4x^3 \\
		x &= \sqrt[3]{\num{1.2e-12}/4} \\
		&= \num{6.69e-5} \equiv \num{7e-5}
	\end{align*}}

	\visible<2-|handout:0>{
	\begin{center}
		\textbf{\alert{The common ion effect!}}
	\end{center}}

	\mode<article>{\clearpage}


\note<2>{
	\textbf{Aluminum chromate solution:}

	\begin{center}
		\ch{Al2(CrO4)3(s) -> 2 Al^{3+}(aq) + 3 CrO4^{2-}(aq)}

		$ \SI{0.0500}{\Molar}~\ch{Al2(CrO4)3} \times
		\dfrac{\SI{3}{\mol}~\ch{CrO4^{2-}}}{\SI{1}{\mole}~\ch{Al2(CrO4)3}}$

		\begin{tabular} {c*{5}{r}}
			& \ch{Ag2CrO4(s)} & \ch{<=>} & \ch{2 Ag^{+}(aq)} &+&
			\ch{CrO4^{2-}(aq)} \\
			I & solid && 0 && 0.1500 \\
			C & $-x$ && $+2x$ && $+x$ \\ \midrule
			E & $\text{solid}-x$ && $2x$ && $0.1500 + x$
		\end{tabular}
	\end{center}

	\begin{columns}
		\column{0.5\linewidth}
		\begin{align*}
			K_{sp} &= [\ch{Ag+}]^2[\ch{CrO4^{2-}}] \\
			\num{1.2e-12} &= (2x)^2(0.1500 + x) \\
			&= 4x^2 \times 0.1500 \\
		\end{align*}
		\column{0.5\linewidth}
		\begin{align*}
			x &= \sqrt{\num{8e-12}/4} \\
			&= \num{1.4e-6}
		\end{align*}

		$x$ is 0.0009\% of 0.1500 -- no worries!

	\end{columns}
	}


\end{frame}

\begin{frame}[allowframebreaks]{Formation: Lewis Acid/Base Reactions}
	If anion \ch{X-} precipitates metal \ch{M+}, often a high concentration
	of \ch{X-} causes solid \ch{MX} to redissolve.
	\begin{itemize}
		\item Increased solubility is due to formation of \emph{complex
			ions}.
		\item Lewis acids and Lewis bases react to form
			\emph{coordinate covalent bonds}.
	\end{itemize}

	For example,
	\begin{align*}
		\ch{Pb^{2+} + OH^{-} &<=> PbOH^{+}} \\
		\shortintertext{but also,}
		\ch{Pb^{2+} + 2 OH- &<=> Pb(OH)2}
	\end{align*}

	\textbf{What product is present?}

	\framebreak

	All will be present!
	\begin{itemize}
		\item $K_i$ will be used for \emph{stepwise} formation
			constants:
			\begin{center}
			\begin{tabular} {l@{ + \ch{OH- <=>} }l c}
				\ch{Pb^{2+}} & \ch{PbOH+} & $K_1$ \\
				\ch{PbOH+} & \ch{Pb(OH)2} & $K_2$ \\
				\ch{Pb(OH)2} & \ch{Pb(OH)3-} & $K_3$ \\
				\ch{Pb(OH)3-} & \ch{Pb(OH)4^{2-}} & $K_4$
			\end{tabular}
			\end{center}
		\item $\beta_i$ will be used for \emph{cumulative} formation
			constants:
			\begin{center}
				\begin{tabular} {l@{ + }r@{ \ch{<=>}
					}l l}
					\ch{Pb^{2+}} & \ch{OH-} & \ch{PbOH+} &
					$K_1$ \\
					\ch{Pb^{2+}} & \ch{2 OH-} & \ch{Pb(OH)2} & $\beta_2 =
				K_1K_2$ \\
					\ch{Pb^{2+}} & \ch{3 OH-} & \ch{Pb(OH)3-} & $\beta_3 =
				K_1K_2K_3$ \\
					\ch{Pb^{2+}} & \ch{4 OH-} & \ch{Pb(OH)4^{2-}} &
				$\beta_4 = K_1K_2K_3K_4$
			\end{tabular}
			\end{center}
	\end{itemize}

	There can only be one concentration of \ch{Pb^{2+}} in solution, so
	solving one equilibrium must satisfy all equilibria.

	\framebreak

	\begin{columns}
		\column{0.5\textwidth}
		\begin{itemize}
			\item As [\ch{I-}] increases, [\ch{Pb_{total}}]
				decreases due to common ion effect.
			\item At very high [\ch{I-}], the \ch{PbI2(s)}
				redissolves due to complexation.
		\end{itemize}
		\column{0.5\textwidth}
		\includegraphics[scale=0.4]{PbI2-complex.jpg}
	\end{columns}
\end{frame}

\begin{frame}[t]{Formation Example}
	Find the concentrations of \ch{PbI+}, \ch{PbI2\aq{}}, \ch{PbI3-}, and
	\ch{PbI4^{2-}} in a solution saturated with \ch{PbI2\sld} and containing
	dissolved \ch{I-} with a concentration of \SI{0.0010}{\Molar}.

	\medskip

	\begin{tabular}{L @{ $=$ } S[table-format=1.1e-1]}
		K_\text{sp}  & 7.9e-9 \\
		K_1     & 1.0e2  \\
		\beta_2 & 1.4e3  \\
		\beta_3 & 8.3e3  \\
		\beta_4 & 3.0e4
	\end{tabular}

	\mode<article>{\vfill}

	\note{ \small
		\begin{align*}
			K_1     &= \frac{[\ch{PbI+     }]}{[\ch{Pb^{2+}}][\ch{I-}]  } &
			\beta_2 &= \frac{[\ch{PbI2\aq  }]}{[\ch{Pb^{2+}}][\ch{I-}]^2} &
			\beta_3 &= \frac{[\ch{PbI3-    }]}{[\ch{Pb^{2+}}][\ch{I-}]^3} &
			\beta_4 &= \frac{[\ch{PbI4^{2-}}]}{[\ch{Pb^{2+}}][\ch{I-}]^4} &
		\end{align*}
		\begin{align*}
			\intertext{We need [\ch{Pb^{2+}}]!}
			K_\text{sp} &= [\ch{Pb^{2+}}][\ch{I-}]^2 &
			\therefore [\ch{Pb^{2+}}] &= \frac{K_\text{sp}}{[\ch{I-}]^2}
			= \frac{\num{7.9e-9}}{\num{0.0010}^2} = \SI{7.9e-3}{\Molar} \\
			&& [\ch{PbI+}] &= K_1[\ch{Pb^{2+}}][\ch{I-}] = \SI{7.9e-4}{\Molar} \\
			&& [\ch{PbI2\aq}] &= \beta_2[\ch{Pb^{2+}}][\ch{I-}]^2 = \SI{1.1e-5}{\Molar} \\
			&& [\ch{PbI3-  }] &= \beta_3[\ch{Pb^{2+}}][\ch{I-}]^3 = \SI{6.6e-8}{\Molar} \\
			&& [\ch{PbI4^{2+}}] &= \beta_4[\ch{Pb^{2+}}][\ch{I-}]^4 = \SI{2.4e-10}{\Molar} \\
		\end{align*}
	}
\end{frame}

\begin{frame}[t]{Acid Dissociation Example}
	What is the pH of a 5.0\% (w/w) vinegar solution? ($K_a = \num{1.8e-5}$)

	\mode<article>{\vspace{15em}}

	\note{\vspace{-2em}\begin{align*}
		\dfrac{\SI{5}{\gram}~\ch{HAc}}{\SI{100.0}{\gram~sol}} \times
		\overbrace{\dfrac{\SI{1000}{\gram~sol}}{\SI{1}{\liter}}}^{\mathclap{\text{assume}~d = d_{\ch{H2O}} = \SI{1}{\gram\per\milli\liter}}} \times
		\dfrac{\SI{1}{\mole}~\ch{HAc}}{\SI{60.06}{\gram}} =
		\SI{0.83}{\Molar}
	\end{align*}

	\begin{center}
		\begin{tabular} {@{}c*{5}{r}}
			& \ch{CH3COOH(aq)} & \ch{+ H2O(l) <=>} &
			\ch{CH3COO^{-}(aq)} &+& \ch{H3O^{+}(aq)} \\
			I & 0.83 && 0 && 0 \\
			C & $-x$ && $+x$ && $+x$ \\ \midrule
			E & $0.83-x$ && $x$ && $x$
		\end{tabular}
	\end{center}

	\begin{columns}
		\column{0.5\linewidth}
		\begin{align*}
			\num{1.8e-5} &= \dfrac{[\ch{CH3COO-}][\ch{H3O+}]}{\ch{CH3COOH}}
			\\
			&= \dfrac{(x)(x)}{0.83-x} = \dfrac{x^2}{0.83} \\
			x &= \SI{3.9e-3}{\Molar}~\ch{H3O+}
		\end{align*}
		\column{0.5\linewidth}
		$x$ is 0.5\% of 0.83
		\begin{align*}
			\text{pH} &= -\log [\ch{H3O+}] \\
			&= -log(0.0039) \\
			&= 2.41
		\end{align*}
	\end{columns}
	}
\end{frame}

\mode<article>{\clearpage}

\begin{frame}[t]{Base Dissociation Example}
	What is the pH of a solution that is \SI{0.175}{\Molar}~\ch{NH3}? ($K_b
	= \num{1.8e-5}$)

	\mode<article>{\vspace{18em}}

\note{
	\begin{center}
		\begin{tabular} {@{}c*{5}{r}}
			& \ch{NH3(aq)} & \ch{ + H2O(l) <=>} &
			\ch{NH4^{+}(aq)} &+& \ch{OH^{-}(aq)} \\
			I & 0.175 && 0 && 0 \\
			C & $-x$ && $+x$ && $+x$ \\ \midrule
			E & $0.175-x$ && $x$ && $x$
		\end{tabular}
	\end{center}

	\begin{columns}
		\column{0.5\linewidth}
		\begin{align*}
			\num{1.8e-5} &= \dfrac{[\ch{NH4+}][\ch{OH-}]}{\ch{NH3}}
			\\
			&= \dfrac{(x)(x)}{0.175-x} = \dfrac{x^2}{0.175} \\
			x &= \SI{1.8e-3}{\Molar}~\ch{OH-}
		\end{align*}
		\column{0.5\linewidth}
		\begin{align*}
			\text{pOH} &= -\log [\ch{OH-}] \\
			&= -\log(0.0018) = 2.75 \\
			\text{pH} &= 14.00 - \text{pOH} = 11.25
		\end{align*}
	\end{columns}
	}
\end{frame}

\begin{frame}{Recall: Acid/Base Strength}
	\begin{itemize}
		\item Acid/base dissociation equilibria are considered for
			\emph{weak} acids and bases.
		\item Strong acids and bases dissociate \emph{completely} --
			i.e. $K_a$ or $K_b$ are so large that it is not worth
			calculating
	\end{itemize}

	\begin{tabularx}{\textwidth} {*{2}{lX}}
		\multicolumn{2}{l}{\textit{Acids}} &
		\multicolumn{2}{l}{\textit{Bases}} \\
		\bfseries Name & \bfseries Formula & \bfseries Name & \bfseries Formula \\
		\midrule
		\ch{HCl} & Hydrochloric acid & \ch{LiOH} & Lithium hydroxide \\
		\ch{HBr} & Hydrogen bromide & \ch{NaOH} & Sodium hydroxide \\
		\ch{HI} & Hydrogen iodide & \ch{KOH} & Potassium hydroxide \\
		\ch{H2SO4} & Sulfuric acid & \ch{RbOH} & Rubidium hydroxide \\
		\ch{HNO3} & Nitric acid & \ch{CsOH} & Cesium hydroxide \\
		\ch{HClO4} & Perchloric acid & \ch{R4NOH} & Quaternary ammonium
		hydroxide
	\end{tabularx}
\end{frame}

\begin{frame}[allowframebreaks]{Polyprotic species}
	\begin{itemize}
		\item Some compounds can donate or accept more than one proton.
			For example,
			\begin{align*}
				\ch{H3PO4 + H2O &<=> H3O+ + H2PO4-} \\
				\ch{H2PO4- + H2O &<=> H3O+ + HPO4^{2-}} \\
				\ch{HPO4^{2-} + H2O &<=> H3O+ + PO4^{3-}}
			\end{align*}
		\item It is harder to remove (or add) each successive proton,
			thus
			\begin{equation*}
				K_\text{a1} >
				K_\text{a2} >
				K_\text{a3} > \cdots \quad\text{and}\quad
				K_\text{b1} >
				K_\text{b2} >
				K_\text{b3} > \cdots
			\end{equation*}
		\item Which \Ka\ and \Kb\ do we use to calculate \Kw?
	\end{itemize}

	\framebreak

	For \ch{H3PO4},
	\begin{reaction*}
		H3PO4     <=>[ $K$_{a1} ][ $K$_{b3} ]
		H2PO4-    <=>[ $K$_{a2} ][ $K$_{b2} ]
		HPO4^{2-} <=>[ $K$_{a3} ][ $K$_{b1} ]
		PO4^{3-}
	\end{reaction*}
	\begin{equation*}
		K_\text{a1}K_\text{b3} =
		K_\text{a2}K_\text{b2} =
		K_\text{a3}K_\text{b1} =
		K_\text{w}
	\end{equation*}

	\vspace{2em}

	For EDTA (\ch{H6Y^{2+}}) -- a hexavalent acid,
	{\fontsize{9pt}{0pt}\selectfont
	\begin{reaction*}
		H6Y^{2+} <=>[ $K$_{a1} ][ $K$_{b6} ]
		H5Y+     <=>[ $K$_{a2} ][ $K$_{b5} ]
		H4Y      <=>[ $K$_{a3} ][ $K$_{b4} ]
		H3Y-     <=>[ $K$_{a4} ][ $K$_{b3} ]
		H2Y^{2-} <=>[ $K$_{a5} ][ $K$_{b2} ]
		HY^{3-}  <=>[ $K$_{a6} ][ $K$_{b1} ] Y^{4-}
	\end{reaction*}
}
	\begin{equation*}
		K_\text{a1}K_\text{b6} = 
		K_\text{a2}K_\text{b5} = 
		K_\text{a3}K_\text{b4} = 
		K_\text{a4}K_\text{b3} =
		K_\text{a5}K_\text{b2} = 
		K_\text{a6}K_\text{b1} =
		K_\text{w}
	\end{equation*}
\end{frame}


\end{document}
