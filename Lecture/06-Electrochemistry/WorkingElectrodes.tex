\documentclass[notes=only]{beamer}
%\usepackage{beamerarticle}

%\usepackage[margin=1in]{geometry}

\usepackage{newtxtext}
\usepackage{analchem}
\usepackage{lecture}
\usepackage{chemmacros}

\title{Types of Working Electrode}
\author{CHEM321}
\date{Fall 2021}

\begin{document}

\maketitle

\begin{frame}{Electrodes of the First Kind}
		A bare metal wire or strip is placed into
			the solution and is used to measure the
			concentration of the same species metal
			ion.
			\begin{align*}
				\ch{M^{$n$+} + $n$\el{} &<=>
				M^0} \\
				\state[pre=,superscript-right=]{E}
				&= \state[pre=]{E} -
				\frac{0.05916}{n} \log
				\frac{1}{[\ch{M^{$n$+}}]}
			\end{align*}
		The simplest of all electrode systems.
\end{frame}

\begin{frame}[t]{Electrodes of the First Kind Example}
	A cell was prepared by dipping a Ag wire and a saturated calomel
	electrode into a \SI{0.10}{\Molar} \ch{AgNO3} solution. The Ag wire was
	attached to the positive terminal of a potentiometer and the SCE was
	attached to the negative (reference) terminal.
	\begin{enumerate}
		\item Write a half-reaction for the Ag electrode.
		\item Write the Nernst equation for the Ag electrode.
		\item Calculate the cell voltage.
			\visible<2->{\textbf{Activity?}}
	\end{enumerate}

\note{\footnotesize
	\ch{Ag+ + 1 \el{} <=> Ag\sld{}} \qquad $E^0 = \SI{0.7993}{\volt}$

	\begin{align*}
		E &= E^{0} - \frac{RT}{nF} \ln
		\frac{1}{[\ch{Ag+}]\gamma_{\ch{Ag+}}} \\
		&= 0.7993 - 0.05916 \log (0.10 \times 0.75) \\
		&= \SI{0.8659}{\volt}~\text{vs. SHE} \\
		&= \SI{0.625}{\volt}~\text{vs. SCE} \\
		\intertext{Without activity:}
		E &= E^{0} - \frac{RT}{nF} \ln
		\frac{1}{[\ch{Ag+}]} \\
		&= 0.7993 - 0.05916 \log (0.10) \\
		&= \SI{0.8585}{\volt}~\text{vs. SHE} \\
		&= \SI{0.617}{\volt}~\text{vs. SCE}
	\end{align*}
	}
\end{frame}

\clearpage

\begin{frame}{Electrodes of the Second Kind}
	\only<+>{%
		The first kind bare metal electrode is now
			coated with an insoluble salt containing
			its ion and some anion. Variations in
			the amount of anion causes variations in
			potential:
			\begin{align*}
				\ch{M_{$n$}X_{$y$} +
				$\frac{y}{n}$ \el{}
				&<=>
				M^0 + $y$ X^{$\sfrac{n}{y}-$}} \\
				\state[pre=,superscript-right=]{E}
				&= \state[pre=]{E} -
				\frac{0.05916}{n} \log
				[\ch{X^{$\sfrac{n}{y}-$}}]^y
			\end{align*}
		Since there are lots of insoluble salts,
			these can expand the number of metallic
			electrodes nicely.
			\begin{align*}
				\ch{Ag+ + \el{} &<=> Ag^0}
				\qquad &&\state[pre=]{E} =
				\SI{0.7993}{\volt} \\
				\ch{AgCl + \el{} &<=> Ag^0 +
				Cl-} \qquad &&\state[pre=]{E} =
				\SI{0.222}{\volt}
			\end{align*}
		}

		\only<+>{%
		An interesting example is to make an EDTA
			electrode using mercury(II) and mercury
			metal:
			\begin{align*}
				\ch{HgY^{2-} + 2 \el{} &<=>
				Hg\lqd{} + Y^{4-}} \qquad
				\state[pre=]{E} =
				\SI{0.21}{\volt} \\
				E &= 0.21 - \frac{0.05916}{2}
				\log
				\frac{\mathcal{A}_{\ch{Y^{4-}}}}{\mathcal{A}_{\ch{HgY^{2-}}}}
				\intertext{To start the system,
				a miniscule amount of
				\ch{HgY^{2-}} is introduced.
				Since $K_{\text{f}}$ is
				\SI{6.3e21}, the amount added is
				stable and
				$\mathcal{A}_{\ch{HgY^{2-}}}$ is
				thus fixed making the response:}
				E &= 0.21 - \frac{0.05916}{2}
				\log
				\mathcal{A}_{\ch{Y^{4-}}} = K +
				0.02958 \p{\ch{Y^{4-}}}
			\end{align*}
			This electrode is useful to establishing
			the endpoint of EDTA titrations.
		}
\end{frame}

\begin{frame}[t]{Electrodes of the Second Kind Example}
	A cell was prepared by dipping a Ag wire, pre-coated with AgCl and a
	saturated calomel electrode into a \SI{0.10}{\Molar} \ch{KCl} solution.
	The AgCl wire was attached to the positive terminal of a potentiometer
	and the SCE was attached to the negative (reference) terminal.
	\begin{enumerate}
		\item Write a half-reaction for the AgCl electrode.
		\item Write the Nernst equation for the AgCl electrode.
		\item Calculate the cell voltage.
			\visible<2->{\textbf{Activity?}}
	\end{enumerate}

	\note{\footnotesize
	\ch{AgCl + 1 \el{} <=> Ag\sld{} + Cl-} \qquad $E^0 = \SI{0.222}{\volt}$

	\begin{align*}
		E &= E^{0} - \frac{RT}{nF} \ln
		[\ch{Cl-}]\gamma_{\ch{Cl-}} \\
		&= 0.222 - 0.05916 \log (0.10 \times 0.755) \\
		&= \SI{0.288}{\volt}~\text{vs. SHE} \\
		&= \SI{0.047}{\volt}~\text{vs. SCE} \\
		\intertext{Without activity:}
		E &= E^{0} - \frac{RT}{nF} \ln
		[\ch{Cl-}] \\
		&= 0.222 - 0.05916 \log (0.10) \\
		&= \SI{0.281}{\volt}~\text{vs. SHE} \\
		&= \SI{0.040}{\volt}~\text{vs. SCE}
	\end{align*}
	}
\end{frame}

\clearpage
	
\begin{frame}{Electrodes of the Third Kind}
		If we can make the metallic electrode
			respond to another cation, it will be an
			electrode of the third kind. Let's
			consider the mercury-EDTA electrode.
			Since
			\begin{align*}
				\ch{Ca^{2+} + Y^{4-} &<=>
				CaY^{2-}} \qquad K_{\text{f}} =
				\frac{\mathcal{A}_{\ch{CaY^{2-}}}}
				{\mathcal{A}_{\ch{Ca^{2+}}}
				\mathcal{A}_{\ch{Y^{4-}}}} \\
				E &= K - \frac{0.05916}{2} \log
				\mathcal{A}_{\ch{Y^{4-}}}
				= K - \frac{0.05916}{2} \log
				\frac{\mathcal{A}_{\ch{CaY^{2-}}}}
				{K_{\text{f}}\mathcal{A}_{\ch{Ca^{2+}}}
				} \\
				\intertext{rearranging yields:}
				E &= K - \frac{0.05916}{2} \log \frac{\mathcal{A}_{\ch{CaY^{2-}}}}
				{K_{\text{f}}} -
				\frac{0.05916}{2} \log
				\frac{1}{\mathcal{A}_{\ch{Ca^{2+}}}}
				\\
				&= K' - 0.02958\p{\ch{Ca}}
			\end{align*}
			Mercury has now become an electrode of
			the third kind --- a sensor for a
			different metal.
\end{frame}

\begin{frame}[t]{Electrodes of the Third Kind Example}
	A cell was prepared by dipping a AgCl wire and a
	saturated calomel electrode into a \SI{0.10}{\Molar} \ch{Pb(NO3)2}
	solution.  The AgCl wire was attached to the positive terminal of a
	potentiometer and the SCE was attached to the negative (reference)
	terminal.
	\begin{enumerate}
		\item Write a half-reaction for the AgCl electrode.
		\item Write the solubility reaction for \ch{PbCl2}.
		\item Write the Nernst equation for the equilibria present.
		\item Calculate the cell voltage.
	\end{enumerate}

\note{
	\begin{align*}
		\ch{AgCl + 1 \el{} &<=> Ag\sld{} + Cl-} \qquad &&E^0 =
		\SI{0.222}{\volt} \\
		\ch{PbCl2 &<=> Pb^{2+} + 2 Cl-} \qquad &&K_\text{sp} =
		\num{1.7e-5}
	\end{align*}

	\begin{align*}
		E &= E^{0} - \frac{RT}{nF} \ln [\ch{Cl-}] \gamma_{\ch{Cl-}} \\
		K_{\text{sp}} &= [\ch{Pb^{2+}}] \gamma_{\ch{Pb^{2+}}}
		[\ch{Cl-}]^2 \gamma_{\ch{Cl-}}^2 \\
		[\ch{Cl-}] \gamma_{\ch{Cl-}} &=
		\sqrt{\frac{K_\text{sp}}{[\ch{Pb^{2+}}] \gamma_{\ch{Pb^{2+}}}}}
	\end{align*}
	}
\end{frame}

\note{
	\begin{align*}
		E &= E^{0} - \frac{RT}{nF} \ln
		\sqrt{\frac{K_\text{sp}}{[\ch{Pb^{2+}}] \gamma_{\ch{Pb^{2+}}}}}
		\\
		E &= 0.222 - 0.05916 \log \sqrt{\frac{\num{1.7e-5}}{0.1 \times
		0.37}} \\
		&= \SI{0.419}{\volt}~\text{vs. SHE} \\
		&= \SI{0.178}{\volt}~\text{vs. SCE}
	\end{align*}
	}

\clearpage

\begin{frame}{Metal Redox Electrodes}
	\begin{itemize}
		\item The inert metal electrodes used to study
			soluble species. Primary are platinum,
			gold, and palladium (inert metals).
			Carbon is also included, although it is
			non-metallic.

		\item An example is the use of platinum in the
			hydrogen electrode or in a solution of
			cerium(III) and cerium(IV):
			\begin{align*}
				\ch{Ce^{4+} + \el{} &<=>
				Ce^{3+}} \\
				E &= \state[pre=]{E} - 0.05916
				\log
				\frac{\mathcal{A}_{\ch{Ce^3+}}}
				{\mathcal{A}_{\ch{Ce^{4+}}}}
			\end{align*}
			In this fashion, platinum can serve as
			the electrode for a titration using
			cerium(IV) as the titrant.
		\item Why aren't these called `Electrodes of the
			Fourth Kind' or `More Electrodes of the
			Third Kind'?
	\end{itemize}
\end{frame}

\begin{frame}{What type of electrode is each reference?}
	\begin{enumerate}
		\item Normal Hydrogen Electrode

			\bigskip

			\note<item>{Metal Redox}

		\item Saturated Calomel Electrode

			\bigskip

			\note<item>{Second Kind}

		\item Silver-Silver Chloride Electrode

			\bigskip

			\note<item>{Second Kind}
	\end{enumerate}

	\bigskip
	\pause

	\begin{block}{Note}
		Reference electrodes are just indicator
		electrodes with \alert{known} potentials!
	\end{block}
\end{frame}

\end{document}
