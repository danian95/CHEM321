%! TEX program = xelatex
\documentclass[notes=only]{beamer}
%\documentclass[notes=hide]{beamer}
%\documentclass[notes=onlyslideswithnotes,notes=hide]{beamer}
%\documentclass[11pt,letterpaper]{article}
%\usepackage{beamerarticle}

\usepackage{analchem}
\usepackage{lecture}
\usepackage{chemmacros}
\usepackage[symbol]{parnotes}
\usepackage{pdfpages}

\usechemmodule{all}

\title{Electrochemistry}
\subtitle{Chapters 14--16}
\institute{CHEM321 --- Analytical Chemistry I \\ Bloomsburg University}
\author{D.A. McCurry}
\date{Fall 2020}

\begin{document}

\maketitle
\mode<article>{\thispagestyle{fancy}}

\frame{\section{Equilibrium and Electrochemistry}
	\begin{learningobjectives}
	\item Explain the difference between oxidation and reduction.
	\item Relate electric charge to stoichiometric quantities.
	\item Derive the relationship between equilibrium and potential
		differences.
	\end{learningobjectives}
}

\begin{frame}{Recall $K$}
	\begin{itemize}
		\item An equilibrium between \ch{Zn^{2+}} cations and
			\ch{Cu^{2+}} cations will be established in solution
			according to the following:

			\begin{align*}
				\ch{Cu^{2+} + Zn\sld{} &<=> Cu\sld{} + Zn^{2+}}
				\qquad K_{\text{eq}} =
				\frac{[\ch{Zn^{2+}}]}{[\ch{Cu^{2+}}]}
			\end{align*}
		
		\item If $K_\text{eq}$ is very large, we know that this will be
			a spontaneous reaction as

			\begin{align*}
				\state{G} = -RT \ln K
			\end{align*}
	\end{itemize}
\end{frame}

\begin{frame}{Redox Reactions}
	A \alert{redox reaction} involves the transfer of electrons from one
	species to another.
\begin{itemize}
		\item A species is said to be \alert{oxidized} when it
			\alert{loses} electrons.
			\begin{reactions*}
				Zn\sld{} &-> Zn^{2+} + 2 \el
				\intertext{\item It is \alert{reduced} when it
					\alert{gains} electrons.}
				Cu^{2+} + 2 \el{} &-> Cu\sld
			\end{reactions*}
		\item An \alert{oxidizing agent} takes electrons from another
			substance and becomes \alert{reduced}.
		\item A \alert{reducing agent} give electrons to another
			substance and is \alert{oxidized} in the process.
	\end{itemize}
\end{frame}

\vspace{\stretch{-1}}

\begin{frame}{Why is oxidation the loss of electrons?}
	\begin{reactions*}
		\intertext{Consider the following:}
		Ca^0 + 1/2 O2 &<=> CaO
		\intertext{Note the calcium is losing electrons:}
		Ca^0 &<=> Ca^{2+} + 2 \el
		\intertext{as it is picking up oxygen, or \emph{being
		oxidized}.}
	\end{reactions*}
\end{frame}

\vspace{\stretch{-1}}

\begin{frame}{Why is reduction the gain of electrons?}
	\begin{reactions*}
		\intertext{Consider the following:}
		Ca^0 + 1/2 O2 &<=> CaO
		\intertext{Note the oxygen is gaining electrons:}
		1/2 O2 + 2 \el &<=> O^{2-}
		\intertext{and the oxidation number is \emph{being
		reduced}.}
	\end{reactions*}
\end{frame}

\begin{frame}{Electrons and Chemistry}
	\begin{itemize}
		\item Electric charge ($q$):
			\begin{itemize}
				\item Charge is measured in coulombs
					(\si{\coulomb}).
			\end{itemize}
		\item A single electron has a charge of
			\SI{1.602177335e-19}{\coulomb}.
		\item Thus, one mole of electrons has a charge of 
			\begin{align*}
				\SI{1.602e-19}{\coulomb} \times
				\SI{6.022e23}{\per\mole} &=
				\underbrace{\SI{96485}{\coulomb\per\mole}}_{\mathclap{\text{
				Faraday constant ($F$)}}} \\
			q &= n \cdot F
			\end{align*}
		\item The quantity of charge flowing each second through a
			circuit is called the \alert{current}.
			\begin{align*}
				\SI{1}{\ampere}~(\text{ampere}) =
				\frac{\SI{1}{\coulomb}}{\SI{1}{\second}}
			\end{align*}
	\end{itemize}
\end{frame}

\vspace{\stretch{-1}}

\begin{frame}[t]{Using current to drive chemical reactions}
	Suppose electrons are forced into a platinum wire immersed in a
	\SI{50.0}{\milli\liter} solution of \SI{0.5}{\Molar} 1,4-benzoquinone
	(\ch{C6H4(=O)2}) which is reduced to hydroquinone (\ch{C6H4(-OH)2}) at a
	constant current of \SI{0.227}{\ampere}.  How long will it take
	to drive the reaction to completion?

\note{
	What's the reaction?
	\begin{center}
		\chemfig{[,0.5]*6((=O)-=-(=O)-=-)} \ch{+ 2 \el{} + 2 H+ <=>}
		\chemfig{[,0.5]*6((-HO)-=-(-OH)=-=-)}
	\end{center}

	How many moles do we have?
	\begin{align*}
		\SI{0.5}{\Molar} \times \SI{0.050}{\liter} = \SI{0.025}{\mole}
	\end{align*}

	Plug in the reaction rate:
	\begin{align*}
		\SI{0.025}{\mole} \times
		\frac{\SI{2}{\mole}~\el}{\SI{1}{\mole}} \times
		\frac{\SI{96485}{\coulomb}}{\SI{1}{\mole}~\el} \times
		\frac{\SI{1}{\second}}{\SI{0.227}{\coulomb}} &=
		\SI{2.1e4}{\second} \\
		&\approx \fbox{\SI{5.9}{\hour}}
	\end{align*}
	}
\end{frame}

\clearpage

\begin{frame}[allowframebreaks]{Voltage, Work, and Free Energy}
	How do we \emph{induce} current?

	\begin{itemize}
		\item The difference in \alert{electric potential} ($\Delta E$)
			between two points is a measure of the work that
			\emph{can be} done or \emph{must be} done when an
			electric charge moves from one point to the other.
			\begin{itemize}
				\item This distinction separates `galvanic'
					cells from `electrolytic' cells!
			\end{itemize}
		\item $\Delta E$ is measured in volts (\si{\volt}).
			\begin{itemize}
				\item From physics, we know work has dimensions
					of energy (\si{\joule}).
				\item One joule is gained or lost when one
					coulomb charge is moved by points whose
					potential differs by one volt.
			\end{itemize}
			\begin{align*}
				\text{work} = \Delta E \cdot q
			\end{align*}

			\framebreak

		\item Free energy ($\Delta G$) for a chemical reaction conducted
			\alert{reversibly} at constant temperature and pressure
			equals the maximum possible electrical work that can be
			done on the surroundings.
			\begin{align*}
				\text{work done on surroundings} &= -\Delta G
				\intertext{Therefore,}
				\Delta G = -\text{work} = -\Delta E \cdot q &=
				-nF\Delta E
			\end{align*}
	\end{itemize}
\end{frame}

\begin{frame}{Equilibrium or Electric Potential?}
	\begin{align*}
		\intertext{So,}
		\state{G} &= -nF\state{E}
		\intertext{and}
		\state{G} &= -RT \ln K
		\intertext{Therefore,}
		\state{E} &= \frac{RT}{nF} \ln K
	\end{align*}

	\pause

	Electric potentials are simply a measure of \alert{chemical
	equilibrium}!
\end{frame}

\begin{frame}{What happens when equilibrium is perturbed?}
	\begin{itemize}
		\item Recall $Q$:
			\begin{align*}
				Q = \frac{[\text{products}]}{[\text{reactants}]}
			\end{align*}
		\item In terms of free energy,
			\begin{align*}
				\Delta G &= -RT \ln K + RT \ln Q \\
				\Delta G &= \state{G} + RT \ln Q
			\end{align*}
		\item To translate in terms of electric potential, divide by
			$-nF$,
			\begin{align*}
				\intertext{Dividing by $-nF$,}
				\frac{\Delta G}{-nF} &= \Delta E = \state{E} -
				\frac{RT}{nF} \ln Q
			\end{align*}
	\end{itemize}
\end{frame}

\vspace{\stretch{-1}}

\begin{frame}{The Nernst Equation}
	\begin{align*}
		\Delta E &= \state{E} - \frac{RT}{nF} \ln
		\frac{\prod_i{\mathcal{A}_i}}{\prod_j{\mathcal{A}_j}}
		\intertext{The Nernst equation describes the concentration
		dependence of the electromotive force (EMF) of an
		electrochemical cell.}
		\visible<2->{
		\intertext{Rewritten as a $\log$ function:}
		\Delta E &\approx \state{E} -
		\underbrace{2.303\frac{RT}{nF}}_{\mathclap{\frac{\SI{59.16}{\milli\volt}}{n}~\text{at
		\SI{25}{\celsius}}
		}}
		\log \frac{\prod_i{\mathcal{A}_i}}{\prod_j{\mathcal{A}_j}}}
	\visible<3->{\intertext{Orders of magnitude change in the activity ratio
		at \SI{25}{\celsius} translates to a \alert{slope} of
		\SI{-59.16}{\milli\volt}
		per decade per $n\el$.}}
	\end{align*}
\end{frame}

\vspace{\stretch{-1}}

\begin{frame}{Walther Hermann Nernst}{1864--1941}
	\begin{minipage}{0.65\linewidth}
		The Nobel Prize in Chemistry 1920 was awarded to Walther Hermann
		Nernst ``in recognition of his work in
		thermochemistry''.\parnote{\url{www.nobelprize.org/prizes/chemistry/1920/summary/}}
	\end{minipage}
	\hfill
	\begin{minipage}{0.25\linewidth}
		\includegraphics[width=\linewidth]{nernst.jpg}
	\end{minipage}

	\bigskip

       \parnotes
\end{frame}

\begin{frame}{The Simplified Nernst Equation}
	We can make the Nernst equation a bit easier to read through a few
	substitutions:
	\begin{align*}
		\Delta E &= \state{E} -
		\frac{RT}{nF}\ln\frac{\prod_i\mathcal{A}_i}{\prod_j\mathcal{A}_j}
		\intertext{at \SI{25}{\celsius},}
		\Delta E &= \state{E} - \frac{\SI{0.02569}{\volt}}{n} \ln Q
		\shortintertext{or}
		\Delta E &= \state{E} - \frac{\SI{0.05916}{\volt}}{n} \log Q
	\end{align*}
\end{frame}

\frame{\section{Calculations using the Nernst Equation}
	\begin{learningobjectives}
	\item Define standard reduction potential.
	\item Calculate the expected potential difference under any set of
		reaction conditions.
	\end{learningobjectives}
}

\begin{frame}{Standard Reduction Potentials, \state[pre=]{E}}
	\begin{itemize}
		\item Potentials measured when all species are at unit activity.
		\item By definition, measured against a standard hydrogen
			electrode (S.H.E.) connected to the negative terminal of
			a potentiometer. The cell of interest is attached to the
			positive terminal:
			\begin{align*}
				\ch{Pt\sld{}}~|~\ch{H2 "\gas[\SI{1}{\atm}]{}"}~|
				~\ch{H+}~&||~
				\text{cell of interest}
			\end{align*}
		\item Since the cell of interest is a reduction, also by
			definition, the oxidation of \ch{H2} must be included (must
			\alert{always} have one of each oxidation and reduction)
			\begin{align*}
				\ch{H2 <=> 2 H+ + 2 \el{}} \qquad
				\state[pre=]{E} = \SI{0.000}{\volt}
			\end{align*}
	\end{itemize}
\end{frame}

\vspace{\stretch{-1}}

\begin{frame}{Simple Nernst Calculations}
	\begin{enumerate}[<+->]
		\item What is the potential at a copper wire sitting in a
			\SI{0.0100}{\Molar}~\ch{Cu(NO3)2} solution given the
			standard reduction potential of copper(II) ion is
			\SI{0.337}{\volt}?
			
			\vfill

		\item What is the potential at a zinc wire sitting in a
			\SI{0.150}{\Molar}~\ch{Zn(NO3)2} solution given
			\state[subscript-right=\ch{Zn^{2+}}]{E} is
			\SI{-0.763}{\volt}?

			\vfill

		\item Which species is reduced when these two wires are shorted
			together?

			\vspace{3em}
	\end{enumerate}

	\note<.>{
	\begin{enumerate}
		\item \begin{align*}
				\state[pre=,subscript-right=\ch{Cu^{2+}},superscript=]{E}
				&= \state[pre=,subscript-right=\ch{Cu^{2+}}]{E} -
				\frac{RT}{nF} \ln \frac{1}{[\ch{Cu^{2+}}]} \\
				&= \SI{0.337}{\volt} - \frac{RT}{2F} \ln
				\frac{1}{\SI{0.0100}{\Molar}} \\
				&= \fbox{\SI{0.278}{\volt}}
			\end{align*}
		\item \begin{align*}
				\state[pre=,subscript-right=\ch{Zn^{2+}},superscript=]{E}
				&= \state[pre=,subscript-right=\ch{Zn^{2+}}]{E} -
				\frac{RT}{nF} \ln \frac{1}{[\ch{Zn^{2+}}]} \\
				&= \SI{-0.763}{\volt} - \frac{RT}{2F} \ln
				\frac{1}{\SI{0.150}{\Molar}} \\
				&= \fbox{\SI{-0.787}{\volt}}
			\end{align*}
		\item \ch{Cu^{2+}}

	\end{enumerate}}
\end{frame}

\clearpage

\begin{frame}[t]{Exercise 14-G, page 336}
	Calculate the voltage of the following cell, in which \ch{KHP} is
	potassium hydrogen phthalate, the monopotassium salt of phthalic acid.
	\begin{center}
		\footnotesize
		\begin{tabular} {@{}*{3}{>{\collectcell\ch}c<{\endcollectcell}|}
			*{3}{|>{\collectcell\ch}c<{\endcollectcell}}}
	
			Hg\lqd{} & Hg2Cl2\sld{} & KCl\, "(\SI{0.1}{\Molar})" &
			KHP\, "(\SI{0.050}{\Molar})" & H2 "\gas[\SI{1.00}{atm}]{}" & Pt\sld{}
		\end{tabular}
	\end{center}

	\note{What reactions do we have to consider?
		\begin{reactions*}
			Hg2Cl2\sld{} + 2 \el{} &<=> 2 Hg\lqd{} + 2 Cl^-\aq{} &&
			"$\state[pre=]{E}=\SI{0.268}{\volt}$" \\
			2 H^+\aq{} + 2 \el{} & <=> H2\gas{} &&
			"$\state[pre=]{E}=\SI{0.00}{\volt}$"
		\end{reactions*}
		Let's consider mercury first:
		\begin{align*}
			E_{\ch{Hg/Hg2Cl2}} &= \state{E} -
			\frac{RT}{nF}\ln\ch{[Cl^-]}^2 \\
					       &= 0.268 -
					       \frac{0.05916}{2}\ln(0.1)^2 \\
					       &= \SI{0.32716}{\volt}
		\end{align*}

		For the hydrogen, we are not given \ch{[H^+]}. How do we find it?
	}
\end{frame}

\note{%
	Weak acid-base equilibria! We are given the intermediate form:
	\begin{reactions*}
		HP^-\aq{} + H2O\lqd{} &<=> P^{2-}\aq{} + H3O^+\aq{} &&
		"$\pKa[1]=2.950$" \\
		HP^-\aq{} + H2O\lqd{} &<=> H2P\aq{} + OH^-\aq{} &&
		"$\pKa[2]=5.408$"
	\end{reactions*}
	Do we know any shortcuts?
	\begin{align*}
		\pH{} = 0.5\left(\pKa[1] + \pKa[2]\right) =
		0.5(2.950+5.408)&=4.179 \\
		\ch{[H^+]}=10^{-\pH{}} = 10^{-4.179} &= \SI{6.62217e-5}{\Molar}
	\end{align*}
	\begin{align*}
		\text{Thus:}\qquad E_{\ch{H^+/H2}} &= \state[pre=]{E} - \frac{RT}{2F} \ln
		\frac{P_{\ch{H2}}}{\ch{[H^+]}^2} \\
							  &= 0 -
							  \frac{0.05916}{2}\ln\frac{1}{(\num{6.62217e-5})^2}
							  \\
							  &=
							  \SI{-0.24721}{\volt}
	\end{align*}
}

\note{%
	Finally, solve for $\Delta E$:
	\begin{align*}
		\Delta E &= E_+ - E_- \\
			 &= E_{\ch{H^+/H2}} - E_{\ch{Hg/Hg2Cl2}} \\
			 &= \SI{-0.24721}{\volt} - \SI{0.32716}{\volt} \\
			 &= \boxed{\SI{-0.574}{\volt}}
	\end{align*}
}

%\mode<presentation>{\setbeamercolor{background canvas}{bg=}\includepdf[pages=-]{dogs.pdf}}
%
%\clearpage

\begin{frame}[t]{Different Descriptions of the Same Reaction}
	A silver/silver iodide wire is used to probe iodide concentration in a
	\SI{0.153}{\milli\Molar} \ch{CdI2} solution.  Calculate the cell's half
	potential  using silver iodide reduction and the silver ion reduction
	equations.  How do the values compare?

	\note{\tiny
		\begin{reactions*}
			AgI\sld{} + \el{} &<=> Ag\sld{} + I^-\aq{} &&
			"$\state[pre=]{E}=\SI{-0.152}{\volt}$"
		\end{reactions*}
		\begin{align*}
			E &= \state[pre=]{E} - \frac{RT}{F} \ln \ch{[I^-]} \\
			&= -0.152 - 0.05916 \log 0.000306 \\
			&= \boxed{\SI{0.0559}{\volt}}
		\end{align*}

		\hrule

		\begin{reactions*}
			Ag^+\aq{} + \el{} &<=> Ag\sld{} &&
			"$\state[pre=]{E}=\SI{0.7993}{\volt}$" \\
			AgI\sld{} &<=> Ag^+\aq{} + I^-\aq{} &&
			"$\Ksp{}=\num{8.3e-17}$"
		\end{reactions*}
		\begin{align*}
			E &= \state[pre=]{E} - \frac{RT}{F} \ln
			\frac{1}{\ch{[Ag^+]}} \\
			  &= 0.7993 - 0.05916 \log \frac{1}{\num{2.71242e-13}} \\
			&= \boxed{\SI{0.0559}{\volt}}
		\end{align*}
		}
\end{frame}

\clearpage

\begin{frame}[t]{Exercise 13-I}
	The formation constant for \ch{Cu(EDTA)^{2-}} is \num{6.3e18}, and
	$\state[pre=]{E} = \SI[retain-explicit-plus]{+0.339}{\volt}$
	for the reduction of copper(II) ion.  From this information, find
	\state{E} for the reaction:
	\begin{reaction*}
		CuY^{2-}  +  2 \el{}  <=>  Cu\sld{}  +  Y^{4-}
	\end{reaction*}

	\note{
		\begin{align*}
			\Delta E &= 0 = \state{E} - \frac{0.05916}{n}\log K
			\shortintertext{and}
			K &= 10^{\frac{n\state{E}}{0.05916}}
		\end{align*}
	}
\end{frame}

\begin{frame}{Nernst Outside of (Fundamental) Electrochemistry}
	\begin{minipage}{0.6\textwidth}
		\raggedright{}
		Between nerves, the potential difference across membranes of about
		\SI{70}{\milli\volt} is the result of different concentrations of
		potassium ions (\ch{K+}).
	\end{minipage}
	\hfill
	\begin{minipage}{0.3\textwidth}
		\includegraphics[width=1.0\linewidth]{nerve.png}
	\end{minipage}

	\bigskip
	
	\begin{footnotesize}
		Blausen.com staff (2014). "Medical gallery of Blausen
		Medical 2014". WikiJournal of Medicine 1 (2).
		DOI: \href{dx.doi.org/10.15347/wjm/2014.010}{10.15347/wjm/2014.010}. ISSN 2002-4436.
	\end{footnotesize}
\end{frame}

\vspace{\stretch{-1}}

\frame{\section{Other Important Electrochemical Terms}
	\begin{learningobjectives}
	\item Define Ohm's law.
	\item Use units of power to determine reaction products.
	\item Explain the difference between galvanic cells and electrolytic
		cells.
	\end{learningobjectives}
}

\vspace{\stretch{-1}}

\clearpage
			
\begin{frame}{Ohm's Law}
	\begin{itemize}
		\item Current can be restricted when the system cannot
			physically supply sufficient electron transfer.
		\item The circuit \alert{resists} the flow of electrons.
		\item The current ($I$) flowing through a circuit is directly
			proportional to the \alert{potential difference} across
			the circuit and inversely proportional to the resistance
			($R$) of the circuit.
			\begin{align*}
				I = \frac{E}{R}
			\end{align*}
		\item By definition, a current of one ampere flows through a
			circuit with a potential difference of one volt if the
			resistance is one ohm (\si{\ohm}).
	\end{itemize}
\end{frame}

\vspace{\stretch{-1}}

\begin{frame}{Power}
	\begin{itemize}
		\item The work done per unit time (\si{\joule\per\second} AKA
			the Watt (\si{\watt})).
			\begin{align*}
				P &= \frac{\text{work}}{\si{\second}} = \frac{E
				\cdot q}{\si{\second}} = E \cdot
				\frac{q}{\si{\second}} = E \cdot I \\
				\text{or,~} P &= IE
			\end{align*}
		\item A cell capable of delivering one ampere at a potential
			difference of one volt has a power output of one Watt.
	\end{itemize}
\end{frame}

\vspace{\stretch{-1}}

\begin{frame}[t]{Example: Exercise 14-A}
	A mercury cell used to power heart pacemakers runs on the following
	reaction:
	\begin{align*}
		\ch{Zn\sld{} + HgO\sld{} <=> ZnO\sld{} + Hg\lqd{}} \qquad
		\state{E} = \SI{1.35}{\volt}
	\end{align*}
	If the power required to operate the pacemaker is \SI{0.0100}{\watt},
	how many kilograms of \ch{HgO} (FW = \SI{216.59}{\gram\per\mole}) will
	be consumed in 365 days? How many pounds \ch{HgO?}
\end{frame}

\note{
	\begin{align*}
		P &= IE \\
		\SI{0.0100}{\watt} &= I \times \SI{1.35}{volt} \\
		I &= \SI{0.007407}{\ampere}
	\end{align*}

	\begin{align*}
		\frac{\SI{0.007407}{\coulomb}}{\si{\second}} &\times
		\SI{365}{\day} \times \frac{\SI{24}{\hour}}{\si{\day}} \times
		\frac{\SI{3600}{\second}}{\si{hour}} \\
		&\times
		\frac{\SI{1}{\mole}~\el}{\SI{96485}{\coulomb}} \times
		\frac{\SI{1}{\mole}~\ch{Hg}}{\SI{2}{\mole}~\el} \times
		\frac{\SI{216.6}{\gram}}{\si{\mole}}
		&= \SI{262}{\gram}~\ch{HgO} \\
		&&= \SI{0.262}{\kilo\gram}~\ch{HgO} \\
		&&= \SI{0.578}{lb}~\ch{HgO} \\
	\end{align*}}

\clearpage

\begin{frame}[allowframebreaks]{Galvanic cells, electrolytic cells, and salt
	bridges}
	\begin{itemize}
		\item Galvanic Cell
			\begin{itemize}
				\item Uses a spontaneous chemical reaction to
					generate electricity.
				\item \alert{One} species must be oxidized and
					\alert{another} reduced.
				\item The two cannot be in contact or electrons
					would flow directly between the species
					(a ``short'').
				\item The oxidizing and reducing species must be
					separated spacially.
			\end{itemize}
		\item Electrolytic Cell
			\begin{itemize}
				\item Can not do work as written.
				\item Require work to be done on them
					(non-spontaneous cells with
					$\state[superscript-right=]{G} > 0$).
			\end{itemize}

			\framebreak

		\item Salt Bridge
			\begin{itemize}
				\item A means of connecting two cells spacially
					using a media saturated with an inert
					salt.
				\item During operation, positive ions migrate
					into the \alert{cathode} cell (where
					reduction occurs) and negative ions
					migrate to the \alert{anode} (oxidation
					side).
				\item Ion migration offsets charge build-up in
					each cell.
			\end{itemize}
			\begin{center}
				\includegraphics[width=0.7\linewidth]{salt-bridge.png}
			\end{center}
	\end{itemize}
\end{frame}

\frame{\section{Fundamentals of Potentiometry}
	\begin{learningobjectives}
	\item Explain how the Nernst equation can help determine concentrations
		of analyte.
	\item Convert potentials between different reference electrodes.
	\item Differentiate between different types of electrodes used in
		potentiometric measurements.
	\end{learningobjectives}
}

\begin{frame}{Potentiometry}
	\begin{itemize}[<+->]
		\item An analytical method in which an electric potential
			difference (a voltage) of a cell is measured.
			\begin{itemize}
				\item A measure of the potential to do work.
				\item A measure of the \alert{chemical
					equilibria}.

					\begin{equation*}
						\state[superscript-right=]{E} =
						\state{E} - \frac{RT}{nF} \ln Q
					\end{equation*}
			\end{itemize}
		\item A potential \alert{difference} means we are measuring the
			system against a \alert{reference} state.
			\begin{itemize}
				\item How do we compare the new vs reference
					systems?
			\end{itemize}
	\end{itemize}
\end{frame}

\begin{frame}{Electrode Basics}
	\begin{itemize}
		\item At a \alert{minimum}, an electrochemical cell must have
			\alert{2} electrodes.
			\begin{itemize}
				\item Indicator or Working electrode
					\begin{itemize}
						\item Where the reaction of
							interest occurs
						\item What we are measuring
					\end{itemize}
				\item Reference electrode
					\begin{itemize}
						\item A known reaction occurs
							here
						\item We measure the reaction at
							the working electrode
							\alert{against} this
							reaction
					\end{itemize}
			\end{itemize}
	\end{itemize}
\end{frame}

\begin{frame}[allowframebreaks]{Reference Electrodes}
	\begin{enumerate}
		\item Normal Hydrogen Electrode (NHE)
			
			Standard Hydrogen Electrode (SHE)
				
			Hydrogen Gas Electrode


			\begin{itemize}
				\item The thermodynamic standard:
					\begin{equation*}
						\ch{2 H+ \aq{} + 2 \el{} <=>
						H2\gas{}} \qquad \state[pre=]E =
						\SI{0.00}{\volt}
					\end{equation*}
				\item \ch{H2\gas{}} is bubbled over a \ch{Pt}
					foil
				\item It's not used very often. Why?
			\end{itemize}

			\framebreak
			
		\item Calomel Electrode
			\begin{itemize}
				\item A very common and very stable reference
					electrode:
					\begin{equation*}
						\ch{Hg2Cl2\sld{} + 2 \el <=> 2
						Hg\lqd{} + 2 Cl- \aq{}} \qquad
						\state[pre=]{E} =
						\SI{0.2676}{\volt}
					\end{equation*}
				\item The potential depends only on the activity
					of the chloride ion:
					\begin{equation*}
						\state[pre=,superscript-right=]{E}
						= \state[pre=]E - \frac{RT}{2F}
						\ln [\ch{Cl-}]^2
					\end{equation*}
				\item A constant activity (thus constant
					potential) is maintained by immersing in
					saturated \ch{KCl} ($\approx
					\SI{4.2}{\formal}$)
					\begin{itemize}
						\item Also known as the
							\alert{Saturated Calomel
							Electrode (SCE)} with a
							potential of
							\SI{0.241}{\volt} vs.
							SHE at
							\SI{25}{\celsius}
					\end{itemize}
				\item Any problems with this electrode?
			\end{itemize}

			\framebreak

		\item Silver-Silver Chloride Electrode (Ag/AgCl)

			\begin{itemize}
				\item Another common and very stable reference
					electrode, but very simple to build and
					maintain:
					\begin{equation*}
						\ch{AgCl\sld{} + \el{} <=>
						Ag\sld{} + Cl- \aq{}} \qquad
						\state[pre=]{E} =
						\SI{0.222}{\volt}
					\end{equation*}
				\item Activity (and potential) also depends
					solely on chloride:
					\begin{equation*}
						\state[pre=,superscript-right=]{E}
						= \state[pre=]{E} -
						\frac{RT}{F} \ln [\ch{Cl-}]
					\end{equation*}
				\item Using saturated \ch{KCl} to maintain a
					constant potential, the electrode has a
					potential of \SI{0.197}{\volt} vs. SHE.
				\item Due to simplicity of fabrication, they can
					be used for studies on the micro scale.
			\end{itemize}
	\end{enumerate}

	\framebreak

	How do we keep the concentration constant for one electrode but not the
	other?

	\begin{center}
		\includegraphics[scale=0.5]{agagcl-saltbridge.png}
	\end{center}

	The \alert{salt bridge} allows for electrons to flow, but keeps the
	solutions separate.

	\framebreak

	We can shrink \alert{one half} of the cell into a convenient package:

	\begin{center}
		\includegraphics[scale=1]{CHI111.png}

		\footnotesize chinstruments.com
	\end{center}

	The frit acts as a salt bridge by preventing solution flow.

	\framebreak

	We can convert between the different reference scales:

	\begin{center}
		\includegraphics[scale=1.8]{potential-scale.png}
	\end{center}
\end{frame}

\begin{frame}[t]{Reference Conversion}
	The oxidation/reduction of ferrocene (\state[pre=]{E} =
	\SI{0.400}{\volt} vs SHE)\footnotemark is often used as an internal standard where
	common reference electrodes are not able to be used (organic media, for
	example). If the oxidation/reduction potential for a ruthenium compound
	is found to be at \SI{-1.157}{\volt} vs.
	\state[pre=,superscript-right=,subscript-right={\ch{Fc+}/\ch{Fc}}]{E}, what
	is this value vs. SHE? vs. Ag/AgCl?

	\vfill

	\footnotetext{Gagne, R. R.; Koval, C. A.; Lisensky, G. C. 
	\textit{Inorg. Chem.}
	\textbf{1980}, 19 (9), 2854–2855.}

\note{
	\begin{itemize}
		\item \SI{0.400}{\volt} vs. SHE, thus all potentials are
			+\SI{0.400}{\volt} for SHE. $E$ = \SI{-0.757}{\volt} vs.
			SHE.

		\item Ag/AgCl is \SI{+0.197}{\volt} vs. SHE, thus potentials are
			-\SI{0.197}{\volt} for Ag/AgCl. $E$ = \SI{-0.954}{\volt}
			vs. Ag/AgCl.
	\end{itemize}
	}
\end{frame}

\clearpage

\begin{frame}{Indicator/Working Electrodes}
	Where the reaction of interest occurs.

	\begin{enumerate}
		\item Electrodes of the First Kind
		\item Electrodes of the Second Kind
		\item Electrodes of the Third Kind
		\item Metal Redox Electrodes
	\end{enumerate}

	There are several different styles:
	\begin{itemize}
		\item Metallic
		\item Membrane or Ion-selective
			\begin{itemize}
				\item Glass
				\item Solid state
				\item Liquid Ion Exchange
				\item Gas Sensing Molecular
				\item Biocatalytic
			\end{itemize}
	\end{itemize}
\end{frame}

\vspace{\stretch{-1}}

\begin{frame}{Electrodes of the First Kind}
		A bare metal wire or strip is placed into
			the solution and is used to measure the
			concentration of the same species metal
			ion.
			\begin{align*}
				\ch{M^{$n$+} + $n$\el{} &<=>
				M^0} \\
				\state[pre=,superscript-right=]{E}
				&= \state[pre=]{E} -
				\frac{0.05916}{n} \log
				\frac{1}{[\ch{M^{$n$+}}]}
			\end{align*}
		The simplest of all electrode systems.
\end{frame}

\vspace{\stretch{-1}}

\begin{frame}[t]{Electrodes of the First Kind Example}
	A cell was prepared by dipping a Ag wire and a saturated calomel
	electrode into a \SI{0.10}{\Molar} \ch{AgNO3} solution. The Ag wire was
	attached to the positive terminal of a potentiometer and the SCE was
	attached to the negative (reference) terminal.
	\begin{enumerate}
		\item Write a half-reaction for the Ag electrode.
		\item Write the Nernst equation for the Ag electrode.
		\item Calculate the cell voltage.
			\visible<2->{\textbf{Activity?}}
	\end{enumerate}

\note{\footnotesize
	\ch{Ag+ + 1 \el{} <=> Ag\sld{}} \qquad $E^0 = \SI{0.7993}{\volt}$

	\begin{align*}
		E &= E^{0} - \frac{RT}{nF} \ln
		\frac{1}{[\ch{Ag+}]\gamma_{\ch{Ag+}}} \\
		&= 0.7993 - 0.05916 \log (0.10 \times 0.75) \\
		&= \SI{0.8659}{\volt}~\text{vs. SHE} \\
		&= \SI{0.625}{\volt}~\text{vs. SCE} \\
		\intertext{Without activity:}
		E &= E^{0} - \frac{RT}{nF} \ln
		\frac{1}{[\ch{Ag+}]} \\
		&= 0.7993 - 0.05916 \log (0.10) \\
		&= \SI{0.8585}{\volt}~\text{vs. SHE} \\
		&= \SI{0.617}{\volt}~\text{vs. SCE}
	\end{align*}
	}
\end{frame}

\clearpage

\begin{frame}{Electrodes of the Second Kind}
	\only<+>{%
		The first kind bare metal electrode is now
			coated with an insoluble salt containing
			its ion and some anion. Variations in
			the amount of anion causes variations in
			potential:
			\begin{align*}
				\ch{M_{$n$}X_{$y$} +
				$\frac{y}{n}$ \el{}
				&<=>
				M^0 + $y$ X^{$\sfrac{n}{y}-$}} \\
				\state[pre=,superscript-right=]{E}
				&= \state[pre=]{E} -
				\frac{0.05916}{n} \log
				[\ch{X^{$\sfrac{n}{y}-$}}]^y
			\end{align*}
		Since there are lots of insoluble salts,
			these can expand the number of metallic
			electrodes nicely.
			\begin{align*}
				\ch{Ag+ + \el{} &<=> Ag^0}
				\qquad &&\state[pre=]{E} =
				\SI{0.7993}{\volt} \\
				\ch{AgCl + \el{} &<=> Ag^0 +
				Cl-} \qquad &&\state[pre=]{E} =
				\SI{0.222}{\volt}
			\end{align*}
		}

		\only<+>{%
		An interesting example is to make an EDTA
			electrode using mercury(II) and mercury
			metal:
			\begin{align*}
				\ch{HgY^{2-} + 2 \el{} &<=>
				Hg\lqd{} + Y^{4-}} \qquad
				\state[pre=]{E} =
				\SI{0.21}{\volt} \\
				E &= 0.21 - \frac{0.05916}{2}
				\log
				\frac{\mathcal{A}_{\ch{Y^{4-}}}}{\mathcal{A}_{\ch{HgY^{2-}}}}
				\intertext{To start the system,
				a miniscule amount of
				\ch{HgY^{2-}} is introduced.
				Since $K_{\text{f}}$ is
				\SI{6.3e21}, the amount added is
				stable and
				$\mathcal{A}_{\ch{HgY^{2-}}}$ is
				thus fixed making the response:}
				E &= 0.21 - \frac{0.05916}{2}
				\log
				\mathcal{A}_{\ch{Y^{4-}}} = K +
				0.02958 \p{\ch{Y^{4-}}}
			\end{align*}
			This electrode is useful to establishing
			the endpoint of EDTA titrations.
		}
\end{frame}

\vspace{\stretch{-1}}

\begin{frame}[t]{Electrodes of the Second Kind Example}
	A cell was prepared by dipping a Ag wire, pre-coated with AgCl and a
	saturated calomel electrode into a \SI{0.10}{\Molar} \ch{KCl} solution.
	The AgCl wire was attached to the positive terminal of a potentiometer
	and the SCE was attached to the negative (reference) terminal.
	\begin{enumerate}
		\item Write a half-reaction for the AgCl electrode.
		\item Write the Nernst equation for the AgCl electrode.
		\item Calculate the cell voltage.
			\visible<2->{\textbf{Activity?}}
	\end{enumerate}

	\note{\footnotesize
	\ch{AgCl + 1 \el{} <=> Ag\sld{} + Cl-} \qquad $E^0 = \SI{0.222}{\volt}$

	\begin{align*}
		E &= E^{0} - \frac{RT}{nF} \ln
		[\ch{Cl-}]\gamma_{\ch{Cl-}} \\
		&= 0.222 - 0.05916 \log (0.10 \times 0.755) \\
		&= \SI{0.288}{\volt}~\text{vs. SHE} \\
		&= \SI{0.047}{\volt}~\text{vs. SCE} \\
		\intertext{Without activity:}
		E &= E^{0} - \frac{RT}{nF} \ln
		[\ch{Cl-}] \\
		&= 0.222 - 0.05916 \log (0.10) \\
		&= \SI{0.281}{\volt}~\text{vs. SHE} \\
		&= \SI{0.040}{\volt}~\text{vs. SCE}
	\end{align*}
	}
\end{frame}

\clearpage
	
\begin{frame}{Electrodes of the Third Kind}
		If we can make the metallic electrode
			respond to another cation, it will be an
			electrode of the third kind. Let's
			consider the mercury-EDTA electrode.
			Since
			\begin{align*}
				\ch{Ca^{2+} + Y^{4-} &<=>
				CaY^{2-}} \qquad K_{\text{f}} =
				\frac{\mathcal{A}_{\ch{CaY^{2-}}}}
				{\mathcal{A}_{\ch{Ca^{2+}}}
				\mathcal{A}_{\ch{Y^{4-}}}} \\
				E &= K - \frac{0.05916}{2} \log
				\mathcal{A}_{\ch{Y^{4-}}}
				= K - \frac{0.05916}{2} \log
				\frac{\mathcal{A}_{\ch{CaY^{2-}}}}
				{K_{\text{f}}\mathcal{A}_{\ch{Ca^{2+}}}
				} \\
				\intertext{rearranging yields:}
				E &= K - \frac{0.05916}{2} \log \frac{\mathcal{A}_{\ch{CaY^{2-}}}}
				{K_{\text{f}}} -
				\frac{0.05916}{2} \log
				\frac{1}{\mathcal{A}_{\ch{Ca^{2+}}}}
				\\
				&= K' - 0.02958\p{\ch{Ca}}
			\end{align*}
			Mercury has now become an electrode of
			the third kind --- a sensor for a
			different metal.
\end{frame}

\vspace{\stretch{-1}}

\begin{frame}[t]{Electrodes of the Third Kind Example}
	A cell was prepared by dipping a AgCl wire and a
	saturated calomel electrode into a \SI{0.10}{\Molar} \ch{Pb(NO3)2}
	solution.  The AgCl wire was attached to the positive terminal of a
	potentiometer and the SCE was attached to the negative (reference)
	terminal.
	\begin{enumerate}
		\item Write a half-reaction for the AgCl electrode.
		\item Write the solubility reaction for \ch{PbCl2}.
		\item Write the Nernst equation for the equilibria present.
		\item Calculate the cell voltage.
	\end{enumerate}

\note{
	\begin{align*}
		\ch{AgCl + 1 \el{} &<=> Ag\sld{} + Cl-} \qquad &&E^0 =
		\SI{0.222}{\volt} \\
		\ch{PbCl2 &<=> Pb^{2+} + 2 Cl-} \qquad &&K_\text{sp} =
		\num{1.7e-5}
	\end{align*}

	\begin{align*}
		E &= E^{0} - \frac{RT}{nF} \ln [\ch{Cl-}] \gamma_{\ch{Cl-}} \\
		K_{\text{sp}} &= [\ch{Pb^{2+}}] \gamma_{\ch{Pb^{2+}}}
		[\ch{Cl-}]^2 \gamma_{\ch{Cl-}}^2 \\
		[\ch{Cl-}] \gamma_{\ch{Cl-}} &=
		\sqrt{\frac{K_\text{sp}}{[\ch{Pb^{2+}}] \gamma_{\ch{Pb^{2+}}}}}
	\end{align*}
	}
\end{frame}

\note{
	\begin{align*}
		E &= E^{0} - \frac{RT}{nF} \ln
		\sqrt{\frac{K_\text{sp}}{[\ch{Pb^{2+}}] \gamma_{\ch{Pb^{2+}}}}}
		\\
		E &= 0.222 - 0.05916 \log \sqrt{\frac{\num{1.7e-5}}{0.1 \times
		0.37}} \\
		&= \SI{0.419}{\volt}~\text{vs. SHE} \\
		&= \SI{0.178}{\volt}~\text{vs. SCE}
	\end{align*}
	}

\clearpage

\begin{frame}{Metal Redox Electrodes}
	\begin{itemize}
		\item The inert metal electrodes used to study
			soluble species. Primary are platinum,
			gold, and palladium (inert metals).
			Carbon is also included, although it is
			non-metallic.

		\item An example is the use of platinum in the
			hydrogen electrode or in a solution of
			cerium(III) and cerium(IV):
			\begin{align*}
				\ch{Ce^{4+} + \el{} &<=>
				Ce^{3+}} \\
				E &= \state[pre=]{E} - 0.05916
				\log
				\frac{\mathcal{A}_{\ch{Ce^3+}}}
				{\mathcal{A}_{\ch{Ce^{4+}}}}
			\end{align*}
			In this fashion, platinum can serve as
			the electrode for a titration using
			cerium(IV) as the titrant.
		\item Why aren't these called `Electrodes of the
			Fourth Kind' or `More Electrodes of the
			Third Kind'?
	\end{itemize}
\end{frame}

\vspace{\stretch{-1}}

\begin{frame}{What type of electrode is each reference?}
	\begin{enumerate}
		\item Normal Hydrogen Electrode

			\bigskip

			\note<item>{Metal Redox}

		\item Saturated Calomel Electrode

			\bigskip

			\note<item>{Second Kind}

		\item Silver-Silver Chloride Electrode

			\bigskip

			\note<item>{Second Kind}
	\end{enumerate}

	\bigskip
	\pause

	\begin{block}{Note}
		Reference electrodes are just indicator
		electrodes with \alert{known} potentials!
	\end{block}
\end{frame}

\vspace{\stretch{-1}}

\frame{\section{Ion Selective Electrodes}
	\begin{learningobjectives}
	\item Explain the origination of the junction potential.
	\item Explain how junction potentials can be both beneficial or
		detrimental to a measurement.
	\item Calculate the selectivity of ion selective electrodes.
	\end{learningobjectives}
}

\vspace{\stretch{-1}}

\begin{frame}{What is $\Delta E$ between two Ag/AgCl electrodes?}
	\begin{itemize}
		\item<1-> We \emph{should} get a \state[superscript-right=]{E}
			= \SI{0}{\volt}.
		\item<2-> In reality, we may see a small potential difference.
	
			\visible<3->{
			\begin{block}{Junction Potential}
				An electric potential that exists at the
				junction (interface) between two different
				electrolyte solutions or substances. It arises
				in solutions as a result of unequal rates of
				diffusion of different ions.
				\begin{equation*}
					E_\text{observed} = E_\text{cell} +
					E_\text{junction}
				\end{equation*}
			\end{block}
			}
		\item<4-> We can \alert{never} be certain about the value of
			$E_\text{cell}$ because it is \alert{impossible} to
			measure $E_\text{junction}$.
			\begin{itemize}
				\item How do we know the junction potential
					across the porous tip of a Ag/AgCl
					electrode?
			\end{itemize}
	\end{itemize}
\end{frame}

\begin{frame}{Junction Potentials across Porous Frits}
	\begin{columns}
		\column{0.5\linewidth}
		\begin{itemize}
			\item Glass presents a negative surface charge due to
				deprotonated silanol groups.
			\item When the pores are very small,\footnotemark{}
				cations will more readily move through the
				pores.
			\item This introduces a separation of charge and thus, a
				junction potential!
		\end{itemize}
		\column{0.4\linewidth}
		\mode<presentation>{
			\begin{center}
				\includegraphics[scale=1]{porous-glass.png}
			\end{center}
		}
		\mode<article>{
		\tikz[overlay]{\node[right,xshift=4.5in,yshift=1in]{\includegraphics[scale=1]{porous-glass.png}};}}
	\end{columns}
	
	\footnotetext{As is often the case in glass frits to prevent solution
	flow!}
\end{frame}

\begin{frame}[t]{Ion Selective Electrodes}
	\only<+>{%
	\begin{itemize}
		\item Junction potentials can often be undesired due to
			erroneous $E_\text{cell}$ measurements.
		\item They can be beneficial, however, if the mobilities of
			\alert{specific} ions can be \alert{preferentially}
			measured.
		\item \textbf{Ion Selective Electrodes (ISE)} have a potential
			that is dependent \alert{only} on one particular ion in
			solution.
			\begin{align*}
				E &= \text{constant} + \frac{RT}{nF} \ln
				\mathcal{A}_\text{o} \\
				&= \text{constant} + \frac{0.05916}{n} \log
				\mathcal{A}_\text{o} \quad \text{(at
				\SI{25}{\celsius})}
			\end{align*}
			An order of magnitude change in concentration alters the
			measured potential by a factor of $\frac{0.05916}{n}$
	\end{itemize}
}

\only<+>{%
	\begin{center}
		\includegraphics[scale=0.5]{ise.png}
	\end{center}
	\begin{itemize}
		\item Ideally, a ligand that can \alert{only} coordinate with
			the analyte of interest is soluble in the membrane.
		\item Realistically, other ions \alert{may} coordinate, causing
			some error.
		\item What ion selective electrode have we used before?

	\end{itemize}
}
\end{frame}

\begin{frame}[allowframebreaks]{Glass Electrodes}
	\begin{columns}
		\column{0.5\linewidth}
	\begin{itemize}
		\item A very thin membrane of glass separates an
			inner ``reference'' solution from the
			outer test solution.
		\item The inner solution contains the ion of interest at some
			\alert{known} activity.
		\item The side with a higher concentration
			forces ions into the membrane, which
			produces a change in charge ---
			measurable as a potential difference.
	\end{itemize}
		\column{0.4\linewidth}
		\mode<presentation>{
			\begin{center}
				\includegraphics[scale=0.45]{glass-electrode.png}
			\end{center}
		}
	\end{columns}
	\mode<article>{\tikz[overlay]{\node[left,xshift=-0.5in,yshift=1in]{\includegraphics[scale=0.45]{glass-electrode.png}};}}
\end{frame}

\begin{frame}{Glass Electrodes for Measuring pH}
	\begin{itemize}
		\item \ch{H+} replaces \ch{Na+} in the (doped) glass via
			\alert{ion exchange equilibrium}.
		\item \ch{H+} does not actually move across the membrane, rather
			\ch{Na+} acts as a salt bridge through the glass.
		\item Because [\ch{H+}] is constant on the reference side, any
			change in [\ch{H+}] in the outer solution will cause a
			change in measured $E$.

			\bigskip

		\item What may affect equilibrium?
			\begin{itemize}
				\item<2-> Temperature
				\item<2-> Ionic strength
			\end{itemize}
	\end{itemize}
\end{frame}

\begin{frame}{Quirks of a Glass pH Electrode}
	\begin{itemize}
		\item The pH cannot be more accurate than the standards.
			\begin{itemize}
				\item 4.01, 7.00, 10.01
			\end{itemize}
		\item A junction potential exists even when the pH of the inner
			solution is the same as the outer --- ionic strength
			effects. (Again, uncertainty of at least \num{0.01}.)
		\item Despite very low current, the electrode is still trying to
			reach equilibria -- electrode drift!
		\item The mobile ion is \ch{Na+}, thus at high [\ch{Na+}] and
			low [\ch{H+}], we may record a lower pH than the true
			value.
		\item In very low pH solutions, the pH may be artificially high
			because the glass membrane is \alert{saturated}.
		\item Care must be taken to avoid drying out the electrode.
	\end{itemize}
\end{frame}

\begin{frame}{Other ISEs}
	\begin{itemize}
		\item Glass membranes
			\begin{itemize}
				\item For \ch{H+} and other monovalent cations.
			\end{itemize}
		\item Solid state electrodes
			\begin{itemize}
				\item Based on inorganic crystals or conductive
					polymers.
			\end{itemize}
		\item Liquid-based electrodes
			\begin{itemize}
				\item Using a hydrophobic polymer membrane
					saturated with a hydrophobic liquid ion
					exchanger.
			\end{itemize}
		\item Compound electrodes
			\begin{itemize}
				\item With an analyte-selective electrode
					enclosed by a membrane that separates
					analyte from other species or that
					generates analyte in a chemical
					reaction.
			\end{itemize}
	\end{itemize}

	All ISEs work based on the concentration of \alert{free} analyte in
	solution. None are perfect, thus electrodes are often discussed in terms
	of their \alert{selectivity coefficient}.

	\end{frame}

\begin{frame}{ISE Selectivity}
	\begin{block}{Selectivity Coefficient}
		\begin{equation*}
			k_{\ch{A,X}}^{\text{pot}} = \frac{\text{response to
			\ch{X}}}{\text{response to \ch{A}}}
		\end{equation*}
	\end{block}

	\begin{itemize}
		\item An electrode intended to measure ion \ch{A} also responds
			to interfering ion \ch{X}.
			\begin{itemize}
				\item The smaller the value of $k$, the less
					interference by \ch{X}.
			\end{itemize}
		\item For the pH electrode, we wanted to measure [\ch{H+}], but
			we saw that [\ch{Na+}] can also have an effect.
		\item For an interfering ion of the \alert{same charge} with a
			Nernstian response, the ISE will respond according to
			\begin{equation*}
				E = \text{constant} \pm
				\frac{0.05916}{z_{\ch{A}}} \log \bigg[
					\mathcal{A}_{\ch{A}} + \sum_{\ch{X}}
					K_{\ch{A,X}}^{\text{Pot}}
					\mathcal{A}_{\ch{X}} \bigg]
			\end{equation*}
	\end{itemize}
\end{frame}

\clearpage

\begin{frame}[t]{ISE Selectivity Example}
	A fluoride ISE has a selectivity coefficient
	$K_{\ch{F- , OH-}}^{\text{Pot}} = 0.1$. What will be the change in
	electrode potential when \SI{1.0e-4}{\Molar}~\ch{F-} at pH~5.5 is raised
	to pH~10.5?

	\vfill

\note{
	\begin{align*}
		\intertext{At pH = 5.5, [\ch{OH-}] is negligible (we're in
		acid).}
		E &= \text{constant} - 0.05916 \log (\num{1.0e-4}) \\
		&= \text{constant} + \SI{236.6}{\milli\volt}
		\intertext{At pH = 10.50, [\ch{OH-}] = \SI{3.2e-4}{\Molar}, so
		the electrode potential is}
		E &= \text{constant} - 0.05916 \log (\num{1.0e-4} + 0.1 \times
		\num{3.2e-4}) \\
		&= \text{constant} + \SI{229.5}{\milli\volt} \\
		\Delta E &= 229.5 - 236.6 = \SI{-7.1}{\milli\volt}
	\end{align*}
	}
\end{frame}

\begin{frame}[t]{Solid State ISE}
	\begin{itemize}[<+->]
		\item Ion exchange equilibria between the solution and
			the surface of the \alert{solid} crystal account
			for the electrode potential.
			\only<.>{\begin{center}
				\includegraphics[scale=0.7]{solid-ise.png}
			\end{center}}
		\item \alert{Vacancies} in the crystal permit ion diffusion
			through the crystal.
			\only<.>{%
				\begin{center}
					\includegraphics[scale=0.9]{vacancy.png}
				\end{center}}
	\end{itemize}
	\visible<+->{%
		\begin{center}
		\scriptsize
		\sisetup{table-parse-only=true,table-column-width=7ex}
		\begin{tabular} { E
			S[table-number-alignment=right]@{--}S[table-number-alignment=left]
		E r@{--}l l}
			\toprule
			"{\bfseries Ion}" & \multicolumn{2}{c}{\bfseries Concentration (\si{\Molar})} & \multicolumn{1}{c}{\bfseries Membrane} &
				\multicolumn{2}{l}{\bfseries pH range} & {\bfseries Interfering species} \\ \midrule
				F-     & e-6 & 1   & LaF3  &  5 &  8 & \ch{OH-} (\SI{0.1}{\Molar}) \\
				Cl-    & e-4 & 1   & AgCl  &  2 & 11 &\ch{CN-}, \ch{S^{2-}}, \ch{I-}, \ch{S2O3^{2-}}, \ch{Br-} \\
				Br-    & e-5 & 1   & AgBr  &  2 & 12 & \ch{CN-}, \ch{S^{2-}}, \ch{I-} \\
				I-     & e-6 & 1   & AgI   &  3 & 12 & \ch{S^{2-}} \\
				SCN-   & e-5 & 1   & AgSCN &  2 & 12 & \ch{S^{2-}}, \ch{I-}, \ch{CN-}, \ch{Br-}, \ch{S2O3^{2-}} \\
				CN-    & e-6 & e-2 & AgI   & 11 & 13 & \ch{S^{2-}}, \ch{I-} \\
				S^{2-} & e-5 & 1   & Ag2S  & 13 & 14 \\
				\bottomrule
		\end{tabular}
	\end{center}}
\end{frame}

\begin{frame}{Liquid-based ISE}
	\only<+>{%
	\begin{columns}
		\column{0.6\textwidth}
		\begin{itemize}
			\item Similar to a solid-state ISE, but uses a
				\alert{liquid junction}.
			\item A hydrophobic membrane is impregnated with a
				hydrophobic ion exchanger -- an
				\alert{ionophore}.
			\item Ideally, neither the ionophore nor the membrane
				will leak in aqueous media.
				\begin{itemize}
					\item In practice, leakage is the
						primary reason for poor LOD.
					\item Lowering inner filling solution
						concentration may help.
				\end{itemize}
			\item Again, the ionophore must be selective for the ion
				of interest.
		\end{itemize}
		\column{0.3\textwidth}
		\mode<presentation>{\includegraphics[scale=0.7]{liquid-ise.png}}
	\end{columns}
}

\only<+>{%
	\begin{center}
	\begin{tabular} {E S[table-format=2.2] l}
		\toprule
		"{\bfseries Ion~(A)}" & {\bfseries LOD (\si{\micro\Molar})} & {\bfseries Selectivity coefficients ($\bm{\log k_{\ch{A,X}}^{\text{Pot}}}$)} \\
		\midrule
		Na+ 	& 30 	& \ch{H+}, -4.8; \ch{K+}, -2.7;	\ch{Ca^{2+}}, -6.0 \\
		K+ 	& 5 	& \ch{Na+}, -4.2; \ch{Mg^{2+}}, -7.6; \ch{Ca^{2+}}, -6.9 \\
		NH3 	& 20	\\
		Cs+ 	& 8 	& \ch{Na+}, -4.7; \ch{Mg^{2+}}, -8.7; \ch{Ca2+}, -8.5 \\
		Ca^{2+}	& 0.1 	& \ch{H+}, -4.9; \ch{Na+}, -4.8; \ch{Mg^{2+}}, -5.3 \\
		Ag+ 	& 0.03 	& \ch{H+}, -10.2; \ch{Na+}, -10.3; \ch{Ca^{2+}}, -11.3 \\
		Pb^{2+}	& 0.06 	& \ch{H+}, -5.6; \ch{Na+}, -5.6; \ch{Mg^{2+}}, -13.8 \\
		Cd^{2+}	& 0.1 	& \ch{H+}, -6.7; \ch{Na+}, -8.4; \ch{Mg^{2+}}, -13.4 \\
		Cu^{2+}	& 2 	& \ch{H+}, -0.7; \ch{Na+}, $<$-5.7; \ch{Mg^{2+}}, $<$-6.9 \\
		ClO4-	& 20 	& \ch{OH-}, -5.0; \ch{Cl-}, -4.9; \ch{NO3-}, -3.1 \\
		I- 	& 2 	& \ch{OH-}, -1.7 \\
		\bottomrule
	\end{tabular}
	\mode<article>{\tikz[overlay]{\node[left,xshift=-4.75in]{\includegraphics[scale=0.7]{liquid-ise.png}};}}
	\end{center}
}
\end{frame}

\vspace{\stretch{-1}}

\begin{frame}[t]{Leakage of a Liquid ISE}
	Suppose a \ch{Pb^{2+}} ISE leaks at a rate of
	\SI{0.5}{\nano\mole\per\hour}. How long will this electrode be useful
	for measuring [\ch{Pb^{2+}}] to a precision of \SI{0.1}{\micro\Molar}?

	\note{}
\end{frame}

\clearpage

\begin{frame}{Compound Electrodes}
	\begin{columns}
		\column{0.5\textwidth}
		\begin{itemize}
			\item Occasionally, we cannot directly measure the
				concentration of an analyte.
				\begin{itemize}
					\item How can we measure a gas dissolved
						in a liquid?
				\end{itemize}
			\item  A conventional electrode is surrounded by a
				membrane that isolates (or generates) the
				analyte to which the electrode responds.
			\item Any dissolved gas that can affect \pH{} may be an
				issue!
		\end{itemize}
		\column{0.5\textwidth}
		\begin{center}
			\includegraphics[scale=0.65]{compound-ise.png}
		\end{center}
	\end{columns}
%	\tikz[overlay]{\node[right,xshift=4.25in,yshift=1in]{\includegraphics[scale=0.65]{compound-ise.png}}}
\end{frame}

\vspace{\stretch{-1}}

\begin{frame}{Practical Use of ISEs}
	\only<+>{%
	\begin{itemize}
		\item Less expensive than competing techniques such as atomic
			spectroscopy and ion chromatography
		\item Linear response to $\log \mathcal{A}$ over 4-6 orders of
			magnitude
		\item Nondestructive 
			\begin{itemize}
				\item We can re-use the sample for later
					analysis
				\item Better yet, we can measure things in vivo!
			\end{itemize}
		\item Noncontaminating
			\begin{itemize}
				\item Excluding minor leakage from liquid ISEs
			\end{itemize}
		\item Short response time unaffected by color or turbidity
			\begin{itemize}
				\item Can use directly in flowing streams
			\end{itemize}
	\end{itemize}
}

\only<+>{%
	\begin{center}
		\includegraphics[scale=1.2]{in-vivo-ph.jpeg}
	\end{center}

	{\footnotesize
	Hao, J.; Xiao, T.; Wu, F.; Yu, P.; Mao, L. \textit{Anal. Chem.}
	\textbf{2016}, 88 (22), 11238–11243.}
}

\only<+>{%
	\begin{itemize}
		\item Precision is 1\% or worse.
		\item Electrodes are easily poisoned by organics and proteins
			that cling, or \emph{foul}, to membranes leading to
			drifting and sluggish responses.
			\begin{block}{Fouling:}
				The adsorption of electroinactive species on an
				electrode surface (sometimes referred to as
				``getting crap on the electrode'') occurs
				frequently.

				\medskip

				\footnotesize{Bard, A. J.; Faulkner, L.
					R.  \textit{Electrochemical Methods:
					Fundamentals and Applications}; Wiley,
				2000, pg. 569.}
			\end{block}
		\item Many membranes are fragile, others are slightly soluble,
			others purposefully leak reagent, and all have a limited
			useful life.
			\begin{itemize}
				\item Many have a limited shelf life!
			\end{itemize}
		\item Potential ligands must be masked.  Ion strengths must be
			adjusted.  Frequently, pH must be maintained within
			limits to protect the membrane or the chemistry.
	\end{itemize}
}

\only<+>{%
	\documentclass{standalone}

\usepackage{tikz}
\usepackage{pgfplots}
\usepackage{mathtools}
\usepackage{chemmacros}
\chemsetup{modules=all}
\usepackage{siunitx}
\DeclareSIUnit{\Molar}{\textsc{m}}

\setlength{\textwidth}{3in}
\setlength{\textheight}{4in}

\begin{document}

\begin{tikzpicture}
	\begin{axis} [
			width=\linewidth,
			height=18em,
			xmax=0,
			xmin=-4.500,
			ymax=0.250,
			ymin=0.000,
			xlabel={$\log [\ch{X^{$n+$}}]$},
			ylabel={Potential (V)},
			label style={font=\footnotesize},
			tick label style={font=\footnotesize,
				/pgf/number format/.cd,
            			fixed,
        			fixed zerofill,
        			precision=2,
				/tikz/.cd},
			xtick={-4.500,-4.000,...,0.000},
			ytick={0.000,0.050,...,0.300},
			no markers,
    			every axis plot/.append style={ultra thick},
			]
		\addplot table [
			x=log,
			y=na,
			col sep=comma
			]
		{ISE-slopes.csv} node[left] {\ch{Na+}};
		\addplot table [
			x=log,
			y=ca,
			col sep=comma
			]
			{ISE-slopes.csv} node[left] {\ch{Ca^{2+}}};
	\end{axis}
\end{tikzpicture}

\end{document}

}
\end{frame}

%\begin{frame}{Standard Addition using ISEs}
%	\only<+>{%
%	\begin{itemize}
%		\item Because ISEs are highly sensitive to ionic strength, we
%			generally want to keep the \alert{matrix} of the
%			calibration the same as the sample.
%%		\item A (small) amount of standard can be added to the sample
%%			and a calibration curve generated.
%		\item If we know
%			\begin{equation*}
%				E  =  k + \beta (2.303)\frac{RT}{nF} \log [\ch{X}]
%			\end{equation*}
%			and using $V_0$ as the volume of unknown, $V_s$ the
%			volume of standard of concentration $C_s$, we can plug
%			in and rearrange the equation as:
%			\begin{equation*}
%				\underbrace{(V_0 + V_s) 10^{E/S}}_{y} =
%				\underbrace{10^{k/S} V_0C_x}_{b} +
%				\underbrace{10^{k/S}
%				C_s}_{m}\underbrace{V_s}_{x}
%			\end{equation*}
%			where $S$ is the slope of the line ($\frac{2.303\beta
%			RT}{nF}$).
%	\end{itemize}
%}
%
%\only<+>{%
%			Recall a standard
%			additions plot yields and $x$-intercept where unknown is
%			related to the negative of known:
%			\begin{equation*}
%				x\text{-intercept} = -V_0C_x/C_s
%			\end{equation*}
%		}
%\end{frame}

\vspace{\stretch{-1}}

\begin{frame}{Solid-State Chemical Sensors}
	\only<+>{%
	\begin{itemize}
		\item New (perhaps, very common at this point in time)
			technology has allowed use of \emph{semiconductors} for
			sensing purposes.
		\item In a \emph{chemical-sensing field effect transistor}, a
			chemically sensitive layer is affected by changes in
			analyte concentration.
		\item This alters the voltage that must be supplied by an
			external circuit to maintain a constant current between
			the drain and source.
		\item Key features:
			\begin{itemize}
				\item Can be very rugged.
				\item Can be made very small!
					\begin{itemize}
						\item Where else can we find
							transistors?
					\end{itemize}
			\end{itemize}
	\end{itemize}
}

\only<+>{%
	\begin{center}
		\includegraphics[scale=0.8]{fet-ise.png}
	\end{center}
}
\end{frame}

\vspace{\stretch{-1}}

\frame{\section{Redox Titrations}
	\begin{learningobjectives}
\item Calculate the cell potential at any point during a redox titration.
\item Know analogous redox titration shortcuts to acid-base titration shortcuts.
\item Find the equivalence point for a redox titration.
\end{learningobjectives}}

\begin{frame}{Redox Titrations vs Acid-Base Titrations}
	\begin{itemize}[<+->]
		\item We generally considered titration of a solution via the
			addition of protons (\ch{H+}).
		\item Can we also consider titrations by adding in electrons?
			\begin{itemize}
				\item We can't add electrons directly\ldots
				\item But we can add oxidizing or reducing
					agents to accept or donate electrons.
			\end{itemize}
	\end{itemize}
\end{frame}

\begin{frame}{A Sample Redox Titration}
	Let's consider the titration of \SI{100.0}{\milli\liter} of
	\SI{0.100}{\Molar} iron(II) using \SI{0.200}{\Molar} cerium(IV) as the
	titrant monitored potentiometrically with a Pt wire and a SCE.
	\begin{reaction*}
		!( "\parbox{\widthof{~titrant~}}{\centering Ceric\\ titrant}" )( Ce^{4+} ) +
		!( "\parbox{\widthof{~analyte~}}{\centering Ferrous\\ analyte}" )( Fe^{2+} ) ->
		!(Cerous)( Ce^{3+} ) + !(Ferric)( Fe^{3+} )
	\end{reaction*}
	\pause
	What are the half reactions? (AKA what's going on here?)
	\pause
	\begin{equation*}
		\left.\begin{aligned}
		\ch{Fe^{3+} + \el{} &<=> Fe^{2+}} &&\quad
		\state[pre=]{E} =
		\SI{0.767}{\volt} \\
		\ch{Ce^{4+} + \el{} &<=> Ce^{3+}} &&\quad
		\state[pre=]{E} =
		\SI{1.70}{\volt}
		\end{aligned}\right\}
		\text{\footnotesize Formal potentials in \SI{1}{\Molar}
	\ch{HClO4}}
	\end{equation*}
	\pause
	How many regions will the titration curve have?
	\pause
	\begin{enumerate}
		\item Before the equivalence point
		\item At the equivalence point
		\item After the equivalence point
	\end{enumerate}
\end{frame}

\begin{frame}{Apparatus for potentiometric titration of \ch{Fe^{2+}} with
	\ch{Ce^{4+}}}
	\begin{center}
		\includegraphics[scale=0.75,trim={3in 0 0 0},clip]{FeCe-apparatus.png}
	\end{center}
\end{frame}

\vspace{\stretch{-1}}

\begin{frame}{Before the Equivalence Point}
	\begin{itemize}[<+->]
		\item As we add \ch{Ce^{4+}}, an equal number of \ch{Ce^{3+}}
			and \ch{Fe^{3+}} ions are formed.
		\item An excess of \ch{Fe^{2+}} exists, thus finding
			[\ch{Fe^{2+}}] and [\ch{Fe^{3+}}] should be simple.
		\item Finding [\ch{Ce^{4+}}] will require solving for
			equilibrium.
		\item We will start with the iron redox pair for convenience:
			\begin{equation*}
				\ch{Fe^{3+} + \el{} <=> Fe^{2+}} \qquad
				\state[pre=]{E} =
				\SI{0.767}{\volt}
			\end{equation*}
		\item What is the Nernst equation?
		\item<.> What is the measured potential at
			\SIlist{10.0;20.0;40.0}{\milli\liter} \ch{Ce^{4+}}?
	\end{itemize}

\note<.>{
	\begin{align*}
		E &= E_+ - E_- \\
		E &= \left(
			0.767
			 - 0.05916 \log
		 \frac{[\ch{Fe^{2+}}]}{[\ch{Fe^{3+}}]} \right) -
		0.241
		\intertext{simplifying}
		E &= 0.526 - 0.05916 \log
			\frac{[\ch{Fe^{2+}}]}{[\ch{Fe^{3+}}]}
	\end{align*}
	}
\end{frame}

\note{
	At \SI{10.0}{\milli\liter}:
	\begin{align*}
		\left(\frac{\SI{0.100}{\mole}}{\SI{1000}{\milli\liter}}\right)(\SI{100.0}{\milli\liter})
		&= \SI{0.0100}{\mole}~\ch{Fe^{2+}} \\
		(\SI{10.0}{\milli\liter})\left(\frac{\SI{0.200}{\mole}}{\SI{1000}{\milli\liter}}\right)
		&= \SI{0.00200}{\mole}~\ch{Ce^{4+}}~\text{added}
		\intertext{Thus, \SI{0.00200}{\mole} \ch{Fe^{2+}} reacted.}
		[\ch{Fe^{2+}}] &= \frac{0.0100 - 0.00200}{100.0 + 10.0} =
		\SI{0.0727}{\Molar} \\
		[\ch{Fe^{3+}}] &= \frac{0.00200}{100.0 + 10.0} =
		\SI{0.0182}{\Molar}
		\intertext{Solve for $E$:}
		E &= \SI{0.490}{\volt}
	\end{align*}
	}

\note{
	At \SI{20.0}{\milli\liter}:
	\begin{align*}
		(\SI{20.0}{\milli\liter})\left(\frac{\SI{0.200}{\mole}}{\SI{1000}{\milli\liter}}\right)
		&= \SI{0.00400}{\mole}~\ch{Ce^{4+}}~\text{added}
		\intertext{Thus, \SI{0.00400}{\mole} \ch{Fe^{2+}} reacted.}
		[\ch{Fe^{2+}}] &= \frac{0.0100 - 0.00400}{100.0 + 20.0} =
		\SI{0.0500}{\Molar} \\
		[\ch{Fe^{3+}}] &= \frac{0.00400}{100.0 + 20.0} =
		\SI{0.0333}{\Molar}
		\intertext{Solve for $E$:}
		E &= \SI{0.516}{\volt}
	\end{align*}
	}

\note{
	At \SI{40.0}{\milli\liter}:
	\begin{align*}
		(\SI{40.0}{\milli\liter})(\frac{\SI{0.200}{\mole}}{\SI{1000}{\milli\liter}})
		&= \SI{0.00800}{\mole}~\ch{Ce^{4+}}~\text{added}
		\intertext{Thus, \SI{0.00800}{\mole} \ch{Fe^{2+}} reacted.}
		[\ch{Fe^{2+}}] &= \frac{0.0100 - 0.00800}{100.0 + 40.0} =
		\SI{0.0143}{\Molar} \\
		[\ch{Fe^{3+}}] &= \frac{0.00800}{100.0 + 40.0} =
		\SI{0.0571}{\Molar}
		\intertext{Solve for $E$:}
		E &= \SI{0.562}{\volt}
	\end{align*}
	}

	\clearpage{}

\begin{frame}{What happens if $E = \SI{0.526}{\volt}$?}
	\begin{align*}
		E &= 0.526 - 0.05916 \log
			\frac{[\ch{Fe^{2+}}]}{[\ch{Fe^{3+}}]} \\
			0.526 &= 0.526 - 0.05916 \log
			\frac{[\ch{Fe^{2+}}]}{[\ch{Fe^{3+}}]} \\
			0 &= - 0.05916 \log
			\frac{[\ch{Fe^{2+}}]}{[\ch{Fe^{3+}}]} = \log
			\frac{[\ch{Fe^{2+}}]}{[\ch{Fe^{3+}}]}
			\visible<2->{\intertext{Where does $\log x = 0$?}}
			\visible<3->{10^0 = 1 &=
			\frac{[\ch{Fe^{2+}}]}{[\ch{Fe^{3+}}]}}
	\end{align*}

	\visible<4->{At $\frac{1}{2}V_e$, when $[\ch{Fe^{2+}}] = [\ch{Fe^{3+}}]$, $E =
	\state[pre=]{E}$ \hfill \textbf{Analogous to \pH = \pKa!}}
\end{frame}


\begin{frame}{At the Equivalence Point}
	\begin{itemize}
		\item Exactly enough \ch{Ce^{4+}} has been added to react all
			the \ch{Fe^{2+}} with \alert{no excess}.
		\item All cerium \alert{must} be \ch{Ce^{3+}} and all iron
			\alert{must} be \ch{Fe^{3+}}.
		%	\begin{itemize}
		%		\item There will be tiny (insignificant) amounts
		%			of \ch{Ce^{4+}} and \ch{Fe^{2+}} from
		%			equilibrium.
		%	\end{itemize}
		%\item Based on stoichiometry
			\begin{align*}
				\ch{Ce^{4+} + Fe^{2+} &<=> Ce^{3+} + Fe^{3+}} \\
				[\ch{Fe^{2+}}] &= [\ch{Ce^{4+}}]
			\end{align*}
		\item We will need to solve for $E$ using both species because
			we don't know the tiny amounts of the two species:
			\begin{align*}
				E_+ &= 0.767 - 0.05916 \log
				\frac{[\ch{Fe^{2+}}]}{[\ch{Fe^{3+}}]} \\
				E_+ &= 1.70 - 0.05916 \log
				\frac{[\ch{Ce^{3+}}]}{[\ch{Ce^{4+}}]}
			\end{align*}
	\end{itemize}

	\mode<article>{\vfill}

\note{
	\vspace{-3em}

	\begin{align*}
		\intertext{Add the two equations:}
		2E_+ &= 0.767 + 1.70 - 0.05916 \log
		\frac{[\ch{Fe^{2+}}]}{[\ch{Fe^{3+}}]} - 0.05916 \log
		\frac{[\ch{Ce^{3+}}]}{[\ch{Ce^{4+}}]} \\
		&= 2.476 - 0.05916 \log \frac{[\ch{Fe^{2+}}][\ch{Ce^{3+}}]}
		{[\ch{Fe^{3+}}][\ch{Ce^{4+}}]}
		\intertext{BUT $[\ch{Fe^{2+}}] = [\ch{Ce^{4+}}]$ and
		$[\ch{Fe^{3+}}] = [\ch{Ce^{3+}}]$}
		2E_+ &= 2.467 - 0.05916 \log 1 = 2.467 \qquad \therefore
		E_+ = \SI{1.23}{\volt}
		\intertext{Solving for cell potential:}
		E_\text{cell} &= E_+ - E_- = 1.23 - 0.241 = \SI{0.99}{\volt}
	\end{align*}

	Independent of concentrations! Analogous to SA/SB
	titration!
	}

\end{frame}


\begin{frame}{After the Equivalence Point}
	\begin{itemize}
		\item All iron is now \ch{Fe^{3+}}, equal to the moles of
			\ch{Ce^{3+}}.
		\item As more titrant is added, we're just getting more
			\ch{Ce^{4+}}.
		\item It is now convenient to calculate $E$ using the
			cerium(III)/(IV) redox couple:
			\begin{align*}
				E &= E_+ -
				\underbrace{E_-}_{\mathclap{\text{SCE}}} \\
				&= \left( 1.70 - 0.05916 \log
				\frac{[\ch{Ce^{3+}}]}{[\ch{Ce^{4+}}]} \right) -
				0.241
			\end{align*}
		\item What is $E_\text{cell}$ at \SI{60.0}{\milli\liter} added
			cerium(IV)?
	\end{itemize}

\note{
	At \SI{10.0}{\milli\liter}:
	\begin{align*}
		\left(\frac{\SI{0.100}{\mole}}{\SI{1000}{\milli\liter}}\right)(\SI{100.0}{\milli\liter})
		&= \SI{0.0100}{\mole}~\ch{Fe^{3+}} \\
		(\SI{60.0}{\milli\liter})\left(\frac{\SI{0.200}{\mole}}{\SI{1000}{\milli\liter}}\right)
		&= \SI{0.0120}{\mole}~\ch{Ce^{4+}}~\text{added} \\
		0.0120 - 0.0100 &= \SI{0.0020}{\mole}~\ch{Ce^{4+}}~\text{excess}
		\\
		[\ch{Ce^{3+}}] &= \frac{0.0100}{100.0 + 60.0} =
		\SI{0.0625}{\Molar} \\
		[\ch{Ce^{4+}}] &= \frac{0.0020}{100.0 + 60.0} =
		\SI{0.0125}{\Molar}
		\intertext{Solve for $E$:}
		E &= \SI{1.42}{\volt}
	\end{align*}
	}

\end{frame}

\begin{frame}{Shape of the Titration Curve}
	\only<+>{\documentclass{standalone}

\usepackage{tikz}
\usepackage{pgfplots}
\usepackage{mathtools}
\usepackage{mhchem}
\usepackage{siunitx}
\DeclareSIUnit{\Molar}{\textsc{m}}

\setlength{\textwidth}{3in}
\setlength{\textheight}{4in}

\begin{document}

\begin{tikzpicture}
	\begin{axis} [
			width=\linewidth,
			height=20em,
			xmax=70,
			xmin=0,
			ymax=1.45,
			ymin=0.275,
			xlabel={$V_{\ch{Ce^{4+}}}$ (\si{\milli\liter})},
			ylabel={$E$ (\si{\volt} vs. SCE)},
			label style={font=\footnotesize},
			tick label style={font=\footnotesize},
			xtick={0,10,...,70},
			ytick={0.3,0.4,...,1.5},
			no markers,
    			every axis plot/.append style={ultra thick},
			]
		\addplot table [
			x=mL,
			y=E,
			col sep=comma
			]
		{fece-titrations.csv};
		\draw[dashed] (axis cs:25,0.275) -- (axis cs:25,0.526)
		node[midway,right] {$\frac{1}{2}V_e$};
		\draw[dashed] (axis cs:0,0.526) -- (axis cs:25,0.526)
			node[midway,above] {$E = \state[pre=,subscript-right={\ch{Fe^{3+}}/\ch{Fe^{2+}}}]{E}$};
		\node[fill,draw,blue,circle,inner sep=2pt](equiv) at (axis
			cs:50,0.99) {};
		\node[left = 1ex of equiv,left] {Equivalence point};
	\end{axis}
\end{tikzpicture}

\end{document}
}

	\mode<article>{\clearpage}

	\only<+>{%
	\begin{itemize}
		\item The previous curve has the equivalence point at the
			half-way point of the steep slope. \alert{This is not
			always the case!}
		\item Consider:
			\begin{reaction*}
				IO3- + 2 Tl+ + 2 Cl- + 6 H+ -> ICl2- + 2 Tl^{3+}
				+ 3 \water
			\end{reaction*}
			This is \alert{not} a 1:1 \el{}-transfer process!
	\end{itemize}
}

\only<+>{%
	\begin{center}
		\begin{tikzpicture}
			\node(tl) {\includegraphics[scale=0.8]{tl-redox-curve.png}};
			\node[right = of tl](fe)
				{\includegraphics[scale=0.75]{fe-redox-curve.png}};
		\end{tikzpicture}
	\end{center}

	In practice, the slope is so steep that this creates negligible error.
}
\end{frame}

\begin{frame}{Finding the End Point}
	\begin{itemize}
		\item We can measure the solution potentiometrically\ldots

			\pause

		\item But we can also use indicators!
			\begin{itemize}
				\item \pH{} indicators respond to \pH{}
				\item Redox indicators must respond to potential
			\end{itemize}

			\begin{reaction*}
				!(oxidized)( In ) + $n$\el{} <=> !(reduced)( In
				)
			\end{reaction*}

			\begin{equation*}
				E = \state[pre=]{E} - \frac{0.05916}{n} \log
				\left( \frac{
					[\ch{In(reduced)}]}
					{[\ch{In(oxidized)}]} \right)
			\end{equation*}
		\item What are the benefits of an indicator?
		\item What are the pitfalls of an indicator?
	\end{itemize}
\end{frame}



\end{document}
